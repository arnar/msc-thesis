\section{\label{sec::pre}Preliminaries}

In this section we present, for sake of completeness, some standard definitions from
the meta-theory of SOS that will be used in the remainder of the paper.

\begin{definition}[Signature and terms]
    We let $V$ represent an infinite set of variables and use $x,x',x_i,y,y',y_i,\dots$ to range over elements of $V$.
    A \emph{signature} $\Sigma$ is a set of function symbols, each with a fixed arity. We call these symbols
    \emph{operators} and usually represent them by $f,g,\dots$. An operator with arity zero is called a
    \emph{constant}. We define the set $\Terms\Sigma$ of \emph{terms} over $\Sigma$ as the smallest set satisfying the following constraints.
    \begin{bullets}
        \item A variable $x\in V$ is a term.
        \item If $f\in \Sigma$ has arity $n$ and $t_1,\dots,t_n$ are terms, then $f(t_1,\dots,t_n)$ is a term.
    \end{bullets}
    We use $t,t',t_i,\dots$ to range over terms. We write $t_1 \equiv t_2$ if $t_1$ and $t_2$ are syntactically equal.
    The function $vars : \Terms\Sigma \rightarrow 2^V$ gives the set of variables appearing in a term.
    The set $\CTerms\Sigma \subseteq \Terms\Sigma$ is the set of \emph{closed terms}, i.e., terms that contain no variables.
    We use $p,p',p_i,\dots$ to range over closed terms.
    A \emph{substitution} $\sigma$ is a function of type $V \rightarrow \Terms\Sigma$. We extend the domain of substitutions to terms
    homomorphically. If the range of a substitution lies in $\CTerms\Sigma$, we say that it is a \emph{closing substitution}.
\end{definition}

\begin{definition}[Transition System Specifications (TSS), formulae and transition relations]
    A \emph{transition system specification} is a triplet $(\Sigma, L, D)$ where
    \begin{bullets}
        \item $\Sigma$ is a signature.
        \item $L$ is a set of labels. If $l \in L$, and $t,t'\in \Terms\Sigma$
              we say that $t \trans{l} t'$ is a \emph{positive formula} and
              $t \ntrans{l}$ is a \emph{negative formula}. A formula, typically denoted by $\phi$, $\psi$, $\phi'$, $\phi_i$, $\ldots$
              is either a negative formula or a positive one.
        \item $D$ is a set of \emph{deduction rules}, i.e., tuples of the form $(\Phi,\phi)$ where $\Phi$ is a set of
              formulae and $\phi$ is a positive formula. We call the formulae contained in $\Phi$ the \emph{premises} of the rule and $\phi$ the
              \emph{conclusion}.
    \end{bullets}
    We write $\vars{r}$ to denote the set of variables appearing in a deduction rule $\DR{r}$.
    We say a formula is \emph{closed} if all of its terms are closed. Substitutions are also extended to formulae
    and sets of formulae in the natural way. A set of positive closed formulae is called a \emph{transition relation}.
\end{definition}

We often refer to a formula $t \trans{l} t'$ as a \emph{transition} with $t$ being its \emph{source}, $l$ its label, and $t'$
its \emph{target}.
A deduction rule $(\Phi,\phi)$ is typically written as $\frac{\Phi}{\phi}$. For a deduction rule $r$, we write $\conc{r}$ to denote its conclusion and $\prem{r}$ to denote its premises.
We call a deduction rule $f$-\emph{defining} when the outermost function symbol appearing in its source of the conclusion is $f$.


The meaning of a TSS is defined by the following notion of least three-valued stable model.
To define this notion, we need two auxiliary definitions, namely provable transition rules and contradiction, which are given below.

\begin{definition}[Provable Transition Rules]
A deduction rule is called a \emph{transition rule} when it is of the form $\frac{N}{\phi}$ with $N$ a set of {\em negative formulae}. A TSS $\tss$ \emph{proves} $\frac{N}{\phi}$, denoted by $\tss \vdash \frac{N}{\phi}$, when there is a well-founded upwardly branching tree with formulae as nodes and of which
\begin{itemize}
\item the root is labelled by $\phi$;
\item if a node is labelled by $\psi$ and the nodes above it form the set $K$ then:
\begin{itemize}
\item $\psi$ is a negative formula and $\psi \in N$, or
\item $\psi$ is a positive formula and $\frac{K}{\psi}$ is an instance of a deduction rule in $\tss$.
\end{itemize}
\end{itemize}
\end{definition}

\begin{definition}[Contradiction and Contingency]
Formula $t \trans{l} t'$ is said to \emph{contradict} $t \ntrans{l}$, and vice versa.
For two sets $\Phi$ and $\Psi$ of formulae,
$\Phi$ \emph{contradicts} $\Psi$, denoted by $\Phi \nvDash \Psi$, when there is a $\phi \in \Phi$ that contradicts a $\psi \in \Psi$.
$\Phi$ is \emph{contingent} w.r.t.\ $\Psi$, denoted by $\Phi \vDash \Psi$, when $\Phi$ does not contradict $\Psi$.
\end{definition}

It immediately follows from the above definition that contradiction and contingency are symmetric relations on (sets of) formulae.
We now have all the necessary ingredients to define the semantics of TSSs in terms of three-valued stable models.

\begin{definition}[The Least Three-Valued Stable Model]
A pair $(C, U)$ of sets of positive closed transition formulae is called a \emph{three-valued stable model} for a TSS $\tss$ when
\begin{itemize}
\item  for all $\phi \in C$, $\tss \vdash \frac{N}{\phi}$ for a set $N$ such that $C \cup U \vDash N$, and
\item  for all $\phi \in U$, $\tss \vdash \frac{N}{\phi}$ for a set $N$ such that $C \vDash N$.
\end{itemize}
$C$ stands for {\em Certainly} and $U$ for {\em Unknown}; the third value is determined by the formulae not in $C \cup U$.
The \emph{least} three-valued stable model is a three valued stable model which is the least with respect to the ordering on
pairs of sets of formulae defined as $(C,U) \leq (C', U')$ iff $C \subseteq C'$ and $U' \subseteq U$.
When for the least three-valued stable model it holds that $U=\emptyset$, we say that $\tss$ is \emph{complete}.
\end{definition}


Complete TSSs univocally define a transition relation, i.e., the $C$ component of their least three-valued stable model.
Completeness is central to almost all meta-results in the SOS meta-theory and,
as it turns out, it also plays an essential role in our meta-results concerning determinism and idempotency.
All practical instances of TSSs are complete and
there are syntactic sufficient conditions guaranteeing completeness, see for example \cite{Groote93}.
