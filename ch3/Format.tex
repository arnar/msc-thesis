\section{\label{sec:idempotency}Idempotency}
Our order of business in this section is to present a rule format
that guarantees the idempotency of certain binary operators.
In the definition of our rule format, we rely implicitly on the work presented in the previous section.


\subsection{Format}

\begin{definition}[Idempotency]
\label{def:idem}
A binary operator $f \in \Sigma$ is \emph{idempotent w.r.t.\ an equivalence} $\sim$ on closed terms if and only if  for each $p \in \CTerms{\Sigma}$, it holds that $f(p,p) \sim p$.
\end{definition}

Idempotency is  defined with respect to a notion of behavioral equivalence.
There are various notions of behavioral equivalence defined in the literature,
which are by and large, weaker than bisimilarity defined below.
Thus, to be as general as possible, we prove our idempotency result
for all notions that contain, i.e., are weaker than, bisimilarity.


\begin{definition}[Bisimulation]
\label{def:bisim}
Let $\tss$ be a TSS with signature $\Sigma$.
A relation $\rel \subseteq \CTerms{\Sigma} \times \CTerms{\Sigma}$ is a \emph{bisimulation relation}
if and only if $\rel$ is symmetric
%\leaveout{(i.e.\ $\forall_{p_0,p_1 \in \CTerms{\Sigma}}~p_0 \rel  p_1 \Leftrightarrow p_1\rel p_0$)}
and for all $p_0,p_1,p'_0 \in \CTerms{\Sigma}$ and $l \in L$
$$(p_0 \Rel p_1 \wedge \tss \vdash p_0 \trans{l} p_0') \Rightarrow \exists_{p_1' \in \CTerms{\Sigma}} (\tss \vdash p_1 \trans{l} p_1'\wedge p_0' \Rel p_1' ).$$
Two terms $p_0, p_1 \in \CTerms{\Sigma}$ are called \emph{bisimilar}, denoted by $p_0 \bisim p_1$, when there exists a bisimulation relation $\Rel$ such that
$p_0 \rel p_1$.
\end{definition}


\begin{definition}[The Idempotency Rule Format]
\label{def::format}
Let $\gamma: L \times L \rightarrow L$ be a partial function such that $\gamma(l_0,l_1) \in \{l_0,l_1\}$ if it is defined.
We define the following two rule forms.

\paragraph{$1_l$. Choice rules}
\[
    \sosrule[i\in\{0,1\}]{\{x_i\trans{l}t\}\cup\Phi}{f(x_0,x_1)\trans{l}t} \\
\]
\paragraph{$2_{l_0,l_1}$. Communication rules}
\[
    \sosrule[t_0\equiv t_1 \mbox{ or } (l_0=l_1 \mbox{ and }  \trans{l_0}\textrm{ is deterministic } )]{\{x_0\trans{l_0}t_0,\, x_1\trans{l_1}t_1\}\cup\Phi}{f(x_0,x_1)\trans{\gamma(l_0,l_1)}f(t_0,t_1)}
\]
In each case, $\Phi$ can be an arbitrary, possibly empty set of (positive or negative) formulae.

In addition, we define the starred version of each form, $1_l^*$ and $2_{l_0,l_1}^*$. The starred version of each rule is the
same as the unstarred one except that $t, t_0$ and $t_1$ are restricted to be concrete variables and the set $\Phi$ must be
empty in each case.

A TSS is in \emph{idempotency format w.r.t. a binary operator $f$} if each \mbox{$f$-defining} rule, i.e., a deduction rule with $f$ appearing in the source of the conclusion, is of the forms $1_l$
or $2_{l_0,l_1}$, for some $l, l_0, l_1 \in L$, and for each label $l \in L$ there exists at least one rule of the forms $1_l^*$ or $2_{l,l}^*$.
\end{definition}

We should note that the starred versions of the forms are special cases of their unstarred counterparts; for example a rule
which has form $1_l^*$ also has form $1_l$.


\begin{theorem}\label{thm:idempotent}
Assume that  a TSS is complete and is in the idempotency format with respect to a binary operator $f$. Then,  $f$ is idempotent w.r.t. to
any equivalence $\sim$ such that  $\bisim \subseteq{\sim}$.
% I.e. for all $p\in\CTerms\Sigma$ it holds that $f(p,p) \sim p$.
\end{theorem}
\begin{proof}
First define the relation $\eqidem \subseteq \CTerms\Sigma \times \CTerms\Sigma$ as follows.
\[
    \eqidem = \{ (p,p), (p,f(p,p)), (f(p,p),p) \,|\, p \in \CTerms\Sigma \}
\]
To prove the theorem it suffices to show that $\eqidem$ is a bisimulation relation. If it is, then $f(p,p)\bisim p$ for any closed term $p$ and since $\bisim \subseteq \sim$ we obtain the theorem.

Let $(C,U)$ be the least three-valued stable model for the TSS under consideration.
First consider a closed term $p$ s.t. $p \trans{l} p' \in C$ for some $l$ and $p'$ (note that $U=\emptyset$ since the TSS is complete).
Next, we argue that $f(p,p) \trans{l} p''$ for some $p''$ such that  $p' \eqidem p''$.
Since $p \trans{l} p' \in C$, there exists a provable transition rule of the form $\frac{N}{p \trans{l} p'}$ for some set of negative formulae $N$ such that $C \vDash N$.
In particular, that means that $p \ntrans{l} \notin N$.
In this case we make use of the requirement that there exists at least one rule of a starred form for label $l$. If there exists a
rule of the form $1_l^*$, i.e.
\[
    \sosrule[i\in\{0,1\}]{x_i\trans{l}x'}{f(x_0,x_1)\trans{l}x'}
\]
then we can instantiate it to prove that $f(p,p)\trans{l}p'\in C$.
In particular, it does not matter if $i=0$ or $i=1$.
Since $\eqidem$ is reflexive, $p'\eqidem p'$ holds.
If there exists a rule of the form $2_{l,l}^*$, we observe that $\gamma(l,l) = l$ so the transition rule becomes
\[
    \sosrule{x_0\trans{l}x_0' \quad x_1\trans{l}x_1'}{f(x_0,x_1) \trans{l} f(x_0',x_1')},
\]
where $x'_0 \equiv x'_1$ or $\trans{l}$ is deterministic.
Now we can use the existence of $p\trans{l}p'$ to satisfy both premises and obtain that $f(p,p)\trans{l}f(p',p')$.
By the definition of $\eqidem$ we also have that $p' \eqidem f(p',p')$.
In either case, if $p \trans{l} p'\in C$, then there exists a $p''$ s.t. $f(p,p) \trans{l} p'' \in C$ and $p' \eqidem p''$.

Now assume that $f(p,p) \trans{k} p' \in C$. Then there exists a provable transition rule $\frac{N}{f(p,p) \trans{k} p'}$
for some set of negative formulae $N$ such that $C\vDash N$. Since all rules for $f$ are either of the form $1_l$ or $2_{l_0,l_1}$,
this provable transition rule must be based on a rule of those forms. We analyze each possibility separately, showing that
in each case $p \trans{k} p''$ for some $p''$ such that $p' \eqidem p''$.

If the rule is based on a rule of form $1_l$, its positive premises must also be provable. In particular it must hold that
$p\trans{k} p' \in C$ since both $x_0$ and $x_1$ in the rule are instantiated to $p$. The other premises are of no
consequence to this conclusion and, again, we observe that $p'\eqidem p'$.

Now consider the case where the transition is a consequence of a rule of the form $2_{l_0,l_1}$.
If $t_0\equiv t_1$, say both are equal to $p''$, we must consider two cases, namely $k=l_0$ and $k=l_1$.
If $k=l_0$ then the first premise of the rule actually states that $p\trans{k}p''$.
If $k=l_1$ then the second premise similarly states that $p\trans{k}p''$.
In either case, we note that $p'\equiv f(p'',p'')$ must hold and again by the definition of $\eqidem$ we have
that $f(p'',p'')\eqidem p''$.

If however $t_0\not\equiv t_1$ the side condition requires that $l_0=l_1 = k$, which also implies $\gamma(l_0,l_1)=l_0=k$,
and that the transition relation $\trans{l_0}$ is deterministic.
In this case it is easy to see that the right-hand sides of the first two premises, namely $t_0$ and $t_1$, evaluate to
the same closed term in the proof structure, say $p''$.
The conclusion then states that $k=l_0$ and $p'\equiv f(p'',p'')$.
It must thus hold that $p\trans{k}p''\in C$ and $f(p'',p'')\eqidem p''$ as before.

From this we obtain that if $f(p,p) \trans{k} p' \in C$ then there exists
a $p''$ such that $p \trans{k} p'' \in C$ and $p' \eqidem p''$.
Thus, $\eqidem$ is a bisimulation.
\end{proof}


\subsection{Relaxing the restrictions} % (fold)

In this section we consider the constraints of the idempotency rule format (Definition~\ref{def::format}) and 
show that they cannot be dropped without jeopardizing the meta-theorem.

% Rule forms $1_l$ and $2_l$ only allow for variables in the targets. To see why arbitrary terms are not allowed, consider the
% following rule where $a$ is a process term.
% \[
%     \sosrule{x \trans{l} a}{f(x,y) \trans{l} a}
% \]
% If this is the only \mbox{$f$-defining} rule and the only outgoing transition for $a$ is $a \trans{l} b$, then the processes
% $a$ and $f(a,a)$ are not bisimilar as the former can do an \mbox{$l$-transition} but the latter is stuck.

First of all we note that, in rule form $1_l$, it is necessary that the label of the premise matches
the label of the conclusion.
If it does not, in general, we cannot prove that $f(p,p)$ simulates $p$ or vice versa.
This requirement can be stated more generally for both rule forms in Definition \ref{def::format};
the label of the conclusion must be among the labels of the premises.
The requirement that $\gamma(l,l') \in \{l,l'\}$ exists to ensure this constraint for form $2_{l,l'}$.
A simple synchronization rule provides a counter-example that shows why this restriction is needed.
Consider the following TSS with constants $0$, $\tau$, $a$ and $\bar{a}$ and two binary operators $+$ and $\parallel$.
\[
\begin{array}{c}
    \sosrule{}{\alpha \trans{\alpha} 0} \quad
    \sosrule{x \trans{\alpha} x'}{x + y \trans{\alpha} x' } \quad \sosrule{y \trans{\alpha} y'}{x + y \trans{\alpha} y' }
    \quad \sosrule{x \trans{a} x' \quad  y \trans{\bar{a}} y'}{x \parallel y \trans{\tau} x' \parallel y'}
\end{array}
\]
where $\alpha$ is $\tau$, $a$ or $\bar{a}$.
Here it is easy to see that although
$(a + \bar{a}) \parallel (a + \bar{a})$ has an outgoing \mbox{$\tau$-transition}, $a + \bar{a}$ does not afford such a transition.

The condition that for each $l$ at least one rule of the forms $1_l^*$ or $2_{l,l}^*$ must exist comprises a few constraints on the rule format.
First of all, it says there must be at least one \mbox{$f$-defining} rule.
If not, it is easy to see that there could exist a process $p$ where $f(p,p)$
deadlocks (since there are no \mbox{$f$-defining} rules) but $p$ does not.
It also states that there must be at
least one rule in the starred form,
where the targets are restricted to variables.
To motivate these constraints, consider the following TSS.
\[
    \sosrule{}{a \trans{a} 0} \quad
    \sosrule{x \trans{a} a}{f(x,y) \trans{a} a}
\]
The processes
$a$ and $f(a,a)$ are not bisimilar as the former can do an \mbox{$a$-transition} but the latter is stuck.
The starred forms also require that $\Phi$ is empty, i.e. there is no testing. This is necessary in the proof because in the presence
of extra premises, we cannot in general instantiate such a rule to show that $f(p,p)$ simulates $p$.
Finally, the condition requires that if we rely on a rule of the form $2_{l,l'}^*$
and $t_0 \equiv\!\!\!\!\!/~ t_1$, then the labels $l$ and $l'$ in the premises of the rule must coincide. To
see why, consider a TSS containing a {\em left synchronize} operator $\rrfloor$, one that synchronizes a step from each operand
but uses the label of the left one. Here we let $\alpha\in\{a,\bar{a}\}$.
\begin{equation*}
    \sosrule{}{\alpha \trans{\alpha} 0} \quad
    \sosrule{x \trans{\alpha} x'}{x + y \trans{\alpha} x' } \quad \sosrule{y \trans{\alpha} y'}{x + y \trans{\alpha} y' } \quad
    \sosrule{x \trans{a} x' \quad y \trans{\bar{a}} y'}{x\rrfloor\, y \trans{a} x'\rrfloor\, y'}
\end{equation*}
In this TSS the processes $(a + \bar{a})$ and $(a + \bar{a})\rrfloor\, (a + \bar{a})$ are not bisimilar
since the latter does not
afford an $\bar{a}$-transition whereas the former does.

% \MAR{Can the above problem be solved by requiring presence of a rule $5_{l',l}$? In the example, presence of a deduction rule
% \begin{equation}
%     \sosrule[]{x \trans{l'} x' \quad y \trans{l} y'}{f(x,y) \trans{l'} f(x',y')}
% \end{equation}
% solves the problem (given the requirement that $x' \equiv y'$!}
%
% \AB{Yes, I believe it can. The condition would then be: $1_l^* \lor 2_{l,l}^* \lor (2_{l,l'}^* \land 2_{l',l}^*)$. Should we go
% with that instead, or present it as an extension later?}

% AB: I liked this example of the trace operator, but it doesn't really apply to the new format
%
% For rules of type $3_{l,l'}$ we require that $\gamma(l,l')=l$ or that $t\equiv t'$. If neither of these conditions hold the format supports the following rule, defining the {\sf trace} operator found in some functional programming languages.
% \[
% \sosrule{x \trans{l} x' \quad y \trans{l'} 0}{\textsf{trace}(x,y) \trans{l'} x'}
% \]
% The idea here is that $\textsf{trace}(p,q)$ performs the input and output of $q$ while evaluating to the value of $p$.
% This operator  is not idempotent as witnessed by the (CCS style) process $p = a.0 + b.c.0$, because while
% $\textsf{trace}(p,p) \trans{a} c.0$ the only possible $a$-successor of $p$ is the nil process. The similar constraint
% on form $4_{l,l'}$ is necessary for the same reasons.

For rules of form $2_{l,l'}$ we require that either $t_0\equiv t_1$, or that the mentioned labels are the same and the
associated transition
relation is deterministic. This requirement is necessary in the proof to ensure that the
target of the conclusion fits our definition of $\eqidem$, i.e. the operator is applied to two identical terms.
Consider the following TSS where $\alpha\in\{a,b\}$.
\[
    \sosrule{}{a\trans{a} a} \quad
    \sosrule{}{a\trans{a} b} \quad
    \sosrule{}{b\trans{b} b} \quad
    \sosrule{x\trans\alpha x' \quad y\trans\alpha y'}{x | y \trans\alpha x' | y'}
\]
For the operator $|$, this violates the condition $t_0\equiv t_1$ (note that $\trans{a}$ is not deterministic).
We observe that $a|a \trans{a} a|b$. The only possibilities for $a$ to simulate this $a$-transition
is either with $a\trans{a}a$
or with $a\trans{a}b$. However, neither $a$ nor $b$ can be bisimilar to $a|b$ because both $a$ and $b$ have outgoing
transitions while $a|b$ is stuck. Therefore $a$ and $a|a$ cannot be bisimilar.
%
If $t_0 \not\equiv t_1$ we must require that the labels match, $l_0 = l_1$, and that $\trans{l_0}$ is deterministic.
We require the labels to match because if they do not, then given only $p\trans{l}p'$ it is impossible to prove that $f(p,p)$
can simulate it using only a $2_{l,l'}^*$ rule. The determinacy of the transition with that label
is necessary when proving that transitions from $f(p,p)$ can, in general, be simulated by $p$;
if we assume that $f(p,p)\trans{l}p'$ then we must be able to conclude that $p'$ has the shape $f(p'',p'')$ for some $p''$,
in order to meet the bisimulation condition for $\eqidem$. Consider the standard choice operator $+$
and prefixing operator $.$ of CCS
with the $|$ operator from the last example, with $\alpha\in\{a,b,c\}$.
\[
    \sosrule{}{\alpha \trans{\alpha} 0} \quad
    \sosrule{}{\alpha.x \trans{\alpha} x} \quad
    \sosrule{x \trans{\alpha} x'}{x + y \trans{\alpha} x' } \quad \sosrule{y \trans{\alpha} y'}{x + y \trans{\alpha} y' } \quad
    \sosrule{x\trans\alpha x' \quad y\trans\alpha y'}{x | y \trans\alpha x' | y'}
\]
If we let $p = a.b + a.c$, then $p|p \trans{a} b|c$ and $b|c$ is stuck.
However, $p$ cannot simulate this transition w.r.t. $\eqidem$.
Indeed, $p$ and $p|p$ are not bisimilar.


% subsection relaxing_the_restrictions (end)

\subsection{Predicates\label{sec::pred}}

There are many examples of TSSs where predicates are used.
The definitions presented in Section \ref{sec::pre} and  \ref{sec:idempotency} can be easily adapted to deal with predicates as well.
In particular, two closed terms are called bisimilar in this setting when,
in addition to the transfer conditions of bisimilarity, they satisfy the same predicates.
%Note that the notion of bisimulation (Definition \ref{def:bisim}), and thus the notion of idempotency (Definition \ref{def:idem}), are
%extended naturally to the setting with predicates,
%by requiring that for each two closed terms and each predicate, one term satisfies the predicate if and only if the other one satisfies the predicate.
To extend the idempotency rule format to a setting with predicates, the following types of rules for predicates are introduced:

\paragraph{$3_P$. Choice rules for predicates}
\[
    \sosrule[i\in\{0,1\}]{\{P x_i\}\cup\Phi}{P f(x_0,x_1)} \\
\]
\paragraph{$4_{P}$. Synchronization rules for predicates}
\[
    \sosrule{\{P x_0,\, P x_1\}\cup\Phi}{P f(x_0,x_1)}
\]

As before, we define the starred version of these forms, $3_P^*$ and $4_{P}^*$. The starred version of each rule is the same as the unstarred one except that the set $\Phi$ must be
empty in each case. With these additional definition the idempotency format is defined as follows.

A TSS is in \emph{idempotency format w.r.t. a binary operator $f$} if each \mbox{$f$-defining} rule, i.e., a deduction rule with $f$ appearing in the source of the conclusion, is of one the forms $1_l$, $2_{l_0,l_1}$, $3_P$ or $4_P$ for some $l, l_0, l_1 \in L$, for each label $l \in L$  and predicate symbol $P$. Moreover, for each $l \in L$, there exists at least one rule of the forms $1_l^*$ or $2_{l,l}^*$, and for each predicate symbol $P$ there is a rule of the form $1^*_P$ or $2_P^*$.

\subsection{Examples} % (fold)

\begin{example}
The most prominent example of an idempotent operator is non-deterministic choice, denoted $+$. It typically has the following deduction rules:
\[
\sosrule{x_0 \trans{a} x_0'}{x_0 + x_1 \trans{a} x_0'} \qquad
\sosrule{x_1 \trans{a} x_1'}{x_0 + x_1 \trans{a} x_1'}
\]
Clearly, these are in the idempotency format w.r.t.\ $+$.
\end{example}

\begin{example}[External choice]
The well-known external choice operator $\Box$ from CSP \cite{Hoare85} has the following deduction rules
\[
\sosrule{x_0 \trans{a} x'_0}{x_0 \Box x_1 \trans{a} x'_0}
\quad
\sosrule{x_1 \trans{a} x'_1}{x_0 \Box x_1 \trans{a} x'_1}
\quad
\sosrule{x_0 \trans{\tau} x'_0}{x_0 \Box x_1 \trans{\tau} x'_0 \Box x_1}
\quad
\sosrule{x_1 \trans{\tau} x'_1}{x_0 \Box x_1 \trans{\tau} x_0 \Box x'_1}
\]
Note that the third and fourth deduction rule are not instances of any of the allowed types of deduction rules.
Therefore, no conclusion about the validity of idempotency can be drawn from our format.
In this case this does not point to a limitation of our format,
because this operator is not idempotent in strong bisimulation semantics \cite{dArgenio95}.
\end{example}

\begin{example}[Strong time-deterministic choice]
The choice operator that is used in the timed process algebra ATP \cite{Nicollin94} has the following deduction rules.
\[
\sosrule{x_0 \trans{a} x'_0}{x_0 \oplus x_1 \trans{a} x'_0}
\qquad
\sosrule{x_1 \trans{a} x'_1}{x_0 \oplus x_1 \trans{a} x'_1}
\qquad
\sosrule{x_0 \trans{\chi} x'_0 \quad x_1 \trans{\chi} x'_1}{x_0 \oplus x_1 \trans{\chi} x'_0 \oplus x'_1}
\]
The idempotency of this operator follows from our format since the last rule for $\oplus$ fits
the form $2_{\chi,\chi}^*$ because, as we remarked in Example~\ref{ex:timetrans1}, the transition relation
$\trans{\chi}$ is deterministic.
\end{example}

\begin{example}[Weak time-deterministic choice]
The choice operator $+$ that is used in most ACP-style timed process algebras~\cite{Baeten02} has the following deduction rules:
\[
\sosrule{x_0 \trans{a} x_0'}{x_0 + x_1 \trans{a} x_0'} \qquad
\sosrule{x_1 \trans{a} x_1'}{x_0 + x_1 \trans{a} x_1'}\]
\[\sosrule{x_0 \trans{1} x'_0 \quad x_1 \trans{1} x'_1}{x_0 + x_1 \trans{1} x'_0 + x'_1} \qquad
\sosrule{x_0 \trans{1} x'_0 \quad x_1 \ntrans{1}}{x_0 + x_1 \trans{1} x'_0} \qquad
\sosrule{x_0 \ntrans{1}  \quad x_1 \trans{1} x'_1}{x_0 + x_1 \trans{1} x'_1}
\]
The third deduction rule is of the form $2^*_{1,1}$, the others are of forms $1^*_a$ and $1^*_1$.
This operator is idempotent (since the transition relation $\trans{1}$ is deterministic, as remarked
in Example~\ref{ex:timetrans2}).
\end{example}

\begin{example}[Conjunctive nondeterminism]
The operator $\lor$ as defined in Example \ref{cnp} by means of the deduction rules
\[
%\sosrule{}{0 \cando{a}}
%\qquad
%\sosrule{}{a.x \cando{a}}
%\qquad
\sosrule{x \cando{a}}{x \lor y \cando{a}}
\qquad
\sosrule{y \cando{a}}{x \lor y \cando{a}}
%\]
%\[
%\sosrule{}{0 \after{a} 0}
%\quad
%\sosrule{}{a.x \after{a} x}
%\quad
%\sosrule{}{a.x \after{b} 0} ~a \neq b
\quad
\sosrule{x \after{a} x' \quad y \after{a} y'}{x \lor y  \after{a} x' \lor y'}
\]
satisfies the idempotency format (extended to a setting with predicates). The first two deduction rules are of the form $3^*_{\cando{a}}$ and the last one is of the form $2^*_{a,a}$. Here we have used the fact that the transition relations $\after{a}$ are deterministic as concluded in Example \ref{cnp}.
\end{example}

\begin{example}[Delayed choice]
Delayed choice can be concluded to be idempotent in the restricted setting without $+$ and $\cdot$ by using the idempotency format and the fact that in this restricted setting the transition relations $\trans{a}$ are deterministic. (See Example \ref{dc}.)
\[
\sosrule{x \downarrow}{x \mp y \downarrow}
\qquad
\sosrule{y \downarrow}{x \mp y \downarrow}
\qquad
\sosrule{x \trans{a} x' \quad y \trans{a} y'}{x \mp y \trans{a} x' \mp y'}
\]
\[
\sosrule{x \trans{a} x' \quad y \ntrans{a}}{x \mp y \trans{a} x'}
\qquad
\sosrule{x \ntrans{a} \quad y \trans{a} y'}{x \mp y \trans{a} y'}
\]
The first two deduction rules are of form $3^*_{\downarrow}$, the third one is a $2^*_{a,a}$ rule, and the others are $1_{a}$ rules. Note that for any label a starred rule is present.

For the extensions discussed in Example \ref{dc} idempotency cannot be established using our rule format
since the transition relations are no longer deterministic. In fact, delayed choice is not idempotent in these cases.
\end{example}
% subsection examples (end)
