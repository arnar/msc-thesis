Instead of proving that $C$ is deterministic for each $l \in L$, we establish the following more general result.
We prove that, for each $l \in L$, 
\begin{equation}\label{eq:det_claim1}
    \textrm{if} \;  p \trans{l} p' \in C \cup U   \;
    \textrm{and} \; p \trans{l} p'' \in C      \quad
    \textrm{then} \quad p' \equiv p''.
\end{equation}

Assume the first two statements.
Since $p \trans{l} p' \in C \cup U$, then there exists a provable transition rule, such that
$\tss \vdash \frac{N}{p \trans{l} p'},$
for some set $N$ of negative formulae such that $C \vDash N$.
We show the claim~\eqref{eq:det_claim1} by an induction on the proof structure
for the transition rule $\frac{N}{p \trans{l} p'}$.
Let $\DR{r}$ be the last deduction rule, and $\sigma$ the substitution,
used in the proof structure for $\frac{N}{p \trans{l} p'}$.

%\AB{I need a little help in stating the induction hypothesis explicitly here, if needed.}

Similarly, since $p \trans{l} p'' \in C$,
there also exists a proof structure such that 
$\tss \vdash \frac{N'}{p \trans{l} p''}$
for some set $N'$ of negative formulae such that $C \cup U \vDash N'$.
Let $\DR{r'}$ be the last deduction rule, and $\sigma'$ the substitution,
used in the proof structure for $\tss \vdash \frac{N'}{p \trans{l} p''}$.

The proof is split in two main cases, the case when both proofs are based on the
same rule ($\DR{r} = \DR{r'}$) and the case when they are based on two distinct
rules ($\DR{r} \ne \DR{r'}$).

\paragraph{Case $\DR{r} = \DR{r'}$.}
%We first consider the case when $\DR{r}$ and $\DR{r'}$ are the same rule, 
In this case, say the rules $\DR{r}$ and $\DR{r'}$ are both the rule $\frac{\Phi}{t\trans{l}t'}$.
Obviously $\sigma(t)\equiv\sigma'(t)$ since both must be equal to $p$.
Since $\sigma(t')$ and $\sigma'(t')$ are equal to $p'$ and $p''$ respectively, 
to show our claim~\eqref{eq:det_claim1} we thus need to show that $\sigma(t') \equiv \sigma'(t')$.

% If the proof structure for $\frac{N}{p\trans{l}p'}$ consists of only one rule,
% $\DR{r}$ is an axiom and thus contains no premises. This means, since all variables
% in $t'$ are source dependent, that they must appear in the source $t$.
% Since $\sigma(t)\equiv p \equiv\sigma'(t)$ we obtain
% that $\sigma(t')\equiv\sigma'(t')$ also holds.

%If the proof structure is more complex, i.e. the set of premises $\Phi$ is non-empty.

We define the \emph{distance} of a source-dependent variable as the length of the shortest 
backward path from the variable, via premises with a label in $L$, to the variables 
in the source of conclusion. A variable in the source of the conclusion is thus 
of distance $0$.

By induction on its distance, we now show that any variable $v$, which is
source-dependent via a subset of $\{ \trans{l} \mid l \in  L\}$,
is assigned the same value by $\sigma$ and $\sigma'$, i.e. $\sigma(v)\equiv\sigma'(v)$.
The first constraint of our rule format dictates that any variable appearing in
the target $t'$ of the rule, is source-dependent via this set. Therefore if
$\sigma$ and $\sigma'$ agree on all such variables, they must also agree on $t'$.

Let $v$ be a variable appearing in the rule, which is source-dependent via some
subset of $\{ \trans{l} \mid l \in  L\}$. The base case of the induction is simple:
If the distance of $v$ is zero, this means $v$ appears in the source $t$. Since
$\sigma(t)\equiv\sigma'(t)$ it must be the case that $\sigma(v)\equiv\sigma'(v)$.

For the inductive step, assume $v$ has a non-zero distance. According
to Definition~\ref{def::varDepend} this means that $v$ appears in the target of some premise
$t_i \trans{l_i} t_i' \in \Phi$ where $l_i\in L$, and all variables appearing
in $t_i$ are also source dependent via the set $\{ \trans{l} \mid l \in L \}$.
However, each variable $w\in vars(t_i)$ has a smaller distance than $v$. By the
induction hypothesis (on variable distances), we thus have that $\sigma(w)\equiv\sigma'(w)$.

At this point, we can invoke the outer induction hypothesis, namely that of our
induction on the proof structure of $p \trans{l} p'$. Since $t_i\trans{l_i}t_i'$
is a premise of the first rule used, it must be provable with a smaller proof structure,
using either the substitution $\sigma$ or $\sigma'$.
By the induction hypothesis, the claim~\eqref{eq:det_claim1} holds for it. In other words
the target of the premise is the same whether we use $\sigma$ or $\sigma'$, i.e.
$\sigma(t_i')\equiv\sigma'(t_i')$. Since the variable $v$ appears in $t_i'$, it
must thus hold that $\sigma(v)\equiv\sigma'(v)$.

% \begin{enumerate}
%     \item Assume that $v$ appears in $t$. In this case, $\sigma(v)\equiv\sigma'(v)$ since $\sigma(t)\equiv\sigma'(t)$.
%     \item Assume that $v$ appears in the target of some premise $t_i \trans{l_i} t_i' \in \Phi$ where $l_i \in L$
%           and all variables in $t_i$ are source dependent via  a subset of $\{ \trans{l} \mid l \in  L\}$.
%           Each variable $w \in vars(t_i)$ has a distance smaller than that of $v$. Therefore, the induction
%           hypothesis (on the distance of variables) applies and we have that $\sigma(w) \equiv\sigma'(w)$.
%           This means that $\sigma(t_i)\equiv\sigma'(t_i)$. This allows
%           us to apply the induction hypothesis on the proof structure, since $\sigma(t_i \trans{l_i} t_i')$ has a proof structure
%           that is smaller than the one for $p \trans{l} p'$, to conclude that $\sigma(t'_i) \equiv \sigma'(t'_i)$.
%           Since $v$ appears in $t_i'$ it must hold that $\sigma(v)\equiv\sigma'(v)$.
% \end{enumerate}
We have thus showed that $\sigma$ and $\sigma'$ agree on the value of $v$ in all cases. 
As noted above, this holds specifically for all variables of $t'$
and we can conclude that $\sigma(t')\equiv\sigma'(t')$, or $p'\equiv p''$, 
which proves the claim~\eqref{eq:det_claim1} in the case of $\DR{r}=\DR{r'}$.

\paragraph{Case $\DR{r}\ne\DR{r'}$.}
We now consider the case where the rules $\DR{r}$ and $\DR{r'}$ are distinct.
Let $\DR{r} = \frac{\Phi}{s\trans{l}s'}$ and $\DR{r'}=\frac{\Phi'}{t\trans{l}t'}$.
We first show that the pair $(\sigma, \sigma')$ is 
determinism-respecting w.r.t.\ $(\Phi, \Phi')$ and $L$.

Assume, towards a contradiction, that the pair is not determinism-respecting.
Then there exist two positive formulae $s_i \trans{l'} s'_i$ and $t_i \trans{l'} t'_i$ 
for some $l' \in L$ among the premises of $\DR{r}$ and$\DR{r'}$, respectively, such that
$\sigma(s_i) \equiv \sigma'(t_i)$ but $\sigma(s'_i) \nequiv \sigma'(t'_i)$.
Since $s_i \trans{l} s'_i$  is a premise of $\DR{r}$, we know that $\sigma(s_i \trans{l} s'_i) \in C \cup U$ 
and it has a smaller proof structure than $p \trans{l} p' \in C \cup U$. 
Following a similar reasoning, $\sigma'(t_i \trans{l} t'_i) \in C$.
But the induction hypothesis (on the proof structure) applies and hence, we have $\sigma(s'_i) \equiv \sigma'(t'_i)$, 
which contradicts our earlier conclusion that $\sigma(s'_i) \nequiv \sigma'(t'_i)$ does not hold. 
Hence, we conclude that our assumption is false and that $(\sigma, \sigma')$ is 
determinism-respecting w.r.t.\ $(\Phi, \Phi')$ and $L$.

Since we have shown that $(\sigma, \sigma')$ is determinism respecting,
it then follows from the second condition of the determinism format that either $\sigma(s') \equiv \sigma'(t')$,
which was to be shown, or
there exist premises $\phi_i \equiv u_i \trans{l_i} u'_i$  in one deduction rule and
$\phi'_i \equiv w_i \ntrans{l_i}$ in the other deduction rule such that
$\sigma(\phi_i)$ contradicts $\sigma'(\phi'_i)$.
We show that the latter possibility leads to a contradiction, thus completing the proof.
Assume $\sigma(\phi_i)$ contradicts $\sigma'(\phi'_i)$, then we have that $\sigma(u_i) \equiv \sigma'(w_i)$.
We distinguish the following two cases based on the status of the positive and negative contradicting 
premises with respect to $\DR{r}$ and $\DR{r'}$.

\begin{enumerate}
\item
Assume that the positive formula is a premise of $\DR{r}$.
Then, $\sigma(u_i \trans{l_i} u'_i) \in C \cup U$ but  from $C \cup U \vDash N'$ and $\sigma'(\phi'_i) \in N'$, it follows that
for no $p''$, we have that $\sigma(u_i) \equiv \sigma'(w_i) \trans{l_i} p'' \in C \cup U$, thus reaching a contradiction.

\item
Assume that the positive formula is a premise of $\DR{r'}$.
Then, $\sigma'(u_i) \trans{l_i} \sigma(u'_i) \in C$ but from $C \vDash N$ and $\sigma(\phi'_i) \in N$, it follows that
for no $p_1$, we have that $\sigma(w_i) \equiv \sigma'(u_i) \trans{l_i} p_1 \in C$, hence reaching a contradiction. \qedhere
\end{enumerate}