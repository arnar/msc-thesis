Instead of proving that $C$ is deterministic for each $l \in L$, we establish the following more general result.
We prove that, for each $l \in L$, if $p \trans{l} p' \in C \cup U$ and $p \trans{l} p'' \in C$, then $p' \equiv p''$.

Since $p \trans{l} p' \in C \cup U$, then there exists a provable transition rule, such that
$\tss \vdash \frac{N}{p \trans{l} p'}$, for some set $N$ of negative formulae such that $C \vDash N$.
We show the claim by an induction on the proof structure
for the transition rule $\frac{N}{p \trans{l} p'}$.
Consider the last deduction rule $\DR{r}$ and substitution $\sigma$
used in the proof structure for $\frac{N}{p \trans{l} p'}$.

%\AB{I need a little help in stating the induction hypothesis explicitly here, if needed.}

Since $p \trans{l} p'' \in C$,
there also exists a proof structure such that $\tss \vdash \frac{N'}{p \trans{l} p''}$
for some set $N'$ of negative formulae such that $C \cup U \vDash N'$.
Again, consider the last deduction rule $\DR{r'}$ and substitution $\sigma'$ used in the proof structure for $\tss \vdash \frac{N'}{p \trans{l} p''}$.

We first consider the case when $\DR{r}$ and $\DR{r'}$ are the same rule, say $\frac{\Phi}{t\trans{l}t'}$.
Obviously $\sigma(t)\equiv\sigma'(t)$ since both must be equal to $p$.
Since $\sigma(t')$ and $\sigma'(t')$ are equal to $p'$ and $p''$ respectively, we need to
show that $\sigma(t') \equiv \sigma'(t')$.

% If the proof structure for $\frac{N}{p\trans{l}p'}$ consists of only one rule,
% $\DR{r}$ is an axiom and thus contains no premises. This means, since all variables
% in $t'$ are source dependent, that they must appear in the source $t$.
% Since $\sigma(t)\equiv p \equiv\sigma'(t)$ we obtain
% that $\sigma(t')\equiv\sigma'(t')$ also holds.

%If the proof structure is more complex, i.e. the set of premises $\Phi$ is non-empty.

We define the distance of a source-dependent variable as the length of the shortest backward path from the variable, via premises with a label in $L$, to the variables in the source of conclusion. A variable in the source of the conclusion is thus of distance 0.

For each variable $v$ that is source dependent via  a subset of $\{ \trans{l} \mid l \in  L\}$, we proceed with another induction
on the distance of $v$ to show that $\sigma(v) \equiv \sigma'(v)$. If we show this, then it follows that $\sigma(t') \equiv \sigma'(t')$ since
all variables in $t'$ are source dependent by the first constraint of our rule format.

We consider the two possible reasons for $v$ being source dependent.
\begin{enumerate}
    \item Assume that $v$ appears in $t$. In this case, $\sigma(v)\equiv\sigma'(v)$ since $\sigma(t)\equiv\sigma'(t)$.
    \item Assume that $v$ appears in the target of some premise $t_i \trans{l_i} t_i' \in \Phi$ where $l_i \in L$
          and all variables in $t_i$ are source dependent via  a subset of $\{ \trans{l} \mid l \in  L\}$.
          Each variable $w \in vars(t_i)$ has a distance smaller than that of $v$. Therefore, the induction
          hypothesis (on the distance of variables) applies and we have that $\sigma(w) \equiv\sigma'(w)$.
          This means that $\sigma(t_i)\equiv\sigma'(t_i)$. This allows
          us to apply the induction hypothesis on the proof structure, since $\sigma(t_i \trans{l_i} t_i')$ has a proof structure
          that is smaller than the one for $p \trans{l} p'$, to conclude that $\sigma(t'_i) \equiv \sigma'(t'_i)$.
          Since $v$ appears in $t_i'$ it must hold that $\sigma(v)\equiv\sigma'(v)$.
\end{enumerate}
In either case, $\sigma$ and $\sigma'$ agree on the value of $v$. Since this holds for all variables of $t'$,
we reach the conclusion we seek, namely that $\sigma(t')\equiv\sigma'(t')$.

We now consider the case where the rules $\DR{r}$ and $\DR{r'}$ are distinct.
We first show that $(\sigma, \sigma')$ is determinism-respecting w.r.t.\ $(\Phi, \Phi')$ and $L$.

Assume,  towards a contradiction, that our claim concerning determinism-respecting substitutions does not hold.
Then there exist  two positive formulae $s_i \trans{l} s'_i$ and $t_i \trans{l} t'_i$ for some $l \in L$ among the premises of $\DR{r}$ and
$\DR{r'}$, respectively, such that
$\sigma(s_i) \equiv \sigma'(t_i)$ but it does not hold that $\sigma(s'_i) \equiv \sigma'(t'_i)$.
Since $s_i \trans{l} s'_i$  is a premise of $\DR{r}$ $\sigma(s_i \trans{l} s'_i) \in C \cup U$ and it has a smaller proof structure than $p \trans{l} p' \in C \cup U$. Following a similar reasoning, $\sigma'(t_i \trans{l} t'_i) \in C$.
But the induction hypothesis (on the proof structure) applies and hence, we have $\sigma(s'_i) \equiv \sigma'(t'_i)$, which contradicts our earlier conclusion that $\sigma(s'_i) \equiv \sigma'(t'_i)$ does not hold. Hence, we conclude that  $(\sigma, \sigma')$ is determinism-respecting w.r.t.\ $(\Phi, \Phi')$ and $L$.

Since we have shown that $(\sigma, \sigma')$ is determinism respecting,
it then follows from the second condition of the determinism format that either $\sigma(conc(r)) \equiv \sigma'(conc(r'))$,
which was to be shown, or
there exist premises $\phi_i \equiv s_i \trans{l_i} s'_i$  in one deduction rule and
$\phi'_i \equiv t_i \ntrans{l_i}$ in the other deduction rule such that
$\sigma(\phi_i)$ contradicts $\sigma'(\phi'_i)$.
We show that the latter possibility leads to a contradiction, thus completing the proof.
Since $\sigma(\phi_i)$ contradicts $\sigma'(\phi'_i)$, we have that $\sigma(s_i) \equiv \sigma'(t_i)$.
We distinguish the following two cases based on the status of the positive and negative contradicting premises with respect to $\DR{r}$ and $\DR{r'}$.

\begin{enumerate}
\item
Assume that the positive formula is a premise of $\DR{r}$.
Then, $\sigma(s_i \trans{l_i} s'_i) \in C \cup U$ but  from $C \cup U \vDash N'$ and $\sigma'(\phi'_i) \in N'$, it follows that
for no $p''$, we have that $\sigma(s_i) \equiv \sigma'(t_i) \trans{l_i} p'' \in C \cup U$, thus reaching a contradiction.

\item
Assume that the positive formula is a premise of $\DR{r'}$.
Then, $\sigma'(s_i) \trans{l_i} \sigma(s'_i) \in C$ but from $C \vDash N$ and $\sigma(\phi'_i) \in N$, it follows that
for no $p_1$, we have that $\sigma(t_i) \equiv \sigma'(s_i) \trans{l_i} p_1 \in C$, hence reaching a contradiction. \qedhere
\end{enumerate}