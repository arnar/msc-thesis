\section{Introduction}
Structural Operational Semantics (SOS) \cite{Plotkin04a} is a popular method for assigning a rigorous meaning to
specification and programming languages.
The meta-theory of SOS provides powerful tools for proving semantic properties for such languages
without investing too much time on the actual proofs; it offers syntactic templates for SOS rules, called
\emph{rule formats},
which guarantee semantic properties once the SOS rules conform to the templates
(see, e.g., the references~\cite{Aceto01,Mousavi07-TCS} for surveys on the meta-theory of SOS).
There are various rule formats  in the literature for many different semantic properties, 
ranging from basic properties such as commutativity \cite{Mousavi05-IPL} and 
associativity \cite{Mousavi08-CONCUR} of operators, and congruence of behavioral 
equivalences (see, e.g., \cite{Verhoef95}) to more technical and involved ones such as 
non-interference \cite{Tini04} and (semi-)stochasticity \cite{Lanotte05}.
In this paper, we propose rule formats for two (related) properties, namely, determinism and idempotency.

Determinism is a semantic property of (a  fragment of) a language
that specifies that a program cannot evolve operationally in several different ways.
It holds for sub-languages of many process calculi and programming languages,
and it is also a crucial property for many formalisms for the description of timed systems, where time transitions are required to be deterministic, because the passage of time should not resolve any choice.

Idempotency is a property of binary composition operators requiring
that the composition of two identical specifications or programs
will result in a piece of specification or program that is equivalent to the original components.
Idempotency of a binary operator $f$ is concisely expressed by the following algebraic equation.
\[
f(x, x) = x
\]
Determinism and idempotency may seem unrelated at first sight.
However, it turns out that in order to obtain a powerful rule format
for idempotency, we need to have the determinism of certain transition relations in place. Therefore,
having a syntactic condition for determinism, apart from its intrinsic value,
results in a powerful, yet syntactic framework for idempotency.


To our knowledge, our rule format for idempotency has no precursor in the literature.
As for determinism, in \cite{Fokkink03a}, a rule format for bounded nondeterminism is presented but the case for determinism is not studied.
Also, in \cite{Ulidowski97b} a rule format is proposed to guarantee several time-related properties, including time determinism, in the settings of
Ordered SOS. In case of time determinism, their format corresponds to a subset of our rule format when translated to the setting of ordinary SOS, by means of the recipe given in \cite{Mousavi06-FSTTCS}.

We made a survey of existing deterministic process calculi and
of idempotent binary operators in the literature and we have applied our formats to them.
Our formats could cover all practical cases that we have discovered so far,
which is an indication of its expressiveness and relevance.

The rest of this paper is organized as follows. In Section \ref{sec::pre} we recall some basic definitions from the meta-theory of SOS.
In Section \ref{sec::det}, we present our rule format for determinism and prove that it does guarantee determinism for certain transition relations.
Section \ref{sec:idempotency} introduces a rule format for idempotency and proves it correct.
In Sections \ref{sec::det} and \ref{sec:idempotency},  we also provide several examples to motivate
the constraints of our rule formats and to demonstrate their practical applications.
Finally, Section \ref{sec::conc} concludes the paper and presents some directions for future research. 