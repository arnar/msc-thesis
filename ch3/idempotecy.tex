\documentclass{llncs}

\usepackage{stmaryrd, amssymb, ifthen, mdwlist, amsmath, graphicx}
\usepackage[usenames]{color}

\def\blfootnote{\xdef\@thefnmark{}\@footnotetext}


\newcommand{\assocdesim}{\textsf{ASSOC-De Simone}}
\newcommand{\Implies}{\mathrel{\Rightarrow}}
\newcommand{\Iff}{\mathrel{\Leftrightarrow}}
\newcommand{\defs}{\mathrel{\doteq}}
\newcommand{\id}[1]{\mathit{#1}}
\newcommand{\MAR}[1]{\textbf{\underline{Michel says}: #1}}
\newcommand{\MRM}[1]{\textbf{\underline{Mohammad says}: #1}}
\newcommand{\AB}[1]{\textbf{\underline{Arnar says}: #1}}


\newcommand{\sosrule}[3][0]{
\ifthenelse{\equal{#1}{0}}
  {\begin{array}[c]{c}
   #2\vspace{0.1em}\\\hline\rule[1em]{0pt}{0.1em}#3
   \end{array}}
  {\displaystyle\frac{#2}{#3}\ifthenelse{\equal{#1}{}}{}{,\quad}#1}
}
\newcommand{\sosproof}[3][0]{
\ifthenelse{\equal{#1}{0}}
  {\begin{array}[b]{c}
   #2\vspace{0.1em}\\\hline\hline\rule[1em]{0pt}{0.1em}#3
   \end{array}}
  {\begin{array}[c]{c}
   #2\vspace{0.1em}\\\hline\hline\rule[1em]{0pt}{0.1em}#3
   \end{array}\ifthenelse{\equal{#1}{}}{}{,\quad}#1}
}

\newcommand{\eqidem}{\mathrel{\simeq_f}}
\newcommand{\isomorphic}{\mathrel{\sim_i}}

%\newcommand{\assocrules}{\mathcal{R}_\textrm{a}}
%\newcommand{\assocrulest}{\mathcal{R}_\textrm{a\term}}
\newcommand{\tss}{\mathcal{T}}
\newcommand{\trans}[1]{\,{\stackrel{{#1}}{\rightarrow}}\,}
\newcommand{\cando}[1]{\mathrel{\mbox{ can}_{#1}\,}}
\newcommand{\after}[1]{\mathrel{\mbox{ after}_{#1}\,}}

\newcommand{\ntrans}[1]{\,{\stackrel{{#1}}{\nrightarrow}}\,}
\newcommand{\bisim}{\mathbin{\mbox{$\underline{\leftrightarrow}$}}}
\newcommand{\rel}{\mathcal{R}}
\newcommand{\Rel}{\,\mathcal{R}\,}
\newcommand{\cond}[1]{\textbf{Condition: }#1}
\newcommand{\term}{\checkmark}
\newcommand{\interrupt}{\vartriangleright}
\newcommand{\disrupt}{\blacktriangleright}
\newcommand{\DR}[1]{\ensuremath \mathrm{(#1)}}
\newcommand{\DRn}[1]{\ensuremath \mathrm{#1}}

\newcommand{\prem}[1]{\ensuremath \id{prem}(#1)}
\newcommand{\conc}[1]{\ensuremath \id{conc}(#1)}

\newcommand{\ntyft}{\textsf{Ntyft}}

\newcommand{\Terms}[1]{\mathbb{T}(#1)}
\newcommand{\CTerms}[1]{\mathbb{C}(#1)}

\newcommand{\vars}[1]{\mathit{vars}(#1)}



\newenvironment{bullets}{\begin{list}{$\bullet$}{\itemsep 0em\topsep 0em}\labelwidth 3mm}{\end{list}}

\definecolor{darkgreen}{rgb}{0,.6,0}
%\newcommand{\leaveout}[1]{{\color{cyan}{#1}}}
\newcommand{\leaveout}[1]{}

\def\lastname{Aceto et.al.}

\title{Rule Formats for Determinism and Idempotency\thanks{The work of Aceto, Birgisson and Ingolfsdottir has been
partially supported by the projects ``The Equational Logic of Parallel
Processes'' (nr.~060013021), and ``New Developments in Operational
Semantics'' (nr.~080039021) of the Icelandic Research Fund. Birgisson
has been further supported by a research-student grant nr.~080890008
of the Icelandic Research Fund.}}
\author{Luca Aceto\inst{1}
\and Arnar Birgisson\inst{1}
\and Anna Ingolfsdottir\inst{1} \and \\
MohammadReza Mousavi\inst{2}
\and Michel A. Reniers\inst{2}
}
\institute{
School of Computer Science, Reykjavik University,
\\ Kringlan 1, IS-103 Reykjavik, Iceland
\and
Department of Computer Science, Eindhoven University of Technology,
\\ {P.O.~Box~513}, NL-5600~MB~~Eindhoven, The Netherlands
}

\bibliographystyle{plain}
%\pagestyle{plain} % for page numbers

\begin{document}
\begin{frontmatter}

\maketitle

\begin{abstract}
Determinism is a semantic property of (a  fragment of) a language
that specifies that a program cannot evolve operationally in several different ways.
Idempotency is a property of binary composition operators requiring
that the composition of two identical specifications or programs
will result in a piece of specification or program that is equivalent to the original components.
In this paper, we propose two (related) meta-theorems for
guaranteeing determinism and idempotency of binary operators.
These meta-theorems are formulated in terms of syntactic templates
for operational semantics, called rule formats.
We show the applicability of our formats by applying them to various operational semantics
from the literature.
\end{abstract}

\end{frontmatter}

%\pagenumbering{arabic}

\section{Introduction}
Structural Operational Semantics (SOS) \cite{Plotkin04a} is a popular method for assigning a rigorous meaning to
specification and programming languages.
The meta-theory of SOS provides powerful tools for proving semantic properties for such languages
without investing too much time on the actual proofs; it offers syntactic templates for SOS rules, called
\emph{rule formats},
which guarantee semantic properties once the SOS rules conform to the templates
(see, e.g., the references~\cite{Aceto01,Mousavi07-TCS} for surveys on the meta-theory of SOS).
There are various rule formats  in the literature for many different semantic properties, ranging from basic properties such as commutativity \cite{Mousavi05-IPL} and associativity \cite{Mousavi08-CONCUR} of operators, and congruence of behavioral equivalences (see, e.g., \cite{Verhoef95}) to more technical and involved ones such as non-interference \cite{Tini04} and (semi-)stochasticity \cite{Lanotte05}.
In this paper, we propose rule formats for two (related) properties, namely, determinism and idempotency.

Determinism is a semantic property of (a  fragment of) a language
that specifies that a program cannot evolve operationally in several different ways.
It holds for sub-languages of many process calculi and programming languages,
and it is also a crucial property for many formalisms for the description of timed systems, where time transitions are required to be deterministic, because the passage of time should not resolve any choice.

Idempotency is a property of binary composition operators requiring
that the composition of two identical specifications or programs
will result in a piece of specification or program that is equivalent to the original components.
Idempotency of a binary operator $f$ is concisely expressed by the following algebraic equation.
\[
f(x, x) = x
\]
Determinism and idempotency may seem unrelated at first sight.
However, it turns out that in order to obtain a powerful rule format
for idempotency, we need to have the determinism of certain transition relations in place. Therefore,
having a syntactic condition for determinism, apart from its intrinsic value,
results in a powerful, yet syntactic framework for idempotency.


To our knowledge, our rule format for idempotency has no precursor in the literature.
As for determinism, in \cite{Fokkink03a}, a rule format for bounded nondeterminism is presented but the case for determinism is not studied.
Also, in \cite{Ulidowski97b} a rule format is proposed to guarantee several time-related properties, including time determinism, in the settings of
Ordered SOS. In case of time determinism, their format corresponds to a subset of our rule format when translated to the setting of ordinary SOS, by means of the recipe given in \cite{Mousavi06-FSTTCS}.

We made a survey of existing deterministic process calculi and
of idempotent binary operators in the literature and we have applied our formats to them.
Our formats could cover all practical cases that we have discovered so far,
which is an indication of its expressiveness and relevance.

The rest of this paper is organized as follows. In Section \ref{sec::pre} we recall some basic definitions from the meta-theory of SOS.
In Section \ref{sec::det}, we present our rule format for determinism and prove that it does guarantee determinism for certain transition relations.
Section \ref{sec:idempotency} introduces a rule format for idempotency and proves it correct.
In Sections \ref{sec::det} and \ref{sec:idempotency},  we also provide several examples to motivate
the constraints of our rule formats and to demonstrate their practical applications.
Finally, Section \ref{sec::conc} concludes the paper and presents some directions for future research. 
\section{\label{sec::pre}Preliminaries}

In this section we present, for sake of completeness, some standard definitions from
the meta-theory of SOS that will be used in the remainder of the paper.

\begin{definition}[Signature and terms]
    We let $V$ represent an infinite set of variables and use $x,x',x_i,y,y',y_i,\dots$ to range over elements of $V$.
    A \emph{signature} $\Sigma$ is a set of function symbols, each with a fixed arity. We call these symbols
    \emph{operators} and usually represent them by $f,g,\dots$. An operator with arity zero is called a
    \emph{constant}. We define the set $\Terms\Sigma$ of \emph{terms} over $\Sigma$ as the smallest set satisfying the following constraints.
    \begin{bullets}
        \item A variable $x\in V$ is a term.
        \item If $f\in \Sigma$ has arity $n$ and $t_1,\dots,t_n$ are terms, then $f(t_1,\dots,t_n)$ is a term.
    \end{bullets}
    We use $t,t',t_i,\dots$ to range over terms. We write $t_1 \equiv t_2$ if $t_1$ and $t_2$ are syntactically equal.
    The function $vars : \Terms\Sigma \rightarrow 2^V$ gives the set of variables appearing in a term.
    The set $\CTerms\Sigma \subseteq \Terms\Sigma$ is the set of \emph{closed terms}, i.e., terms that contain no variables.
    We use $p,p',p_i,\dots$ to range over closed terms.
    A \emph{substitution} $\sigma$ is a function of type $V \rightarrow \Terms\Sigma$. We extend the domain of substitutions to terms
    homomorphically. If the range of a substitution lies in $\CTerms\Sigma$, we say that it is a \emph{closing substitution}.
\end{definition}

\begin{definition}[Transition System Specifications (TSS), formulae and transition relations]
    A \emph{transition system specification} is a triplet $(\Sigma, L, D)$ where
    \begin{bullets}
        \item $\Sigma$ is a signature.
        \item $L$ is a set of labels. If $l \in L$, and $t,t'\in \Terms\Sigma$
              we say that $t \trans{l} t'$ is a \emph{positive formula} and
              $t \ntrans{l}$ is a \emph{negative formula}. A formula, typically denoted by $\phi$, $\psi$, $\phi'$, $\phi_i$, $\ldots$
              is either a negative formula or a positive one.
        \item $D$ is a set of \emph{deduction rules}, i.e., tuples of the form $(\Phi,\phi)$ where $\Phi$ is a set of
              formulae and $\phi$ is a positive formula. We call the formulae contained in $\Phi$ the \emph{premises} of the rule and $\phi$ the
              \emph{conclusion}.
    \end{bullets}
    We write $\vars{r}$ to denote the set of variables appearing in a deduction rule $\DR{r}$.
    We say a formula is \emph{closed} if all of its terms are closed. Substitutions are also extended to formulae
    and sets of formulae in the natural way. A set of positive closed formulae is called a \emph{transition relation}.
\end{definition}

We often refer to a formula $t \trans{l} t'$ as a \emph{transition} with $t$ being its \emph{source}, $l$ its label, and $t'$
its \emph{target}.
A deduction rule $(\Phi,\phi)$ is typically written as $\frac{\Phi}{\phi}$. For a deduction rule $r$, we write $\conc{r}$ to denote its conclusion and $\prem{r}$ to denote its premises.
We call a deduction rule $f$-\emph{defining} when the outermost function symbol appearing in its source of the conclusion is $f$.


The meaning of a TSS is defined by the following notion of least three-valued stable model.
To define this notion, we need two auxiliary definitions, namely provable transition rules and contradiction, which are given below.

\begin{definition}[Provable Transition Rules]
A deduction rule is called a \emph{transition rule} when it is of the form $\frac{N}{\phi}$ with $N$ a set of {\em negative formulae}. A TSS $\tss$ \emph{proves} $\frac{N}{\phi}$, denoted by $\tss \vdash \frac{N}{\phi}$, when there is a well-founded upwardly branching tree with formulae as nodes and of which
\begin{itemize}
\item the root is labelled by $\phi$;
\item if a node is labelled by $\psi$ and the nodes above it form the set $K$ then:
\begin{itemize}
\item $\psi$ is a negative formula and $\psi \in N$, or
\item $\psi$ is a positive formula and $\frac{K}{\psi}$ is an instance of a deduction rule in $\tss$.
\end{itemize}
\end{itemize}
\end{definition}

\begin{definition}[Contradiction and Contingency]
Formula $t \trans{l} t'$ is said to \emph{contradict} $t \ntrans{l}$, and vice versa.
For two sets $\Phi$ and $\Psi$ of formulae,
$\Phi$ \emph{contradicts} $\Psi$, denoted by $\Phi \nvDash \Psi$, when there is a $\phi \in \Phi$ that contradicts a $\psi \in \Psi$.
$\Phi$ is \emph{contingent} w.r.t.\ $\Psi$, denoted by $\Phi \vDash \Psi$, when $\Phi$ does not contradict $\Psi$.
\end{definition}

It immediately follows from the above definition that contradiction and contingency are symmetric relations on (sets of) formulae.
We now have all the necessary ingredients to define the semantics of TSSs in terms of three-valued stable models.

\begin{definition}[The Least Three-Valued Stable Model]
A pair $(C, U)$ of sets of positive closed transition formulae is called a \emph{three-valued stable model} for a TSS $\tss$ when
\begin{itemize}
\item  for all $\phi \in C$, $\tss \vdash \frac{N}{\phi}$ for a set $N$ such that $C \cup U \vDash N$, and
\item  for all $\phi \in U$, $\tss \vdash \frac{N}{\phi}$ for a set $N$ such that $C \vDash N$.
\end{itemize}
$C$ stands for {\em Certainly} and $U$ for {\em Unknown}; the third value is determined by the formulae not in $C \cup U$.
The \emph{least} three-valued stable model is a three valued stable model which is the least with respect to the ordering on
pairs of sets of formulae defined as $(C,U) \leq (C', U')$ iff $C \subseteq C'$ and $U' \subseteq U$.
When for the least three-valued stable model it holds that $U=\emptyset$, we say that $\tss$ is \emph{complete}.
\end{definition}


Complete TSSs univocally define a transition relation, i.e., the $C$ component of their least three-valued stable model.
Completeness is central to almost all meta-results in the SOS meta-theory and,
as it turns out, it also plays an essential role in our meta-results concerning determinism and idempotency.
All practical instances of TSSs are complete and
there are syntactic sufficient conditions guaranteeing completeness, see for example \cite{Groote93}.

\section{\label{sec::det}Determinism}
\begin{definition}[Determinism]
A transition relation $T$ is called deterministic for label $l$, when if
$p \trans{l} p' \in T$ and $p \trans{l} p'' \in T$, then $p' \equiv p''$.
\end{definition}

Before we define a format for determinism, we need two auxiliary definitions.
The first one is
the definition of source dependent variables,
which we borrow from~\cite{Mousavi05-ICALP} with minor additions.
%\MAR{It is not clear what are those additions and why we have those.}

\begin{definition}[\label{def::varDepend}Source dependency]
For a deduction rule, we define the set of \emph{source dependent} variables as the smallest set that contains
\begin{enumerate}
    \item all variables appearing in the source of the conclusion, and
    \item all variables that appear in the target of a premise where all variables in the source of that premise are source dependent.
\end{enumerate}
For a source dependent variable $v$, let $\mathcal{R}$ be the collection of transition relations appearing in a set of premises needed to show source dependency through condition 2. We say that $v$ is source dependent \emph{via} the relations in $\mathcal{R}$.
\end{definition}

Note that for a source dependent variable, the set $\mathcal{R}$ is not necessarily unique. For example, in the rule
\[
    \sosrule{y\trans{l_1}y' \quad x\trans{l_2}z \quad z\trans{l_3}y'}{f(x,y) \trans{l} y'}
\]
the variable $y'$ is source dependent both via the set $\{\trans{l_1}\}$ as well as $\{\trans{l_2},\trans{l_3}\}$.

%We also note that for any source dependent variable, it is possible to construct a chain of transition relations and
%other source dependent variables, such that by following these relations backwards through the premises

The second auxiliary definition needed for our determinism format is the definition of determinism-respecting substitutions.

\begin{definition}[\label{def::determinismRespecting}Determinism-Respecting Pairs of Substitutions]
Given a set $L$ of labels, a pair of substitutions $(\sigma,\sigma')$ is determinism-respecting w.r.t.\
a pair of sets of formulae $(\Phi, \Phi')$ and $L$ when for all two positive formulae $s \trans{l} s' \in \Phi$ and $t \trans{l} t' \in \Phi'$ such that $l \in L$, $\sigma(s) \equiv \sigma'(t)$ only if  $\sigma(s') \equiv \sigma'(t')$.
%\MAR{What is $\equiv$?}
\end{definition}

%\MAR{Would it be possible to avoid the term ``format'' for the following constraint. It is not syntactic and thus should not be called a format.}

\begin{definition}[\label{def::detHard}Determinism Format]
A TSS $\tss$ is in the determinism format w.r.t.\ a set of labels $L$,
when for each $l \in L$ the following conditions hold.
\begin{enumerate}
    \item In each deduction rule $\frac{\Phi}{t\trans{l}t'}$, each variable $v\in \vars{t'}$ is source dependent
          via a subset of $\{ \trans{l} \mid l \in L\}$, and
    \item for each pair of distinct deduction rules $\frac{\Phi_0}{t_0\trans{l}t_0'}$ and $\frac{\Phi_1}{t_1\trans{l}t_1'}$ and
          for each determinism-respecting pair of substitutions $(\sigma, \sigma')$ w.r.t. $(\Phi_0, \Phi_1)$ and $L$
          such that $\sigma(t_0)\equiv\sigma'(t_1)$, it holds that
          either $\sigma(t_0')\equiv\sigma'(t_1')$ or $\sigma(\Phi_0)$ contradicts $\sigma'(\Phi_1)$.
\end{enumerate}
\end{definition}

The first constraint in the definition above ensures that each rule in a TSS
in the determinism format, with some $l \in L$ as its label of conclusion,
can be used to prove at most one outgoing transition for each closed term.
The second requirement guarantees that no two different rules can be used to prove two distinct
$l$-labelled transitions for any closed term.

\begin{theorem}\label{th::det}
Consider a TSS with $(C, U)$ as its least three-valued stable model and a subset $L$ of its labels. If the TSS is in the determinism format w.r.t.\ $L$, then $C$ is deterministic for each $l\in L$.
\end{theorem}
\begin{proof}
Instead of proving that $C$ is deterministic for each $l \in L$, we establish the following more general result.
We prove that, for each $l \in L$, 
\begin{equation}\label{eq:det_claim1}
    \textrm{if} \;  p \trans{l} p' \in C \cup U   \;
    \textrm{and} \; p \trans{l} p'' \in C      \quad
    \textrm{then} \quad p' \equiv p''.
\end{equation}

Assume the first two statements.
Since $p \trans{l} p' \in C \cup U$, then there exists a provable transition rule, such that
$\tss \vdash \frac{N}{p \trans{l} p'},$
for some set $N$ of negative formulae such that $C \vDash N$.
We show the claim~\eqref{eq:det_claim1} by an induction on the proof structure
for the transition rule $\frac{N}{p \trans{l} p'}$.
Let $\DR{r}$ be the last deduction rule, and $\sigma$ the substitution,
used in the proof structure for $\frac{N}{p \trans{l} p'}$.

%\AB{I need a little help in stating the induction hypothesis explicitly here, if needed.}

Similarly, since $p \trans{l} p'' \in C$,
there also exists a proof structure such that 
$\tss \vdash \frac{N'}{p \trans{l} p''}$
for some set $N'$ of negative formulae such that $C \cup U \vDash N'$.
Let $\DR{r'}$ be the last deduction rule, and $\sigma'$ the substitution,
used in the proof structure for $\tss \vdash \frac{N'}{p \trans{l} p''}$.

The proof is split in two main cases, the case when both proofs are based on the
same rule ($\DR{r} = \DR{r'}$) and the case when they are based on two distinct
rules ($\DR{r} \ne \DR{r'}$).

\paragraph{Case $\DR{r} = \DR{r'}$.}
%We first consider the case when $\DR{r}$ and $\DR{r'}$ are the same rule, 
In this case, say the rules $\DR{r}$ and $\DR{r'}$ are both the rule $\frac{\Phi}{t\trans{l}t'}$.
Obviously $\sigma(t)\equiv\sigma'(t)$ since both must be equal to $p$.
Since $\sigma(t')$ and $\sigma'(t')$ are equal to $p'$ and $p''$ respectively, 
to show our claim~\eqref{eq:det_claim1} we thus need to show that $\sigma(t') \equiv \sigma'(t')$.

% If the proof structure for $\frac{N}{p\trans{l}p'}$ consists of only one rule,
% $\DR{r}$ is an axiom and thus contains no premises. This means, since all variables
% in $t'$ are source dependent, that they must appear in the source $t$.
% Since $\sigma(t)\equiv p \equiv\sigma'(t)$ we obtain
% that $\sigma(t')\equiv\sigma'(t')$ also holds.

%If the proof structure is more complex, i.e. the set of premises $\Phi$ is non-empty.

We define the \emph{distance} of a source-dependent variable as the length of the shortest 
backward path from the variable, via premises with a label in $L$, to the variables 
in the source of conclusion. A variable in the source of the conclusion is thus 
of distance $0$.

By induction on its distance, we now show that any variable $v$, which is
source-dependent via a subset of $\{ \trans{l} \mid l \in  L\}$,
is assigned the same value by $\sigma$ and $\sigma'$, i.e. $\sigma(v)\equiv\sigma'(v)$.
The first constraint of our rule format dictates that any variable appearing in
the target $t'$ of the rule, is source-dependent via this set. Therefore if
$\sigma$ and $\sigma'$ agree on all such variables, they must also agree on $t'$.

Let $v$ be a variable appearing in the rule, which is source-dependent via some
subset of $\{ \trans{l} \mid l \in  L\}$. The base case of the induction is simple:
If the distance of $v$ is zero, this means $v$ appears in the source $t$. Since
$\sigma(t)\equiv\sigma'(t)$ it must be the case that $\sigma(v)\equiv\sigma'(v)$.

For the inductive step, assume $v$ has a non-zero distance. According
to Definition~\ref{def::varDepend} this means that $v$ appears in the target of some premise
$t_i \trans{l_i} t_i' \in \Phi$ where $l_i\in L$, and all variables appearing
in $t_i$ are also source dependent via the set $\{ \trans{l} \mid l \in L \}$.
However, each variable $w\in vars(t_i)$ has a smaller distance than $v$. By the
induction hypothesis (on variable distances), we thus have that $\sigma(w)\equiv\sigma'(w)$.

At this point, we can invoke the outer induction hypothesis, namely that of our
induction on the proof structure of $p \trans{l} p'$. Since $t_i\trans{l_i}t_i'$
is a premise of the first rule used, it must be provable with a smaller proof structure,
using either the substitution $\sigma$ or $\sigma'$.
By the induction hypothesis, the claim~\eqref{eq:det_claim1} holds for it. In other words
the target of the premise is the same whether we use $\sigma$ or $\sigma'$, i.e.
$\sigma(t_i')\equiv\sigma'(t_i')$. Since the variable $v$ appears in $t_i'$, it
must thus hold that $\sigma(v)\equiv\sigma'(v)$.

% \begin{enumerate}
%     \item Assume that $v$ appears in $t$. In this case, $\sigma(v)\equiv\sigma'(v)$ since $\sigma(t)\equiv\sigma'(t)$.
%     \item Assume that $v$ appears in the target of some premise $t_i \trans{l_i} t_i' \in \Phi$ where $l_i \in L$
%           and all variables in $t_i$ are source dependent via  a subset of $\{ \trans{l} \mid l \in  L\}$.
%           Each variable $w \in vars(t_i)$ has a distance smaller than that of $v$. Therefore, the induction
%           hypothesis (on the distance of variables) applies and we have that $\sigma(w) \equiv\sigma'(w)$.
%           This means that $\sigma(t_i)\equiv\sigma'(t_i)$. This allows
%           us to apply the induction hypothesis on the proof structure, since $\sigma(t_i \trans{l_i} t_i')$ has a proof structure
%           that is smaller than the one for $p \trans{l} p'$, to conclude that $\sigma(t'_i) \equiv \sigma'(t'_i)$.
%           Since $v$ appears in $t_i'$ it must hold that $\sigma(v)\equiv\sigma'(v)$.
% \end{enumerate}
We have thus showed that $\sigma$ and $\sigma'$ agree on the value of $v$ in all cases. 
As noted above, this holds specifically for all variables of $t'$
and we can conclude that $\sigma(t')\equiv\sigma'(t')$, or $p'\equiv p''$, 
which proves the claim~\eqref{eq:det_claim1} in the case of $\DR{r}=\DR{r'}$.

\paragraph{Case $\DR{r}\ne\DR{r'}$.}
We now consider the case where the rules $\DR{r}$ and $\DR{r'}$ are distinct.
Let $\DR{r} = \frac{\Phi}{s\trans{l}s'}$ and $\DR{r'}=\frac{\Phi'}{t\trans{l}t'}$.
We first show that the pair $(\sigma, \sigma')$ is 
determinism-respecting w.r.t.\ $(\Phi, \Phi')$ and $L$.

Assume, towards a contradiction, that the pair is not determinism-respecting.
Then there exist two positive formulae $s_i \trans{l'} s'_i$ and $t_i \trans{l'} t'_i$ 
for some $l' \in L$ among the premises of $\DR{r}$ and$\DR{r'}$, respectively, such that
$\sigma(s_i) \equiv \sigma'(t_i)$ but $\sigma(s'_i) \nequiv \sigma'(t'_i)$.
Since $s_i \trans{l} s'_i$  is a premise of $\DR{r}$, we know that $\sigma(s_i \trans{l} s'_i) \in C \cup U$ 
and it has a smaller proof structure than $p \trans{l} p' \in C \cup U$. 
Following a similar reasoning, $\sigma'(t_i \trans{l} t'_i) \in C$.
But the induction hypothesis (on the proof structure) applies and hence, we have $\sigma(s'_i) \equiv \sigma'(t'_i)$, 
which contradicts our earlier conclusion that $\sigma(s'_i) \nequiv \sigma'(t'_i)$ does not hold. 
Hence, we conclude that our assumption is false and that $(\sigma, \sigma')$ is 
determinism-respecting w.r.t.\ $(\Phi, \Phi')$ and $L$.

Since we have shown that $(\sigma, \sigma')$ is determinism respecting,
it then follows from the second condition of the determinism format that either $\sigma(s') \equiv \sigma'(t')$,
which was to be shown, or
there exist premises $\phi_i \equiv u_i \trans{l_i} u'_i$  in one deduction rule and
$\phi'_i \equiv w_i \ntrans{l_i}$ in the other deduction rule such that
$\sigma(\phi_i)$ contradicts $\sigma'(\phi'_i)$.
We show that the latter possibility leads to a contradiction, thus completing the proof.
Assume $\sigma(\phi_i)$ contradicts $\sigma'(\phi'_i)$, then we have that $\sigma(u_i) \equiv \sigma'(w_i)$.
We distinguish the following two cases based on the status of the positive and negative contradicting 
premises with respect to $\DR{r}$ and $\DR{r'}$.

\begin{enumerate}
\item
Assume that the positive formula is a premise of $\DR{r}$.
Then, $\sigma(u_i \trans{l_i} u'_i) \in C \cup U$ but  from $C \cup U \vDash N'$ and $\sigma'(\phi'_i) \in N'$, it follows that
for no $p''$, we have that $\sigma(u_i) \equiv \sigma'(w_i) \trans{l_i} p'' \in C \cup U$, thus reaching a contradiction.

\item
Assume that the positive formula is a premise of $\DR{r'}$.
Then, $\sigma'(u_i) \trans{l_i} \sigma(u'_i) \in C$ but from $C \vDash N$ and $\sigma(\phi'_i) \in N$, it follows that
for no $p_1$, we have that $\sigma(w_i) \equiv \sigma'(u_i) \trans{l_i} p_1 \in C$, hence reaching a contradiction. \qedhere
\end{enumerate}
\end{proof}

For a TSS in the determinism format with $(C, U)$ as its least three-valued stable model, $U$ and thus $C \cup U$ need not be deterministic.
The following counter-example illustrates this phenomenon.

\begin{example}
Consider the TSS given by the following deduction rules.
\[
\sosrule[]{a \trans{l} a}{a \trans{l} b} \quad \sosrule[]{a \ntrans{l}}{a \trans{l} a}
\]
The above-given TSS is in the determinism format since $a \trans{l} a$ and $a \ntrans{l}$ contradict each other (under any substitution).
Its least three-valued stable model is, however, $(\emptyset, \{a \trans{l} a, a \trans{l} b\})$ and $\{a \trans{l} a, a \trans{l} b\}$ is not deterministic.
\end{example}

\begin{corollary}
Consider a complete TSS with $L$ as a subset of its labels. If the TSS is in the determinism format w.r.t. $L$, then its defined transition relation is deterministic for each $l \in L$.
\end{corollary}

Constraint 2 in Definition \ref{def::detHard} may seem difficult to verify,
since it requires checks for all possible (determinism-respecting) substitutions.
However, in practical cases, to be quoted in the remainder of this paper,
variable names are chosen in such a way that constraint 2
can be checked syntactically.
For example, consider the following two deduction rules.
\[
\sosrule{x \trans{a} x'}{f(x,y) \trans{a} x'} \qquad
\sosrule{y \ntrans{a} \quad x \trans{b} x'}{f(y,x) \trans{a} x'}
\]
If in both deduction rules $f(x,y)$ (or symmetrically $f(y,x)$) was used,
it could have been easily seen from the syntax of the rules that the premises of one deduction rule
always (under all pairs of substitutions agreeing on the value of $x$) contradict the premises of the other deduction rule and, hence, constraint 2 is trivially satisfied.
Based on this observation, we next present a rule format, whose constraints
have a purely syntactic form,
and that is sufficiently powerful to handle all the examples we discuss
in Section \ref{sec::detExamples}.
(Note that, for the examples in Section \ref{sec::detExamples}, checking the constraints of Definition \ref{def::detHard} is
not too hard either.)



\begin{definition}[Normalized TSSs]\label{def::normalized}
A TSS is normalized w.r.t. $L$ if each deduction rule is $f$-defining for some function symbol $f$, and for each label $l \in L$, each function symbol $f$ and each pair of deduction rules of the form
\[
\DR{r} \sosrule{\Phi_r}{f(\overrightarrow{s}) \trans{l} s'} \quad
\DR{r'} \sosrule{\Phi_{r'}}{f(\overrightarrow{t}) \trans{l} t'}
\]
the following constraints are satisfied:
\begin{enumerate}
\item the sources of the conclusions coincide, i.e., $f(\overrightarrow{s}) \equiv f(\overrightarrow{t})$,
\item each variable $v \in \vars{s'}$ (symmetrically $v \in \vars{t'}$) is source dependent in $\DR{r}$ (respectively in $\DR{r'}$) via some subset of $\{ \trans{l} \mid l \in L\}$,

\item for each variable $v \in \vars{r} \cap \vars{r'}$ there is a set of formulae in $\Phi_r \cap \Phi_{r'}$ proving its source dependency (both in $\DR{r}$ and $\DR{r'})$) via some subset of $\{ \trans{l} \mid l \in L\}$.
\end{enumerate}
\end{definition}

The second and third constraint in Definition \ref{def::simplifiedDet} guarantee that
the syntactic equivalence of relevant terms (the target of the conclusion or the premises negating each other)
will lead to syntactically equivalent closed terms under all determinism-respecting pairs of substitutions.



%\begin{definition}[Source-normalized rules]
%Given a countably infinite list $\widetilde{x} \equiv x_0, x_1, \ldots$ of variables.
%A $\tss$ contains only {\em source-normalized rules}, when all its deduction rules
%are of the form:
%\[
%\sosrule[]{\Phi}{f(\overrightarrow{x}) \trans{l} t'}
%\]
%where $\overrightarrow{x}$ is a prefix of $\widetilde{x}$.
%Henceforth, we assume a fixed yet arbitrary list $\widetilde{x} \equiv x_0, x_1, \ldots$ and only speak
%of source-normalized rules without mentioning the list.
%
%We refer to each rule with $f$ as the main operator of the source and $l$ as the label of the conclusion as an $(f,l)$\emph{-defining rule}.
%\end{definition}
%
%\MAR{We should indicate that for any TSS there is an equivalent (definition?) TSS with only source-normalized rules.}
The reader can check that all the examples quoted from the literature in Section \ref{sec::detExamples} are indeed normalized TSSs.


\begin{definition}[\label{def::simplifiedDet}Syntactic Determinism Format]
A normalized TSS is in the (syntactic) determinism format w.r.t.\ $L$, when
%\begin{enumerate}
%    \item in each deduction rule $\frac{\Phi}{t\trans{l}t'}$, each variable $v\in vars(t')$ is source dependent
%          via some subset of $\{ \trans{l} \mid l \in L\}$, and
%    \item
    for each two deduction rules $\frac{\Phi_0}{f(\overrightarrow{s}) \trans{l} s'}$ and $\frac{\Phi_1}{f(\overrightarrow{s})  \trans{l} s''}$, it holds that $s' \equiv s''$ or $\Phi_0$ contradicts $\Phi_1$.
%\end{enumerate}
\end{definition}

The following theorem states that for normalized TSSs, Definition \ref{def::simplifiedDet} implies Definition \ref{def::detHard}.


\begin{theorem}\label{th::simplifiedImpliesHard}
Each normalized TSS in the syntactic determinism format w.r.t.\ $L$ is also in the determinism format w.r.t.\ $L$.
\end{theorem}

For brevity, we omit the proof of Theorem \ref{th::simplifiedImpliesHard}.
%(The proof is included for reviewers' convenience in Appendix \ref{proof::simpImpHard}.)
The following statement is thus a corollary to Theorems \ref{th::simplifiedImpliesHard} and \ref{th::det}.

\begin{corollary}
Consider a normalized TSS with $(C, U)$ as its least three-valued stable model and a subset $L$ of its labels. If the TSS is in the (syntactic) determinism format w.r.t.\ $L$ (according to Definition \ref{def::simplifiedDet}), then $C$ is deterministic w.r.t.\ any $l\in L$.
\end{corollary}


\subsection{Examples\label{sec::detExamples}}
In this section, we present some examples of various TSSs from the literature and apply our (syntactic) determinism format to them.
Some of the examples we discuss below are based on TSSs with predicates. The extension of our formats with predicates is
straightforward and we discuss it in Section \ref{sec::pred} to follow.


\begin{example}[Conjunctive Nondeterministic Processes]\label{cnp}
Hennessy and Plotkin, in \cite{Hennessy87},  define a language, called conjunctive nondeterministic processes,
for studying logical characterizations of processes.
The signature of the language consists of a constant $0$, a unary action prefixing operator $a.\_$ for each $a \in A$, and
a binary conjunctive nondeterminism operator $\lor$.
The operational semantics of this language is defined by the following deduction rules.
\[
\sosrule{}{0 \cando{a}}
\qquad
\sosrule{}{a.x \cando{a}}
\qquad
\sosrule{x \cando{a}}{x \lor y \cando{a}}
\qquad
\sosrule{y \cando{a}}{x \lor y \cando{a}}
\]
\[
\sosrule{}{0 \after{a} 0}
\quad
\sosrule{}{a.x \after{a} x}
\quad
\sosrule{}{a.x \after{b} 0} ~a \neq b
\quad
\sosrule{x \after{a} x' \quad y \after{a} y'}{x \lor y  \after{a} x' \lor y'}
\]
The above TSS is in the (syntactic) determinism format with respect to the transition relation $\after{a}$ (for each $a \in A$).
Hence, we can conclude that the transition relations $\after{a}$ are deterministic.
\end{example}

\begin{example}[Delayed choice]\label{dc}
The second example we discuss is a subset of the process algebra $\mathrm{BPA}_{\delta\epsilon}+\mbox{DC}$ \cite{BaetenMauw94}, i.e., Basic Process Algebra with deadlock and empty process extended with delayed choice. First we restrict attention to the fragment of this process algebra without non-deterministic choice $+$ and with action prefix $a.\_$ instead of general sequential composition $\cdot$. This altered process algebra has the following deduction rules, where $a$ ranges over the set of actions $A$:
\[
\sosrule{}{\epsilon\downarrow}
\qquad
\sosrule{}{a.x \trans{a} x}
\qquad
\sosrule{x \downarrow}{x \mp y \downarrow}
\qquad
\sosrule{y \downarrow}{x \mp y \downarrow}
\]
\[
\sosrule{x \trans{a} x' \qquad y \trans{a} y'}{x \mp y \trans{a} x' \mp y'}
\qquad
\sosrule{x \trans{a} x' \qquad y \ntrans{a}}{x \mp y \trans{a} x'}
\qquad
\sosrule{x \ntrans{a} \qquad y \trans{a} y'}{x \mp y \trans{a} y'}
\]
In the above specification, predicate $p \downarrow$ denotes the possibility of termination for process $p$.
The intuition behind the delayed choice operator, denoted by $\_\ \mp\ \_$, is that the choice between
two components is only resolved when one performs an action that the other cannot perform.
When both components can perform an action, the delayed choice between them remains unresolved and
the two components synchronize on the common action.
This transition system specification is in the (syntactic) determinism format w.r.t.\ $\{ a \mid a \in A\}$.

Addition of non-deterministic choice $+$ or sequential composition $\cdot$ results in deduction rules that do not satisfy the determinism format. For example, addition of sequential composition comes with the following deduction rules:
\[ \sosrule{x \trans{a} x'}{x \cdot y \trans{a} x' \cdot y}
\qquad
\sosrule{x \downarrow \quad y \trans{a} y'}{x \cdot y \trans{a} y'}
\]
The sets of premises of these rules do not contradict each other.
The extended TSS is indeed non-deterministic since, for example, $(\epsilon \mp (a.\epsilon))\cdot (a.\epsilon) \trans{a} \epsilon$ and
$(\epsilon \mp (a.\epsilon))\cdot (a.\epsilon) \trans{a} \epsilon \cdot (a.\epsilon)$.
\end{example}

\begin{example}[Time transitions I]\label{ex:timetrans1}
This example deals with the Algebra of Timed Processes, ATP, of Nicollin and Sifakis \cite{Nicollin94}.
In the TSS given below, we specify the time transitions (denoted by label $\chi$) of delayable deadlock $\delta$,
non-deterministic choice $\_~\oplus~\_$, unit-delay operator $\lfloor\_\rfloor\_$ and parallel composition $\_~\parallel~\_$.
\[
\sosrule{}{\delta \trans{\chi} \delta}
\qquad
\sosrule{x \trans{\chi} x' \quad y \trans{\chi} y'}{x \oplus y \trans{\chi} x' \oplus y'}
\qquad
\sosrule{}{\lfloor x \rfloor (y) \trans{\chi} y}
\qquad
\sosrule{x \trans{\chi} x' \quad y \trans{\chi} y'}{x \parallel y \trans{\chi} x' \parallel y'}
\]
These deduction rules all trivially satisfy the determinism format for time transitions since the sources of conclusions of different deduction rules cannot be unified. Also the additional operators involving time, namely, the delay operator $\lfloor \_ \rfloor^{d}\_$, execution delay operator
$\lceil \_ \rceil^{d}\_$ and unbounded start delay operator $\lfloor \_ \rfloor^{\omega}$, satisfy the determinism format for time transitions. The deduction rules are given below, for $d \geq 1$:
%start delay within d
\[
\sosrule{}{\lfloor x \rfloor^1(y) \trans{\chi} y}
\qquad
\sosrule{x \trans{\chi} x'}{\lfloor x \rfloor^{d+1} (y) \trans{\chi} \lfloor x' \rfloor^{d}(y)}
\qquad
\sosrule{x \ntrans{\chi}}{\lfloor x \rfloor^{d+1} (y) \trans{\chi} \lfloor x \rfloor^{d}(y)}
\]
%unbounded start delay
\[
\sosrule{x \trans{\chi} x'}{\lfloor x \rfloor^{\omega} \trans{\chi} \lfloor x' \rfloor^{\omega}}
\qquad
\sosrule{x \ntrans{\chi}}{\lfloor x \rfloor^{\omega} \trans{\chi} \lfloor x \rfloor^{\omega}}
\]
%execution delay within d
\[
\sosrule{x \trans{\chi} x'}{\lceil x \rceil^1(y) \trans{\chi} y}
\qquad
\sosrule{x \trans{\chi} x'}{\lceil x \rceil^{d+1} (y) \trans{\chi} \lceil x' \rceil^{d}(y)}
\]
\end{example}

\begin{example}[Time transitions II]\label{ex:timetrans2}
Most of the timed process algebras that originate from the Algebra of Communicating Processes (ACP) from \cite{Bergstra84,Baeten90} such as those reported in \cite{Baeten02} have a deterministic time transition relation as well.

In the TSS given below, the time unit delay operator is denoted by $\sigma_{\mathrm{rel}} \_$, nondeterministic choice is denoted by $\_~+~\_$,
and sequential composition is denoted by $\_ \cdot \_$.
The deduction rules for the time transition relation for this process algebra are the following:
\[ \sosrule{}{\sigma_{\mathrm{rel}}(x) \trans{1} x}
\qquad
\sosrule{x \trans{1} x' \quad y \trans{1} y'}{x + y \trans{1} x'+y'}
\qquad
\sosrule{x \trans{1} x' \quad y \ntrans{1} }{x + y \trans{1} x'}
\qquad
\sosrule{x \ntrans{1} \quad y \trans{1} y'}{x + y \trans{1} y'}
\]
\[
\sosrule{x \trans{1} x' \quad x \not\downarrow}{x \cdot y \trans{1} x' \cdot y}
\qquad
\sosrule{x \trans{1} x' \quad y \ntrans{1}}{x \cdot y \trans{1} x' \cdot y}
\qquad
\sosrule{x \trans{1} x' \quad x \downarrow \quad y \trans{1} y'}{x \cdot y \trans{1} x' \cdot y + y'}
\qquad
\sosrule{x \ntrans{1} \quad x \downarrow \quad y \trans{1} y'}{x \cdot y \trans{1} y'}
\]
Note that here we have an example of deduction rules, the first two deduction rules for time transitions of a sequential composition, for which the premises do not contradict each other. Still these deduction rules satisfy the determinism format since the targets of the conclusions are identical. In the syntactically richer framework of \cite{Reniers08}, where arbitrary first-order logic formulae over transitions are allowed, those two deduction rules are presented by a single rule with premise $x \trans{1} x' \land (x \not\downarrow \lor y \ntrans{1})$.

Sometimes such timed process algebras have an operator for specifying an arbitrary delay, denoted by $\sigma^*_{\mathrm{rel}} \_$, with the following deduction rules.
\[ \sosrule{x \ntrans{1}}{\sigma^*_{\mathrm{rel}}(x) \trans{1} \sigma^*_{\mathrm{rel}}(x)}
\qquad
\sosrule{x \trans{1} x'}{\sigma^*_{\mathrm{rel}}(x) \trans{1} x' + \sigma^*_{\mathrm{rel}}(x)}
\]
The premises of these rules contradict each other and so, the semantics of this operator also satisfies the constraints of our (syntactic)
determinism format.
\end{example}

\section{\label{sec:idempotency}Idempotency}
Our order of business in this section is to present a rule format
that guarantees the idempotency of certain binary operators.
In the definition of our rule format, we rely implicitly on the work presented in the previous section.


\subsection{Format}

\begin{definition}[Idempotency]
\label{def:idem}
A binary operator $f \in \Sigma$ is \emph{idempotent w.r.t.\ an equivalence} $\sim$ on closed terms if and only if  for each $p \in \CTerms{\Sigma}$, it holds that $f(p,p) \sim p$.
\end{definition}

Idempotency is  defined with respect to a notion of behavioral equivalence.
There are various notions of behavioral equivalence defined in the literature,
which are by and large, weaker than bisimilarity defined below.
Thus, to be as general as possible, we prove our idempotency result
for all notions that contain, i.e., are weaker than, bisimilarity.


\begin{definition}[Bisimulation]
\label{def:bisim}
Let $\tss$ be a TSS with signature $\Sigma$.
A relation $\rel \subseteq \CTerms{\Sigma} \times \CTerms{\Sigma}$ is a \emph{bisimulation relation}
if and only if $\rel$ is symmetric
%\leaveout{(i.e.\ $\forall_{p_0,p_1 \in \CTerms{\Sigma}}~p_0 \rel  p_1 \Leftrightarrow p_1\rel p_0$)}
and for all $p_0,p_1,p'_0 \in \CTerms{\Sigma}$ and $l \in L$
$$(p_0 \Rel p_1 \wedge \tss \vdash p_0 \trans{l} p_0') \Rightarrow \exists_{p_1' \in \CTerms{\Sigma}} (\tss \vdash p_1 \trans{l} p_1'\wedge p_0' \Rel p_1' ).$$
Two terms $p_0, p_1 \in \CTerms{\Sigma}$ are called \emph{bisimilar}, denoted by $p_0 \bisim p_1$, when there exists a bisimulation relation $\Rel$ such that
$p_0 \rel p_1$.
\end{definition}


\begin{definition}[The Idempotency Rule Format]
\label{def::format}
Let $\gamma: L \times L \rightarrow L$ be a partial function such that $\gamma(l_0,l_1) \in \{l_0,l_1\}$ if it is defined.
We define the following two rule forms.

\paragraph{$1_l$. Choice rules}
\[
    \sosrule[i\in\{0,1\}]{\{x_i\trans{l}t\}\cup\Phi}{f(x_0,x_1)\trans{l}t} \\
\]
\paragraph{$2_{l_0,l_1}$. Communication rules}
\[
    \sosrule[t_0\equiv t_1 \mbox{ or } (l_0=l_1 \mbox{ and }  \trans{l_0}\textrm{ is deterministic } )]{\{x_0\trans{l_0}t_0,\, x_1\trans{l_1}t_1\}\cup\Phi}{f(x_0,x_1)\trans{\gamma(l_0,l_1)}f(t_0,t_1)}
\]
In each case, $\Phi$ can be an arbitrary, possibly empty set of (positive or negative) formulae.

In addition, we define the starred version of each form, $1_l^*$ and $2_{l_0,l_1}^*$. The starred version of each rule is the
same as the unstarred one except that $t, t_0$ and $t_1$ are restricted to be concrete variables and the set $\Phi$ must be
empty in each case.

A TSS is in \emph{idempotency format w.r.t. a binary operator $f$} if each \mbox{$f$-defining} rule, i.e., a deduction rule with $f$ appearing in the source of the conclusion, is of the forms $1_l$
or $2_{l_0,l_1}$, for some $l, l_0, l_1 \in L$, and for each label $l \in L$ there exists at least one rule of the forms $1_l^*$ or $2_{l,l}^*$.
\end{definition}

We should note that the starred versions of the forms are special cases of their unstarred counterparts; for example a rule
which has form $1_l^*$ also has form $1_l$.


\begin{theorem}\label{thm:idempotent}
Assume that  a TSS is complete and is in the idempotency format with respect to a binary operator $f$. Then,  $f$ is idempotent w.r.t. to
any equivalence $\sim$ such that  $\bisim \subseteq{\sim}$.
% I.e. for all $p\in\CTerms\Sigma$ it holds that $f(p,p) \sim p$.
\end{theorem}

%\begin{proof}
%For review purposes the proof is given in Appendix \ref{proof:idempotent}. \hfill \qed
%\end{proof}


\subsection{Relaxing the restrictions} % (fold)

In this section we consider the constraints of the idempotency rule format and show that they cannot be dropped without jeopardizing the meta-theorem.

% Rule forms $1_l$ and $2_l$ only allow for variables in the targets. To see why arbitrary terms are not allowed, consider the
% following rule where $a$ is a process term.
% \[
%     \sosrule{x \trans{l} a}{f(x,y) \trans{l} a}
% \]
% If this is the only \mbox{$f$-defining} rule and the only outgoing transition for $a$ is $a \trans{l} b$, then the processes
% $a$ and $f(a,a)$ are not bisimilar as the former can do an \mbox{$l$-transition} but the latter is stuck.

First of all we note that, in rule form $1_l$, it is necessary that the label of the premise matches
the label of the conclusion.
If it does not, in general, we cannot prove that $f(p,p)$ simulates $p$ or vice versa.
This requirement can be stated more generally for both rule forms in Definition \ref{def::format};
the label of the conclusion must be among the labels of the premises.
The requirement that $\gamma(l,l') \in \{l,l'\}$ exists to ensure this constraint for form $2_{l,l'}$.
A simple synchronization rule provides a counter-example that shows why this restriction is needed.
Consider the following TSS with constants $0$, $\tau$, $a$ and $\bar{a}$ and two binary operators $+$ and $\parallel$.
\[
\begin{array}{c}
    \sosrule{}{\alpha \trans{\alpha} 0} \quad
    \sosrule{x \trans{\alpha} x'}{x + y \trans{\alpha} x' } \quad \sosrule{y \trans{\alpha} y'}{x + y \trans{\alpha} y' }
    \quad \sosrule{x \trans{a} x' \quad  y \trans{\bar{a}} y'}{x \parallel y \trans{\tau} x' \parallel y'}
\end{array}
\]
where $\alpha$ is $\tau$, $a$ or $\bar{a}$.
Here it is easy to see that although
$(a + \bar{a}) \parallel (a + \bar{a})$ has an outgoing \mbox{$\tau$-transition}, $a + \bar{a}$ does not afford such a transition.

The condition that for each $l$ at least one rule of the forms $1_l^*$ or $2_{l,l}^*$ must exist comprises a few constraints on the rule format.
First of all, it says there must be at least one \mbox{$f$-defining} rule.
If not, it is easy to see that there could exist a process $p$ where $f(p,p)$
deadlocks (since there are no \mbox{$f$-defining} rules) but $p$ does not.
It also states that there must be at
least one rule in the starred form,
where the targets are restricted to variables.
To motivate these constraints, consider the following TSS.
\[
    \sosrule{}{a \trans{a} 0} \quad
    \sosrule{x \trans{a} a}{f(x,y) \trans{a} a}
\]
The processes
$a$ and $f(a,a)$ are not bisimilar as the former can do an \mbox{$a$-transition} but the latter is stuck.
The starred forms also require that $\Phi$ is empty, i.e. there is no testing. This is necessary in the proof because in the presence
of extra premises, we cannot in general instantiate such a rule to show that $f(p,p)$ simulates $p$.
Finally, the condition requires that if we rely on a rule of the form $2_{l,l'}^*$
and $t_0 \equiv\!\!\!\!\!/~ t_1$, then the labels $l$ and $l'$ in the premises of the rule must coincide. To
see why, consider a TSS containing a {\em left synchronize} operator $\rrfloor$, one that synchronizes a step from each operand
but uses the label of the left one. Here we let $\alpha\in\{a,\bar{a}\}$.
\begin{equation*}
    \sosrule{}{\alpha \trans{\alpha} 0} \quad
    \sosrule{x \trans{\alpha} x'}{x + y \trans{\alpha} x' } \quad \sosrule{y \trans{\alpha} y'}{x + y \trans{\alpha} y' } \quad
    \sosrule{x \trans{a} x' \quad y \trans{\bar{a}} y'}{x\rrfloor\, y \trans{a} x'\rrfloor\, y'}
\end{equation*}
In this TSS the processes $(a + \bar{a})$ and $(a + \bar{a})\rrfloor\, (a + \bar{a})$ are not bisimilar
since the latter does not
afford an $\bar{a}$-transition whereas the former does.

% \MAR{Can the above problem be solved by requiring presence of a rule $5_{l',l}$? In the example, presence of a deduction rule
% \begin{equation}
%     \sosrule[]{x \trans{l'} x' \quad y \trans{l} y'}{f(x,y) \trans{l'} f(x',y')}
% \end{equation}
% solves the problem (given the requirement that $x' \equiv y'$!}
%
% \AB{Yes, I believe it can. The condition would then be: $1_l^* \lor 2_{l,l}^* \lor (2_{l,l'}^* \land 2_{l',l}^*)$. Should we go
% with that instead, or present it as an extension later?}

% AB: I liked this example of the trace operator, but it doesn't really apply to the new format
%
% For rules of type $3_{l,l'}$ we require that $\gamma(l,l')=l$ or that $t\equiv t'$. If neither of these conditions hold the format supports the following rule, defining the {\sf trace} operator found in some functional programming languages.
% \[
% \sosrule{x \trans{l} x' \quad y \trans{l'} 0}{\textsf{trace}(x,y) \trans{l'} x'}
% \]
% The idea here is that $\textsf{trace}(p,q)$ performs the input and output of $q$ while evaluating to the value of $p$.
% This operator  is not idempotent as witnessed by the (CCS style) process $p = a.0 + b.c.0$, because while
% $\textsf{trace}(p,p) \trans{a} c.0$ the only possible $a$-successor of $p$ is the nil process. The similar constraint
% on form $4_{l,l'}$ is necessary for the same reasons.

For rules of form $2_{l,l'}$ we require that either $t_0\equiv t_1$, or that the mentioned labels are the same and the
associated transition
relation is deterministic. This requirement is necessary in the proof to ensure that the
target of the conclusion fits our definition of $\eqidem$, i.e. the operator is applied to two identical terms.
Consider the following TSS where $\alpha\in\{a,b\}$.
\[
    \sosrule{}{a\trans{a} a} \quad
    \sosrule{}{a\trans{a} b} \quad
    \sosrule{}{b\trans{b} b} \quad
    \sosrule{x\trans\alpha x' \quad y\trans\alpha y'}{x | y \trans\alpha x' | y'}
\]
For the operator $|$, this violates the condition $t_0\equiv t_1$ (note that $\trans{a}$ is not deterministic).
We observe that $a|a \trans{a} a|b$. The only possibilities for $a$ to simulate this $a$-transition
is either with $a\trans{a}a$
or with $a\trans{a}b$. However, neither $a$ nor $b$ can be bisimilar to $a|b$ because both $a$ and $b$ have outgoing
transitions while $a|b$ is stuck. Therefore $a$ and $a|a$ cannot be bisimilar.
%
If $t_0 \not\equiv t_1$ we must require that the labels match, $l_0 = l_1$, and that $\trans{l_0}$ is deterministic.
We require the labels to match because if they do not, then given only $p\trans{l}p'$ it is impossible to prove that $f(p,p)$
can simulate it using only a $2_{l,l'}^*$ rule. The determinacy of the transition with that label
is necessary when proving that transitions from $f(p,p)$ can, in general, be simulated by $p$;
if we assume that $f(p,p)\trans{l}p'$ then we must be able to conclude that $p'$ has the shape $f(p'',p'')$ for some $p''$,
in order to meet the bisimulation condition for $\eqidem$. Consider the standard choice operator $+$
and prefixing operator $.$ of CCS
with the $|$ operator from the last example, with $\alpha\in\{a,b,c\}$.
\[
    \sosrule{}{\alpha \trans{\alpha} 0} \quad
    \sosrule{}{\alpha.x \trans{\alpha} x} \quad
    \sosrule{x \trans{\alpha} x'}{x + y \trans{\alpha} x' } \quad \sosrule{y \trans{\alpha} y'}{x + y \trans{\alpha} y' } \quad
    \sosrule{x\trans\alpha x' \quad y\trans\alpha y'}{x | y \trans\alpha x' | y'}
\]
If we let $p = a.b + a.c$, then $p|p \trans{a} b|c$ and $b|c$ is stuck.
However, $p$ cannot simulate this transition w.r.t. $\eqidem$.
Indeed, $p$ and $p|p$ are not bisimilar.


% subsection relaxing_the_restrictions (end)

\subsection{Predicates\label{sec::pred}}

There are many examples of TSSs where predicates are used.
The definitions presented in Section \ref{sec::pre} and  \ref{sec:idempotency} can be easily adapted to deal with predicates as well.
In particular, two closed terms are called bisimilar in this setting when,
in addition to the transfer conditions of bisimilarity, they satisfy the same predicates.
%Note that the notion of bisimulation (Definition \ref{def:bisim}), and thus the notion of idempotency (Definition \ref{def:idem}), are
%extended naturally to the setting with predicates,
%by requiring that for each two closed terms and each predicate, one term satisfies the predicate if and only if the other one satisfies the predicate.
To extend the idempotency rule format to a setting with predicates, the following types of rules for predicates are introduced:

\paragraph{$3_P$. Choice rules for predicates}
\[
    \sosrule[i\in\{0,1\}]{\{P x_i\}\cup\Phi}{P f(x_0,x_1)} \\
\]
\paragraph{$4_{P}$. Synchronization rules for predicates}
\[
    \sosrule{\{P x_0,\, P x_1\}\cup\Phi}{P f(x_0,x_1)}
\]

As before, we define the starred version of these forms, $3_P^*$ and $4_{P}^*$. The starred version of each rule is the same as the unstarred one except that the set $\Phi$ must be
empty in each case. With these additional definition the idempotency format is defined as follows.

A TSS is in \emph{idempotency format w.r.t. a binary operator $f$} if each \mbox{$f$-defining} rule, i.e., a deduction rule with $f$ appearing in the source of the conclusion, is of one the forms $1_l$, $2_{l_0,l_1}$, $3_P$ or $4_P$ for some $l, l_0, l_1 \in L$, for each label $l \in L$  and predicate symbol $P$. Moreover, for each $l \in L$, there exists at least one rule of the forms $1_l^*$ or $2_{l,l}^*$, and for each predicate symbol $P$ there is a rule of the form $1^*_P$ or $2_P^*$.

\subsection{Examples} % (fold)

\begin{example}
The most prominent example of an idempotent operator is non-deterministic choice, denoted $+$. It typically has the following deduction rules:
\[
\sosrule{x_0 \trans{a} x_0'}{x_0 + x_1 \trans{a} x_0'} \qquad
\sosrule{x_1 \trans{a} x_1'}{x_0 + x_1 \trans{a} x_1'}
\]
Clearly, these are in the idempotency format w.r.t.\ $+$.
\end{example}

\begin{example}[External choice]
The well-known external choice operator $\Box$ from CSP \cite{Hoare85} has the following deduction rules
\[
\sosrule{x_0 \trans{a} x'_0}{x_0 \Box x_1 \trans{a} x'_0}
\qquad
\sosrule{x_1 \trans{a} x'_1}{x_0 \Box x_1 \trans{a} x'_1}
\qquad
\sosrule{x_0 \trans{\tau} x'_0}{x_0 \Box x_1 \trans{\tau} x'_0 \Box x_1}
\qquad
\sosrule{x_1 \trans{\tau} x'_1}{x_0 \Box x_1 \trans{\tau} x_0 \Box x'_1}
\]
Note that the third and fourth deduction rule are not instances of any of the allowed types of deduction rules.
Therefore, no conclusion about the validity of idempotency can be drawn from our format.
In this case this does not point to a limitation of our format,
because this operator is not idempotent in strong bisimulation semantics \cite{dArgenio95}.
\end{example}

\begin{example}[Strong time-deterministic choice]
The choice operator that is used in the timed process algebra ATP \cite{Nicollin94} has the following deduction rules.
\[
\sosrule{x_0 \trans{a} x'_0}{x_0 \oplus x_1 \trans{a} x'_0}
\qquad
\sosrule{x_1 \trans{a} x'_1}{x_0 \oplus x_1 \trans{a} x'_1}
\qquad
\sosrule{x_0 \trans{\chi} x'_0 \quad x_1 \trans{\chi} x'_1}{x_0 \oplus x_1 \trans{\chi} x'_0 \oplus x'_1}
\]
The idempotency of this operator follows from our format since the last rule for $\oplus$ fits
the form $2_{\chi,\chi}^*$ because, as we remarked in Example~\ref{ex:timetrans1}, the transition relation
$\trans{\chi}$ is deterministic.
\end{example}

\begin{example}[Weak time-deterministic choice]
The choice operator $+$ that is used in most ACP-style timed process algebras has the following deduction rules:
\[
\sosrule{x_0 \trans{a} x_0'}{x_0 + x_1 \trans{a} x_0'} \qquad
\sosrule{x_1 \trans{a} x_1'}{x_0 + x_1 \trans{a} x_1'}\]
\[\sosrule{x_0 \trans{1} x'_0 \quad x_1 \trans{1} x'_1}{x_0 + x_1 \trans{1} x'_0 + x'_1} \qquad
\sosrule{x_0 \trans{1} x'_0 \quad x_1 \ntrans{1}}{x_0 + x_1 \trans{1} x'_0} \qquad
\sosrule{x_0 \ntrans{1}  \quad x_1 \trans{1} x'_1}{x_0 + x_1 \trans{1} x'_1}
\]
The third deduction rule is of the form $2^*_{1,1}$, the others are of forms $1^*_a$ and $1^*_1$.
This operator is idempotent (since the transition relation $\trans{1}$ is deterministic, as remarked
in Example~\ref{ex:timetrans2}).
\end{example}

\begin{example}[Conjunctive nondeterminism]
The operator $\lor$ as defined in Example \ref{cnp} by means of the deduction rules
\[
%\sosrule{}{0 \cando{a}}
%\qquad
%\sosrule{}{a.x \cando{a}}
%\qquad
\sosrule{x \cando{a}}{x \lor y \cando{a}}
\qquad
\sosrule{y \cando{a}}{x \lor y \cando{a}}
%\]
%\[
%\sosrule{}{0 \after{a} 0}
%\quad
%\sosrule{}{a.x \after{a} x}
%\quad
%\sosrule{}{a.x \after{b} 0} ~a \neq b
\quad
\sosrule{x \after{a} x' \quad y \after{a} y'}{x \lor y  \after{a} x' \lor y'}
\]
satisfies the idempotency format (extended to a setting with predicates). The first two deduction rules are of the form $3^*_{\cando{a}}$ and the last one is of the form $2^*_{a,a}$. Here we have used the fact that the transition relations $\after{a}$ are deterministic as concluded in Example \ref{cnp}.
\end{example}

\begin{example}[Delayed choice]
Delayed choice can be concluded to be idempotent in the restricted setting without $+$ and $\cdot$ by using the idempotency format and the fact that in this restricted setting the transition relations $\trans{a}$ are deterministic. (See Example \ref{dc}.)
\[
\sosrule{x \downarrow}{x \mp y \downarrow}
\qquad
\sosrule{y \downarrow}{x \mp y \downarrow}
\qquad
\sosrule{x \trans{a} x' \quad y \trans{a} y'}{x \mp y \trans{a} x' \mp y'}
\qquad
\sosrule{x \trans{a} x' \quad y \ntrans{a}}{x \mp y \trans{a} x'}
\qquad
\sosrule{x \ntrans{a} \quad y \trans{a} y'}{x \mp y \trans{a} y'}
\]
The first two deduction rules are of form $3^*_{\downarrow}$, the third one is a $2^*_{a,a}$ rule, and the others are $1_{a}$ rules. Note that for any label a starred rule is present.

For the extensions discussed in Example \ref{dc} idempotency cannot be established using our rule format
since the transition relations are no longer deterministic. In fact, delayed choice is not idempotent in these cases.
\end{example}
% subsection examples (end)

%\input{Extensions}
%\input{Isomorphism}
\section{\label{sec::conc}Conclusions}
In this paper, we presented two rule formats  guaranteeing determinism of certain transitions and idempotency of binary operators.
Our rule formats cover all practical cases of determinism and idempotency that we have thus far encountered in the literature.

We plan to extend our rule formats with the addition of data/store.
Also, it is interesting to study the addition of structural congruences pertaining to idempotency
to the TSSs in our idempotency format.  


%\bibliography{lit}
\begin{thebibliography}{10}

\bibitem{Aceto01}
L.~ Aceto, W.J.~Fokkink, and C.~Verhoef.
\newblock Structural operational semantics.
\newblock In {\em  Handbook of Process Algebra, {Chapter} 3}, pages 197--292. Elsevier, 2001.

\bibitem{Baeten02}
J.C.M.~Baeten and C.A.~Middelburg.
\newblock {\em Process Algebra with Timing}.
\newblock EATCS Monographs. Springer-Verlag, Berlin, Germany, 2002.

\bibitem{Baeten90}
J.C.M.~Baeten and W.P.~Weijland.
\newblock {\em Process Algebra}, volume~18 of {\em Cambridge Tracts in
  Theoretical Computer Science}.
\newblock Cambridge University Press, 1990.

\bibitem{BaetenMauw94}
J.C.M.~Baeten and S.~Mauw.
\newblock Delayed choice: An operator for joining {M}essage {S}equence
  {C}harts.
\newblock In {\em Proceedings of Formal Description Techniques}, volume~6 of {\em
  IFIP Conference Proceedings}, pages 340--354. Chapman {\&} Hall, 1995.

\bibitem{Bergstra84}
J.A.~ Bergstra and J.W.~ Klop.
\newblock Process algebra for synchronous communication.
\newblock {\em Information and Control}, 60(1-3):109--137, 1984.

\bibitem{Mousavi08-CONCUR}
S.~ Cranen, M.R.~ Mousavi, and M.A.~ Reniers.
\newblock A rule format for associativity.
\newblock In {\em Proceedings  of CONCUR'08},
  volume 5201 of {\em Lecture Notes in Computer Science}, pages 447--461,
   Springer-Verlag, 2008.

\bibitem{dArgenio95}
P.R.~ D'Argenio.
\newblock $\tau$-angelic choice for process algebras (revised version).
\newblock Technical report, LIFIA, Depto. de Inform\'{a}tica, Fac. de Cs. Exactas, Universidad Nacional de La Plata. March 1995.

\bibitem{Fokkink03a}
W.J.~Fokkink and T.~Duong Vu.
\newblock Structural operational semantics and bounded nondeterminism.
\newblock {\em Acta Informatica}, 39(6--7):501--516, 2003.

\bibitem{Groote93}
J.F.~Groote.
\newblock Transition system specifications with negative premises.
\newblock {\em Theoretical Computer Science (TCS)}, 118(2):263--299, 1993.

\bibitem{Hennessy87}
M.~ Hennessy and G.D.~ Plotkin.
\newblock Finite conjuncitve nondeterminism.
\newblock In {\em Concurrency and Nets, Advances in Petri Nets}, pages 233--244.
  Springer, 1987.

\bibitem{Hoare85}
C.A.R.~Hoare.
\newblock {\em Communicating Sequential Processes}.
\newblock Prentice Hall, 1985.

\bibitem{Lanotte05}
R.~ Lanotte and S.~Tini.
\newblock Probabilistic congruence for semistochastic generative processes.
\newblock In {\em Proceedings of FOSSACS'05}, volume 3441 of {\em Lecture Notes in Computer
  Science}, pages 63--78. Springer, 2005.

\bibitem{Mousavi06-FSTTCS}
M.R.~ Mousavi, I.C.C.~ Phillips, M.A.~ Reniers, and I.~
  Ulidowski.
\newblock The meaning of ordered {SOS}.
\newblock In {\em Proceedings of FSTTCS'06}, volume 4337 of {\em Lecture Notes in Computer Science},
  pages 334--345, Springer, 2006.

\bibitem{Mousavi05-IPL}
M.R.~ Mousavi, M.A.~ Reniers, and J.F.~Groote.
\newblock A syntactic commutativity format for {SOS}.
\newblock {\em Information Processing Letters}, 93:217--223, 2005.

\bibitem{Mousavi05-ICALP}
M.R.~ Mousavi and M.A.~ Reniers.
\newblock Orthogonal extensions in structural operational semantics.
\newblock In {\em Proceedings of the ICALP'05}, volume 3580 of {\em Lecture Notes in
  Computer Science}, pages 1214--1225. Springer, 2005.

\bibitem{Mousavi07-TCS}
M.R.~ Mousavi, M.A.~ Reniers, and J.F.~Groote.
\newblock {SOS} formats and meta-theory: 20 years after.
\newblock {\em Theoretical Computer Science}, (373):238--272, 2007.

\bibitem{Nicollin94}
X.~ Nicollin and J.~Sifakis.
\newblock The algebra of timed processes {ATP}: Theory and application.
\newblock {\em Information and Computation}, 114(1):131--178,  1994.

\bibitem{Plotkin04a}
G.D.~ Plotkin.
\newblock A structural approach to operational semantics.
\newblock {\em Journal of Logic and Algebraic Progamming}, 60:17--139,
  2004.
\newblock This article first appeared as Technical Report DAIMI FN-19, Computer Science Department, Aarhus
  University.

\bibitem{Tini04}
S.~Tini.
\newblock Rule formats for compositional non-interference properties.
\newblock {\em Journal of Logic and Algebraic Progamming}, 60:353--400,
  2004.

\bibitem{Ulidowski97b}
I.~Ulidowski and S.~Yuen.
\newblock Extending process languages with time.
\newblock In {\em Proceedings of {AMAST}'97},
  volume 1349 of {\em Lecture Notes in Computer Science}, pages 524--538.
  Springer, 1997.

\bibitem{Reniers08}
M. van Weerdenburg and M.A.~ Reniers.
\newblock Structural operational semantics with first-order logic.
\newblock In  {\em Pre-proceedings of SOS'08}, pages 48--62, 2008.

\bibitem{Verhoef95}
C. Verhoef.
\newblock A congruence theorem for structured operational semantics with
  predicates and negative premises.
\newblock {\em Nordic Journal of Computing}, 2(2):274--302, 1995.

\end{thebibliography}


%\appendix
%\newpage
\section{Proof of Theorem \ref{th::det}}
\label{proof::det}

In this section, we present a proof of Theorem \ref{th::det}.\footnote{Due to space restrictions, the proofs are omitted from the main text. They are provided as an appendix, however, for reviewers' convenience. If the paper is accepted, the full version of the paper, including the proofs, will appear as a technical report and will be made publicly available.}
%\begin{proof}
Instead of proving that $C$ is deterministic for each $l \in L$, we establish the following more general result.
We prove that, for each $l \in L$, if $p \trans{l} p' \in C \cup U$ and $p \trans{l} p'' \in C$, then $p' \equiv p''$.

Since $p \trans{l} p' \in C \cup U$, then there exists a provable transition rule, such that
$\tss \vdash \frac{N}{p \trans{l} p'}$, for some set $N$ of negative formulae such that $C \vDash N$.
We show the claim by an induction on the proof structure
for the transition rule $\frac{N}{p \trans{l} p'}$.
Consider the last deduction rule $\DR{r}$ and substitution $\sigma$
used in the proof structure for $\frac{N}{p \trans{l} p'}$.

%\AB{I need a little help in stating the induction hypothesis explicitly here, if needed.}

Since $p \trans{l} p'' \in C$,
there also exists a proof structure such that $\tss \vdash \frac{N'}{p \trans{l} p''}$
for some set $N'$ of negative formulae such that $C \cup U \vDash N'$.
Again, consider the last deduction rule $\DR{r'}$ and substitution $\sigma'$ used in the proof structure for $\tss \vdash \frac{N'}{p \trans{l} p''}$.

We first consider the case when $\DR{r}$ and $\DR{r'}$ are the same rule, say $\frac{\Phi}{t\trans{l}t'}$.
Obviously $\sigma(t)\equiv\sigma'(t)$ since both must be equal to $p$.
Since $\sigma(t')$ and $\sigma'(t')$ are equal to $p'$ and $p''$ respectively, we need to
show that $\sigma(t') \equiv \sigma'(t')$.

% If the proof structure for $\frac{N}{p\trans{l}p'}$ consists of only one rule,
% $\DR{r}$ is an axiom and thus contains no premises. This means, since all variables
% in $t'$ are source dependent, that they must appear in the source $t$.
% Since $\sigma(t)\equiv p \equiv\sigma'(t)$ we obtain
% that $\sigma(t')\equiv\sigma'(t')$ also holds.

%If the proof structure is more complex, i.e. the set of premises $\Phi$ is non-empty.

We define the distance of a source-dependent variable as the length of the shortest backward path from the variable, via premises with a label in $L$, to the variables in the source of conclusion. A variable in the source of the conclusion is thus of distance 0.

For each variable $v$ that is source dependent via  a subset of $\{ \trans{l} \mid l \in  L\}$, we proceed with another induction
on the distance of $v$ to show that $\sigma(v) \equiv \sigma'(v)$. If we show this, then it follows that $\sigma(t') \equiv \sigma'(t')$ since
all variables in $t'$ are source dependent by the first constraint of our rule format.

We consider the two possible reasons for $v$ being source dependent.
\begin{enumerate}
    \item Assume that $v$ appears in $t$. In this case, $\sigma(v)\equiv\sigma'(v)$ since $\sigma(t)\equiv\sigma'(t)$.
    \item Assume that $v$ appears in the target of some premise $t_i \trans{l_i} t_i' \in \Phi$ where $l_i \in L$
          and all variables in $t_i$ are source dependent via  a subset of $\{ \trans{l} \mid l \in  L\}$.
          Each variable $w \in vars(t_i)$ has a distance smaller than that of $v$. Therefore, the induction
          hypothesis (on the distance of variables) applies and we have that $\sigma(w) \equiv\sigma'(w)$.
          This means that $\sigma(t_i)\equiv\sigma'(t_i)$. This allows
          us to apply the induction hypothesis on the proof structure, since $\sigma(t_i \trans{l_i} t_i')$ has a proof structure
          that is smaller than the one for $p \trans{l} p'$, to conclude that $\sigma(t'_i) \equiv \sigma'(t'_i)$.
          Since $v$ appears in $t_i'$ it must hold that $\sigma(v)\equiv\sigma'(v)$.
\end{enumerate}
In either case, $\sigma$ and $\sigma'$ agree on the value of $v$. Since this holds for all variables of $t'$,
we reach the conclusion we seek, namely that $\sigma(t')\equiv\sigma'(t')$.

We now consider the case where the rules $\DR{r}$ and $\DR{r'}$ are distinct.
We first show that $(\sigma, \sigma')$ is determinism-respecting w.r.t.\ $(\Phi, \Phi')$ and $L$.

Assume,  towards a contradiction, that our claim concerning determinism-respecting substitutions does not hold.
Then there exist  two positive formulae $s_i \trans{l} s'_i$ and $t_i \trans{l} t'_i$ for some $l \in L$ among the premises of $\DR{r}$ and
$\DR{r'}$, respectively, such that
$\sigma(s_i) \equiv \sigma'(t_i)$ but it does not hold that $\sigma(s'_i) \equiv \sigma'(t'_i)$.
Since $s_i \trans{l} s'_i$  is a premise of $\DR{r}$ $\sigma(s_i \trans{l} s'_i) \in C \cup U$ and it has a smaller proof structure than $p \trans{l} p' \in C \cup U$. Following a similar reasoning, $\sigma'(t_i \trans{l} t'_i) \in C$.
But the induction hypothesis (on the proof structure) applies and hence, we have $\sigma(s'_i) \equiv \sigma'(t'_i)$, which contradicts our earlier conclusion that $\sigma(s'_i) \equiv \sigma'(t'_i)$ does not hold. Hence, we conclude that  $(\sigma, \sigma')$ is determinism-respecting w.r.t.\ $(\Phi, \Phi')$ and $L$.

Since we have shown that $(\sigma, \sigma')$ is determinism respecting,
it then follows from the second condition of the determinism format that either $\sigma(conc(r)) \equiv \sigma'(conc(r'))$,
which was to be shown, or
there exist premises $\phi_i \equiv s_i \trans{l_i} s'_i$  in one deduction rule and
$\phi'_i \equiv t_i \ntrans{l_i}$ in the other deduction rule such that
$\sigma(\phi_i)$ contradicts $\sigma'(\phi'_i)$.
We show that the latter possibility leads to a contradiction, thus completing the proof.
Since $\sigma(\phi_i)$ contradicts $\sigma'(\phi'_i)$, we have that $\sigma(s_i) \equiv \sigma'(t_i)$.
We distinguish the following two cases based on the status of the positive and negative contradicting premises with respect to $\DR{r}$ and $\DR{r'}$.

\begin{enumerate}
\item
Assume that the positive formula is a premise of $\DR{r}$.
Then, $\sigma(s_i \trans{l_i} s'_i) \in C \cup U$ but  from $C \cup U \vDash N'$ and $\sigma'(\phi'_i) \in N'$, it follows that
for no $p''$, we have that $\sigma(s_i) \equiv \sigma'(t_i) \trans{l_i} p'' \in C \cup U$, thus reaching a contradiction.

\item
Assume that the positive formula is a premise of $\DR{r'}$.
Then, $\sigma'(s_i) \trans{l_i} \sigma(s'_i) \in C$ but from $C \vDash N$ and $\sigma(\phi'_i) \in N$, it follows that
for no $p_1$, we have that $\sigma(t_i) \equiv \sigma'(s_i) \trans{l_i} p_1 \in C$, hence reaching a contradiction. \hfill\qed
\end{enumerate}
%\end{proof}

\section{Proof of Theorem~\ref{th::simplifiedImpliesHard}}
\label{proof::simpImpHard}

Let $\tss$ be a normalized TSS in the syntactic determinism format w.r.t. $L$.
Condition 1
of Definition~\ref{def::detHard} is satisfied since $\tss$ is normalized. To
see this, consider item 2 of Definition~\ref{def::normalized}, by taking $\DR{r}$ and $\DR{r'}$ to be the same rule.

To prove condition 2 of Definition~\ref{def::detHard}
let $(r) = \frac{\Phi_0}{t_0\trans{l}t_0'}$ and
$(r') = \frac{\Phi_1}{t_1\trans{l}t_1'}$ be distinct rules of $\tss$ and $(\sigma,\sigma')$
be a determinism-respecting pair of substitutions w.r.t. $(\Phi_0, \Phi_1)$ and $L$
such that $\sigma(t_0) \equiv \sigma'(t_1)$. Since $\tss$ is normalized,
both $\DR{r}$ and $\DR{r'}$ are $f$-defining for some function symbol $f$, i.e., $t_0 = f(\overrightarrow{s})$
and $t_1 = f(\overrightarrow{t})$.
Furthermore, since $f(\overrightarrow{s}) \equiv f(\overrightarrow{t})$ we have that
$\sigma$ and $\sigma'$
agree on all variables appearing
in $f(\overrightarrow{s}) = f(\overrightarrow{t})$.

For each variable $v \in \vars{r} \cap \vars{r'}$, we define
its {\em common source distance} to be the source distance of $v$
when only taking the formulae in $\Phi_0 \cap \Phi_1$ into account.
Note that such a source distance exists since by constraint 3 of Definition \ref{def::normalized}
all $v \in \vars{r} \cap \vars{r'}$ are source dependent via a subset of $\{ \trans{l} \mid l \in  L\}$ included in $\Phi_0 \cap \Phi_1$.

We prove for each $v \in \vars{r} \cap \vars{r'}$ that $\sigma(v) \equiv \sigma'(v)$ by
an induction on the common source distance of variables $v$.
Suppose that we show the above claim, then we can prove the theorem as follows.
It follows from Definition \ref{def::simplifiedDet} that either $t_0' \equiv t'_1$ or $\Phi_0$ contradicts $\Phi_1$.
If $t'_0 \equiv t'_1$, then variables in $\vars{t'_0} = \vars{t'_1}$ are all source dependent via transitions in $L$
that are common to both $\Phi_0$ and $\Phi_1$
(by constraint 3 of Definition \ref{def::normalized}).
By the above-mentioned claim, $\sigma(t'_0) \equiv \sigma'(t'_1)$, thus, constraint 2 of Definition \ref{def::detHard} follows, which was to be shown.
If  $\Phi_0$ contradicts $\Phi_1$, then assume that the premises negating each other are $\phi_j \equiv s_j \trans{l_j} s'_j$ and $\phi_{j'} \equiv t_{j'} \ntrans{l_j}$ and it holds that $s_j \equiv t_{j'}$.
All variables in $t_j \equiv s_j$ are source dependent via transitions in $L$ (by constraint 3   of Definition \ref{def::normalized}).
It follows from the claim that $\sigma(s_j) \equiv \sigma'(t_{j'})$
and thus, $\sigma(\phi_j)$ contradicts $\sigma'(\phi_{j'})$, which implies constraint 2 of Definition \ref{def::detHard}.


Hence, it only remains to prove, by an induction on the common source distance of $v$, that $\sigma(v) \equiv \sigma'(v)$.
If $v\in vars(f(\overrightarrow s))$ then we know that $\sigma(v) \equiv \sigma'(v)$ (since $t_0 \equiv t_1$ and $\sigma(t_0) \equiv \sigma'(t_1)$.
Otherwise, since $v$ is source dependent in $(r)$
via transitions with labels in $L$,
there is a positive premise
$u \trans{l} u'$ in $\Phi_0$ with $l\in L$
such that $v\in vars(u')$ and all variables in $u$ are source
dependent with a shorter common source distance.
Furthermore,
since $v$ appears in both rules, i.e., $v \in vars(r) \cap vars(r')$,
this premise also appears in $\Phi_1$ according
to item 3 of Definition~\ref{def::normalized} and thus $\vars{u} \subseteq vars(r) \cap vars(r')$.
By the induction hypothesis we have that $\sigma(u)\equiv \sigma'(u)$ and since $(\sigma,\sigma')$
is determinism-respecting w.r.t. $(\Phi_0, \Phi_1)$ and $L$, we know that
$\sigma(u') \equiv \sigma'(u')$. Specifically, the substitutions
must agree on the value of $v$, i.e.
$\sigma(v) \equiv \sigma'(v)$. \hfill\qed


\iffalse
Now assume that $\sigma(t_0') \not\equiv \sigma'(t_1')$.
To prove condition 2 of definition~\ref{def::detHard} we show that
$\sigma(\Phi_0)$ must contradict $\sigma'(\Phi_1)$.
The first step is to show that $t_0' \not\equiv t_1'$ must hold.
%We show by an induction
Assume, towards
a contradiction, that $t_0' \equiv t_1'$ and take $v\in vars(t_0')$.
If $v\in vars(f(\overrightarrow s))$ then we know that $\sigma(v) \equiv \sigma'(v)$.
If not, then because $v$ is source dependent in $(r)$
via transitions with labels in $L$
there is a positive premise
$u \trans{l} u'$ in $\Phi_0$ with $l\in L$
such that $v\in vars(u')$ and all variables in $u$ are source
dependent by a shorter distance.
Furthermore,
since $v$ appears in both rules, i.e. $v \in vars(r) \cap vars(r')$,
this premise also appears in $\Phi_1$ according
to item 3 of definition~\ref{def::normalized} and thus $u \in vars(r) \cap vars(r')$.
By induction we can argue that $\sigma(u)\equiv \sigma'(u)$ and since $(\sigma,\sigma')$
is determinism-respecting w.r.t. $(\Phi_0, \Phi_1)$ and $L$, we know that
$\sigma(u')\equiv\sigma'(u')$. Specifically, the substitutions
must agree on the value of $v$, i.e.
$\sigma(v) \equiv \sigma'(v)$.
Since we have this for all variables in $t_0'\equiv t_1'$ we obtain that
$\sigma(t_0') \equiv \sigma'(t_1')$. This provides the direct contradiction we seek
and thus proves that $t_0' \not\equiv t_1'$.

Since $\tss$ is in the syntactic determinsism format, the fact that $t_0' \not\equiv t_1'$
means that $\Phi_0$ must contradict $\Phi_1$. Without loss of generality we can assume that
there is a positive premise $w \trans{l} w' \in \Phi_0$ and a negative premise
$w \ntrans{l} \in \Phi_1$. We now use the same argument as we
used to obtain the contradiction above to show that
$\sigma(w) \equiv \sigma'(w)$.
By condition 2 of definition~\ref{def::normalized} any variable
$v' \in vars(w)$ is source dependent in $(r')$ via a set of transisitions with labels from
$L$. If $v'$ appears in the source of the conclusion of $(r')$, and therefore also $(r)$
we have immediately that $\sigma(v') \equiv \sigma'(v')$. If it does not appear in the
source of the conlusion then there again exists a positive premise
$u \trans{l} u'$ in $\Phi_1$
with $l\in L$ and $v' \in vars(u')$.%
\footnote{Note that we are re-using the names $u, u'$ and $l$. They do not necessarily refer
to the same terms or label as in the previous paragraph}
Since $w \in vars(r) \cap vars(r')$ this premise also appears in $\Phi_0$.
Again we may inductively reason that $\sigma(u)
\equiv \sigma'(u)$ and since $(\sigma,\sigma')$ is determinism-respecting w.r.t $(\Phi_0,
\Phi_1)$ and $L$, we have that $\sigma(u')\equiv\sigma'(u')$, in particular $\sigma(v')
\equiv \sigma'(v')$. Since this holds for any variable $v' \in vars(w)$ we find that
$\sigma(w)\equiv \sigma'(w)$. Specifically this means that $\sigma(\Phi_0)$ contradicts
$\sigma'(\Phi_1)$, i.e. condition 2 of definition~\ref{def::detHard} is met.

Since both conditions of definition~\ref{def::detHard} are met, $\tss$ is in the
determinism format with respect to $L$. \hfill\qed
\fi

\section{Proof of Theorem \ref{thm:idempotent}}
\label{proof:idempotent}
%\begin{proof}
First define the relation $\eqidem \subseteq \CTerms\Sigma \times \CTerms\Sigma$ as follows.
\[
    \eqidem = \{ (p,p), (p,f(p,p)), (f(p,p),p) \,|\, p \in \CTerms\Sigma \}
\]
To prove the theorem it suffices to show that $\eqidem$ is a bisimulation relation. If it is, then $f(p,p)\bisim p$ for any closed term $p$ and since $\bisim \subseteq \sim$ we obtain the theorem.

Let $(C,U)$ be the least three-valued stable model for the TSS under consideration.
First consider a closed term $p$ s.t. $p \trans{l} p' \in C$ for some $l$ and $p'$ (note that $U=\emptyset$ since the TSS is complete).
Next, we argue that $f(p,p) \trans{l} p''$ for some $p''$ such that  $p' \eqidem p''$.
Since $p \trans{l} p' \in C$, there exists a provable transition rule of the form $\frac{N}{p \trans{l} p'}$ for some set of negative formulae $N$ such that $C \vDash N$.
In particular, that means that $p \ntrans{l} \notin N$.
In this case we make use of the requirement that there exists at least one rule of a starred form for label $l$. If there exists a
rule of the form $1_l^*$, i.e.
\[
    \sosrule[i\in\{0,1\}]{x_i\trans{l}x'}{f(x_0,x_1)\trans{l}x'}
\]
then we can instantiate it to prove that $f(p,p)\trans{l}p'\in C$.
In particular, it does not matter if $i=0$ or $i=1$.
Since $\eqidem$ is reflexive, $p'\eqidem p'$ holds.
If there exists a rule of the form $2_{l,l}^*$, we observe that $\gamma(l,l) = l$ so the transition rule becomes
\[
    \sosrule{x_0\trans{l}x_0' \quad x_1\trans{l}x_1'}{f(x_0,x_1) \trans{l} f(x_0',x_1')},
\]
where $x'_0 \equiv x'_1$ or $\trans{l}$ is deterministic.
Now we can use the existence of $p\trans{l}p'$ to satisfy both premises and obtain that $f(p,p)\trans{l}f(p',p')$.
By the definition of $\eqidem$ we also have that $p' \eqidem f(p',p')$.
In either case, if $p \trans{l} p'\in C$, then there exists a $p''$ s.t. $f(p,p) \trans{l} p'' \in C$ and $p' \eqidem p''$.

Now assume that $f(p,p) \trans{k} p' \in C$. Then there exists a provable transition rule $\frac{N}{f(p,p) \trans{k} p'}$
for some set of negative formulae $N$ such that $C\vDash N$. Since all rules for $f$ are either of the form $1_l$ or $2_{l_0,l_1}$,
this provable transition rule must be based on a rule of those forms. We analyze each possibility separately, showing that
in each case $p \trans{k} p''$ for some $p''$ such that $p' \eqidem p''$.

If the rule is based on a rule of form $1_l$, its positive premises must also be provable. In particular it must hold that
$p\trans{k} p' \in C$ since both $x_0$ and $x_1$ in the rule are instantiated to $p$. The other premises are of no
consequence to this conclusion and, again, we observe that $p'\eqidem p'$.

Now consider the case where the transition is a consequence of a rule of the form $2_{l_0,l_1}$.
If $t_0\equiv t_1$, say both are equal to $p''$, we must consider two cases, namely $k=l_0$ and $k=l_1$.
If $k=l_0$ then the first premise of the rule actually states that $p\trans{k}p''$.
If $k=l_1$ then the second premise similarly states that $p\trans{k}p''$.
In either case, we note that $p'\equiv f(p'',p'')$ must hold and again by the definition of $\eqidem$ we have
that $f(p'',p'')\eqidem p''$.

If however $t_0\not\equiv t_1$ the side condition requires that $l_0=l_1 = k$, which also implies $\gamma(l_0,l_1)=l_0=k$,
and that the transition relation $\trans{l_0}$ is deterministic.
In this case it is easy to see that the right-hand sides of the first two premises, namely $t_0$ and $t_1$, evaluate to
the same closed term in the proof structure, say $p''$.
The conclusion then states that $k=l_0$ and $p'\equiv f(p'',p'')$.
It must thus hold that $p\trans{k}p''\in C$ and $f(p'',p'')\eqidem p''$ as before.

From this we obtain that if $f(p,p) \trans{k} p' \in C$ then there exists
a $p''$ such that $p \trans{k} p'' \in C$ and $p' \eqidem p''$.
Thus, $\eqidem$ is a bisimulation. \hfill \qed
%\end{proof}



\end{document}


\begin{definition}[De Simone format \cite{deSimone85}]
A deduction rule is in the \emph{De Simone format} if it is of the form
$$\sosrule[P]{\{x_i\trans[l_i]y_i~|~i\in I\}}{f(x_1, \ldots, x_n)\trans[l]t}$$
where $f \in \Sigma$ has arity $n$, $I \subseteq \{1, \ldots, n\}$ is a finite set of indices, $P$ is a predicate on the labels in the deduction rule and moreover,

\begin{bullets}
\item for $1 \leq i < j \leq n$, $x_i$ and $x_j$ are different variables and for $1 \leq i \leq n$ and $j \in I$, $x_i$ and $y_j$ are distinct variables,
%There are no duplicates in the set $\{x_i~|~i \in I\} \cup \{y_i~|~i \in I\}$
\item $t$ is a term in which the variables from $\{ x_i ~|~i \not\in I\} \cup \{y_i~|~i\in I\}$ occur at most once.
\end{bullets}
A TSS is in the De Simone format if and only if all of its deduction rules are.
\end{definition}



\begin{definition}[Bisimulation]
\label{def:bisim}
Let $\tss$ be a TSS with signature $\Sigma$.
A relation $\rel \subseteq \CTerms{\Sigma} \times \CTerms{\Sigma}$ is a \emph{bisimulation relation}
if and only if $\rel$ is symmetric
%\leaveout{(i.e.\ $\forall_{p_0,p_1 \in \CTerms{\Sigma}}~p_0 \rel  p_1 \Leftrightarrow p_1\rel p_0$)}
and for all $p_0,p_1,p'_0 \in \CTerms{\Sigma}$ and $l \in L$
$$(p_0 \Rel p_1 \wedge \tss \vdash p_0 \trans[l] p_0') \Rightarrow \exists_{p_1' \in \CTerms{\Sigma}} (\tss \vdash p_1 \trans[l] p_1'\wedge p_0' \Rel p_1' ).$$
Two terms $p_0, p_1 \in \CTerms{\Sigma}$ are called \emph{bisimilar}, denoted by $p_0 \bisim p_1$ when there exists a bisimulation relation $\Rel$ such that
$p_0 \rel p_1$.
\end{definition}


It is easy to check that bisimilarity is indeed an equivalence.
Bisimilarity can be extended to open terms by requiring that $t_0 \bisim t_1$
when $\sigma(t_0) \bisim \sigma(t_1)$ for all closing substitutions $\sigma: V \rightarrow \CTerms{\Sigma}$.
In the remainder of this paper, we restrict our attention to the notions of equivalence on \emph{closed terms}
that contain strong bisimilarity. However, all our results carry over (without any change)
to the notions on \emph{open terms} that contain strong bisimilarity on open terms in the above sense.
Another notion of equivalence that we use in the remainder of this paper is isomorphism, as defined below.

%\Michel{Of definitie isomorphism weglaten en alleen omschrijven in tekst, of een duidelijke definitie geven. Dan dus ook duidelijker uitleggen wat de %LTS is die geassocieerd kan worden met een closed term.}
%\Mohammad{Herformuleert en verplaatst naar hier; check of dat goed is.}

\begin{definition}(Isomorphism)
\label{def::isomorphism}
Two closed terms $p$ and $q$ are \emph{isomorphic}, denoted by $p \isomorphic q$, when
there exists a bijective function $h: reach(p) \rightarrow reach(q)$
such that $h(p) = q$ and if $h(p_0) = q_0$ and $h(p_1) = q_1$, then $p_0 \trans[l]{p_1}$ if and only if $q_0 \trans[l]{q_1}$,
where $reach(p)$ is the smallest set satisfying $p \in reach(p)$ and if $p' \in reach(p)$ and $p' \trans[l] q'$, then $q' \in reach(p)$ (i.e., the set of closed terms reachable from $p$).
\end{definition}

