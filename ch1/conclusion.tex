\section{Discussion and future work}
\label{sec:conclusion}

We have presented both a formal semantics and an implementation of TMI over the
Haskell STM system. During this work, we discovered that there are many design
decisions to be made and the design we have presented here is only one of many
possibilities. The variants we experimented with in the design process did not
always exhibit the behaviour that we expected or wanted. For this task, defining
the formal semantics proved to be an essential tool to understand and evaluate
different design decisions as well as spotting special cases that were not so
obvious in the actual implementation. 
Indeed, constructing the semantics helped us discover bugs
and unexpected behaviour in the code, even after weeks of careful consideration.
In addition the formal semantics gives a clear and unambiguous
description of the TMI architecture.

%
% What do you mean by 'the design' ?  Do you mean the TMI architecture (which
% we describe and clearly isn't limited in this way), or the formal semantics,
% or the Haskell implementation?  If it is the last (which I think it is), then
% it isn't quite true, since we have the privilege stuff.
%
% Also, this is kind of a downer to end on.  Why not just skip, or say that we would
% like to explore better implementations for stateful policies (although our semantics supports it).
%

The TMI architecture, as described in Section~\ref{ssub:overview_of_tmi}, 
can support the enforcement of stateful security policies that depend on the 
execution history over multiple transactions.  In our paper \cite{tmi}, we have experimented 
with such policies in another implementation of the TMI architecture.  
However, we have not explored the addition of such facilities here, 
in order to simplify the exposition of our semantics and Haskell implementation.

The privilege amplification by nested TMI actions naturally relies on the programmer
to ensure that the nested authorization manager does not violate the enclosing
policy. The objective of the TMI architecture is to provide facilities for writing
policy enforcement code, not to prevent injection of malicious code. In the case of library
development where one deals with untrusted code, such nesting may not be desirable and can
be disabled. We have experimented with other ways of implementing privilege amplification
without relying on this nesting, with good results.
% The policy state must be protected by the STM mechanisms to avoid
% race conditions or interference between threads, but we must take care that the accesses
% to this state are only performed by trusted code, as they cannot be subject to authorization,
% lest we introduce a circular dependency.

In our implementation we are maintaining the introspection log by hand. This works well for
prototypical purposes, but we would like to investigate the possibility of making use
of the real underlying transaction log. This requires modifications to the STM framework
provided in the GHC runtime library. Having the formal semantics as the definition of the desired
behaviour should make such an implementation easier to construct and check.

Future work in this context also involves writing or porting complex software to the
architecture to obtain realistic performance measurements.
Also, our semantics may still be simplified, while allowing the same behavior; 
for example, the transaction log seems redundant in STM transitions, 
and may possibly be eliminated.


Most importantly, we think that having clear semantics for TMI and an implementation
over a production-ready STM system, further validates our claim that TMI architecture
is very relevant to practical software development.
