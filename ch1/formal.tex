\section{Formal semantics of TMI} % (fold)
\label{sec:formal_semantics_of_tim}

\begin{figure}
\flushleft{\begin{minipage}{\linewidth}
    \small
\color{old}
\vspace{-1.5ex}
\begin{gather*}
\begin{array}{rrcl}
\textrm{Thread soup} & P,Q & ::= & M_t \alt (P\,|\,Q) \\
\new{\textrm{Descriptors}} & \new{D_\bot} & \new{::=} & \new{M \cup \{\bot\}} \\
\textrm{Heap} & \Theta & ::= & r \hookrightarrow M \new{\,\times\, D_\bot} \\
\textrm{Allocations} & \Delta & ::= & r \hookrightarrow M \new{\,\times\, D_\bot} \\
\new{\textrm{Access types}} & \new{T} & \new{::=} & \new{\{ \textsc{create}, \textsc{read}, \textsc{write} \}} \\
\textrm{\new{Log}} & \new{\Sigma} & \new{::=} & \new{\textrm{list monoid } ([], \oplus) \textrm{ over } T\times D} \\
\\
\textrm{Evaluation} & \EE & ::= & [\cdot] \alt \EE \,\prg{>{}>=}\,M \alt \prg{catch }\EE\;M \\
\textrm{contexts} & \SS & ::= & [\cdot] \alt \SS \,\prg{>{}>=}\,M \\
                  & \PP & ::= & \SS_t \alt (\PP|P) \alt (P|\PP) \\
\textrm{Action} & a & ::= & \prg{!}c \alt \prg{?}c \alt \epsilon
\end{array}
\end{gather*}
\end{minipage}}\caption{Program state and evaluation contexts}
\vspace{2ex}
\label{fig:contexts}
\end{figure}




To formalize the semantics of TMI, we build on the semantics for the Haskell STM
presented in~\cite{haskellstm}. The semantics is a structural operational semantics
in the style of Plotkin~\cite{plotkin_04_structural}. 
For the sake of completeness and to help the reader understand our extensions,
we give a cursory
explanation of the concepts of the semantics from~\cite{haskellstm} 
so that a reader not familiar with it may understand our extensions.

Figures~\ref{fig:sosadmin} through \ref{fig:sos2} give the operational rules that describe the steps a program may
take. At the top level, a program transforms a state of the form $P;\Theta$ via labelled transition.
\[
    P;\,\Theta \stackrel{a}{\tarrow} Q;\,\Theta'
\]
$P$ represents a program term in the syntax of Figure~\ref{fig:syntax2} while $\Theta$ stands for a
memory store, a partial function from variable names to annotated terms. An annotated value is a tuple
$(t,d)$ where $t$ is a program term and $d$ is a value that holds the security relevant description of
the relevant variable. The labels on transitions represent the programs input and output actions. $Q$
and $\Theta'$ represent the term that is left unevaluated and the updated store after a transition, 
respectively.

To model atomicity of transactions, separate relations represent the top level I/O transitions
and the STM actions. We extend this by adding a third relation representing the security relevant TMI
actions. Furthermore we add a simple relation for evaluation of security managers under the context of
an immutable transaction log.

Execution of a program proceeds by
non-deterministically
picking a program term from a collection of terms, each representing
a separate thread of execution. One I/O transition of this term combined with the current store is performed
and then the process is repeated. This models interleaved concurrency at the level of I/O transitions.
STM transitions however can only be performed as a required premise of the \lstinline+atomically+ operator
at the I/O level, and thus appear in this model as a single atomic step.

As mentioned in~\cite{haskellstm} there is no need to represent rollback, but contrary to the semantics
in that paper,
our extensions do need to formalize the notion of the transaction log as it is no longer purely an 
implementation detail. For simplicity though, we only model the log for security sensitive operations as
they are the only ones relevant to the semantics of TMI.

\begin{figure}
\flushleft{\begin{minipage}{\linewidth}
    \small
\color{old}

\center{\framebox{Administrative transitions $\qquad M \tarrow N $}}
\begin{gather*}
    M \quad\tarrow\quad V  \quad \textrm{if $\,\mathcal{V}\lsem M \rsem = V$ and $M \nequiv V$}
    \qquad\soslbl{EVAL}
\\
\begin{array}{rcll}
\prg{return }N \,\prg{>{}>=}\, M
    & \tarrow
    & M\;N
    & \soslbl{BIND}
    \\
\prg{throw }N \,\prg{>{}>=}\, M
    & \tarrow
    & \prg{throw }N
    & \soslbl{THROW}
    \\
\prg{retry >{}>=}\, M
    & \tarrow
    & \prg{retry}
    & \soslbl{RETRY}
%\prg{catch retry }N
%    & \tarrow
%    & \prg{retry}
%    & \soslbl{CATCH3}
%    \\
%\prg{abort }N\,\prg{>{}>=}\, M
%    & \tarrow
%    & \prg{abort }N
%    & \soslbl{ABORT}
%    & &
%\prg{catch }(\prg{abort }M)\,N
%    & \tarrow
%    & \prg{abort }M
%    & \soslbl{CATCH4}
\end{array}
%
\end{gather*}

\end{minipage}}\caption{Evaluation of terms and monad operations}
\label{fig:sosadmin}
\end{figure}


\begin{figure*}
\flushleft{\begin{minipage}{\textwidth}
    \small

\color{old}

\center{\framebox{I/O transitions $\qquad P;\,\Theta \stackrel{a}{\tarrow} Q;\,\Theta'$}}
\begin{gather*}
\begin{array}{rcll}
    \PP[\prg{putChar } c];\, \Theta & \stackrel{!c}{\tarrow}
                                       & \PP[\prg{return ()}];\, \Theta
                                       & \soslbl{PUTC} \\
    \PP[\prg{getChar }];\, \Theta& \stackrel{?c}{\tarrow}
                                       & \PP[\prg{return } c];\, \Theta
                                       & \soslbl{GETC} \\
    \PP[\prg{forkIO } M];\, \Theta& \tarrow
                                       & (\PP[\prg{return } t] \,|\, M_t);\, \Theta
                                            \quad t\notin \PP, \Theta, M
                                       & \soslbl{FORK} \\
    \PP[\prg{catch }(\prg{return }M)\, N];\,\Theta& \tarrow 
                                       & \PP[\prg{return } M];\,\Theta
                                       & \soslbl{CATCH1} \\
    \PP[\prg{catch }(\prg{throw }M)\, N];\,\Theta& \tarrow 
                                      & \PP[N\,M];\,\Theta
                                      & \soslbl{CATCH2}
\end{array}
\\
\sos{M \tarrow N}
    {\PP[M];\, \Theta \tarrow \PP[N];\, \Theta}
\quad\soslbl{ADMIN}
\\
%
% <<< Atomic
%
\sos{M;\,\Theta,\emptyset
     \stackrel{*}{\tdarrow}
     \prg{return }N;\, \Theta',\Delta'}
   {\PP[\prg{atomically }M];\,\Theta \tarrow
    \PP[\prg{return }N];\,\Theta'}
  \quad\soslbl{ARET}
\qquad
\sos{M;\,\Theta,\emptyset
     \stackrel{*}{\tdarrow}
     \prg{throw }N;\, \Theta',\Delta'}
   {\PP[\prg{atomically }M];\,\Theta \tarrow
    \PP[\prg{throw }N];\,\Theta\cup\Delta'}
  \quad\soslbl{ATHROW}
%\\
%\sos{M;\,\Theta,\emptyset
%     \stackrel{*}{\tdarrow}
%     \prg{abort }N;\, \Theta',\Delta',\Sigma'}
%   {\PP[\prg{atomically }M];\,\Theta \tarrow
%    \PP[\prg{throw }N];\,\Theta\cup\Delta'}
%  \quad\soslbl{ATHROW1}
% >>>
\end{gather*}

\rule{0.95\linewidth}{0.5pt}

\center{\framebox{STM transitions $\qquad M;\,\Theta, \Delta\new{,\Sigma}
                                 \tdarrow N;\,\Theta',\Delta'\new{,\Sigma'}$}}
\begin{gather*}
\begin{array}{rclll}
%
   \SS[\prg{readTVar } r];\,\Theta,\Delta\new{,\Sigma}
 & \tdarrow
 & \SS[\prg{return }\Theta_1(r)];\,\Theta,\Delta\new{,\Sigma}
 & \textrm{if $r\in \dom(\Theta)$ \new{and $\Theta_2(s) = \bot$}}
 & \soslbl{READ} \\
%
   \SS[\prg{writeTVar } r\;M];\,\Theta,\Delta\new{,\Sigma}
 & \tdarrow
 & \SS[\prg{return ()}];\,\Theta[r\mapsto \new{(}M\new{,\bot)}],\Delta\new{,\Sigma}
 & \textrm{if $r\in \dom(\Theta)$ \new{and $\Theta_2(s) = \bot$}}
 & \soslbl{WRITE} \\
%
   \SS[\prg{newTVar } M];\,\Theta,\Delta\new{,\Sigma}
 & \tdarrow
 & \SS[\prg{return }r];\,\Theta[r\mapsto \new{(}M\new{,\bot)}],\Delta[r\mapsto \new{(}M\new{,\bot)}]\new{,\Sigma}
 & \textrm{$r\notin \dom(\Theta)$}
 & \soslbl{NEW} \\
%
\end{array}
%
\\
\sos{M \tarrow N}
    {\SS[M];\,\Theta,\Delta\new{,\Sigma} \tdarrow \SS[N];\,\Theta,\Delta\new{,\Sigma}}
    \quad\soslbl{AADMIN}
\\[5pt]
\sos{M_1;\,\Theta,\Delta\new{,\Sigma} \stackrel{*}{\tdarrow} 
      \prg{return }N;\,\Theta',\Delta'\new{,\Sigma'}}
      {\SS[M_1 \prg{ `orElse` } M_2];\,\Theta,\Delta\new{,\Sigma} \tdarrow 
      \SS[\prg{return }N];\,\Theta',\Delta'\new{,\Sigma'}} 
    \quad\soslbl{OR1}
\\[5pt]
\sos{M_1;\,\Theta,\Delta\new{,\Sigma} \stackrel{*}{\tdarrow} 
      \prg{throw }N;\,\Theta',\Delta'\new{,\Sigma'}}
      {\SS[M_1 \prg{ `orElse` } M_2];\,\Theta,\Delta\new{,\Sigma} \tdarrow 
      \SS[\prg{throw }N];\,\Theta',\Delta'\new{,\Sigma'}} 
    \quad\soslbl{OR2}
\\[5pt]
\sos{M_1;\,\Theta,\Delta\new{,\Sigma} \stackrel{*}{\tdarrow} 
      \prg{retry};\,\Theta',\Delta'\new{,\Sigma'}}
      {\SS[M_1 \prg{ `orElse` } M_2];\,\Theta,\Delta\new{,\Sigma} \tdarrow 
      \SS[M_2];\,\Theta,\Delta\new{,\Sigma}} 
    \quad\soslbl{OR3}
\\[5pt]
%
% <<< XSTM catch
%
\sos{M;\,\Theta,\emptyset\new{,[]}
     \stackrel{*}{\tdarrow}
     \prg{return }M';\, \Theta', \Delta'\new{,\Sigma'}}
    {\SS[\prg{catch }M\;N];\,\Theta,\Delta\new{,\Sigma} \tdarrow
     \SS[\prg{return }M'];\, \Theta',\Delta\cup\Delta'\new{,\Sigma\oplus\Sigma'}}
    \quad\soslbl{XSTM1}
\\[5pt]
\sos{M;\,\Theta,\emptyset\new{,[]}
     \stackrel{*}{\tdarrow}
     \prg{throw }M';\, \Theta', \Delta'\new{, \Sigma'}}
    {\SS[\prg{catch }M\;N];\,\Theta,\Delta\new{,\Sigma} \tdarrow
     \SS[N \,M'];\, \Theta\cup\Delta',\Delta\cup\Delta'\new{,\Sigma\oplus(\Sigma'|_{\Delta'})}}
    \quad\soslbl{XSTM2}
\\[5pt]
\sos{M;\,\Theta,\emptyset\new{,[]}
     \stackrel{*}{\tdarrow}
     \prg{retry};\, \Theta', \Delta'\new{, \Sigma'}}
    {\SS[\prg{catch }M\;N];\,\Theta,\Delta\new{,\Sigma} \tdarrow
     \SS[\prg{retry}];\, \Theta,\Delta\new{,\Sigma}}
    \quad\soslbl{XSTM3}
\\[5pt]
% >>>
% <<< Authorized
%
\new{
\sos{M;\,\Theta,\emptyset,[]
     \stackrel{*}{\tddarrow}
     \prg{return }M';\, \Theta',\Delta',\Sigma'
   \qquad
   \Sigma' \vdash N \stackrel{*}{\leadsto} \prg{return }N' }
   {\SS[\prg{authorized }N\;M];\,\Theta,\Delta,\Sigma \tdarrow
    \SS[\prg{return }M'];\Theta',\Delta',\Sigma}
  \quad\soslbl{AURET1}
}
\\[5pt]
\new{
\sos{M;\Theta,\emptyset,[]
     \stackrel{*}{\tddarrow}
     \prg{throw }M';\, \Theta',\Delta',\Sigma'
   \qquad
   \Sigma' \vdash N \stackrel{*}{\leadsto} \prg{return }N' }
   {\SS[\prg{authorized }N\;M];\,\Theta,\Delta,\Sigma \tdarrow
    \SS[\prg{throw }M'];\,\Theta',\Delta',\Sigma}
  \quad\soslbl{AUTHROW1}
}
\\[5pt]
\new{
\sos{M;\,\Theta,\emptyset,[]
     \stackrel{*}{\tddarrow}
     \prg{return }M';\, \Theta',\Delta',\Sigma'
   \qquad
   \Sigma' \vdash N \stackrel{*}{\leadsto} \prg{throw }N' }
   {\SS[\prg{authorized }N\;M];\,\Theta,\Delta,\Sigma \tdarrow
    \SS[\prg{throw UnathorizedError}];\Theta',\Delta',\Sigma}
  \quad\soslbl{AURET2}
}
\\[5pt]
\new{
\sos{M;\Theta,\emptyset,[]
     \stackrel{*}{\tddarrow}
     \prg{throw }M';\, \Theta',\Delta',\Sigma'
   \qquad
   \Sigma' \vdash N \stackrel{*}{\leadsto} \prg{throw }N' }
   {\SS[\prg{authorized }N\;M];\,\Theta,\Delta,\Sigma \tdarrow
    \SS[\prg{throw UnauthorizedError}];\,\Theta',\Delta',\Sigma}
  \quad\soslbl{AUTHROW1}
}
\\[5pt]
\new{
\sos{M;\Theta,\emptyset,[]
     \stackrel{*}{\tddarrow}
     \prg{retry};\, \Theta',\Delta',\Sigma'
   }
   {\SS[\prg{authorized }N\;M];\,\Theta,\Delta,\Sigma \tdarrow
    \SS[\prg{retry}];\,\Theta',\Delta'\Sigma}
  \quad\soslbl{AURETRY}
}
% >>>
%
\end{gather*}

\end{minipage}}
\caption{Operational semantics for IO and STM actions}
\label{fig:sos1}
\end{figure*}


\begin{figure*}
\flushleft{\begin{minipage}{\textwidth}
    \small

\center{\framebox{TMI transitions $\qquad M;\,\Theta, \Delta, \Sigma
                                \tddarrow N;\,\Theta',\Delta',\Sigma'$}}
\begin{gather*}
\begin{array}{rcll}
%
   \SS[\prg{readTMIVar } r];\,\Theta,\Delta,\Sigma
 & \tddarrow
 & \SS[\prg{return }\Theta_1(r)];\,\Theta,\Delta,\Sigma\oplus[(\textsc{read},\Theta_2(r))]
 & \\ & & \hspace{4cm}
   \textrm{if $r\in \dom(\Theta)$ and $\Theta_2(r)) \ne \bot$}
 & \soslbl{TMIREAD} \\[.7em]
%
   \SS[\prg{writeTMIVar } r\;M];\,\Theta,\Delta,\Sigma
 & \tddarrow
 & \SS[\prg{return ()}];\,\Theta[r\mapsto (M,\Theta_2(r))],\Delta,\Sigma\oplus[(\textsc{write},\Theta_2(r))]
 & \\ & & \hspace{4cm}
   \textrm{if $r\in \dom(\Theta)$ and $\Theta_2(r)) \ne \bot$}
 & \soslbl{TMIWRITE} \\[.7em]
%
   \SS[\prg{newTMIVar }N\;M];\,\Theta,\Delta,\Sigma
 & \tddarrow
 & \SS[\prg{return }r];\,\Theta[r\mapsto (M,N)],\Delta[r\mapsto (M,N)],\Sigma\oplus[(\textsc{create},N)]
 & \\ & & \hspace{4cm}
   \textrm{$r\notin \dom(\Theta)$}
 & \soslbl{TMINEW} \\[.7em]
%
\end{array}
%
\\
\sos{M \tarrow N}
    {\SS[M];\,\Theta,\Delta\new{,\Sigma} \tddarrow \SS[N];\,\Theta,\Delta\new{,\Sigma}}
    \quad\soslbl{TADMIN}
\qquad
\sos{M;\,\Theta,\Delta,\Sigma \stackrel{*}{\tdarrow} N;\,\Theta',\Delta',\Sigma'}
    {\SS[\prg{liftSTM }M];\,\Theta,\Delta,\Sigma \tddarrow
     \SS[N];\,\Theta',\Delta',\Sigma'}
    \quad\soslbl{LIFTSTM}
\\[5pt]
\sos{M_1;\,\Theta,\Delta\new{,\Sigma} \stackrel{*}{\tddarrow} 
      \prg{return }N;\,\Theta',\Delta'\new{,\Sigma'}}
      {\SS[M_1 \prg{ `orElse` } M_2];\,\Theta,\Delta\new{,\Sigma} \tddarrow 
      \SS[\prg{return }N];\,\Theta',\Delta'\new{,\Sigma'}} 
    \quad\soslbl{TOR1}
\\[5pt]
\sos{M_1;\,\Theta,\Delta\new{,\Sigma} \stackrel{*}{\tddarrow} 
      \prg{throw }N;\,\Theta',\Delta'\new{,\Sigma'}}
      {\SS[M_1 \prg{ `orElse` } M_2];\,\Theta,\Delta\new{,\Sigma} \tddarrow 
      \SS[\prg{throw }N];\,\Theta',\Delta'\new{,\Sigma'}} 
    \quad\soslbl{TOR2}
\\[5pt]
\sos{M_1;\,\Theta,\Delta\new{,\Sigma} \stackrel{*}{\tddarrow} 
      \prg{retry};\,\Theta',\Delta'\new{,\Sigma'}}
      {\SS[M_1 \prg{ `orElse` } M_2];\,\Theta,\Delta\new{,\Sigma} \tddarrow 
      \SS[M_2];\,\Theta,\Delta\new{,\Sigma}} 
    \quad\soslbl{TOR3}
\\[5pt]
%
% <<< XSTM catch
%
\sos{M;\,\Theta,\emptyset\new{,[]}
     \stackrel{*}{\tddarrow}
     \prg{return }M';\, \Theta', \Delta'\new{,\Sigma'}}
    {\SS[\prg{catch }M\;N];\,\Theta,\Delta\new{,\Sigma} \tddarrow
     \SS[\prg{return }M'];\, \Theta',\Delta\cup\Delta'\new{,\Sigma\oplus\Sigma'}}
    \quad\soslbl{XTMI1}
\\[5pt]
\sos{M;\,\Theta,\emptyset\new{,[]}
     \stackrel{*}{\tddarrow}
     \prg{throw }M';\, \Theta', \Delta'\new{, \Sigma'}}
    {\SS[\prg{catch }M\;N];\,\Theta,\Delta\new{,\Sigma} \tddarrow
     \SS[N \,M'];\, \Theta\cup\Delta',\Delta\cup\Delta'\new{,\Sigma\oplus(\Sigma'|_{\Delta'})}}
    \quad\soslbl{XTMI2}
\\[5pt]
\sos{M;\,\Theta,\emptyset\new{,[]}
     \stackrel{*}{\tddarrow}
     \prg{retry};\, \Theta', \Delta'\new{, \Sigma'}}
    {\SS[\prg{catch }M\;N];\,\Theta,\Delta\new{,\Sigma} \tddarrow
     \SS[\prg{retry}];\, \Theta,\Delta\new{,\Sigma}}
    \quad\soslbl{XTMI3}
\end{gather*}

\rule{0.95\linewidth}{0.5pt}

\center{\framebox{Authorization transitions $\qquad \Sigma \vdash M \leadsto N$}}
\begin{gather*}
\sos{M\tarrow N}
    {\Sigma \vdash \EE[M] \,\leadsto\, \EE[N]}
\quad\soslbl{AUADMIN}
\qquad\qquad
\Sigma \vdash \EE[\prg{getlog}] \,\leadsto\, \EE[\prg{return } hs(\Sigma)]
\quad\soslbl{GETLOG}
\\[0.5em]
\Sigma \vdash \EE[\prg{catch }(\prg{return }M)\, N]
    \,\leadsto\,
    \EE[\prg{return } M]
    \quad\soslbl{ACATCH1}
\\
\Sigma \vdash \EE[\prg{catch }(\prg{throw }M)\, N]
    \,\leadsto\,
    \EE[N\,M]
    \quad\soslbl{ACATCH2}
\end{gather*}

\end{minipage}}
\caption{Operational semantics for TMI actions}
\label{fig:sos2}
\end{figure*}


\subsection{Syntax, states and evaluation contexts} % (fold)
\label{sub:syntax}

The syntax of terms for a subset of STM Haskell is given in Figure~\ref{fig:syntax2} with our TMI-related
extensions (highlighted). Terms and values are standard except that the application of some monadic
operators are considered values, a technique again lifted from~\cite{haskellstm}. The \lstinline+do+-notation
used up until now is standard syntactic sugar for the monad bind and return operations.

\vspace{0.5em}
\begin{tabular}{rcl}
    \lstinline[mathescape=true]+do {$x$<-$e$; $Q$}+ & $\equiv$ &
    \lstinline[mathescape=true]+$e$ >>= ($\backslash x$ -> do {$Q$})+ \\

    \lstinline[mathescape=true]+do {$e$; $Q$}+ & $\equiv$ &
    \lstinline[mathescape=true]+$e$ >>= ($\backslash$_ -> do {$Q$})+ \\

    \lstinline[mathescape=true]+do {$e$}+ & $\equiv$ &
    \lstinline[mathescape=true]+$e$+
\end{tabular}
\vspace{0.5em}

Figure~\ref{fig:contexts} defines some symbols used in the semantics. The metavariable $D$ represents a set
of terms used to describe the security properties of variables. We extend this set with an invalid value $\bot$ and
write $D_\bot$ for the extended set. A state of a computation is a pair $(M,\Theta)$ of a term that remains
to be evaluated and a store $\Theta$. The store maps variable names to terms and their variable descriptor. If
a variable does not have a suitable descriptor, we use $\bot$ as a fill-in.

The set of access types, $T$, consists of three constants, each
representing an operation performed on variables. An introspection log $\Sigma$
is a list monoid of pairs $(t,d)$ where $t$ is an access type and $d$ is a descriptor term; we use $[]$ for the empty
list and $\oplus$ for concatenation, and in the semantics we use $[\cdot]$ as a constructor.
Other symbols are conventional and taken from~\cite{haskellstm}.

For a (partial) function $f$ whose co-domain is a cross-product of two or more sets, and an integer $i$;
we write $f_i$ instead of $\pi_i \circ f$ where $\pi_i$ is the standard $i$-th projection function.
For convenience, we introduce the following notation for filtering logs. If $\Delta$ is a store and $\Sigma$ is
an introspection log, we define the \emph{$\Delta$-restriction of $\Sigma$}, indicated by $\Sigma |_\Delta$, thus
\[
    \begin{array}{rl}
        [] |_\Delta &= [] \\
        ([(t,d)] \oplus \Sigma')|_\Delta &=
        \begin{cases}
            [(t,d)] \oplus \Sigma'|_\Delta & \textrm{if $d \in \img(\Delta_2)$} \\
            \Sigma'|_\Delta & \textrm{otherwise}
        \end{cases}
    \end{array}
\]
Intuitively, $\Sigma|_\Delta$ is the list of entries from $\Sigma$ which apply to variables defined by $\Delta$,
where variables are identified by their security descriptors.

Interleaving of operations is modelled with the evaluation context $\mathbb{P}$, often referred to as
a {\em thread soup}. Through this evaluation context the semantics can non-deterministically choose a
term for reduction from the parallel construct, each term representing a thread. Haskell terms are usually
reduced according to the evaluation context $\mathbb{E}$, which allows for reductions of the right
hand side of the \lstinline+>>=+ operator as well as the body of a \lstinline+catch+ term. However,
since we want to handle exceptions in a specific manner for STM and TMI actions, we will use the
simpler context $\mathbb{S}$ which requires the operational semantics rules to specify explicitly
how \lstinline+catch+ terms are handled.

% subsection syntax (end)

\subsection{Operational semantics} % (fold)
\label{sub:operational_semantics}

Figures~\ref{fig:sosadmin} through \ref{fig:sos2} detail the transition relations of our semantics.
Figures~\ref{fig:sosadmin} and \ref{fig:sos1} are mostly the same as in the semantics
of~\cite{haskellstm}, parts added for TMI are indicated with a darker ink. 
The semantics uses several different
transition systems that are layered such that a sequence of reductions in one layer becomes one reduction
in the next layer above. Namely, there are three main layers - the top level I/O context, the STM context
and the TMI context. An auxiliary transition system is used to reduce authorization functions.

\paragraph{\emph{Values and I/O transitions:}}
The {\em admin} transitions of Figure~\ref{fig:sosadmin} 
define the evaluation of terms to values via a function $\mathcal{V}$. This
function is standard and its definition omitted here. Administrative transitions also include the
behaviour of the monadic bind operator \lstinline+>>=+.

The top level I/O actions are described
by the labelled $\tarrow$ relation. They 
operate on the $\mathbb{P}$ context, which allows for picking any program
term from the thread soup for reduction. The first two rules are I/O primitives. The rule {\em FORK} is used
to create a new thread and enter it into the thread soup. The rules {\em CATCH1} and {\em CATCH2} 
deal with exception
handling as described in the appendix of the post-publication, extended version of the Haskell
semantics~\cite{haskellstm}.
The {\em ADMIN} rule
allows for lifting of administrative transitions to the I/O transition relation.
This is done to reduce
repetition, as the administrative 
rules also apply to the STM and TMI transition relations, which have a similar
lifting rule.
Finally, the rules {\em ARET} and {\em ATHROW} 
enable the use of the \lstinline+atomic+ combinator to lift a sequence
of reductions in the STM transition relation to a single I/O transition.

If the series of STM reductions results in a \lstinline+return+ value, the effects on the store are retained.
If it however results in an exception (i.e. a \lstinline+throw+ value), the modifications to existing variables
are discarded but any new allocations are retained. This is necessary as the exception value may hold references
to newly allocated variables.

\paragraph{\emph{STM transitions:}}
The STM transitions define the behavior of STM actions. The states used in this transitions is extended
from the I/O transitions by adding a separate store for new allocations $\Delta$, and an introspection log
$\Sigma$. A transition of the form
\[
M;\,\Theta, \Delta,\Sigma \tdarrow N;\,\Theta',\Delta',\Sigma'
\]
represents a reduction of the term $M$ to the term $N$. Some variables in $\Theta$ may be introduced
or altered to yield $\Theta'$. $\Delta$ is a store similar to $\Theta$, that only tracks newly allocated
variables while $\Sigma$ is a log of accesses to TMI variables. $\Delta$ and $\Sigma$ are transaction
local, i.e. they are reset at the start of each atomic sequence of reductions in the STM system, see
e.g. rule~$\soslbl{ARET}$.

The first three rules define actions on transactional variables. They operate on the store
$\Theta$ but only on those variables where the second component (i.e. the security descriptor) is $\bot$.
This is what differentiates regular transactional variables from security sensitive variables. 

Other rules in the STM transition system define the behavior of STM combinators as described in
Section~\ref{sub:haskell_stm}. We have used the revised semantics of exception handling from the
later versions of~\cite{haskellstm}, namely the rule {\em XTM2} ensure that when a term reduction
results in a caught exception, all effects of that reduction are rolled back except for new 
allocations.

% �g er ekkert rosa happy me� �essi paragraph heading, en �a� ver�ur a� brj�ta eitthva� upp �essa 
% bla�s��u einhvernvegin
\paragraph{\emph{TMI transitions:}}
A key TMI addition to the STM transitions is the handling of \lstinline+authorized+.
The rules {\em AURETn} and {\em AUTHROWn} specify that for a term of the form {\tt (authorized}
$N\;M${\tt )}
if $M$ evaluates to a \lstinline+return+ or \lstinline+throw+ value, then that value is propagated only
if $N$ evaluates to a \lstinline+return+ value under the authorization relation (see later) 
If evaluation of the {\em authorization term} $N$ raises an exception, a fixed exception containing
no information about the local transaction state is thrown. This triggers a rollback of any updates
performed during that invocation of \lstinline+authorized+.

We should note that in the case of an exception, including authorization failure, new allocations
are retained for the reason described above. In the implementation, this is not done explicitly
as deallocation of references are handled by the garbage collector. Any new allocations that are
actually referenced by exception values are thus retained, but others are discarded. Thus, since
we don't allow any variable references in our special exception for authorization failures, no
new allocations will leak in practice.

Figure~\ref{fig:sos2} shows the two new TMI transition relations. The first one deals with
TMI actions and is indicated by the symbol $\tddarrow$. The configurations of this transition
system are identical to those of the STM system.
Indeed, TMI actions behave very much like STM actions, the main difference is that variable
operations in TMI actions can operate on security sensitive variables. A variable $v$ is security
sensitive if and only if $\Theta_2(v) \neq \bot$.
The first three rules of
Figure~\ref{fig:sos2} describe the variable operations. 
When these operations are performed, a log entry is added to
$\Sigma$ with the contents of the variable's security descriptor. Another addition over STM
behavior is rule {\em LIFTSTM}. This rule states that any sequence of STM reductions can be lifted
to the TMI level. 
This is necessary to allow TMI code to access regular transactional variables, and should be 
possible, since TMI actions are always performed in the context of an enclosing STM action.

Finally, we add a separate transition system to evaluate authorization functions. In this system
a transition of the form
\[
\Sigma \vdash M \leadsto N
\]
represents the reduction term $M$ to term $N$, under the context of an introspection log $\Sigma$.
The reason for this notation is that the introspection log is fixed, i.e. read-only, for these
transitions.
This system only allows pure operations and monad binding via the administrative transitions, the
usual exception handling and one special term \lstinline+getlog+. 
The \lstinline+getlog+ term is reduced to a list representation (in the Haskell sense) of the access
log $\Sigma$. The terms reduced with this system can thus examine the log and make decisions based
on its contents.

\paragraph{\emph{An example:}}
As an example of reading and applying the rules, consider the program
\begin{lstlisting}[style=small]
atomically 
  (
     authorized (assert (isEmpty getlog)) 
                (writeTMIVar x 10)
  )
\end{lstlisting}
Working from the inside out, we can see that the innermost expression of \lstinline+writeTMIVar x 10+
will update the value of $x$ in $\Theta$ as well as enter an entry to the introspection log $\Sigma$, by applying
rule $\soslbl{TMIWRITE}$. The resulting term is \lstinline+return ()+. % breaks accross lines
As the resulting log is non-empty, the authorization term 
\lstinline+assert (isEmpty getlog)+ will throw an exception. Thus, for the \lstinline+authorized+
term, rule $\soslbl{AURET2}$ is the only applicable one, so that term evaluates to
\lstinline+throw UnauthorizedError+. The \lstinline+atomically+ term is therefore evaluated to
the same via rule $\soslbl{ATHROW}$, but this rule does not preserve updates to the store $\Theta$,
meaning the transaction has been aborted.

\paragraph{\emph{Nested TMI actions:}} As we mentioned in the previous section, the capability
of lifting STM actions up to the TMI levels allows us to nest TMI actions. An inner TMI action
can be authorized with a separate authorization term. Consider the following example of an
action that provides a student with information about her grade for a course, as well as the
average of all grades of other students. 
Naturally, the student doesn't have access to other students' grades
but for the purpose of calculating the average we may allow such access in a nested action.

Assume that we have defined the following terms.
\begin{mybullet}
\item \lstinline+ownGradesRead s+ is an authorization term that succeeds only if
    the input log only contains reading of grades that belong to student \lstinline+s+
\item \lstinline+allGradesRead+ is an authorization term that succeeds only if the log only contains
    reading of grades, but regardless of the owner of the grades. This may be considered
    a kind of a {\em system} read access to grades.
\item \lstinline+readGrade s+ is a TMI action that reads a grade of a student from the
    relevant \lstinline+TMIVar+ and returns it. The introspection log will contain an 
    appropriate entry afterwards.
\item \lstinline+averageGrades+ is a TMI action that reads grades of all students from the
    appropriate \lstinline+TMIVars+ and returns their average. The introspection log will 
    contain an entry for every read grade.
\end{mybullet}
Now it is possible to define the following TMI action that provides a student with her
own grade as well as the average grades of all students.
\begin{lstlisting}[style=small]
gradeInformation :: Student -> TMI (Grade, Grade)
gradeInformation s =
  do own <- readGrade s
     avg <- liftSTM (authorized allGradesRead
                                averageGrades)
     return (own, avg)
\end{lstlisting}
Now this function may be called used with the appropriate authorization function,
namely one that only allows a student access to her own grades.
\begin{lstlisting}[style=small]
atomically (authorized (ownGradesRead s)
                       (gradeInformation s))
\end{lstlisting}

By applying the operational semantics rules to this term, one can find that the innermost action
\lstinline+averageGrades+ will by authorized by \lstinline+allGradesRead+ {\em before}
turning it into an STM transition. This may, for example, happen through rule $\soslbl{AURET1}$.
Note that such a rule does not keep the log entries of already authorized actions, i.e.
the $\Sigma$ is not affected in the $\Rightarrow$ transition below the line. 
Thus the log of the nested
action is not contained in the log authorized by the outer authorization function
\lstinline+ownGradesRead+.

This use of nesting constitutes a {\em privilege amplification} 
in a manner similar to stack inspection~\cite{GordonFournetPOPL}.

% subsection operational_semantics (end)


% section formal_semantics_of_tim (end)
