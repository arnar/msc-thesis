\section{Introduction} % (fold)
\label{sec:introduction}

The implementation of security enforcement mechanisms requires special care,
as any flaw may open the door to malicious attacks. This is especially true
in the case of multithreaded software, as the designer must consider all possible
interleavings of code paths. In~\cite{tmi} we presented Transactional Memory Introspection
(TMI), an architecture that greatly simplifies the implementation of correct
reference monitors on 
mechanisms that
implement Software Transactional Memory (STM)~\cite{harrisFraserSTM,hm93} support.
%
In this paper we present a formal semantics for TMI,
    as well as a reference TMI implementation over the Haskell STM.
These specifications clarify the TMI architecture
    and helps identify and resolve ambiguities 
    in its implementation.


STM systems provide many useful guarantees that make the implementation of multithreaded
software easier and less error-prone. In particular, STM offers 
atomicity and isolation
through optimistically executing concurrent code and monitoring
for conflicting accesses to resources.
By providing
rollback mechanisms, STM systems can resolve conflicting accesses by undoing the work of a transaction
and retrying that transaction again, from the beginning.

All STM implementations must perform bookkeeping of accesses (such as reads and writes)
to shared resources. By interposing on 
this bookkeeping, and the necessary monitoring and validation steps,
TMI provides facilities to support the creation of robust and correct
enforcement mechanisms. 
TMI provides \emph{complete mediation} by enhancing the STM
runtime checks against conflicting, concurrent accesses, and TMI
adds the requirement that all accesses must have been successfully
authorized before a transaction is committed.
TMI also \emph{simplifies error handling}. When unauthorized accesses are detected in a transaction,
the transaction is rolled back and not retried. 
This saves the programmer from the onerous and error-prone task of performing clean-up after
a failed authorization.

Another common problem in traditional authorization is {\em time of check to time of use} bugs.
Such bugs arise when an authorization check is used to decide if a
dangerous operation should be performed, and when the interleaving
of code execution may cause state changes that invalidate that
decision, before the dangerous operation is actually performed.
TMI resolves this problem, by making use of STM mechanisms to
execute both the authorization check and the dangerous operation
within a single transaction.

In~\cite{tmi} we give a comprehensive, informal overview of the TMI architecture and also evaluate 
a Java TMI implementation built on the DSTM2 library~\cite{hlm06}.
In order to provide clear, well-defined semantics for TMI, this paper provides a more
formal treatment and defines a structural operational semantics~\cite{plotkin_04_structural} 
for the TMI architecture.
For this we need an STM system with equally well founded semantics. This is provided by the Haskell
STM, whose semantics is given in~\cite{haskellstm}. Our semantics of TMI builds on this foundation
and is also accompanied by an implementation over the Haskell STM system. We believe having formal
semantics for TMI is important to guide possible implementations and disambiguate design choices.

The structure of the paper is as follows. In section 2 we give the necessary background, including
STM systems and how TMI makes use of their mechanisms, as well as an overview of the Haskell STM
implementation. Section 3 covers TMI in greater detail and describes its Haskell implementation
from the user standpoint. Section 4 defines the formal semantics of TMI, building on existing
semantics for STM Haskell. Section 5 describes the key elements of our Haskell implementation
and section 6 discusses future work.

% section introduction (end)
