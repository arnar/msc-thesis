\section{Background} % (fold)
\label{sec:background}

\subsection{STM and TMI} % (fold)
\label{sub:stm_and_tmi}

STM provides attractive guarantees for multithreaded software, namely atomicity,
consistency and isolation of {\em transactions}, specifically marked blocks of code.
In general, STM implementations must do so by performing

\begin{mybullet}
    \item careful monitoring of the resources that are accessed within a transaction,
    \item validation of the accesses of concurrent transactions, and
    \item complete rollback of the effects of aborted transactions.
\end{mybullet}

TMI builds on this machinery and allows security enforcement to benefit from
the STM guarantees. TMI helps the programmer to write correct enforcement mechanisms
and simplifies error-handling. In~\cite{tmi} we outline three main benefits of TMI:

\paragraph{Complete mediation.} TMI provides complete mediation by implicitly invoking
the reference monitor before any effects of a transaction are permanently committed. The
reference monitor validation checks
are able to inspect the resource access logs of the STM and may veto the
commit if an application specific policy is violated. 
In general, this requires that STM mechanisms provide
strong atomicity, i.e. resources marked for transactional scrutiny may not be accessed outside
the scope of a transaction.

\paragraph{Freedom from TOCTTOU bugs.} {\em Time of check to time of use} (TOCTTOU) bugs arise
in conventional enforcement mechanisms when interleaved threads may affect the policy decisions
of each other. For example, a thread may make a policy-based decision to allow access to a
certain resource, e.g. reading a memory location. Before that operation is actually performed, 
execution may be preempted by another thread. That thread can change the global state so that the 
policy decision becomes invalid,
e.g. by writing privileged information into the memory location.

This problem is implicitly solved by using STM, which guarantees that transactions are 
isolated and cannot affect the policy decisions of each other.

\paragraph{Simplified error handling.} In the event of an authorization failure, TMI uses the STM
facilities to completely roll back the effects of the transaction in question and raise an appropriate
exception to the code that initiated the transaction. This frees the programmer from having to
undo state changes leading up to the unauthorized operation, a common source of 
errors~\cite{errorHandlingMistakes}.

% subsection stm_and_tmi (end)

\subsection{Haskell STM} % (fold)
\label{sub:haskell_stm}

For a formal treatment, we build our semantics and implementation on those of the Haskell STM~\cite{haskellstm},
which in turn is built on Concurrent Haskell~\cite{concurrenthaskell}. Concurrent Haskell is an extension to Haskell 98,
a lazy (i.e. call-by-name), pure, functional language. It supports concurrent threads and communications
between them. Non-pure computations are modelled with {\em monads}~\cite{monads}, this includes computations
with side-effects such as input/output and mutable state.

The main entry to a Haskell program is an instantiation of the I/O monad, i.e. a value that represents an
{\em action} of the type \lstinline+IO ()+. An action of this type can, in addition to performing pure
computation, perform other I/O actions by way of composing smaller actions into larger ones. For an example,
Haskell standard libraries define the basic I/O actions \lstinline+getChar+ and \lstinline+putChar+, which
read from standard input and write to standard output, respectively. The most common composition is simple
sequencing. For example the composed I/O action
\begin{lstlisting}[style=small]
main = do { c <- getChar; putChar c; putChar c }
\end{lstlisting}
defines an action, that when executed will perform the three actions listed in sequence.

In general a value of type \lstinline+IO a+ represents an action that when executed, may perform some
I/O operations as defined by the Haskell libraries and then result in a value of type \lstinline+a+.
Pure functions cannot execute such actions without jeopardizing their purity and this is neatly enforced
by the Haskell type system. Naturally, I/O actions are however free to run pure computations. Thus
the only way to get at the value of an action, is if it is a part of a bigger I/O action. The Haskell
runtime bootstraps the whole process by executing the special I/O action called \lstinline+main+.

In addition to conventional input and output, I/O actions can perform reads and updates of mutable memory
cells. The type \lstinline+IORef a+ represents a mutable cell that contains a value of type \lstinline+a+. 
Haskell provides the basic I/O actions \lstinline+newIORef+, \lstinline+readIORef+ and \lstinline+writeIORef+
for manipulation of such cells. As with other I/O actions, these operations can only be used when composing
larger I/O actions.

Concurrent Haskell supports explicit forking of threads through the I/O action \lstinline+forkIO+.
\begin{lstlisting}[style=small]
forkIO :: IO a -> IO ThreadID
\end{lstlisting}
\lstinline+forkIO+ takes another I/O action as a parameter and spawns a new thread to execute the action, 
immediately
returning a newly allocated thread identifier. For further discussion of concurrency we refer to~\cite{awkward}
or tutorials such as~\cite{partutorial}.

The Haskell STM is based on a monadic type similar to the one for I/O actions, namely \lstinline+STM a+.
A value of this type represents an {\em STM action}, which when executed may perform smaller STM actions and
result in a value of type \mbox{\lstinline+a+.} %  dot falls on a second line. I tried mbox (dunno if it works)
STM actions may contain pure computations as well, but note that
they {\em cannot} contain e.g. I/O actions. The main STM actions provided are actions that allow manipulations
of another kind of memory cells, which have the type \mbox{\lstinline+TVar a+}, % This breaks accross lines, tried mbox
where \lstinline+a+ is the type of
value that the cell holds. The actions are \lstinline+newTVar+, \lstinline+readTVar+ and \lstinline+writeTVar+,
so they have the same power as their I/O counterparts. The important thing to note is that sets of \lstinline+IORef+s
and \lstinline+TVar+s are kept separate, one can only be used in I/O actions and the other in STM actions.

STM actions can be composed. Similar to I/O actions, the most common composition is sequencing,
but in addition Haskell STM provides the basic STM action \lstinline+retry+ and a combinator
\lstinline+orElse+. The action \lstinline+retry+ is a blocking operation for 
STM actions, which restarts the current transaction with potentially updated \lstinline+TVar+ contents. 
By issuing \lstinline+retry+, the programmer is stating
that the current transaction cannot finish for the state of \lstinline+TVar+s it started in. 
The Haskell STM provides an optimized implementation of \lstinline+retry+. This implementation captures
the set of \lstinline+TVar+s that a transaction has read before the retry, and suspends the transaction
until at least one of those \lstinline+TVar+s has been updated.
This makes sense because the {\em only} outside factors that can
affect the execution of an STM action are the values of the \lstinline+TVar+s it reads.

If \lstinline+t1+ and \lstinline+t2+ are STM actions, then \lstinline+t1 `orElse` t2+
is an STM action that first tries performing \lstinline+t1+ on its own. 
If \lstinline+t1+ invokes the \lstinline+retry+ action,
then the combined action rolls back the effects of \lstinline+t1+ 
and tries \lstinline+t2+ instead. If that one retries also, the whole
action retries, but waits for updates on the variables read by {\em both} \lstinline+t1+ and \lstinline+t2+.

For an example how the above can be used for synchronization primitives such as communication channels, see Section
4 of~\cite{haskellstm}.

While the basic STM actions and their compositions give us a way to build larger STM actions, we have not
discussed how those actions can be run or how they relate to transactions. For this, STM Haskell provides us
with the \lstinline+atomically+ function\footnote{While~\cite{haskellstm} uses the name \lstinline+atomic+, the actual implementation
of the Glasgow Haskell Compiler uses \lstinline+atomically+.}. This function has the type
\begin{lstlisting}[style=small]
atomically :: STM a -> IO a
\end{lstlisting}
This function gives an I/O action that, when performed, will execute the input STM action.
The atomicity comes from the fact that STM Haskell will guarantee that the effects that the STM action has on
\lstinline+TVar+s are atomic, i.e. they all become visible at once and that what happens
inside the STM action is not affected by concurrent threads.  

The Haskell STM system does this by monitoring concurrent invocations of STM actions, taking care of rolling
them back if they conflict with each other and retrying them.
As an example, the following program creates
a transactional variable holding a counter and spawns three threads that each increments the counter atomically.
\begin{lstlisting}[style=small]
increment :: TVar Int -> STM ()
increment counter = do x <- readTVar counter
                       writeTVar counter (x + 1)

main = do c <- atomically (newTVar 0)
          forkIO (atomically (increment c))
          forkIO (atomically (increment c))
          forkIO (atomically (increment c))
\end{lstlisting}
The \lstinline+increment+ action is a classical example of where a race condition might occur 
in a traditional setting,
but in our situation the STM system will guarantee the atomicity of each invocation.

% subsubsection haskell_stm (end)

% section background (end)
