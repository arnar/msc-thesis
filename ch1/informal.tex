\section{Transactional Memory Introspection} % (fold)
\label{sec:transactional_memory_introspection}

In this section we give an overview of the TMI architecture and how it is implemented. We then describe
our Haskell implementation from a user standpoint.

\subsection{Overview of TMI} % (fold)
\label{ssub:overview_of_tmi}

As described in our previous work~\cite{tmi}, the TMI architecture aims to  raise the level
of abstraction in the implementation of security enforcement mechanisms. It allows the programmer
to decouple application logic from security enforcement. Just as STM frees the programmer from worrying
about lock acquisition order and other synchronization efforts, TMI can be used to eliminate concerns about check
placement, race conditions and exceptional execution paths.

TMI provides these guarantees by imposing on the STM system. 
The programmer marks certain variables as security
sensitive. This implicitly indicates to the STM system that these variables are shared,
and ensures that the STM system will protect against race conditions in accesses to the variables.
TMI enhances the monitoring of these security sensitive variables
by ensuring that an access-control reference monitor is invoked every time
that the variables are accessed.

\paragraph{\emph{Time of policy evaluation with TMI:}} % Changed proposed header
% Didn't apply changes in below para
The TMI architecture only loosely constrains when policy must be evaluated,
and in~\cite{tmi} we consider a number of alternatives. In particular, TMI enforcement
can be \emph{eager} or \emph{lazy}.
With {\em eager} enforcement,
every access to a variable triggers the reference monitor, which immediately checks it against the
relevant policy. If authorization is denied, the transaction is immediately aborted. With lazy enforcement,
accesses to variables are simply logged (often they are already logged by the STM) and 
the logs are inspected by the reference monitor only at the end
of the transaction. If any of the logged accesses
are invalid, the whole transaction is aborted. 

% Applied some, not all, changes in below para
A key property of TMI enforcement is that policy decisions
can be evaluated at any time, as long as they are evaluated in a serialized fashion,
and evaluation is fully complete before the transaction commits.
A good STM system will ensure that each transaction is executed in isolation,
such that aborting one will have the same semantics as not having started it. 
This said, for our formal treatment and Haskell implementation, we focus 
on lazy enforcement only. Thus, the following
discussion only deals with the lazy variant unless otherwise noted.

\paragraph{\emph{Utilizing TMI enforcement:}}  % Changed proposed header
% Didn't apply changes in below para
To use TMI, the programmer declares a set of variables as security relevant.
This implicitly indicates to the underlying STM system
that those variables should be protected against race conditions.
This means the values held by these variables can only
be read or modified within a transaction, and that the STM system takes care of resolving conflicting accesses
by concurrent transactions. 
This also means that, upon every variable access,
TMI appends information identifying the variable in 
question to a transaction-specific {\em introspection log}. 
In particular, the introspection log will contain information about the creation, reading, and writing
of the security sensitive variables.

% Changed below para
In addition, TMI requires that all sections
of code that access security-sensitive variables
must be explicitly marked as \emph{atomic}.
To execute such atomic code sections,
programmers initiate a TMI transaction 
and provide a reference to the atomic block and a
{\em security manager}. The security manager is a block of code (or closure) that 
encodes the intended, application-specific security policy, 
and is able to determine
whether a transaction introspection log
satisfies the security policy.
The security manager closure includes the active principal, and other  auxiliary information that is 
needed to check policy compliance.

% Changed below para
% NOTE: I don't have a good spell checker, so please spell check.
Finally, transaction commit plays a special role in TMI enforcement.
TMI runs atomic blocks as transactions in the underlying STM system,
but changes the semantics of transaction commit.
After a TMI atomic block has finished execution, but before it is committed,
TMI ensures that the security manager has fully evaluated whether the transaction introspection log 
complies with the intended security policy.
Importantly, this evaluation occurs within the same STM transaction
as the execution of the atomic block,
and a commit of the transaction is attempted only if 
the security manager returns success.

Even when the transaction has complied with the security policies,
the attempted commit may still fail,
and the transaction is retried,
in the case when the STM system 
finds conflicting concurrent accesses.
Also,
if the security manager finds the transaction in violation of policy,
all state changes are rolled back---including changes to the security manager state,
in the case of history-based policies---and,
instead of retrying the transaction,
an exception is raised to the invoker of the atomic block.

\paragraph{\emph{A simple example:}} 
The following pseudo code shows what software
that makes use of TMI-based security enforcement might look like.
(Note that the code makes use of function-argument currying.)
\begin{lstlisting}[style=small,language=pseudo]
declare sensitive accounts = array of Account

function withdraw(account, amount):
    account.balance = account.balance - amount

function security_manager(user, log):
    if log contains <withdrawal from account>:
        if account.owner == user:
            return Allowed
    return Denied

main program:
    user = aquire_login_credentials()
    try:
        transaction with security_manager(user):
            withdraw(get_account(123456), 42)
    catch AuthorizationFailed:
        tell user about error
\end{lstlisting}
In this code, two aspects are especially noteworthy. First, 
security enforcement code is completely decoupled from the application logic
and the function \lstinline+withdraw+ performs no authorization. 
Even so, complete mediation is ensured,
since the introspection log is implicitly updated by the TMI reference monitor
upon each access to account variables.

Second, in the case of 
authorization failure (e.g. a withdrawal from a different user's account), the error handler 
need only consider how to indicate the error to the user.
The error handler need not clean up any mess:
the state changes that happened during the
transaction (if any)
have already been rolled back when the error handler starts execution. 
Although perhaps not apparent in this simplified example, 
there is ample evidence that writing correct cleanup code is difficult,
especially when multiple security-relevant operations are involved~\cite{errorHandlingMistakes}.

While the above observations form the two main benefits of the TMI architecture,
the third is freedom from TOCTTOU bugs.
Without TMI,
this example code 
might suffer from 
TOCTTOU race conditions,
e.g., if accounts could change owners. 
However, with TMI,
such account-ownership changes would be isolated,
and a transaction 
would be guaranteed to see the same owner throughout its execution.

% subsubsection overview_of_tmi (end)

\subsection{TMI in Haskell} % (fold)
\label{ssub:tmi_in_haskell}

We saw earlier how Concurrent Haskell uses the type system to confine operations on shared variables to
STM actions, and provides a single function to wrap STM actions into an atomic I/O action.
For TMI, we do something very similar. We confine operations on security sensitive variables to {\em TMI
actions}, and provide a single function to turn a TMI action into an STM action and associating it with a
security manager at the same time. 

Figure~\ref{fig:syntax2} shows the extensions of STM Haskell with the TMI extensions (highlighted).
We define a new monad that represents TMI actions and operations on sensitive variables. In addition,
we lift all standard STM functions to their TMI counterparts. This is done so that an existing Haskell STM program can
be easily adapted to TMI with minimal changes to their code. The TMI monad also encapsulates state, namely the
introspection log of a transaction. The introspection log contains entries which specify the access type (create,
read or write) of a variable and the {\em security descriptor} of a variable. Security descriptors are provided by the
programmer when she creates sensitive variables and contain the metadata about the variable 
that is necessary for authorization,
such as the owner of an account, permissions of a file, etc.

\begin{figure}
\flushleft{\begin{minipage}{\linewidth}
\small
\color{old}
\begin{gather*}
\begin{array}{rcl}
    x,y & \in & Variable \\
    r,t & \in & Name \\
    c & \in & Char \\
\\
V & ::= & r \alt c \alt \backslash x\prg{->}M \\
& | & \prg{return }M \alt M \,\prg{>{}>=}\, N \\
& | & \prg{putChar }c \alt \prg{getChar} \\
& | & \prg{throw }M \alt \prg{catch }M\;N \\
& | & \prg{retry} \alt M \;\textrm{\lstinline{`orElse`}}\; N \\
& | & \prg{forkIO }M \alt \prg{atomically }M \\
& | & \prg{newTVar }M \\
& | & \prg{readTVar } r \alt \prg{writeTVar }r\;M \\
& | & \new{ \prg{newTMIVar }N\;M } \\
& | & \new{ \prg{readTMIVar } r } \alt \new{ \prg{writeTMIVar }r\;M } \\
& | & \new{ \prg{authorized }N\;M } \alt \new{ \prg{liftSTM } M} \\
& | & \new{ \prg{getlog} } \alt \new{ \prg{UnauthorizedError} } \\
% \alt \new{ \prg{abort} }\\
\\
M,N & ::= & x \alt V \alt M\;N \alt \dots \\
\end{array}
\end{gather*}
\end{minipage}}\caption{Syntax of values ($V$) and terms ($N,M$)}
\label{fig:syntax2}
\end{figure}



Since the type of security descriptors is application specific, our new monad type
is polymorphic,
\begin{lstlisting}[style=small]
TMI d a
\end{lstlisting}
where \lstinline+d+ is the type of descriptors and \lstinline+a+ is the type returned by the action. For
security sensitive variables, we have a type similar to \lstinline+IORef a+ and \lstinline+TVar a+,
\begin{lstlisting}[style=small]
TMIVar d a
\end{lstlisting}
An instance of this type is a cell with a security descriptor of type \lstinline+d+ and a value of type \lstinline+a+.
While the value can change over time, the security descriptor is specified when the cell is created and cannot
change after that. Creation, reading and writing of cells is performed with the following set of functions.
\begin{lstlisting}[style=small]
  newTMIVar :: d -> a -> TMI d (TMIVar d a)
 readTMIVar :: TMIVar d a -> TMI d a
writeTMIVar :: TMIVar d a -> a -> TMI d ()
\end{lstlisting}

For an example, the following code defines a descriptor type for a bank account. The account itself is represented
by a simple integer.
\begin{lstlisting}[style=small]
-- Security descriptor for accounts
data AccountDescr = AccountDescr {
    acctOwner  :: String,
    acctNumber :: Int
}
type Account = TMIVar AccountDescr Int

createAccount :: String -> Int -> Int 
                               -> TMI Account
createAccount owner number balance = 
    newTMIVar (AccountDescr owner number) balance
\end{lstlisting}
The next function demonstrates reading and writing of the security-relevant account variables.
\begin{lstlisting}[style=small]
deposit :: Account -> Int -> TMI AccountDescr ()
deposit acct amount = 
    do balance <- readTMIVar acct
       writeTMIVar acct (balance + amount)
\end{lstlisting}

To turn a TMI action into an STM action, we need to associate it with a security manager, i.e. a boolean function
that evaluates the transaction introspection log of security-relevant accesses and determines if the transaction should be aborted. As an input
to this function, TMI defines the type of an introspection log.
\begin{lstlisting}[style=small]
data AccessType = CreateVar | ReadVar | WriteVar
type TMILog d = [(AccessType, d)]
\end{lstlisting}
To specify the application specific policy, the programmer must supply the security manager, a function
of the type \lstinline+TMILog d -> Bool+. This function, along with a TMI action is passed to the
\lstinline+authorized+ function. The simplest security manager is one that performs no authorization and
simply allows all operations.
\begin{lstlisting}[style=small]
allowAll :: forall d. TMI d a -> STM a
allowAll tx = authorized (const True) tx
\end{lstlisting}
A slightly more complex example is a security manager that looks at all \lstinline+Account+s touched
by a transaction and verifies that they belong to the current user. The current user is passed to the
security manager as the first argument, and this currying ensures 
that we satisfy the type required by \lstinline+authorized+.
% NOTE: the comment in the code said ``Defined my the TMI module'' and I changed to ``by''
\begin{lstlisting}[style=small]
auth :: String -> TMILog AccountDescr -> Bool
auth user thelog = all checkowner thelog
    where
      checkowner :: (AccessType, AccountDescr) 
                 -> Bool
      checkowner (_,descr) = 
            user == (acctOwner descr)

-- Defined by the TMI module:
-- authorized :: (TMILog d -> Bool) 
--            -> TMI d a 
--            -> STM a

main = 
  do acct <- atomically (allowAll mkAccount)
     atomically (doDeposit acct "alice")  -- OK
     atomically (doDeposit acct "bob")    -- FAILS
  where
     mkAccount = createAccount "alice" 123456 0
     doDeposit acct user =
         authorized (auth user) (deposit acct 42)
\end{lstlisting}

Since TMI actions are ultimately executed as STM actions, we also provide a lifting operation to lift
STM operations into TMI operations, \lstinline+liftSTM+. 
This allows for the embedding of an STM action inside a TMI action.
Once the TMI action is turned into an STM action via \lstinline+authorized+, the embedded action is
just composed with it in the normal way. An interesting effect of this is that it allows for nested
calls to \lstinline+authorized+. While this might cause ambiguity for other implementations,
in this Haskell-based implementation such nesting has 
clear and well-defined semantics, and can therefore be permitted.
In fact, we will make explicit use of such nesting in the following sections to
implement {\em privilege amplification}.
% �etta h�r a� ofan er eitt sem v�ri f�nt a� s�na � merkingafr��inni.
% Ef �etta er svona s�per clear, �� g�tum vi� s�nt �a� og gert a� s�luv�ru.
% Eins og �a� er n�na, �� er �etta ekkert s�per clear fyrir mig, btw.

% Hvar eru orElseTMI og vinir � k��anum?  Eiga �eir ekki a� vera neinsta�ar?
TMI actions are also composable in the same way STM actions are. This means the monadic bind acts as
sequential composition and we provide \lstinline+orElseTMI+ and \lstinline+retryTMI+ that behave as
their STM counterparts. When TMI actions composed with \lstinline+orElseTMI+ are turned into STM actions,
via \lstinline+atomically+, the security manager only sees the log entries for \lstinline+TMIVar+-actions
that are actually committed or could have affected the committed actions.
% subsubsection tmi_in_haskell (end)

% section transactional_memory_introspection (end)
