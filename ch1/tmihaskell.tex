\documentclass[preprint,natbib,10pt]{sigplanconf}

\usepackage{amsmath}
\usepackage{amssymb}
\usepackage{latexsym}

% This is now the recommended way for checking for PDFLaTeX:
\usepackage{ifpdf}

\ifpdf
\usepackage[pdftex]{graphicx}
\else
\usepackage{graphicx}
\fi

\ifpdf
\usepackage{pdfsync}
\fi

\usepackage{rotating}

\usepackage{color}

\makeatletter
\newenvironment{tablehere}
  {\def\@captype{table}}
  {}

\newenvironment{figurehere}
  {\def\@captype{figure}}
  {}
\makeatother

% Other packages and settings
\usepackage{fancyvrb}
\usepackage{shortvrb}
\usepackage{hyperref}
\usepackage{textcomp}
\usepackage{listings}

% Listings setup
\lstset{%
  language=Haskell,
  aboveskip=0.5\baselineskip,
  belowskip=0.5\baselineskip,
  basicstyle=\ttfamily,
  commentstyle=\itshape,
  emphstyle=\spotcolor,
  numberstyle=\scriptsize,
  breaklines=true,
  basewidth=0.5em,
  numbersep=8pt,
  fontadjust=true,
  flexiblecolumns=true,
  keepspaces=false,
  extendedchars=true,
  tabsize=4,
  upquote=true}
\lstdefinestyle{small}{basicstyle=\small\ttfamily}
\lstdefinestyle{numbered}{numbers=left,stepnumber=1}

\lstdefinelanguage{pseudo}{%
    morekeywords={function, if, return, try, catch, main, program, transaction, declare, sensitive, array, of},
}

%------------------------------------------------------------------------------
%                                Space savers. 
%------------------------------------------------------------------------------
%
% Space saving List environment for enumerations. Does not indent any of the
% items.
\newcounter{myctrsquish}
\newenvironment{mylistsquish}{\begin{list}{(\textbf{\arabic{myctrsquish}})}
{\usecounter{myctrsquish}
\setlength{\topsep}{1mm}\setlength{\itemsep}{0.5mm}
\setlength{\parsep}{0.5mm}
\setlength{\itemindent}{0mm}\setlength{\partopsep}{0mm}
\setlength{\labelwidth}{-2mm}
\setlength{\leftmargin}{0mm}}}{\end{list}}

% This mylist environment indents items, and saves less space than the above.
\newcounter{myctr}
\newenvironment{mylist}{\begin{list}{(\textbf{\arabic{myctr}})}
{\usecounter{myctr}
\setlength{\topsep}{1mm}\setlength{\itemsep}{0.5mm}
\setlength{\parsep}{0.5mm}
\setlength{\listparindent}{\parindent}
\setlength{\itemindent}{-.5ex}\setlength{\partopsep}{0mm}
\setlength{\labelwidth}{-2mm}
\setlength{\leftmargin}{0mm}}}{\end{list}}

% Space saving List environment for itemizing.
\newenvironment{mybullet}{\begin{list}{$\bullet$}
{\setlength{\topsep}{1mm}\setlength{\itemsep}{0.5mm}
\setlength{\parsep}{0.5mm}
\setlength{\listparindent}{\parindent}
\setlength{\itemindent}{0mm}\setlength{\partopsep}{0mm}
\setlength{\labelwidth}{15mm}
\setlength{\leftmargin}{4mm}}}{\end{list}}

% Space saving List environment for subitemizing.
\newenvironment{mysubbullet}{\begin{list}{--}
{\setlength{\topsep}{1mm}\setlength{\itemsep}{0.5mm}
\setlength{\parsep}{0.5mm}
\setlength{\itemindent}{0mm}\setlength{\partopsep}{0mm}
\setlength{\labelwidth}{15mm}
\setlength{\leftmargin}{4mm}}}{\end{list}}


% Definitions for SOS rules

\newcommand{\sos}[2]{\frac{#1}{#2}}
\newcommand{\trans}[1]{\;#1\;}
\newcommand{\prg}[1]{\texttt{#1}}
\newcommand{\soslbl}[1]{\textit{(#1)}}

\newcommand{\alt}{\;\;|\;\;}

\newcommand{\tarrow}{\trans{\rightarrow}}
\newcommand{\tdarrow}{\trans{\Rightarrow}}
\newcommand{\tddarrow}{\trans{\rightharpoonup}}

\newcommand{\PP}{\mathbb{P}}
\newcommand{\EE}{\mathbb{E}}
\renewcommand{\SS}{\mathbb{S}}

\DeclareMathOperator{\img}{img}
\DeclareMathOperator{\dom}{dom}

\newcommand{\lsem}{[\![}
\newcommand{\rsem}{]\!]}

\newcommand{\nequiv}{\not\equiv}

% Comment this out if we prefer slashed o:
\renewcommand{\emptyset}{\{\}}

% for shading the background
%\definecolor{old}{gray}{0}
%\definecolor{newcolor}{gray}{0.85}
%\newcommand{\new}[1]{\colorbox{newcolor}{#1}}

% for coloring the text
%\definecolor{old}{gray}{0}
%\definecolor{newcolor}{rgb}{0.8,0,0}
%\newcommand{\new}[1]{\textcolor{newcolor}{#1}}

% new is black, old is gray
\definecolor{old}{gray}{0.3}
\newcommand{\new}[1]{\textcolor{black}{#1}}

\newcommand{\todo}[1]{[{\textcolor{red}{\bf TODO:}} #1]}

\begin{document}

\ifpdf
\DeclareGraphicsExtensions{.pdf, .jpg, .tif}
\else
\DeclareGraphicsExtensions{.eps, .jpg}
\fi

\conferenceinfo{PLAS '09}{June 15, Dublin.} 
\copyrightyear{2009} 
\copyrightdata{[to be supplied]} 

\titlebanner{}        % These are ignored unless
\preprintfooter{TMI Semantics and Haskell implementation}   % 'preprint' option specified.

\title{An Implementation and Semantics for Transactional Memory Introspection in Haskell}
\subtitle{}

\authorinfo{Arnar Birgisson}
           {Reykjav\'{\i}k University}
           {arnarb07@ru.is}
\authorinfo{\'Ulfar Erlingsson}
           {Reykjav\'{\i}k University \\ Microsoft Research, Silicon Valley}
           {ulfar@ru.is}

\maketitle

\begin{abstract}
    Transactional Memory Introspection (TMI) is a novel 
    reference monitor architecture that provides 
    complete mediation,
    freedom from {\em time of check to time of use} bugs and 
    improved failure handling for authorization.
    TMI builds on and integrates with implementations of the 
    Software Transactional Memory (STM) architecture~\cite{harrisFraserSTM}. 
    In this paper we present a formal
    definition of TMI and a concrete implementation over the Haskell STM.
We find that this specification and reference implementation
    establishes clear semantics for the TMI architecture.
In particular, they help identify and resolve ambiguities 
    that apply to implementations
    such in our prior work~\cite{tmi}.
\end{abstract}

\category{D.4.6}{Operating Systems}{Access controls}
\category{D.1.3}{Software}{Concurrent programming}

\terms {Languages, Security}

\keywords {Reference monitors, Transactional memory}

Languages are among the most important and ubiquitous concepts in Computer Science.
In almost every sub-field of Computer Science, one can find specific languages for
describing and reasoning about the concepts of that field. Programming languages
are the best known example of course, but file formats, network protocols,
instruction sets and even the various diagrammatic techniques used in Software
Engineering can be considered as (visual) languages. Theoretical Computer Science has
specification languages and various logics. Natural Language Processing has markup
languages for describing voicing and sentence structure. Artificial Intelligence
uses specific languages for describing behaviour, the rules of games and the
constraints of planning problems. Indeed, it seems to be common practice in 
Computer Science to invent formalisms for specific problems, and these formalisms
very often involve some kinds of languages.

Any language consists of two parts: its \emph{syntax} and its \emph{semantics}.
The syntax defines what strings of symbols are valid, i.e. part of the language,
while the semantics defines the actual \emph{meaning} of any valid string.
Formal specification of syntax is very common, even in non-academic use of 
Computer Science. However, one can argue that what really 
defines the true nature
of a a language
is its semantics. Different languages are much rather set apart by different
semantics than different syntax.
This thesis is a study, by way of example, of one specific technique of specifying
semantics formally, namely \emph{Structural Operational Semantics}. To put things
in context, we'll start by an informal overview of this field.


\section{Structural Operational Semantics}

Structural Operational Semantics~\cite{Plotkin04a, Plotkin04b},
SOS or simply Operational Semantics
\footnote{The term Operational Semantics is sometimes used as a synonym for
Structural Operational Semantics (\emph{small-step} semantics) and Natural
Semantics (\emph{large-step} semantics). In this thesis we are mostly
concerned with the former.}
for short,
is a way of defining the meaning of terms in formal languages. By
\emph{formal languages} we mean any language for specifying ideas formally in the mathematical
sense. This includes programming languages, process languages for modelling and
verification as well as many others. By \emph{terms} we usually mean programs
or specifications written in these languages.

As the name indicates, SOS describes semantics in terms of program \emph{structure}
and the \emph{operations} a program carries out as it computes. 
An SOS specification of the semantics for a certain
language is a collection of rules. These rules specify how a term with a certain
structure behaves, by describing the operation that the next step of execution
of this term performs
(often on a hypothetical machine), and what is the term that should be executed
for the next step. An example of an SOS rule is
\begin{equation}\label{eq:intro_rule1}
    \frac{t_1 \trans{a} t_1' \quad t_2 \trans{b} t_2'}{\textsf{op(}t_1,t_2\textsf{)}\trans{b} t_2'}
\end{equation}
This rule specifies how a term of the form $\textsf{op(}t_1,t_2\textsf{)}$ behaves, where $t_1$
and $t_2$ can be any sub-terms as allowed by the syntax. This is seen by looking
at the left-hand side of the \emph{conclusion}, the part appearing below the line. 
The two expressions above
the line are called \emph{premises}. These are operations of the sub-terms
that describe when the rule is applicable to derive a step of computation
of the composite program $\textsf{op(}t_1,t_2\textsf{)}$.
This particular rule only applies if the operations in the premises can be
deduced from the collection of rules. Naturally, this depends in each case on what
the terms $t_1$ and $t_2$ actually are. If the operations in all the premises
are valid for given sub-terms, the label of the arrow and the right-hand side below the line
specify what is the operation performed by executing $\textsf{op(}t_1,t_2\textsf{)}$
and what is the term that results after doing so. As one can see, both of these may
be parameterised with information from the premises.

The hypothetical execution of terms proceeds by finding a rule that matches the
term to be executed and whose premises are met. This rule then specifies an operation,
i.e. \emph{a single step} of execution, and the term to use for finding the
next step. This process is repeated to create a sequence of operations.
It is important to note that execution in this context does not necessarily mean
execution on a real machine, but rather it is a useful abstract metaphor for reasoning about
program behaviour. We say the sequences of operations are steps in the execution
of a program on a hypothetical machine.

Often it is useful to indicate when such a sequence may stop, i.e. when the
program terminates. We often do this by designating a specific set of terms as
\emph{terminal}, meaning that when a sequence of operations reaches such a term,
the application of rules stops.
Sometimes this is the empty term, e.g. a program of the form
\textsf{print "Hello"; print "World"} might perform the following sequence of
operations
\[
    \textsf{print "Hello"; print "World"}
    \;\trans{!"Hello"}\;
    \textsf{print "World"}
    \;\trans{!"World"}\;
    \epsilon
\]
where the operation $!string$ stands for writing $string$ to the screen, and
$\epsilon$ is the empty program. In other settings the terminal terms may represent
values. For example, a functional programming language might specify the meaning
of the term $50 - 4 \times 2$ with the following sequence of operations.
\[
    50 - 4 \times 2 \;\trans{}\; 50 - 8 \;\trans{}\; 42
\]
In this case, the term $42$ is terminal since it contains no operators.

There is an important difference between the two approaches; in the former case the natural meaning
of the program \textsf{print "Hello"; print "World"} is determined by the sequence
of operations that its execution goes through, while in the latter the meaning
of the program $50 - 4 \times 2$ is represented by the final value
that the sequence reaches. Which one we choose depends on the particular setting
in which we are using SOS.

For the latter interpretation, where the meaning of a term is taken to be the
final value reached by a sequence of operations, there is an important thing to
note about an SOS specification (collection of rules). In the general case, there
is nothing that prevents the specification to contain rules that allow us to
deduce \emph{multiple} sequences of operations. For example, consider a system
that contains the rule~\ref{eq:intro_rule1} above, but also contains the following
rule.
\begin{equation}\label{eq:intro_rule2}
    \frac{t_1 \trans{a} t_1' \quad t_2 \trans{b} t_2'}{\textsf{op(}t_1,t_2\textsf{)}\trans{a} t_1'}
\end{equation}
Presented with a term of the form $\textsf{op(}t_1,t_2\textsf{)}$, we can see
that both rules may apply (depending on $t_1$ and $t_2$). If they do, we have
a case of non-determinism where the term may either be executed according to
rule~\ref{eq:intro_rule1} or rule~\ref{eq:intro_rule2}. In fact, an operator
with this pair of rules is known as a \emph{choice operator}, i.e. the execution
of the term $\textsf{op(}t_1,t_2\textsf{)}$ may choose whether it behaves like
$t_1$ or like $t_2$.

It is not difficult to see that, when we take the meaning of a program to be its
final value, if such non-determinism exists in the SOS specification, this meaning
is not well defined. A term might give rise to multiple sequences of operations and
thus multiple final values. Thus, when this view of meaning is taken, which is
common when dealing with functional languages, we often make the requirement that
the language's specification given by the rules is \emph{deterministic}, 
i.e. for each term
there is at most one operation and subsequent term that can be deduced from the
collection of rules. Such collection of rules are the topic of Chapter~\ref{ch:formats}
of this thesis.

In the other setting, where meaning of a term is taken to be the sequence of
operations it gives rise to, non-determinism is generally allowed. This is for example
the case in Process Algebra, where the meaning of a term is simply taken to be
determined, in some formal sense, by
the set of all possible behaviours it may generate. Two terms might for instance be considered
equal if they generate the same set of operation sequences.

\vspace{1em}

Sometimes the terms of the language don't contain enough information themselves
to model execution. This is for example the case in programming languages that have
variables which are globally bound. To find the value of a program term $3 + x$,
one needs to know the value of $x$. In SOS specifications, this is solved by using
\emph{configurations} instead of terms in the rules. A configuration is a predefined
structure which models the state of the execution completely. In the case of languages
with variables, a common formulation is to represent the states as pairs of a
term (with the same meaning as described above) and a \emph{variable store}, written
$\langle t, \Theta \rangle$. The variable store is in turn a function from the
set of variables to actual values.\footnote{Or terms in the case of lazy languages.}
A typical rule in such a language might look like this.
\begin{equation}\label{eq:intro_rule3}
    \frac{}{\langle \textsf{x := }n, \Theta\rangle \trans{}
            \langle \epsilon, \Theta[x\mapsto n]\rangle} \qquad n\in \mathbb{N}
\end{equation}
Note that this rule has no premises, which means that it applies whenever the term
to be executed matches the left-hand side of the conclusion.
The rule specifies that the operation of executing an assignment term, e.g.
\textsf{x := 28}, under a store $\Theta$, results in a configuration with an
empty term and a store that is identical to $\Theta$ except for its value in $x$,
which is mapped to $28$ (this is the conventional meaning of the $[\cdot\mapsto\cdot]$ syntax).
Formally there is nothing special about using configurations instead of terms;
configurations can themselves be considered ``terms'' of an extended language.

Another interesting thing to note about rule~\ref{eq:intro_rule3} is that it is
in fact a \emph{rule schema}. In other words, it represents a countably infinite
number of rules, indexed by the natural number $n$. This is common when a part
of the syntax of the language comes from a large domain such as $\mathbb{N}$.

\vspace{1em}

This thesis consists of three main chapters, each of which is an independent paper.
While their topics are in essence not related to each other, they all make use of
operational semantics in a central manner. Although familiarity with SOS helps, the
informal introduction above should provide the reader with enough background to read
Chapter~\ref{ch:tmi}, which exemplifies a fairly complex use case of SOS. 
Chapters~\ref{ch:decomp} and especially~\ref{ch:formats} use semantics in a more
formal way; these chapters will each introduce the necessary preliminaries needed
for their discussion. The following section introduces each chapter and highlights
their ties to operational semantics.


\section{Thesis contributions}

Over the course of 12 months, rather than working solely on one single MSc study project,
I have participated in several research projects at
Reykjavik University and at the Technical University of Eindhoven. The result of this work
are research contributions made by me and my co-authors to a few different fields
of Computer Science. Each of these projects have built on the theory of SOS; in fact
one of the projects (Chapter~\ref{ch:formats}) is only about the theory of SOS rule
systems, independent of their use.

The papers are arranged in order of increasing abstraction. Chapter~\ref{ch:tmi} uses SOS
to specify the semantics of an authorisation framework in a functional programming language.
The SOS specification presented is non-trivial, but is intended solely for clarifying
the semantics of this particular framework. The specification is accompanied by a
detailed discussion of the semantics as well as an implementation of the framework
in question.

Chapter~\ref{ch:decomp} is in the field of Process Algebra. It 
uses SOS to provide quotienting techniques a la~\cite{Larsen91} for extensions
to the process specification language CCS and Hennessy-Milner logic. CCS has a simple
operational semantics and the paper proves, using the semantics, a powerful theorem
for studying properties that include past modalities in a decompositional manner.

Finally, Chapter~\ref{ch:formats} goes one abstraction level above SOS specifications
and provides so-called \emph{meta-theorems} about rule systems that guarantee their
determinism and the idempotence of certain operators. The meta-theorems consist of
syntactic conditions on the rules themselves, such conditions are generally referred
to as \emph{SOS rule formats}.

In order to give the reader enough background for each chapter, we will now describe
the general field of each paper, its contributions as well as highlight the specific
contributions I made to each.


\subsection{Semantics of Transactional Memory Introspection}

Chapter~\ref{ch:tmi} builds on previous work of \cite{tmi}. In that paper we present
an authorisation architecture called Transactional Memory Introspection, or TMI.
The motivation for this architecture comes from the fact that Software Transactional
Memory has recently become a popular way of avoiding race conditions in concurrent
programs. Software Transactional Memory, or STM for short, tackles the issue of
shared memory by replacing programmer managed locks with transactions. Where programmers
would conventionally manage access to shared resources by careful lock placement,
they may use STM instead to offload this responsibility to a machine controlled
framework.

When using STM, programmers do away with lock management and instead mark sections
of code as \emph{atomic}. At run-time, an STM system will, as part of the program
in question, ensure that the accesses a single thread makes to shared resources
inside such atomic sections, appear atomic to other threads and moreover, are
completely isolated from the effects concurrent threads may have on this shared
state. Semantically this is equivalent to enforcing a rule which says that only
one thread may be running in an atomic section at each time.

The beauty of STM comes from the fact that the actual implementation 
does not enforce such strict policies, 
as that would hurt performance. Instead, multiple threads are allowed to
execute simultaneously inside atomic sections. Meanwhile, the STM system will carefully
monitor the actions of each one of the threads. Generally, the threads will be
accessing disjoint sets of resources, so most of the time this simultaneous execution
poses no problems. However, in the cases where threads in atomic sections do conflict
in their accesses, the STM system will notice and simply roll back some or all
of the threads involved, and restart their execution at the start of their atomic
sections. A rollback consists of undoing all work done by the threads, and will
be triggered in cases when the execution of a thread has violated the isolation
guarantee of the STM system. In practice, such violations happen in the minority
of cases, so often the overhead of this approach will be paid for by the overhead
saved in not using fine grained locking.

To implement the above, an STM system generally must provide
\begin{itemize}
    \item isolation of concurrent threads in atomic sections,
    \item monitoring of resource access to detect conflicts,
    \item the ability to abort and rollback execution of an atomic section.
\end{itemize}
In~\cite{tmi} we argue that these mechanisms can be very beneficial to the problem
of \emph{policy enforcement}. Policy enforcement (or authorisation) is required
in programs that handle sensitive data, to ensure that no illegal operations are
performed, such as releasing confidential data or otherwise violate the applications
policy. Traditionally this is done by careful code scrutiny and great effort on
the programming side. Just as with locking, this practice is prone to errors.

Since programs that use STM systems for synchronisation purposes are already paying
the price of monitoring and maintaining the ability to abort code, we conjectured
that these mechanisms could be used to simplify policy enforcement at a relatively
little extra cost. We identified three common problems (or errors) in modern policy
enforcement code.
\begin{itemize}
    \item \emph{Time of check to time of use} (TOCTTOU) bugs. These happen when
    a policy decision is made prior to access, but the state used for the decisions
    is mutated in between by a concurrent thread.
    \item Difficulty in guaranteeing \emph{complete mediation}, i.e. ensuring that
    any access, explicit or implicit, is accompanied by the relevant policy check.
    This is non-trivial in complex systems and empirical studies show that this
    is a source of several security holes in critical software.
    \item Difficulty in dealing with authorisation errors, when a policy violation
    has been detected, the system state must be carefully reset in order not to leak
    sensitive information or implicitly cause other policy violations.
\end{itemize}
The first of these is simply a synchronisation issue, and could be solved with locking.
However, STM systems provide a synchronisation mechanism with added benefits; we
can make use of its careful monitoring and abort capabilities to severely reduce
the second and third difficulties.

When an application that makes use of TMI (which implies the use of STM) runs,
any accesses made inside atomic sections are inspected by the STM system. TMI
hooks into this inspection and also notifies an application specific security
manager, which checks if the access is allowed by the application policy. At any
time, the security manager has the capability to veto an access due to policy
violation, in which case the abort mechanism of the STM is invoked. In one fell
swoop this solves the issue of complete mediation, since the STM diligently
inspects every access, as well as the issue of error handling since the rollback
puts the system back into a consistent state and the isolation guarantees of the
STM make sure that no concurrent thread gained knowledge of the actions leading
up to the policy violation.

Our previous work of~\cite{tmi} consists of an extended discussion of the above,
accompanied by a proof-of-concept implementation based on a prototype STM framework
for Java~\cite{hlm06}. However, while working with TMI and STM systems in general,
we discovered that there are a great number of subtleties in the behaviour of
unusual edge cases. An informal discussion, and even an implementation, did not
provide a thorough understanding of the semantics of TMI. Thus the contribution
presented in Chapter~\ref{ch:tmi} consists of the formal specification of the semantics
of our architecture, in the form of an extension to the semantics of the Haskell
STM system~\cite{haskellstm}. The semantics is accompanied by a matching implementation.

My specific contributions to Chapter~\ref{ch:tmi} consist of most of the technical
work involved. I built the extension of the Haskell STM semantics, which went
through several iterations of discussions with my co-author and revisions. In
parallel I wrote the implementation in Haskell, which provided a lot of insight
into the design decisions behind the semantic specification. I wrote the initial
versions of most of the text, except for the introduction and the background on
STM and TMI. All sections underwent a rewriting phase carried out jointly by
me and my co-author.

This work has been submitted to the ACM SIGPLAN Fourth Workshop on Programming
Languages and Analysis for Security (PLAS 2009), scheduled for June 15th 2009
in Dublin, Ireland. The author notification will arrive before the final version
of this thesis is prepared. The submission is identical to the version presented
here.
\section{Background} % (fold)
\label{sec:background}

\subsection{STM and TMI} % (fold)
\label{sub:stm_and_tmi}

STM provides attractive guarantees for multithreaded software; namely atomicity,
consistency and isolation of specifically marked blocks of code in {\em transactions}.
In general, STM implementations must do so by performing

\begin{mybullet}
    \item careful monitoring of the resources that are accessed within a transaction,
    \item validation of the accesses of concurrent transactions, and
    \item complete rollback of the effects of aborted transactions.
\end{mybullet}

TMI builds on this machinery and allows security enforcement to benefit from
the STM guarantees. TMI helps the programmer to write correct enforcement mechanisms
and simplifies error-handling. In~\cite{tmi} we outline three main benefits of TMI:

\paragraph{Complete mediation.} TMI provides complete mediation by implicitly invoking
the reference monitor before any effects of a transaction are permanently committed. The
reference monitor validation checks
are able to inspect the resource access logs of the STM and may veto the
commit if an application specific policy is violated. 
In general, this requires that STM mechanisms provide
strong atomicity, i.e. resources marked for transactional scrutiny may not be accessed outside
the scope of a transaction.

\paragraph{Freedom from TOCTTOU bugs.} {\em Time of check to time of use} (TOCTTOU) bugs arise
in conventional enforcement mechanisms when interleaved threads may affect the policy decisions
of each other. For example, a thread may make a policy-based decision to allow access to a
certain resource, e.g. reading a memory location. Before that operation is actually performed, 
execution may be preempted by another thread. That thread can change the global state so that the 
policy decision becomes invalid,
e.g. by writing privileged information into the memory location.

This problem is implicitly solved by using STM, which guarantees that transactions are 
isolated and cannot affect the policy decisions of each other.

\paragraph{Simplified error handling.} In the event of an authorization failure, TMI uses the STM
facilities to completely roll back the effects of the transaction in question and raise an appropriate
exception to the code that initiated the transaction. This frees the programmer from having to
undo state changes leading up to the unauthorized operation, a common source of 
errors~\cite{errorHandlingMistakes}.

% subsection stm_and_tmi (end)

\subsection{Haskell STM} % (fold)
\label{sub:haskell_stm}

For a formal treatment, we build our semantics and implementation on those of the Haskell STM~\cite{haskellstm},
which in turn is built on Concurrent Haskell~\cite{concurrenthaskell}. Concurrent Haskell is an extension to Haskell 98,
a lazy (i.e. call-by-name), pure, functional language. It supports concurrent threads and communications
between them. Non-pure computations are modelled with {\em monads}~\cite{monads}; this includes computations
with side-effects such as input/output and mutable state.

The main entry to a Haskell program is an instantiation of the I/O monad, i.e. a value that represents an
{\em action} of the type \lstinline+IO ()+. An action of this type can, in addition to performing pure
computation, perform other I/O actions by way of composing smaller actions into larger ones. For an example,
Haskell standard libraries define the basic I/O actions \lstinline+getChar+ and \lstinline+putChar+, which
read from standard input and write to standard output, respectively. The most common composition is simple
sequencing. For example the composed I/O action
\begin{lstlisting}[style=small]
main = do { c <- getChar; putChar c; putChar c }
\end{lstlisting}
defines an action that, when executed, will perform the three actions listed in sequence.

In general a value of type \lstinline+IO a+ represents an action that when executed, may perform some
I/O operations as defined by the Haskell libraries and then result in a value of type \lstinline+a+.
Pure functions cannot execute such actions without jeopardizing their purity and this is neatly enforced
by the Haskell type system. Naturally, I/O actions are however free to run pure computations. Thus
the only way to get at the value of an action is if it is a part of a bigger I/O action. The Haskell
runtime bootstraps the whole process by executing the special I/O action called \lstinline+main+.

In addition to conventional input and output, I/O actions can perform reads and updates of mutable memory
cells. The type \lstinline+IORef a+ represents a mutable cell that contains a value of type \lstinline+a+. 
Haskell provides the basic I/O actions \lstinline+newIORef+, \lstinline+readIORef+ and \lstinline+writeIORef+
for manipulation of such cells. As with other I/O actions, these operations can only be used when composing
larger I/O actions.

Concurrent Haskell supports explicit forking of threads through the I/O action \lstinline+forkIO+.
\begin{lstlisting}[style=small]
forkIO :: IO a -> IO ThreadID
\end{lstlisting}
\lstinline+forkIO+ takes another I/O action as a parameter and spawns a new thread to execute the action, 
immediately
returning a newly allocated thread identifier. For further discussion of concurrency we refer to~\cite{awkward}
or tutorials such as~\cite{partutorial}.

The Haskell STM is based on a monadic type similar to the one for I/O actions, namely \lstinline+STM a+.
A value of this type represents an {\em STM action}, which when executed may perform smaller STM actions and
result in a value of type \mbox{\lstinline+a+.} %  dot falls on a second line. I tried mbox (dunno if it works)
STM actions may contain pure computations as well, but note that
they {\em cannot} contain e.g. I/O actions. The main STM actions provided are actions that allow manipulations
of another kind of memory cells, which have the type \mbox{\lstinline+TVar a+}, % This breaks accross lines, tried mbox
where \lstinline+a+ is the type of
value that the cell holds. The actions are \lstinline+newTVar+, \lstinline+readTVar+ and \lstinline+writeTVar+,
so they have the same power as their I/O counterparts. The important thing to note is that sets of \lstinline+IORef+s
and \lstinline+TVar+s are kept separate; one can only be used in I/O actions and the other in STM actions.

STM actions can be composed. Similar to I/O actions, the most common composition is sequencing,
but in addition Haskell STM provides the basic STM action \lstinline+retry+ and a combinator
\lstinline+orElse+. The action \lstinline+retry+ is a blocking operation for 
STM actions, which restarts the current transaction with potentially updated \lstinline+TVar+ contents. 
By issuing \lstinline+retry+, the programmer is stating
that the current transaction cannot finish for the state of \lstinline+TVar+s it started in. 
The Haskell STM provides an optimized implementation of \lstinline+retry+. This implementation captures
the set of \lstinline+TVar+s that a transaction has read before the retry, and suspends the transaction
until at least one of those \lstinline+TVar+s has been updated.
This makes sense because the {\em only} outside factors that can
affect the execution of an STM action are the values of the \lstinline+TVar+s it reads.

If \lstinline+t1+ and \lstinline+t2+ are STM actions, then \lstinline+t1 `orElse` t2+
is an STM action that first tries performing \lstinline+t1+ on its own. 
If \lstinline+t1+ invokes the \lstinline+retry+ action,
then the combined action rolls back the effects of \lstinline+t1+ 
and tries \lstinline+t2+ instead. If that one retries also, the whole
action retries, but waits for updates on the variables read by {\em both} \lstinline+t1+ and \lstinline+t2+.

For an example how the above can be used for synchronization primitives such as communication channels, see Section
4 of~\cite{haskellstm}.

While the basic STM actions and their compositions give us a way to build larger STM actions, we have not
discussed how those actions can be run or how they relate to transactions. For this, STM Haskell provides us
with the \lstinline+atomically+ function,\footnote{While~\cite{haskellstm} uses the name \lstinline+atomic+, the actual implementation
of the Glasgow Haskell Compiler uses \lstinline+atomically+.} whose type is
\begin{lstlisting}[style=small]
atomically :: STM a -> IO a
\end{lstlisting}
This function gives an I/O action that, when performed, will execute the input STM action.
The atomicity comes from the fact that STM Haskell will guarantee that the effects that the STM action has on
\lstinline+TVar+s are atomic, i.e. they all become visible at once and that what happens
inside the STM action is not affected by concurrent threads.  

The Haskell STM system does this by monitoring concurrent invocations of STM actions, taking care of rolling
them back if they conflict with each other and retrying them.
As an example, the following program creates
a transactional variable holding a counter and spawns three threads that each increments the counter atomically.
\begin{lstlisting}[style=small]
increment :: TVar Int -> STM ()
increment counter = do x <- readTVar counter
                       writeTVar counter (x + 1)

main = do c <- atomically (newTVar 0)
          forkIO (atomically (increment c))
          forkIO (atomically (increment c))
          forkIO (atomically (increment c))
\end{lstlisting}
The \lstinline+increment+ action is a classical example of where a race condition might occur 
in a traditional setting,
but in our situation the STM system will guarantee the atomicity of each invocation.

% subsubsection haskell_stm (end)

% section background (end)

\section{Transactional Memory Introspection} % (fold)
\label{sec:transactional_memory_introspection}

In this section we give an overview of the TMI architecture and how it is implemented. We then describe
our Haskell implementation from a user standpoint.

\subsection{Overview of TMI} % (fold)
\label{ssub:overview_of_tmi}

As described in our previous work~\cite{tmi}, the TMI architecture aims to  raise the level
of abstraction in the implementation of security enforcement mechanisms. It allows the programmer
to decouple application logic from security enforcement. Just as STM frees the programmer from worrying
about lock acquisition order and other synchronization efforts, TMI can be used to eliminate concerns about check
placement, race conditions and exceptional execution paths.

TMI provides these guarantees by imposing on the STM system. 
The programmer marks certain variables as security
sensitive. This implicitly indicates to the STM system that these variables are shared,
and ensures that the STM system will protect against race conditions in accesses to the variables.
TMI enhances the monitoring of these security sensitive variables
by ensuring that an access-control reference monitor is invoked every time
that the variables are accessed.

\paragraph{\emph{Time of policy evaluation with TMI:}} % Changed proposed header
% Didn't apply changes in below para
The TMI architecture only loosely constrains when policy must be evaluated,
and in~\cite{tmi} we consider a number of alternatives. In particular, TMI enforcement
can be \emph{eager} or \emph{lazy}.
With {\em eager} enforcement,
every access to a variable triggers the reference monitor, which immediately checks it against the
relevant policy. If authorization is denied, the transaction is immediately aborted. With lazy enforcement,
accesses to variables are simply logged (often they are already logged by the STM) and 
the logs are inspected by the reference monitor only at the end
of the transaction. If any of the logged accesses
are invalid, the whole transaction is aborted. 

% Applied some, not all, changes in below para
A key property of TMI enforcement is that policy decisions
can be evaluated at any time, as long as they are evaluated in a serialized fashion,
and evaluation is fully complete before the transaction commits.
A good STM system will ensure that each transaction is executed in isolation,
such that aborting one will have the same semantics as not having started it. 
This said, for our formal treatment and Haskell implementation, we focus 
on lazy enforcement only. Thus, the following
discussion only deals with the lazy variant unless otherwise noted.

\paragraph{\emph{Utilizing TMI enforcement:}}  % Changed proposed header
% Didn't apply changes in below para
To use TMI, the programmer declares a set of variables as security relevant.
This implicitly indicates to the underlying STM system
that those variables should be protected against race conditions.
This means the values held by these variables can only
be read or modified within a transaction, and that the STM system takes care of resolving conflicting accesses
by concurrent transactions. 
This also means that, upon every variable access,
TMI appends information identifying the variable in 
question to a transaction-specific {\em introspection log}. 
In particular, the introspection log will contain information about the creation, reading, and writing
of the security sensitive variables.

% Changed below para
In addition, TMI requires that all sections
of code that access security-sensitive variables
must be explicitly marked as \emph{atomic}.
To execute such atomic code sections,
programmers initiate a TMI transaction 
and provide a reference to the atomic block and a
{\em security manager}. The security manager is a block of code (or closure) that 
encodes the intended, application-specific security policy, 
and is able to determine
whether a transaction introspection log
satisfies the security policy.
The security manager closure includes the active principal, and other  auxiliary information that is 
needed to check policy compliance.

% Changed below para
% NOTE: I don't have a good spell checker, so please spell check.
Finally, transaction commit plays a special role in TMI enforcement.
TMI runs atomic blocks as transactions in the underlying STM system,
but changes the semantics of transaction commit.
After a TMI atomic block has finished execution, but before it is committed,
TMI ensures that the security manager has fully evaluated whether the transaction introspection log 
complies with the intended security policy.
Importantly, this evaluation occurs within the same STM transaction
as the execution of the atomic block,
and a commit of the transaction is attempted only if 
the security manager returns success.

Even when the transaction has complied with the security policies,
the attempted commit may still fail,
and the transaction is retried,
in the case when the STM system 
finds conflicting concurrent accesses.
Also,
if the security manager finds the transaction in violation of policy,
all state changes are rolled back---including changes to the security manager state,
in the case of history-based policies---and,
instead of retrying the transaction,
an exception is raised to the invoker of the atomic block.

\paragraph{\emph{A simple example:}} 
The following pseudo code shows what software
that makes use of TMI-based security enforcement might look like.
(Note that the code makes use of function-argument currying.)
\begin{lstlisting}[style=small,language=pseudo]
declare sensitive accounts = array of Account

function withdraw(account, amount):
    account.balance = account.balance - amount

function security_manager(user, log):
    if log contains <withdrawal from account>:
        if account.owner == user:
            return Allowed
    return Denied

main program:
    user = aquire_login_credentials()
    try:
        transaction with security_manager(user):
            withdraw(get_account(123456), 42)
    catch AuthorizationFailed:
        tell user about error
\end{lstlisting}
In this code, two aspects are especially noteworthy. First, 
security enforcement code is completely decoupled from the application logic
and the function \lstinline+withdraw+ performs no authorization. 
Even so, complete mediation is ensured,
since the introspection log is implicitly updated by the TMI reference monitor
upon each access to account variables.

Second, in the case of 
authorization failure (e.g. a withdrawal from a different user's account), the error handler 
need only consider how to indicate the error to the user.
The error handler need not clean up any mess:
the state changes that happened during the
transaction (if any)
have already been rolled back when the error handler starts execution. 
Although perhaps not apparent in this simplified example, 
there is ample evidence that writing correct cleanup code is difficult,
especially when multiple security-relevant operations are involved~\cite{errorHandlingMistakes}.

While the above observations form the two main benefits of the TMI architecture,
the third is freedom from TOCTTOU bugs.
Without TMI,
this example code 
might suffer from 
TOCTTOU race conditions,
e.g., if accounts could change owners. 
However, with TMI,
such account-ownership changes would be isolated,
and a transaction 
would be guaranteed to see the same owner throughout its execution.

% subsubsection overview_of_tmi (end)

\subsection{TMI in Haskell} % (fold)
\label{ssub:tmi_in_haskell}

We saw earlier how Concurrent Haskell uses the type system to confine operations on shared variables to
STM actions, and provides a single function to wrap STM actions into an atomic I/O action.
For TMI, we do something very similar. We confine operations on security sensitive variables to {\em TMI
actions}, and provide a single function to turn a TMI action into an STM action and associating it with a
security manager at the same time. 

Figure~\ref{fig:syntax2} shows the extensions of STM Haskell with the TMI extensions (highlighted).
We define a new monad that represents TMI actions and operations on sensitive variables. In addition,
we lift all standard STM functions to their TMI counterparts. This is done so that an existing Haskell STM program can
be easily adapted to TMI with minimal changes to their code. The TMI monad also encapsulates state, namely the
introspection log of a transaction. The introspection log contains entries which specify the access type (create,
read or write) of a variable and the {\em security descriptor} of a variable. Security descriptors are provided by the
programmer when she creates sensitive variables and contain the metadata about the variable 
that is necessary for authorization,
such as the owner of an account, permissions of a file, etc.

\begin{figure}
\flushleft{\begin{minipage}{\linewidth}
\small
\color{old}
\begin{gather*}
\begin{array}{rcl}
    x,y & \in & Variable \\
    r,t & \in & Name \\
    c & \in & Char \\
\\
V & ::= & r \alt c \alt \backslash x\prg{->}M \\
& | & \prg{return }M \alt M \,\prg{>{}>=}, N \\
& | & \prg{putChar }c \alt \prg{getChar} \\
& | & \prg{throw }M \alt \prg{catch }M\;N \\
& | & \prg{retry} \alt M \;\textrm{\lstinline{`orElse`}}\; N \\
& | & \prg{forkIO }M \alt \prg{atomically }M \\
& | & \prg{newTVar }M \\
& | & \prg{readTVar } r \alt \prg{writeTVar }r\;M \\
& | & \new{ \prg{newTMIVar }N\;M } \\
& | & \new{ \prg{readTMIVar } r } \alt \new{ \prg{writeTMIVar }r\;M } \\
& | & \new{ \prg{authorized }N\;M } \alt \new{ \prg{liftSTM } M} \\
& | & \new{ \prg{getlog} } \alt \new{ \prg{UnauthorizedError} } \\
% \alt \new{ \prg{abort} }\\
\\
M,N & ::= & x \alt V \alt M\;N \alt \dots \\
\end{array}
\end{gather*}
\end{minipage}}\caption{Syntax of values ($V$) and terms ($N,M$)}
\label{fig:syntax2}
\end{figure}



Since the type of security descriptors is application specific, our new monad type
is polymorphic,
\begin{lstlisting}[style=small]
TMI d a
\end{lstlisting}
where \lstinline+d+ is the type of descriptors and \lstinline+a+ is the type returned by the action. For
security sensitive variables, we have a type similar to \lstinline+IORef a+ and \lstinline+TVar a+,
\begin{lstlisting}[style=small]
TMIVar d a
\end{lstlisting}
An instance of this type is a cell with a security descriptor of type \lstinline+d+ and a value of type \lstinline+a+.
While the value can change over time, the security descriptor is specified when the cell is created and cannot
change after that. Creation, reading and writing of cells is performed with the following set of functions.
\begin{lstlisting}[style=small]
  newTMIVar :: d -> a -> TMI d (TMIVar d a)
 readTMIVar :: TMIVar d a -> TMI d a
writeTMIVar :: TMIVar d a -> a -> TMI d ()
\end{lstlisting}

For an example, the following code defines a descriptor type for a bank account. The account itself is represented
by a simple integer.
\begin{lstlisting}[style=small]
-- Security descriptor for accounts
data AccountDescr = AccountDescr {
    acctOwner  :: String,
    acctNumber :: Int
}
type Account = TMIVar AccountDescr Int

createAccount :: String -> Int -> Int 
                               -> TMI Account
createAccount owner number balance = 
    newTMIVar (AccountDescr owner number) balance
\end{lstlisting}
The next function demonstrates reading and writing of the security-relevant account variables.
\begin{lstlisting}[style=small]
deposit :: Account -> Int -> TMI AccountDescr ()
deposit acct amount = 
    do balance <- readTMIVar acct
       writeTMIVar acct (balance + amount)
\end{lstlisting}

To turn a TMI action into an STM action, we need to associate it with a security manager, i.e. a boolean function
that evaluates the transaction introspection log of security-relevant accesses and determines if the transaction should be aborted. As an input
to this function, TMI defines the type of an introspection log.
\begin{lstlisting}[style=small]
data AccessType = CreateVar | ReadVar | WriteVar
type TMILog d = [(AccessType, d)]
\end{lstlisting}
To specify the application specific policy, the programmer must supply the security manager, a function
of the type \lstinline+TMILog d -> Bool+. This function, along with a TMI action is passed to the
\lstinline+authorized+ function. The simplest security manager is one that performs no authorization and
simply allows all operations.
\begin{lstlisting}[style=small]
allowAll :: forall d. TMI d a -> STM a
allowAll tx = authorized (const True) tx
\end{lstlisting}
A slightly more complex example is a security manager that looks at all \lstinline+Account+s touched
by a transaction and verifies that they belong to the current user. The current user is passed to the
security manager as the first argument, and this currying ensures 
that we satisfy the type required by \lstinline+authorized+.
% NOTE: the comment in the code said ``Defined my the TMI module'' and I changed to ``by''
\begin{lstlisting}[style=small]
auth :: String -> TMILog AccountDescr -> Bool
auth user thelog = all checkowner thelog
    where
      checkowner :: (AccessType, AccountDescr) 
                 -> Bool
      checkowner (_,descr) = 
            user == (acctOwner descr)

-- Defined by the TMI module:
-- authorized :: (TMILog d -> Bool) 
--            -> TMI d a 
--            -> STM a

main = 
  do acct <- atomically (allowAll mkAccount)
     atomically (doDeposit acct "alice")  -- OK
     atomically (doDeposit acct "bob")    -- FAILS
  where
     mkAccount = createAccount "alice" 123456 0
     doDeposit acct user =
         authorized (auth user) (deposit acct 42)
\end{lstlisting}

Since TMI actions are ultimately executed as STM actions, we also provide a lifting operation to lift
STM operations into TMI operations, \lstinline+liftSTM+. 
This allows for the embedding of an STM action inside a TMI action.
Once the TMI action is turned into an STM action via \lstinline+authorized+, the embedded action is
just composed with it in the normal way. An interesting effect of this is that it allows for nested
calls to \lstinline+authorized+. While this might cause ambiguity for other implementations,
in this Haskell-based implementation such nesting has 
clear and well-defined semantics, and can therefore be permitted.
In fact, we will make explicit use of such nesting in the following sections to
implement {\em privilege amplification}.
% �etta h�r a� ofan er eitt sem v�ri f�nt a� s�na � merkingafr��inni.
% Ef �etta er svona s�per clear, �� g�tum vi� s�nt �a� og gert a� s�luv�ru.
% Eins og �a� er n�na, �� er �etta ekkert s�per clear fyrir mig, btw.

% Hvar eru orElseTMI og vinir � k��anum?  Eiga �eir ekki a� vera neinsta�ar?
TMI actions are also composable in the same way STM actions are. This means the monadic bind acts as
sequential composition and we provide \lstinline+orElseTMI+ and \lstinline+retryTMI+ that behave as
their STM counterparts. When TMI actions composed with \lstinline+orElseTMI+ are turned into STM actions,
via \lstinline+atomically+, the security manager only sees the log entries for \lstinline+TMIVar+-actions
that are actually committed or could have affected the committed actions.
% subsubsection tmi_in_haskell (end)

% section transactional_memory_introspection (end)

\section{Formal semantics of TMI} % (fold)
\label{sec:formal_semantics_of_tim}

\begin{figure}
\flushleft{\begin{minipage}{\linewidth}
    \small
\color{old}
\vspace{-1.5ex}
\begin{gather*}
\begin{array}{rrcl}
\textrm{Thread soup} & P,Q & ::= & M_t \alt (P\,|\,Q) \\
\new{\textrm{Descriptors}} & \new{D_\bot} & \new{::=} & \new{M \cup \{\bot\}} \\
\textrm{Heap} & \Theta & ::= & r \hookrightarrow M \new{\,\times\, D_\bot} \\
\textrm{Allocations} & \Delta & ::= & r \hookrightarrow M \new{\,\times\, D_\bot} \\
\new{\textrm{Access types}} & \new{T} & \new{::=} & \new{\{ \textsc{create}, \textsc{read}, \textsc{write} \}} \\
\textrm{\new{Log}} & \new{\Sigma} & \new{::=} & \new{\textrm{list monoid } ([], \oplus) \textrm{ over } T\times D} \\
\\
\textrm{Evaluation} & \EE & ::= & [\cdot] \alt \EE \,\prg{>{}>=}\,M \alt \prg{catch }\EE\;M \\
\textrm{contexts} & \SS & ::= & [\cdot] \alt \SS \,\prg{>{}>=}\,M \\
                  & \PP & ::= & \SS_t \alt (\PP|P) \alt (P|\PP) \\
\textrm{Action} & a & ::= & \prg{!}c \alt \prg{?}c \alt \epsilon
\end{array}
\end{gather*}
\end{minipage}}\caption{Program state and evaluation contexts}
\vspace{2ex}
\label{fig:contexts}
\end{figure}




To formalize the semantics of TMI, we build on the semantics for the Haskell STM
presented in~\cite{haskellstm}. The semantics is a structural operational semantics
in the style of Plotkin~\cite{Plotkin04a}. 
For the sake of completeness and to help the reader understand our extensions,
we give a cursory
explanation of the concepts of the semantics from~\cite{haskellstm} 
so that a reader not familiar with it may understand our extensions.

Figures~\ref{fig:sosadmin} through \ref{fig:sos2} give the operational rules that describe the steps a program may
take. At the top level, a program transforms a state of the form $P;\Theta$ via labelled transition.
\[
    P;\,\Theta \stackrel{a}{\tarrow} Q;\,\Theta'
\]
$P$ represents a program term in the syntax of Figure~\ref{fig:syntax2} while $\Theta$ stands for a
memory store, a partial function from variable names to annotated terms. An annotated value is a tuple
$(t,d)$ where $t$ is a program term and $d$ is a value that holds the security relevant description of
the relevant variable. The labels on transitions represent the programs input and output actions. $Q$
and $\Theta'$ represent the term that is left unevaluated and the updated store after a transition, 
respectively.

To model atomicity of transactions, separate relations represent the top level I/O transitions
and the STM actions. We extend this by adding a third relation representing the security relevant TMI
actions. Furthermore we add a simple relation for evaluation of security managers under the context of
an immutable transaction log.

Execution of a program proceeds by
non-deterministically
picking a program term from a collection of terms, each representing
a separate thread of execution. One I/O transition of this term combined with the current store is performed
and then the process is repeated. This models interleaved concurrency at the level of I/O transitions.
STM transitions however can only be performed as a required premise of the \lstinline+atomically+ operator
at the I/O level, and thus appear in this model as a single atomic step.

As mentioned in~\cite{haskellstm} there is no need to represent rollback, but contrary to the semantics
in that paper,
our extensions do need to formalize the notion of the transaction log as it is no longer purely an 
implementation detail. For simplicity though, we only model the log for security sensitive operations as
they are the only ones relevant to the semantics of TMI.

\begin{figure}[t]
\flushleft{\begin{minipage}{\linewidth}
    \small
\color{old}

\center{\framebox{Administrative transitions $\qquad M \tarrow N $}}
\begin{gather*}
    M \quad\tarrow\quad V  \quad \textrm{if $\,\mathcal{V}\lsem M \rsem = V$ and $M \nequiv V$}
    \qquad\soslbl{EVAL}
\\
\begin{array}{rcll}
\prg{return }N \,\prg{>{}>=}\, M
    & \tarrow
    & M\;N
    & \soslbl{BIND}
    \\
\prg{throw }N \,\prg{>{}>=}\, M
    & \tarrow
    & \prg{throw }N
    & \soslbl{THROW}
    \\
\prg{retry >{}>=}\, M
    & \tarrow
    & \prg{retry}
    & \soslbl{RETRY}
%\prg{catch retry }N
%    & \tarrow
%    & \prg{retry}
%    & \soslbl{CATCH3}
%    \\
%\prg{abort }N\,\prg{>{}>=}\, M
%    & \tarrow
%    & \prg{abort }N
%    & \soslbl{ABORT}
%    & &
%\prg{catch }(\prg{abort }M)\,N
%    & \tarrow
%    & \prg{abort }M
%    & \soslbl{CATCH4}
\end{array}
%
\end{gather*}

\rule{0.95\linewidth}{0.33pt}

\center{\framebox{I/O transitions $\qquad P;\,\Theta \stackrel{a}{\tarrow} Q;\,\Theta'$}}
\begin{gather*}
\begin{array}{rcll}
    \PP[\prg{putChar } c];\, \Theta & \stackrel{!c}{\tarrow}
                                       & \PP[\prg{return ()}];\, \Theta
                                       & \soslbl{PUTC} \\
    \PP[\prg{getChar }];\, \Theta& \stackrel{?c}{\tarrow}
                                       & \PP[\prg{return } c];\, \Theta
                                       & \soslbl{GETC} \\
    \PP[\prg{forkIO } M];\, \Theta& \tarrow
                                       & (\PP[\prg{return } t] \,|\, M_t);\, \Theta
                                            \quad t\notin \PP, \Theta, M
                                       & \soslbl{FORK} \\
    \PP[\prg{catch }(\prg{return }M)\, N];\,\Theta& \tarrow 
                                       & \PP[\prg{return } M];\,\Theta
                                       & \soslbl{CATCH1} \\
    \PP[\prg{catch }(\prg{throw }M)\, N];\,\Theta& \tarrow 
                                      & \PP[N\,M];\,\Theta
                                      & \soslbl{CATCH2}
\end{array}
\\
\sos{M \tarrow N}
    {\PP[M];\, \Theta \tarrow \PP[N];\, \Theta}
\quad\soslbl{ADMIN}
\\
%
% <<< Atomic
%
\sos{M;\,\Theta,\emptyset
     \stackrel{*}{\tdarrow}
     \prg{return }N;\, \Theta',\Delta'}
   {\PP[\prg{atomically }M];\,\Theta \tarrow
    \PP[\prg{return }N];\,\Theta'}
  \quad\soslbl{ARET}
\\
\sos{M;\,\Theta,\emptyset
     \stackrel{*}{\tdarrow}
     \prg{throw }N;\, \Theta',\Delta'}
   {\PP[\prg{atomically }M];\,\Theta \tarrow
    \PP[\prg{throw }N];\,\Theta\cup\Delta'}
  \quad\soslbl{ATHROW}
%\\
%\sos{M;\,\Theta,\emptyset
%     \stackrel{*}{\tdarrow}
%     \prg{abort }N;\, \Theta',\Delta',\Sigma'}
%   {\PP[\prg{atomically }M];\,\Theta \tarrow
%    \PP[\prg{throw }N];\,\Theta\cup\Delta'}
%  \quad\soslbl{ATHROW1}
% >>>
\end{gather*}

\end{minipage}}\caption{Evaluation of terms and monad operations and IO actions}
\label{fig:sosadmin}
\end{figure}



\subsection{Syntax, states and evaluation contexts} % (fold)
\label{sub:syntax}

The syntax of terms for a subset of STM Haskell is given in Figure~\ref{fig:syntax2} with our TMI-related
extensions (highlighted). Terms and values are standard except that the application of some monadic
operators are considered values, a technique again lifted from~\cite{haskellstm}. The \lstinline+do+-notation
used up until now is standard syntactic sugar for the monad bind and return operations.

\vspace{0.5em}
\begin{tabular}{rcl}
    \lstinline[mathescape=true]+do {$x$<-$e$; $Q$}+ & $\equiv$ &
    \lstinline[mathescape=true]+$e$ >>= ($\backslash x$ -> do {$Q$})+ \\

    \lstinline[mathescape=true]+do {$e$; $Q$}+ & $\equiv$ &
    \lstinline[mathescape=true]+$e$ >>= ($\backslash$_ -> do {$Q$})+ \\

    \lstinline[mathescape=true]+do {$e$}+ & $\equiv$ &
    \lstinline[mathescape=true]+$e$+
\end{tabular}
\vspace{0.5em}

Figure~\ref{fig:contexts} defines some symbols used in the semantics. The metavariable $D$ represents a set
of terms used to describe the security properties of variables. We extend this set with an invalid value $\bot$ and
write $D_\bot$ for the extended set. A state of a computation is a pair $(M,\Theta)$ of a term that remains
to be evaluated and a store $\Theta$. The store maps variable names to terms and their variable descriptor. If
a variable does not have a suitable descriptor, we use $\bot$ as a fill-in.

The set of access types, $T$, consists of three constants, each
representing an operation performed on variables. An introspection log $\Sigma$
is a list monoid of pairs $(t,d)$ where $t$ is an access type and $d$ is a descriptor term; we use $[]$ for the empty
list and $\oplus$ for concatenation, and in the semantics we use $[\cdot]$ as a constructor.
Other symbols are conventional and taken from~\cite{haskellstm}.

For a (partial) function $f$ whose co-domain is a cross-product of two or more sets, and an integer $i$,
we write $f_i$ instead of $\pi_i \circ f$ where $\pi_i$ is the standard $i$-th projection function.
For convenience, we introduce the following notation for filtering logs. If $\Delta$ is a store and $\Sigma$ is
an introspection log, we define the \emph{$\Delta$-restriction of $\Sigma$}, indicated by $\Sigma |_\Delta$, thus
\[
    \begin{array}{rl}
        [] |_\Delta &= [] \\
        ([(t,d)] \oplus \Sigma')|_\Delta &=
        \begin{cases}
            [(t,d)] \oplus \Sigma'|_\Delta & \textrm{if $d \in \img(\Delta_2)$} \\
            \Sigma'|_\Delta & \textrm{otherwise}
        \end{cases}
    \end{array}
\]
Intuitively, $\Sigma|_\Delta$ is the list of entries from $\Sigma$ which apply to variables defined by $\Delta$,
where variables are identified by their security descriptors.

Interleaving of operations is modelled with the evaluation context $\mathbb{P}$, often referred to as
a {\em thread soup}. Through this evaluation context the semantics can non-deterministically choose a
term for reduction from the parallel construct, each term representing a thread. Haskell terms are usually
reduced according to the evaluation context $\mathbb{E}$, which allows for reductions of the right
hand side of the \lstinline+>>=+ operator as well as within the body of a \lstinline+catch+ term. However,
since we want to handle exceptions in a specific manner for STM and TMI actions, we will use the
simpler context $\mathbb{S}$ which requires the operational semantics rules to specify explicitly
how \lstinline+catch+ terms are handled.

% subsection syntax (end)

\begin{figure*}
\flushleft{\begin{minipage}{\textwidth}
    \small

\color{old}

\center{\framebox{STM transitions $\qquad M;\,\Theta, \Delta\new{,\Sigma}
                                 \tdarrow N;\,\Theta',\Delta'\new{,\Sigma'}$}}
\begin{gather*}
\begin{array}{rclll}
%
   \SS[\prg{readTVar } r];\,\Theta,\Delta\new{,\Sigma}
 & \tdarrow
 & \SS[\prg{return }\Theta_1(r)];\,\Theta,\Delta\new{,\Sigma}
 & \\ & & \hspace{1cm}
   \textrm{if $r\in \dom(\Theta)$ \new{and $\Theta_2(s) = \bot$}}
   \hspace{.7cm} \soslbl{READ} \\[.7em]
%
   \SS[\prg{writeTVar } r\;M];\,\Theta,\Delta\new{,\Sigma}
 & \tdarrow
 & \SS[\prg{return ()}];\,\Theta[r\mapsto \new{(}M\new{,\bot)}],\Delta\new{,\Sigma}
 & \\ & & \hspace{1cm}
   \textrm{if $r\in \dom(\Theta)$ \new{and $\Theta_2(s) = \bot$}}
   \hspace{.7cm} \soslbl{WRITE} \\[.7em]
%
   \SS[\prg{newTVar } M];\,\Theta,\Delta\new{,\Sigma}
 & \tdarrow
 & \SS[\prg{return }r];\,\Theta[r\mapsto \new{(}M\new{,\bot)}],\Delta[r\mapsto \new{(}M\new{,\bot)}]\new{,\Sigma}
 & \\ & & \hspace{1cm}
   \textrm{$r\notin \dom(\Theta)$}
   \hspace{2.9cm} \soslbl{NEW} \\[.7em]
%
\end{array}
%
\\
\sos{M \tarrow N}
    {\SS[M];\,\Theta,\Delta\new{,\Sigma} \tdarrow \SS[N];\,\Theta,\Delta\new{,\Sigma}}
    \quad\soslbl{AADMIN}
\\[5pt]
\sos{M_1;\,\Theta,\Delta\new{,\Sigma} \stackrel{*}{\tdarrow} 
      \prg{return }N;\,\Theta',\Delta'\new{,\Sigma'}}
      {\SS[M_1 \prg{ `orElse` } M_2];\,\Theta,\Delta\new{,\Sigma} \tdarrow 
      \SS[\prg{return }N];\,\Theta',\Delta'\new{,\Sigma'}} 
    \quad\soslbl{OR1}
\\[5pt]
\sos{M_1;\,\Theta,\Delta\new{,\Sigma} \stackrel{*}{\tdarrow} 
      \prg{throw }N;\,\Theta',\Delta'\new{,\Sigma'}}
      {\SS[M_1 \prg{ `orElse` } M_2];\,\Theta,\Delta\new{,\Sigma} \tdarrow 
      \SS[\prg{throw }N];\,\Theta',\Delta'\new{,\Sigma'}} 
    \quad\soslbl{OR2}
\\[5pt]
\sos{M_1;\,\Theta,\Delta\new{,\Sigma} \stackrel{*}{\tdarrow} 
      \prg{retry};\,\Theta',\Delta'\new{,\Sigma'}}
      {\SS[M_1 \prg{ `orElse` } M_2];\,\Theta,\Delta\new{,\Sigma} \tdarrow 
      \SS[M_2];\,\Theta,\Delta\new{,\Sigma}} 
    \quad\soslbl{OR3}
\\[5pt]
%
% <<< XSTM catch
%
\sos{M;\,\Theta,\emptyset\new{,[]}
     \stackrel{*}{\tdarrow}
     \prg{return }M';\, \Theta', \Delta'\new{,\Sigma'}}
    {\SS[\prg{catch }M\;N];\,\Theta,\Delta\new{,\Sigma} \tdarrow
     \SS[\prg{return }M'];\, \Theta',\Delta\cup\Delta'\new{,\Sigma\oplus\Sigma'}}
    \quad\soslbl{XSTM1}
\\[5pt]
\sos{M;\,\Theta,\emptyset\new{,[]}
     \stackrel{*}{\tdarrow}
     \prg{throw }M';\, \Theta', \Delta'\new{, \Sigma'}}
    {\SS[\prg{catch }M\;N];\,\Theta,\Delta\new{,\Sigma} \tdarrow
     \SS[N \,M'];\, \Theta\cup\Delta',\Delta\cup\Delta'\new{,\Sigma\oplus(\Sigma'|_{\Delta'})}}
    \quad\soslbl{XSTM2}
\\[5pt]
\sos{M;\,\Theta,\emptyset\new{,[]}
     \stackrel{*}{\tdarrow}
     \prg{retry};\, \Theta', \Delta'\new{, \Sigma'}}
    {\SS[\prg{catch }M\;N];\,\Theta,\Delta\new{,\Sigma} \tdarrow
     \SS[\prg{retry}];\, \Theta,\Delta\new{,\Sigma}}
    \quad\soslbl{XSTM3}
\\[5pt]
% >>>
% <<< Authorized
%
\new{
\sos{M;\,\Theta,\emptyset,[]
     \stackrel{*}{\tddarrow}
     \prg{return }M';\, \Theta',\Delta',\Sigma'
   \qquad
   \Sigma' \vdash N \stackrel{*}{\leadsto} \prg{return }N' }
   {\SS[\prg{authorized }N\;M];\,\Theta,\Delta,\Sigma \tdarrow
    \SS[\prg{return }M'];\Theta',\Delta',\Sigma}
  \quad\soslbl{AURET1}
}
\\[5pt]
\new{
\sos{M;\Theta,\emptyset,[]
     \stackrel{*}{\tddarrow}
     \prg{throw }M';\, \Theta',\Delta',\Sigma'
   \qquad
   \Sigma' \vdash N \stackrel{*}{\leadsto} \prg{return }N' }
   {\SS[\prg{authorized }N\;M];\,\Theta,\Delta,\Sigma \tdarrow
    \SS[\prg{throw }M'];\,\Theta',\Delta',\Sigma}
  \quad\soslbl{AUTHROW1}
}
\\[5pt]
\new{
\sos{M;\,\Theta,\emptyset,[]
     \stackrel{*}{\tddarrow}
     \prg{return }M';\, \Theta',\Delta',\Sigma'
   \qquad
   \Sigma' \vdash N \stackrel{*}{\leadsto} \prg{throw }N' }
   {\SS[\prg{authorized }N\;M];\,\Theta,\Delta,\Sigma \tdarrow
    \SS[\prg{throw UnathorizedError}];\Theta',\Delta',\Sigma}
  \quad\soslbl{AURET2}
}
\\[5pt]
\new{
\sos{M;\Theta,\emptyset,[]
     \stackrel{*}{\tddarrow}
     \prg{throw }M';\, \Theta',\Delta',\Sigma'
   \qquad
   \Sigma' \vdash N \stackrel{*}{\leadsto} \prg{throw }N' }
   {\SS[\prg{authorized }N\;M];\,\Theta,\Delta,\Sigma \tdarrow
    \SS[\prg{throw UnauthorizedError}];\,\Theta',\Delta',\Sigma}
  \quad\soslbl{AUTHROW1}
}
\\[5pt]
\new{
\sos{M;\Theta,\emptyset,[]
     \stackrel{*}{\tddarrow}
     \prg{retry};\, \Theta',\Delta',\Sigma'
   }
   {\SS[\prg{authorized }N\;M];\,\Theta,\Delta,\Sigma \tdarrow
    \SS[\prg{retry}];\,\Theta',\Delta'\Sigma}
  \quad\soslbl{AURETRY}
}
% >>>
%
\end{gather*}

\end{minipage}}
\caption{Operational semantics for STM actions}
\label{fig:sos1}
\end{figure*}


\begin{figure*}
\flushleft{\begin{minipage}{\textwidth}
    \small

\center{\framebox{TMI transitions $\qquad M;\,\Theta, \Delta, \Sigma
                                \tddarrow N;\,\Theta',\Delta',\Sigma'$}}
\begin{gather*}
\begin{array}{rcll}
%
   \SS[\prg{readTMIVar } r];\,\Theta,\Delta,\Sigma
 & \!\!\tddarrow\!\!
 & \SS[\prg{return }\Theta_1(r)];\,\Theta,\Delta,\Sigma\oplus[(\textsc{read},\Theta_2(r))]
 & \\ & & \hspace{1cm}
   \textrm{if $r\in \dom(\Theta)$ and $\Theta_2(r)) \ne \bot$}
   \hspace{.7cm} \soslbl{TMIREAD} \\[.7em]
%
   \SS[\prg{writeTMIVar } r\;M];\,\Theta,\Delta,\Sigma
 & \!\!\tddarrow\!\!
 & \SS[\prg{return ()}];\,\Theta[r\mapsto (M,\Theta_2(r))],\Delta,\Sigma\oplus[(\textsc{write},\Theta_2(r))]
 & \\ & & \hspace{1cm}
   \textrm{if $r\in \dom(\Theta)$ and $\Theta_2(r)) \ne \bot$}
   \hspace{.7cm} \soslbl{TMIWRITE} \\[.7em]
%
   \SS[\prg{newTMIVar }N\;M];\,\Theta,\Delta,\Sigma
 & \!\!\tddarrow\!\!
 & \SS[\prg{return }r];\,\Theta[r\mapsto (M,N)],\Delta[r\mapsto (M,N)],\Sigma\oplus[(\textsc{create},N)]
 & \\ & & \hspace{1cm}
   \textrm{$r\notin \dom(\Theta)$}
   \hspace{.7cm} \soslbl{TMINEW} \\[.7em]
%
\end{array}
%
\\
\sos{M \tarrow N}
    {\SS[M];\,\Theta,\Delta\new{,\Sigma} \tddarrow \SS[N];\,\Theta,\Delta\new{,\Sigma}}
    \quad\soslbl{TADMIN}
\\[5pt]
\sos{M;\,\Theta,\Delta,\Sigma \stackrel{*}{\tdarrow} N;\,\Theta',\Delta',\Sigma'}
    {\SS[\prg{liftSTM }M];\,\Theta,\Delta,\Sigma \tddarrow
     \SS[N];\,\Theta',\Delta',\Sigma'}
    \quad\soslbl{LIFTSTM}
\\[5pt]
\sos{M_1;\,\Theta,\Delta\new{,\Sigma} \stackrel{*}{\tddarrow} 
      \prg{return }N;\,\Theta',\Delta'\new{,\Sigma'}}
      {\SS[M_1 \prg{ `orElse` } M_2];\,\Theta,\Delta\new{,\Sigma} \tddarrow 
      \SS[\prg{return }N];\,\Theta',\Delta'\new{,\Sigma'}} 
    \quad\soslbl{TOR1}
\\[5pt]
\sos{M_1;\,\Theta,\Delta\new{,\Sigma} \stackrel{*}{\tddarrow} 
      \prg{throw }N;\,\Theta',\Delta'\new{,\Sigma'}}
      {\SS[M_1 \prg{ `orElse` } M_2];\,\Theta,\Delta\new{,\Sigma} \tddarrow 
      \SS[\prg{throw }N];\,\Theta',\Delta'\new{,\Sigma'}} 
    \quad\soslbl{TOR2}
\\[5pt]
\sos{M_1;\,\Theta,\Delta\new{,\Sigma} \stackrel{*}{\tddarrow} 
      \prg{retry};\,\Theta',\Delta'\new{,\Sigma'}}
      {\SS[M_1 \prg{ `orElse` } M_2];\,\Theta,\Delta\new{,\Sigma} \tddarrow 
      \SS[M_2];\,\Theta,\Delta\new{,\Sigma}} 
    \quad\soslbl{TOR3}
\\[5pt]
%
% <<< XSTM catch
%
\sos{M;\,\Theta,\emptyset\new{,[]}
     \stackrel{*}{\tddarrow}
     \prg{return }M';\, \Theta', \Delta'\new{,\Sigma'}}
    {\SS[\prg{catch }M\;N];\,\Theta,\Delta\new{,\Sigma} \tddarrow
     \SS[\prg{return }M'];\, \Theta',\Delta\cup\Delta'\new{,\Sigma\oplus\Sigma'}}
    \quad\soslbl{XTMI1}
\\[5pt]
\sos{M;\,\Theta,\emptyset\new{,[]}
     \stackrel{*}{\tddarrow}
     \prg{throw }M';\, \Theta', \Delta'\new{, \Sigma'}}
    {\SS[\prg{catch }M\;N];\,\Theta,\Delta\new{,\Sigma} \tddarrow
     \SS[N \,M'];\, \Theta\cup\Delta',\Delta\cup\Delta'\new{,\Sigma\oplus(\Sigma'|_{\Delta'})}}
    \quad\soslbl{XTMI2}
\\[5pt]
\sos{M;\,\Theta,\emptyset\new{,[]}
     \stackrel{*}{\tddarrow}
     \prg{retry};\, \Theta', \Delta'\new{, \Sigma'}}
    {\SS[\prg{catch }M\;N];\,\Theta,\Delta\new{,\Sigma} \tddarrow
     \SS[\prg{retry}];\, \Theta,\Delta\new{,\Sigma}}
    \quad\soslbl{XTMI3}
\end{gather*}

\rule{0.95\linewidth}{0.33pt}

\center{\framebox{Authorization transitions $\qquad \Sigma \vdash M \leadsto N$}}
\begin{gather*}
\sos{M\tarrow N}
    {\Sigma \vdash \EE[M] \,\leadsto\, \EE[N]}
\quad\soslbl{AUADMIN}
\\[0.5em]
\Sigma \vdash \EE[\prg{getlog}] \,\leadsto\, \EE[\prg{return } hs(\Sigma)]
\quad\soslbl{GETLOG}
\\[0.5em]
\Sigma \vdash \EE[\prg{catch }(\prg{return }M)\, N]
    \,\leadsto\,
    \EE[\prg{return } M]
    \quad\soslbl{ACATCH1}
\\[0.5em]
\Sigma \vdash \EE[\prg{catch }(\prg{throw }M)\, N]
    \,\leadsto\,
    \EE[N\,M]
    \quad\soslbl{ACATCH2}
\end{gather*}

\end{minipage}}
\caption{Operational semantics for TMI actions}
\label{fig:sos2}
\end{figure*}


\subsection{Operational semantics} % (fold)
\label{sub:operational_semantics}

Figures~\ref{fig:sosadmin} through \ref{fig:sos2} detail the transition relations of our semantics.
Figures~\ref{fig:sosadmin} and \ref{fig:sos1} are mostly the same as in the semantics
of~\cite{haskellstm}, parts added for TMI are indicated with a darker ink. 
The semantics uses several different
transition systems that are layered such that a sequence of reductions in one layer becomes one reduction
in the next layer above. This makes a sequence of transitions in a lower layer appear as 
one atomic transition at the higher level.
There are three main layers - the top level I/O context, the STM context
and the TMI context. An auxiliary transition system is used to reduce authorization functions.

\paragraph{\emph{Values and I/O transitions:}}
The {\em admin} transitions of Figure~\ref{fig:sosadmin} 
define the evaluation of terms to values via a function $\mathcal{V}$. This
function is standard and its definition omitted here. Administrative transitions also include the
behaviour of the monadic bind operator \lstinline+>>=+.

The top level I/O actions are described
by the labelled $\tarrow$ relation. They 
operate on the $\mathbb{P}$ context, which allows for picking any program
term from the thread soup for reduction. The first two rules are I/O primitives. The rule {\em FORK} is used
to create a new thread and enter it into the thread soup, choosing a fresh thread id $t$.
The rules {\em CATCH1} and {\em CATCH2} 
deal with exception
handling as described in the appendix of the post-publication, extended version of the Haskell
semantics~\cite{haskellstm}.
The {\em ADMIN} rule
allows for lifting of administrative transitions to the I/O transition relation.
This is done to reduce
repetition, as the administrative 
rules also apply to the STM and TMI transition relations, which have a similar
lifting rule.
Finally, the rules {\em ARET} and {\em ATHROW} 
enable the use of the \lstinline+atomic+ combinator to lift a sequence
of reductions in the STM transition relation to a single I/O transition.

If the series of STM reductions results in a \lstinline+return+ value, the effects on the store are retained.
If it however results in an exception (i.e. a \lstinline+throw+ value), the modifications to existing variables
are discarded but any new allocations are retained. This is necessary as the exception value may hold references
to newly allocated variables.

\paragraph{\emph{STM transitions:}}
The STM transitions define the behaviour of STM actions. The states used in these transitions are extended
from the I/O transitions by adding a separate store for new allocations $\Delta$, and an introspection log
$\Sigma$. A transition of the form
\[
M;\,\Theta, \Delta,\Sigma \tdarrow N;\,\Theta',\Delta',\Sigma'
\]
represents a reduction of the term $M$ to the term $N$. Some variables in $\Theta$ may be introduced
or altered to yield $\Theta'$. $\Delta$ is a store similar to $\Theta$, that only tracks newly allocated
variables while $\Sigma$ is a log of accesses to TMI variables. $\Delta$ and $\Sigma$ are transaction
local, i.e. they are reset at the start of each atomic sequence of reductions in the STM system, see
e.g. rule~$\soslbl{ARET}$.

The first three rules define actions on transactional variables. They operate on the store
$\Theta$ but only on those variables where the second component (i.e. the security descriptor) is $\bot$.
This is what differentiates regular transactional variables from security sensitive variables. 

Other rules in the STM transition system define the behavior of STM combinators as described in
Section~\ref{sub:haskell_stm}. We have used the revised semantics of exception handling from the
later versions of~\cite{haskellstm}, namely the rule {\em XTM2} ensure that when a term reduction
results in a caught exception, all effects of that reduction are rolled back except for new 
allocations.

% �g er ekkert rosa happy me� �essi paragraph heading, en �a� ver�ur a� brj�ta eitthva� upp �essa 
% bla�s��u einhvernvegin
\paragraph{\emph{TMI transitions:}}
A key TMI addition to the STM transitions is the handling of \lstinline+authorized+.
The rules {\em AURETn} and {\em AUTHROWn} specify that for a term of the form {\tt (authorized}
$N\;M${\tt )}
if $M$ evaluates to a \lstinline+return+ or \lstinline+throw+ value, then that value is propagated only
if $N$ evaluates to a \lstinline+return+ value under the authorization relation (see later).
If evaluation of the {\em authorization term} $N$ raises an exception, a fixed exception containing
no information about the local transaction state is thrown. This triggers a rollback of any updates
performed during that invocation of \lstinline+authorized+.

We should note that in the case of an exception, including authorization failure, new allocations
are retained for the reason described above. In the implementation, this is not done explicitly
as deallocation of references is handled by the garbage collector. Any new allocations that are
actually referenced by exception values are thus retained, but others are discarded. Thus, since
we don't allow any variable references in our special exception for authorization failures, no
new allocations will leak in practice.

Figure~\ref{fig:sos2} shows the two new TMI transition relations. The first one deals with
TMI actions and is indicated by the symbol $\tddarrow$. The configurations of this transition
system are identical to those of the STM system.
Indeed, TMI actions behave very much like STM actions, the main difference is that variable
operations in TMI actions can operate on security sensitive variables. A variable $v$ is security
sensitive if and only if $\Theta_2(v) \neq \bot$.
The first three rules of
Figure~\ref{fig:sos2} describe the variable operations. 
When these operations are performed, a log entry is added to
$\Sigma$ with the contents of the variable's security descriptor. Another addition over STM
behavior is rule {\em LIFTSTM}. This rule states that any sequence of STM reductions can be lifted
to the TMI level. 
This is necessary to allow TMI code to access regular transactional variables, and should be 
possible, since TMI actions are always performed in the context of an enclosing STM action.

Finally, we add a separate transition system to evaluate authorization functions. In this system
a transition of the form
\[
\Sigma \vdash M \leadsto N
\]
represents the reduction of term $M$ to term $N$, under the context of an introspection log $\Sigma$.
The reason for this notation is that the introspection log is fixed, i.e. read-only, for these
transitions.
This system only allows pure operations and monad binding via the administrative transitions, the
usual exception handling and one special term \lstinline+getlog+. 
The \lstinline+getlog+ term is reduced to a list representation (in the Haskell sense) of the access
log $\Sigma$. The terms reduced with this system can thus examine the log and make decisions based
on its contents.

\paragraph{\emph{An example:}}
As an example of reading and applying the rules, consider the program
\begin{lstlisting}[style=small]
atomically 
  (
     authorized (assert (isEmpty getlog)) 
                (writeTMIVar x 10)
  )
\end{lstlisting}
Working from the inside out, we can see that the innermost expression of \lstinline+writeTMIVar x 10+
will update the value of $x$ in $\Theta$ as well as enter an entry to the introspection log $\Sigma$, by applying
rule $\soslbl{TMIWRITE}$. The resulting term is \lstinline+return ()+. % breaks accross lines
As the resulting log is non-empty, the authorization term 
\lstinline+assert (isEmpty getlog)+ will throw an exception. Thus, for the \lstinline+authorized+
term, rule $\soslbl{AURET2}$ is the only applicable one, so that term evaluates to
\lstinline+throw UnauthorizedError+. The \lstinline+atomically+ term is therefore evaluated to
the same result via rule $\soslbl{ATHROW}$, but this rule does not preserve updates to the store $\Theta$,
meaning that the transaction has been aborted.

\paragraph{\emph{Nested TMI actions:}} As we mentioned in the previous section, the capability
of lifting STM actions up to the TMI levels allows us to nest TMI actions. An inner TMI action
can be authorized with a separate authorization term. Consider the following example of an
action that provides a student with information about her grade for a course, as well as the
average of all grades of other students. 
Naturally, the student doesn't have access to other students' grades
but for the purpose of calculating the average we may allow such access in a nested action.

Assume that we have defined the following terms.
\begin{mybullet}
\item \lstinline+ownGradesRead s+ is an authorization term that succeeds only if
    the input log only contains reading of grades that belong to student \lstinline+s+
\item \lstinline+allGradesRead+ is an authorization term that succeeds only if the log only contains
    reading of grades, but regardless of the owner of the grades. This may be considered
    a kind of a {\em system} read access to grades.
\item \lstinline+readGrade s+ is a TMI action that reads a grade of a student from the
    relevant \lstinline+TMIVar+ and returns it. The introspection log will contain an 
    appropriate entry afterwards.
\item \lstinline+averageGrades+ is a TMI action that reads grades of all students from the
    appropriate \lstinline+TMIVars+ and returns their average. The introspection log will 
    contain an entry for every read grade.
\end{mybullet}
Now it is possible to define the following TMI action that provides a student with her
own grade as well as the average grades of all students.
\begin{lstlisting}[style=small]
gradeInformation :: Student -> TMI (Grade, Grade)
gradeInformation s =
  do own <- readGrade s
     avg <- liftSTM (authorized allGradesRead averageGrades)
     return (own, avg)
\end{lstlisting}
This function may be called with the appropriate authorization function,
namely one that only allows a student access to her own grades.
\begin{lstlisting}[style=small]
atomically (authorized (ownGradesRead s) (gradeInformation s))
\end{lstlisting}

By applying the operational semantics rules to this term, one can find that the innermost action
\lstinline+averageGrades+ will by authorized be \lstinline+allGradesRead+ {\em before}
turning it into an STM transition. This may, for example, happen through rule $\soslbl{AURET1}$.
Note that such a rule does not keep the log entries of already authorized actions, i.e.
the $\Sigma$ is not affected in the $\Rightarrow$ transition below the line. 
Thus the log of the nested
action is not contained in the log authorized by the outer authorization function
\lstinline+ownGradesRead+.

This use of nesting constitutes a {\em privilege amplification} 
in a manner similar to stack inspection~\cite{GordonFournetPOPL}.

% subsection operational_semantics (end)


% section formal_semantics_of_tim (end)

\section{Implementation}
\label{sec:implementation}

Our Haskell implementation is comprised of one module, \verb+TMI+. 
The most important components are the
monad \lstinline+TMI d a+ and the type for TMI variables, \lstinline+TMIVar d a+. 
Both are parameterized on the descriptor type \lstinline+d+, which is chosen by the user
of this module. A TMI variable is represented by a descriptor value and an STM \lstinline+TVar+,
\begin{lstlisting}[style=small]
data TMIVar d a = TMIVar {
      getTVar       :: TVar a,
      getDescriptor :: d
}
\end{lstlisting}
The field accessors \lstinline+getTVar+ and \lstinline+getDescriptor+ are not exported and
only available inside the \lstinline+TMI+ module.

The TMI monad is a stack of the standard {\em writer monad} on top of the regular STM
monad. The writer monad has the ability of collecting accumulating information in a
sequence, which is exactly what we need to maintain the introspection log.
In the TMI module, the regular STM module is imported under the name \lstinline+T+.
\begin{lstlisting}[style=small]
newtype TMI d a = TM {
   unwrapTM :: WriterT (TMILog d) T.STM a
} deriving (Monad)
\end{lstlisting}
The type \lstinline+TMILog d+ represents a log of all accesses to TMI variables.
It is defined by the following declarations.
\begin{lstlisting}[style=small]
data TMIAccess = CreateVar | ReadVar | WriteVar
type TMILog d = [(TMIAccess, d)]
\end{lstlisting}
For inserting entries in the log, we define the following shortcut, where \lstinline+tell+
is the standard function that the writer monad uses to collect information.
\begin{lstlisting}[style=small]
log :: (TMIAccess, d) -> STM d ()
log entry = (TM . tell) [entry]
\end{lstlisting}
\lstinline+log+ returns an STM action which has the
only effect of appending its argument to the introspection log. Note that this
helper is not exported, so users of the TMI module cannot append to the log
directly.
As expected, a log entry of the form \lstinline+(ReadVar, x)+ just means that a variable with descriptor
value \lstinline+x+ was read. 

The \lstinline+liftSTM+ function lifts an STM operation to a TMI operation. The log
is not affected by the work performed in the STM action.
\begin{lstlisting}[style=small]
liftSTM :: STM a -> TMI d a
liftSTM = TM . lift
\end{lstlisting}

The functions to create, read and write TMI variables are now simple to define. They
all enter the relevant entries to the log and then call the underlying functions
from the STM module.
\begin{lstlisting}[style=small]
newTMIVar :: d -> a -> TMI d (TMIVar d a)
newTMIVar description val =
    do log (CreateVar, description)
       var <- liftSTM (T.newTVar val)
       return (TMIVar var description)

readTMIVar :: TMIVar d a -> TMI d a
readTMIVar tv = 
    do log (ReadVar, getDescriptor tv)
       liftSTM (T.readTVar (getTVar tv))

writeTMIVar :: TMIVar d a -> a -> TMI d ()
writeTMIVar tv val = 
    do log (WriteVar, getDescriptor tv)
       liftSTM (T.writeTVar (getTVar tv) val)
\end{lstlisting}

The combinators from the STM world are defined thus.
\begin{lstlisting}[style=small]
retryTMI = liftSTM T.retry

runTMI = runWriterT . unwrapTM   -- helper

orElseTMI t1 t2 = TM . WriterT (runTMI t1 `T.orElse` runTMI t2)
\end{lstlisting}

What is left is to define the crucial \lstinline+authorized+ function. This function
accepts an authorization function with the type \lstinline+TMILog d -> Bool+
and a TMI action; it should return an STM action that performs the operation
of the TMI action, validates the resulting log with the authorization function and
either returns the result or throws an exception. With this description in mind,
the implementation is pretty straight-forward.
\begin{lstlisting}[style=small]
authorized :: (TMILog d -> Bool) -> TMI d a -> T.STM a
authorized auth act =  do
   (result, log) <- runTMI act
   if not (auth log)
      then throw (AssertionFailed "Access denied")
      else return result
\end{lstlisting}
Note that a custom exception may be more appropriate but for sake of clarity
we simply use the standard assertion failure to trigger a transaction abort.

Our Haskell TMI implementation can support variants of
history-based policy enforcement~\cite{HBAC},
in particular
allowing \emph{privilege amplification} with nested TMI actions as described in
the previous section.
%
In particular a call to \lstinline+authorized+ will return
an STM action which {\em contains} a TMI operation and an associated
authorization closure. When
code inside a TMI action needs increased privileges it can nest a call to \lstinline+authorized+
with a different authorization manager and use \lstinline+liftSTM+ to lift the resulting
STM operation back to the TMI level.

As in stack inspection~\cite{GordonFournetPOPL},
privilege amplification provides 
TMI security managers with
a useful 
escape hatch to perform operations as a more powerful ``application principal''.
%
% For instance, we have used it 
% in cases where applications need to compute aggregates, 
% such as in an grade-management application
% that shows a student their grade and a class grade average,
% but prevents access to other student's grades.


\section{Discussion and future work}
\label{sec:conclusion}

We have presented both a formal semantics and an implementation of TMI over the
Haskell STM system. During this work, we discovered that there are many design
decisions to be made and the design we have presented here is only one of many
possibilities. The variants we experimented with in the design process did not
always exhibit the behaviour that we expected or wanted. For this task, defining
the formal semantics proved to be an essential tool to understand and evaluate
different design decisions as well as spotting special cases that were not so
obvious in the actual implementation. 
Indeed, constructing the semantics helped us discover bugs
and unexpected behaviour in the code, even after weeks of careful consideration.
In addition the formal semantics gives a clear and unambiguous
description of the TMI architecture.

%
% What do you mean by 'the design' ?  Do you mean the TMI architecture (which
% we describe and clearly isn't limited in this way), or the formal semantics,
% or the Haskell implementation?  If it is the last (which I think it is), then
% it isn't quite true, since we have the privilege stuff.
%
% Also, this is kind of a downer to end on.  Why not just skip, or say that we would
% like to explore better implementations for stateful policies (although our semantics supports it).
%

The TMI architecture, as described in Section~\ref{ssub:overview_of_tmi}, 
can support the enforcement of stateful security policies that depend on the 
execution history over multiple transactions.  In our paper \cite{tmi}, we have experimented 
with such policies in another implementation of the TMI architecture.  
However, we have not explored the addition of such facilities here, 
in order to simplify the exposition of our semantics and Haskell implementation.

The privilege amplification by nested TMI actions naturally relies on the programmer
to ensure that the nested authorization manager does not violate the enclosing
policy. The objective of the TMI architecture is to provide facilities for writing
policy enforcement code, not to prevent injection of malicious code. In the case of library
development where one deals with untrusted code, such nesting may not be desirable and can
be disabled. We have experimented with other ways of implementing privilege amplification
without relying on this nesting, with good results.
% The policy state must be protected by the STM mechanisms to avoid
% race conditions or interference between threads, but we must take care that the accesses
% to this state are only performed by trusted code, as they cannot be subject to authorization,
% lest we introduce a circular dependency.

In our implementation we are maintaining the introspection log by hand. This works well for
prototypical purposes, but we would like to investigate the possibility of making use
of the real underlying transaction log. This requires modifications to the STM framework
provided in the GHC runtime library. Having the formal semantics as the definition of the desired
behaviour should make such an implementation easier to construct and check.

Future work in this context also involves writing or porting complex software to the
architecture to obtain realistic performance measurements.
Also, our semantics may still be simplified, while allowing the same behavior; 
for example, the transaction log seems redundant in STM transitions, 
and may possibly be eliminated.


Most importantly, we think that having clear semantics for TMI and an implementation
over a production-ready STM system, further validates our claim that TMI architecture
is very relevant to practical software development.


%\appendix
%\section{Appendix Title}
%
%This is the text of the appendix, if you need one.

\acks

We would like to thank Luca Aceto for his helpful comments on our
use of Structural Operational Semantics. We also thank 
Wouter Swierstra and Josef Svenningsson for their detailed comments
on our Haskell code. 

\bibliographystyle{plainnat}
\bibliography{references}

%\begin{thebibliography}{}
%
%\bibitem{smith02}
%Smith, P. Q. reference text
%
%\end{thebibliography}

\end{document}

