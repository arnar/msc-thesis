\section{Decompositional Reasoning} % (fold)
\label{sec:decomp_hml}

We now define the quotienting construction over formulae structurally. 
Quotienting distributes over the boolean operators.
\begin{align*}
\true / \rho &= \true \\
(\phi_1 \land \phi_2)/\rho &= \phi_1/\rho  \land \phi_2 / \rho \\
(\neg\phi)/\rho &= \neg(\phi/\rho) \\
\end{align*}
The modal operators however need more attention. We start with $\dmnd{\alpha}\phi$
and consider separately the cases where $\alpha\in A$ and $\alpha = \tau$. In
the following we assume $p' = \last(\rho)$.
\begin{align*}
    (\dmnd{a}\phi) / \rho &= \dmnd{a} \left(\phi / \rho(p'\ptrans{} p')\right)
        \lor \left(\bigvee_{\rho' : \rho\transshort{a}\rho'} \pdmnd{} (\phi/\rho') \right)
    \\[1em]
    (\dmnd{\tau}\phi) / \rho &= \dmnd{\tau} \left(\phi / \rho(p'\ptrans{} p')\right)
        \lor \left(\bigvee_{\rho' : \rho\transshort{\tau}\rho'} \pdmnd{} (\phi/\rho') \right)
        \lor \left( \bigvee_{\rho',a : \rho \transshort{a} \rho'} \dmnd{\bar{a}} (\phi/\rho')  \right)
\end{align*}
Intuitively, the first case states that when we expect the composed computation to be
able to perform an $a$-transition, there are two possibilities. The first possibility
is that the component
we intend to test with the quotient formula can perform an $a$-transition. The rest of the
formula must then be quotiented with $\rho$ plus a pseudo-step representing that this
component remained idle.
The second possibility is that there is an $a$-transition from $\rho$. In this case
the component we want to test must proceed with a pseudo-step.
The same holds when we look
for a $\tau$-transition, with one addition.  If $\rho$ can advance with a non-$\tau$
action, then we should look for a matching action in the other component that may
have caused the two components to communicate.

To define the transformation for formulas of the form $\dmndback{\alpha}\phi$, we
again look at several cases separately. First we consider the case when $\rho$ has
the empty path. In this case it is obvious that no backward step is possible.
\[
    (\dmndback{\alpha}\phi) / (p,\lambda) = \false
\]
The second case to consider is when $\rho$ ends with a pseudo-transition, or a ``gap''.
In this case the only possibility is that the other component (the one we are testing)
is able to perform the backward transition.
\[
    (\dmndback{\alpha}\phi) / \rho'(p'\ptrans{} p') =
        \dmndback{\alpha} (\phi / \rho')
\]
The third case applies when $\rho$ does indeed end with the transition we look for.
In this case the other component must end with a matching gap.
\begin{equation}\label{eq:rewrite_back_dmnd_same}
    (\dmndback{\alpha}\phi) / \rho'(p''\trans{\alpha} p') = \pdmndback{} (\phi / \rho')
\end{equation}
The only remaining case to consider is when $\rho$ ends with a transition different from
the one we look for. We split this case further and consider
again separately the cases when $\alpha\in A$ and when $\alpha = \tau$. The former case
is simple: if $\rho$ indicates that the last transition has a label other than the one
specified
in the diamond operator, there is no way that the composed computation satisfies it.
\[
    (\dmndback{a}\phi) / \rho'(p'' \trans{\beta} p') = \false \quad\textrm{where $a\ne \beta$}
\]
If however the diamond operator mentions a $\tau$ transition, then we must look for a
transition in the other component that can synchronise with the last one of $\rho$. Note that this
case does not include computations ending with a $\tau$ transition, as that case is
covered by equation~\ref{eq:rewrite_back_dmnd_same}.
\[
    (\dmndback{\tau}\phi) / \rho'(p'' \trans{b} p') = \dmndback{\bar{b}} (\phi/\rho')
\]
This covers all possible cases for $\dmndback{\alpha}\phi / \rho$.

For the new operators, $\pdmnd{}$ and $\pdmndback{}$, the transformation is simple.
First, if a composed computation should satisfy $\pdmnd{}\phi$, then it must be
because both components are able to add a pseudo step and satisfy $\phi$. I.e.
\[
    (\pdmnd{}\phi) / \rho = \pdmnd{} \left( \phi / \rho(p'\ptrans{} p') \right)
\]
where again $p' = \last(\rho)$. If the composition should end with a $\ptrans{}$
pseudo-transition, then both components must also end with $\ptrans{}$. Thus, we
have to consider two cases, where $\rho$ does end with such a pseudo-transition
and when it doesn't.
\[
    (\pdmndback{}\phi) / \rho = \begin{cases}
        \pdmndback{} (\phi / \rho') & \textrm{if $\rho = \rho'(p' \ptrans{} p')$} \\
        \false & \textrm{otherwise}
    \end{cases}
\]
%
The complete quotienting transformation is summarised in the Table~\ref{tab:quotients}.

\begin{table}[t]
    \begin{align*}
        \true / \rho &= \true \\
        (\phi_1 \land \phi_2)/\rho &= \phi_1/\rho  \land \phi_2 / \rho \\
        (\neg\phi)/\rho &= \neg(\phi/\rho) \\
        %
        (\dmnd{a}\phi) / \rho &= \dmnd{a} \left(\phi / \rho(p'\ptrans{} p')\right)
            \lor \left(\bigvee_{\rho' : \rho\transshort{a}\rho'} \pdmnd{} (\phi/\rho') \right)
        \\[1em]
        (\dmnd{\tau}\phi) / \rho &= \dmnd{\tau} \left(\phi / \rho(p'\ptrans{} p')\right)
            \lor \left(\bigvee_{\rho' : \rho\transshort{\tau}\rho'} \pdmnd{} (\phi/\rho') \right) \\
            &\phantom{= \dmnd{\tau} \left(\phi / \rho(p'\ptrans{} p')\right)\;}
            \lor \left( \bigvee_{\rho',a : \rho \transshort{a} \rho'} \dmnd{\bar{a}} (\phi/\rho')  \right) \\
        %
        (\dmndback{\alpha}\phi) / (p,\lambda) &= \false \\
        (\dmndback{\alpha}\phi) / \rho'(p'\ptrans{} p') &= \dmndback{\alpha} (\phi / \rho') \\
        (\dmndback{\alpha}\phi) / \rho'(p''\trans{\alpha} p') &= \pdmndback{} (\phi / \rho') \\
        (\dmndback{a}\phi) / \rho'(p'' \trans{\beta} p') &= \false \quad\textrm{where $a\ne \beta$} \\
        (\dmndback{\tau}\phi) / \rho'(p'' \trans{b} p') &= \dmndback{\bar{b}} (\phi/\rho') \\
        %
        (\pdmnd{}\phi) / \rho &= \pdmnd{} \left( \phi / \rho(p'\ptrans{} p') \right) \\
        %
        (\pdmndback{}\phi) / \rho &= \begin{cases}
            \pdmndback{} (\phi / \rho') & \textrm{if $\rho = \rho'(p' \ptrans{} p')$} \\
            \false & \textrm{otherwise}
        \end{cases}
    \end{align*}
    \caption{Quotienting transformations of formulas in \HMLpp}
    \label{tab:quotients}
\end{table}

% subsection transformation_of_formulas (end)

\subsection{Decomposition theorem}\label{sub:theorem}

Before we state and prove our main theorem, we establish a few useful lemmas.

\begin{lemma}\label{lemma:both_step_means_tau}
    If $p\parallel q \trans{\alpha} p'\parallel q'$ where $p\not\equiv p'$ and
    $q\not\equiv q'$ then $\alpha = \tau$.
\end{lemma}
\begin{proof}
    Consider the proof tree for the transition $p\parallel q \trans{\alpha}
    p'\parallel q'$ and, in particular, the last rule used in the proof. This
    rule can be one of the three rules for the parallel operator. The first two,
    where only one component advances, are ruled out since then either $p\equiv p'$
    or $q\equiv q'$ must hold. Therefore the last rule used in the proof
    must be the communication rule, in which case the
    label of the proved transition can only be $\tau$.
\end{proof}

\begin{lemma}\label{lemma:last_step_decomp}
    Let $p,q$ be processes, $(p\parallel q, \pi)\in\C(p\parallel q)$ and
    $(\mu_1,\mu_2)\in D(\pi)$.

    \noindent (i) If $(p\parallel q, \pi) \trans{\alpha} (p\parallel q, \pi')$
    then there exists a pair $(\mu_1', \mu_2')\in D(\pi')$ such that one of the following
    holds.
    \begin{enumerate}
        \item $(p,\mu_1) \trans\alpha (p,\mu_1')$ and $(q,\mu_2) \ptrans{} (q,\mu_2')$,
        \item $(p,\mu_1) \ptrans{} (p,\mu_1')$ and $(q,\mu_2) \trans\alpha (q,\mu_2')$ or
        \item $\alpha = \tau$, $(p,\mu_1) \trans{a} (p,\mu_1')$ and $(q,\mu_2) \trans{\bar{a}} (q,\mu_2')$ for some $a\in A$.
    \end{enumerate}

    \noindent (ii) Symmetrically,
    \begin{enumerate}
        \item If there exists a $\mu_1'$ s.t. $(p, \mu_1) \trans\alpha (p,\mu_1')$
            then there exists a $\pi'$ s.t. $(p\parallel q, \pi) \trans\alpha
            (p\parallel q, \pi')$ and $(\mu_1', \mu_2(q'\ptrans{}q')) \in D(\pi')$
            where $q' = \last(\mu_2)$.
        \item If there exists a $\mu_2'$ s.t. $(q, \mu_2) \trans\alpha (q,\mu_2')$
            then there exists a $\pi'$ s.t. $(p\parallel q, \pi) \trans\alpha
            (p\parallel q, \pi')$ and $(\mu_1(p' \ptrans{} p'), \mu_2') \in D(\pi')$
            where $p' = \last(\mu_1)$.
        \item If there exist $\mu_1'$ and $\mu_2'$ s.t. $(p,\mu_1) \trans{a} (p,\mu_1')$
            and $(q,\mu_2) \trans{\bar{a}} (q,\mu_2')$ for some $a\in A$, then there
            exists $\pi'$ s.t. $(p\parallel q, \pi) \trans{\tau} (p\parallel q, \pi')$
            and $(\mu_1', \mu_2') \in D(\pi')$.
    \end{enumerate}
\end{lemma}
\begin{proof}
    (i) Assume that $(p\parallel q, \pi) \trans{\alpha} (p\parallel q, \pi')$ and let
    $(\mu_1,\mu_2)\in D(\pi)$. This means there exist processes $p',q',p'',q''$
    with $\pi' = \pi(p''\parallel q'' \trans{\alpha} p'\parallel q')$,
    $p''=\last(\mu_1), q''=\last(\mu_2)$.
    Since $p''\parallel q'' \not\equiv
    p'\parallel q'$ we observe that $p''\equiv p'$ and $q''\equiv q'$ cannot hold
    simultaneously, so we consider the remaining cases.
    \begin{enumerate}
        \item $p''\not\equiv p'$ and $q''\equiv q'$. In this case the transition
            $p''\parallel q' \trans{\alpha} p'\parallel q'$ was proven using the first
            rule for $\parallel$. Its only premise must hold, namely $p''\trans{\alpha} p'$.
            We therefore let $\mu_1' = \mu_1(p''\trans{\alpha}p')$ and $\mu_2' = \mu_2
            (q' \ptrans{} q')$. From the inductive definition of $D$ it is easy to see
            that $(\mu_1',\mu_2') \in D(\pi')$.
        \item $p''\equiv p'$ and $q''\not\equiv q'$. This case is entirely symmetric
            to the previous one where the proof is based on the second rule for $\parallel$.
        \item $p''\not\equiv p'$ and $q''\not\equiv q'$. Here the proof
            of the transition $p''\parallel q'' \trans{\alpha} p'\parallel q'$ must be based
            on the third rule for $\parallel$, 
            namely the communication rule and $\alpha=\tau$, as seen by
            Lemma~\ref{lemma:both_step_means_tau}. By the premises of this rule there
            exists an $a\in A$ such that $p''\trans{a}p'$ and $q''\trans{\bar{a}}q'$. We simply
            let $\mu_1' = \mu_1(p''\trans{a}p')$ and $\mu_2' = \mu_2(q''\trans{\bar{a}}q')$.
            Again it is clear from the definition of $D$ that $(\mu_1',\mu_2')\in D(\pi')$.
    \end{enumerate}

    (ii) The construction of $\pi'$ in all cases is straightforward and unique
    (cfr. Lemma~\ref{thm:paths_compose_uniquely}). The rest is simple to check
    with the definition of $D$.
\end{proof}

\begin{lemma}\label{lemma:removing_last_trans}
    Let $(p\parallel q,\pi) \in \C(p\parallel q)$ with $\pi$ non-empty and $(\mu_1,\mu_2)
    \in D(\pi)$. Let $\pi',\mu_1'$ and $\mu_2'$ be the prefixes of length $\abs{\pi}-1$ of $\pi,
    \mu_1$ and $\mu_2$ respectively. Then $(\mu_1',\mu_2') \in D(\pi')$.
\end{lemma}
\begin{proof}
    Follows directly from the definition of $D$.
\end{proof}

We are now ready to prove the main theorem in this chapter, to the effect that
the quotienting of a formula $\phi$ with respect to a computation $\rho$ is
properly defined.

\begin{theorem}\label{thm:decomposition}
    For CCS processes $p,q$ and a computation $(p \parallel q, \pi)\in \C(p\parallel q)$
    and a formula $\phi \in \HMLpp$ we have
    %\sublabon{equation}
    \begin{equation}\label{eq:decomp_ltr}
        (p\parallel q, \pi) \vDash^* \phi  \quad\Rightarrow\quad
        \forall (\mu_1,\mu_2) \in D(\pi) : (p, \mu_1) \vDash^* \phi/(q, \mu_2)
    \end{equation}
    and conversely (note the direction of the implication),
    \begin{equation}\label{eq:decomp_rtl}
        (p\parallel q, \pi) \vDash^* \phi
        \quad\Leftarrow\quad
        \exists (\mu_1,\mu_2) \in D(\pi) : (p, \mu_1) \vDash^* \phi/(q, \mu_2).
    \end{equation}
    %\sublaboff{equation}
\end{theorem}
\begin{proof}
    We prove both implications simultaneously by induction on the structure of $\phi$. 
    In the following text, the terms
    {\em ``the left-hand side''} and {\em ``the right-hand side''} refer respectively
    to the left- and right-hand sides of the above implications where the quantifier
    used in the right-hand side will be made clear by the context.

    \proofcase{Case $\phi = \true$} Then $\phi/(q, \mu_2) = \true$ and both sides
    of both~\eqref{eq:decomp_ltr} and~\eqref{eq:decomp_rtl} are trivially satisfied.


    \proofcase{Case $\phi = \psi_1 \land \psi_2$}\par\nobreak
    %
    \ltr First assume $(p\parallel q, \pi)
    \vDash^* \psi_1 \land \psi_2$
    and let $(\mu_1,\mu_2)\in D(\pi)$.
    Since both $\psi_1$ and $\psi_2$ are smaller than $\phi$ and both are satisfied by
    $(p\parallel q, \pi)$ we have by induction that $(p,\mu_1) \vDash^* \psi_i/(q, \mu_2)$
    for $i\in\{1,2\}$. Since $\phi/(q,\mu_2) = (\psi_1 \land \psi_2)/(q,\mu_2) =
    \psi_1 / (q, \mu_2) \land \psi_2 / (q,\mu_2)$ we obtain $(p,\mu_1) \vDash^*
    \phi/(q,\mu_2)$.

    \rtl Now assume the right side of~\eqref{eq:decomp_rtl}, 
    \[
        \exists (\mu_1,\mu_2)\in D(\pi) : (p,\mu_1)
        \vDash^* (\psi_1 \land \psi_2)/(q, \mu_2).
    \] 
    By definition the formula is equal
    to $\psi_1/(q,\mu_1) \land \psi_2/(q,\mu_2)$. By induction \mbox{$(p\parallel q, \pi)$}
    satisfies both $\psi_1$ and $\psi_2$ and thus also $\psi_1 \land \psi_2 = \phi$.


    \proofcase{Case $\phi = \neg \psi$}\par\nobreak
    %
    \ltr First assume the left side $(p\parallel q, \pi) \vDash^* \neg\psi$.
    %Since $\psi$ is smaller than $\phi$ the
    %implication~\eqref{eq:decomp_rtl} holds for it by induction. 
    Assume towards
    contradiction that there does exist a decomposition $(\mu_1',\mu_2')\in D(\pi)$
    such that $(p, \mu_1') \vDash^* \psi/(q, \mu_2')$. 
    Then by induction~\eqref{eq:decomp_rtl} gives $(p\parallel q, \pi) \vDash^* \psi$,
    which is in direct contradiction with our assumption. Since no such decomposition
    can exist, it holds for all $(\mu_1,\mu_2)\in D(\pi)$ that
    $(p,\mu_1) \vDash^* \neg\psi/(q,\mu_2) = \phi/(q,\mu_2)$.

    \rtl Assume the right side of~\eqref{eq:decomp_rtl}, namely there exists a
    decomposition $(\mu_1,\mu_2)\in D(\pi)$ such that $(p,\mu_1)\vDash^* \neg\psi/(q,\mu_2)$.
    Assume, again towards a contradiction, that $(p\parallel q,\pi)\vDash^*\psi$.
    By induction, \eqref{eq:decomp_ltr} then gives that for all $(\mu_1',\mu_2')\in D(\pi)$,
    $(p,\mu_1')\vDash^* \psi/(q,\mu_2')$. In particular, this holds for the
    decomposition $(\mu_1,\mu_2)$, which contradicts our assumption. Therefore
    we must have that $(p\parallel q,\pi) \vDash^* \neg\psi = \phi$.


    \proofcase{Case $\phi = \dmnd\alpha\psi$}\par\nobreak
    %
    \ltr Again, first assume the left side and take $(\mu_1,\mu_2)\in D(\pi)$. Then
    there exists a computation $(p\parallel q, \pi')$ s.t. $(p\parallel q, \pi) \trans{\alpha}
    (p\parallel q, \pi')$ and $(p\parallel q, \pi') \vDash^* \psi$.
    % This means that $\pi' = \pi (p''\parallel q'' \trans{\alpha}
    % p' \parallel q')$ for some processes $p'', q'', p'$ and $q'$.   xx
    By part (i) of
    Lemma~\ref{lemma:last_step_decomp} there exists a pair $(\mu_1',\mu_2')\in D(\pi')$.
    Since $\psi$ is a subformula of $\phi$ we have by induction that
    \begin{equation}\label{eq:dmnd_psi}
        (p,\mu_1') \vDash^* \psi / (q,\mu_2')
    \end{equation}
    Lemma~\ref{lemma:last_step_decomp} also states that one of the following three cases holds.
    \begin{enumerate}
        \item $(p,\mu_1)\trans{\alpha}(p,\mu_1')$ and $(q,\mu_2)\ptrans{}(q,\mu_2')$.
            %The first clause of $\phi/(q,\mu_2)$ is $\dmnd{\alpha}(\psi/(q,\mu_2')$.
            From~\eqref{eq:dmnd_psi} we have that $(p,\mu_1) \vDash^*
            \dmnd{\alpha}(\psi/(q,\mu_2'))$ and since 
            the formula $\dmnd\alpha(\psi/(q,\mu_2'))$
            is the first clause of the
            disjunction defining 
            $\phi/(q,\mu_2)$ then also $(p,\mu_1) \vDash^* \phi/(q,\mu_2)$.
        \item $(p,\mu_1)\ptrans{}(p,\mu_1')$ and $(q,\mu_2)\trans{\alpha}(q,\mu_2')$.
            Again from~\eqref{eq:dmnd_psi} we have that $(p,\mu_1) \vDash^*
            \pdmnd{}(\psi/(q,\mu_2'))$, and again
            the formula $\pdmnd{}(\psi/(q,\mu_2'))$
            is a clause of the disjunction defining
            $\phi/(q,\mu_2)$ so $(p,\mu_1) \vDash^* \phi/(q,\mu_2)$.
        \item $\alpha = \tau$, $(p,\mu_1) \trans{a} (p,\mu_1')$ and $(q,\mu_2)
            \trans{\bar{a}} (q,\mu_2')$ for some $a\in A$. Then the disjunction
            $\phi/(q,\mu_2)$ has a clause $\dmnd{a}\left( \psi / (q,\mu_2') \right)$
            (note that $\bar{\bar{a}} = a$). By~\eqref{eq:dmnd_psi} we get that
            $(p,\mu_1) \vDash^* \phi / (q,\mu_2)$.
    \end{enumerate}
    In all cases the result is the same, namely $(p,\mu_1) \vDash^* \phi / (q,\mu_2)$
    which is what we wanted to prove.

    \rtl Now assume the right side of~\eqref{eq:decomp_rtl}, i.e. there exists a $(\mu_1,\mu_2)
    \in D(\pi)$ such that that $(p,\mu_1) \vDash^* \dmnd{\alpha}\psi / (q,\mu_2)$.
    We know $\dmnd{\alpha}\psi / (q,\mu_2)$ is a disjunction of one or more clauses
    so $(p,\mu_1)$ must satisfy at least one of them. Each clause has one of three
    forms, and we analyze the possible cases.
    Let $\phi'$ be a clause that $(p,\mu_1)$ satisfies.
    \begin{enumerate}
        \item Assume that $\phi' = \dmnd{\alpha} \left(\psi / (q,\mu_2)(q' \ptrans{} q') \right)$
            where $q' = \last(q,\mu_2)$.
            Then there is a $\mu_1'$ such that $(p,\mu_1) \trans{\alpha} (p,\mu_1')$
            and $(p,\mu_1') \vDash^* \psi/(q,\mu_2(q'\ptrans{}q'))$. If
            we let $\mu_2' = \mu_2(q' \ptrans{} q')$ then 
            part (ii) of
            Lemma~\ref{lemma:last_step_decomp}
            gives that there exists a $\pi'$ with $(\mu_1',\mu_2')\in D(\pi')$ and
            $(p\parallel q, \pi) \trans{\alpha} (p\parallel q, \pi')$.
            Since $(p,\mu_1')
            \vDash^* \psi / (q,\mu_2')$ then by induction, since $\psi$ is smaller than
            $\phi$, $(p\parallel q, \pi') \vDash^* \psi$. This in turn means that
            $(p\parallel q, \pi) \vDash^* \dmnd{\alpha}\psi = \phi$.
        \item Assume that $\phi' = \pdmnd{} \left(\psi / (q,\mu_2')\right)$ for some $\mu_2'$
            such that $(q,\mu_2) \trans{\alpha} (q,\mu_2')$. Let $\mu_1' = \mu_1(p'
            \ptrans{} p')$ where $p' = \last(p,\mu_1)$. Lemma~\ref{lemma:last_step_decomp}
            gives the existence of $\pi'$ with $(\mu_1',\mu_2')\in D(\pi')$ and $(p\parallel
            q,\pi) \trans{\alpha} (p\parallel q, \pi')$. Since $(p,\mu_1')\vDash^* \psi
            / (q,\mu_2')$ then by induction $(p\parallel q,\pi') \vDash^* \psi$ and thus
            $(p\parallel q, \pi) \vDash^* \dmnd{\alpha}\psi = \phi$.
        \item Assume that $\alpha = \tau$ and $\phi' = \dmnd{\bar{a}} \left( \psi / (q,\mu_2') \right)$
            for some $\mu_2'$ s.t. $(q,\mu_2)\trans{a}(q,\mu_2')$ and $a\in A$.
            This means there is a
            $\mu_1'$ with $(p,\mu_1) \trans{\bar{a}} (p,\mu_1')$.
            Lemma~\ref{lemma:last_step_decomp} then says that there exists $\pi'$ with $(p
            \parallel q, \pi) \trans{\tau} (p\parallel q, \pi')$ and $(\mu_1',\mu_2')
            \in D(\pi')$. Since $(p,\mu_1')\vDash^* \psi / (q,\mu_2')$ then by induction
            $(p\parallel q,\pi')\vDash^* \psi$ and thus $(p\parallel q, \pi) \vDash^*
            \dmnd{\tau}\psi = \phi$.
    \end{enumerate}
    In all cases we obtain what we wanted to prove, namely $(p\parallel q, \pi) \vDash^* \phi$.


    \proofcase{Case $\phi = \dmndback\alpha\psi$}\par\nobreak
    %
    \ltr Assume that $(p\parallel q, \pi) \vDash^* \dmndback{\alpha}\psi$ and take
    $(\mu_1,\mu_2)\in D(\pi)$. Since \mbox{$(p\parallel q, \pi') \trans{\alpha} (p\parallel
    q, \pi)$} for some $\pi'$ such that $(p\parallel q,\pi') \vDash^* \psi$,
    there exist processes $p',q',p'',q''$ such that $\pi
    = \pi'(p''\parallel q'' \trans{\alpha} p'\parallel q')$. By analysing the definition
    of $D$, we can gain some information about $\mu_1$ and $\mu_2$, in particular by
    comparing $p''$ to $p'$ and $q''$ to $q'$. Since $p''\parallel q'' \not\equiv
    p'\parallel q'$ we must consider three cases.
    \begin{enumerate}
        \item $p''\not\equiv p'$ and $q''\equiv q'$. Then $(\mu_1,\mu_2) = (\mu_1'
            (p'' \trans{\alpha} p'), \mu_2'(q' \ptrans{} q'))$ for some $(\mu_1',\mu_2')
            \in D(\pi')$. Given this form of $\mu_2$ we also know that $(\dmndback{\alpha}
            \psi) / (q,\mu_2) = \dmndback{\alpha} \left( \psi / (q,\mu_2') \right)$. Since
            $(p\parallel q, \pi') \vDash^* \psi$, we get by induction that $(p,\mu_1')
            \vDash^* \psi / (q,\mu_2')$, which in turn means that $(p,\mu_1)\vDash^*
            \dmndback{\alpha} \left(\psi / (q,\mu_2)\right)$ and since the last step of
            $\mu_2$ is a pseudo-step, $\dmndback{\alpha} \left(\psi / (q,\mu_2)\right)
            = (\dmndback{\alpha}\psi)/(q,\mu_2)$.
        \item $p''\equiv p'$ and $q''\not\equiv q'$. In this case $(\mu_1,\mu_2)
            = (\mu_1'(p'\ptrans{}p'), \mu_2'(q''\trans{\alpha}q'))$ where $(\mu_1',
            \mu_2') \in D(\pi')$. This form of $\mu_2$ means that $\dmndback{\alpha}
            \psi / (q,\mu_2) = \pdmndback{} \left( \psi / (q,\mu_2') \right)$.
            By induction, the fact that $(p\parallel q, \pi') \vDash^* \psi$ gives
            that $(p,\mu_1') \vDash^* \psi / (q,\mu_2')$, so
            since $\mu_2 = \mu_2'(q'' \trans\alpha q')$ holds,
            $(p,\mu_1) \vDash^*
            \pdmndback{} \left( \psi / (q,\mu_2') \right) = (\dmndback{\alpha}\psi)
            / (q,\mu_2)$.
        \item $p''\not\equiv p'$ and $q''\not\equiv q'$. By Lemma~\ref{lemma:both_step_means_tau}
            $\alpha$ must be equal to $\tau$. Thus we have that $(\mu_1,\mu_2) =
            (\mu_1'(p''\trans{a}p'), \mu_2'(q''\trans{\bar a}q'))$ for some $a\in A$ and
            $(\mu_1',\mu_2') \in D(\pi')$. This also means that $(\dmndback{\alpha}\psi)
            / (q,\mu_2) = \dmndback{a}\left( \psi / (q,\mu_2') \right)$. Since $(p\parallel q,
            \pi') \vDash^* \psi$ we again have by induction that $(p,\mu_1')\vDash^*\psi/(q,
            \mu_2')$. We therefore obtain that $(p,\mu_1) \vDash^* \dmndback{a}\left(
            \psi / (q,\mu_2') \right) = (\dmndback{\alpha}\psi) / (q,\mu_2)$.
    \end{enumerate}
    In all cases we obtain the same result, namely $(p,\mu_1) \vDash^* (\dmndback{\alpha}\psi)
    / (q,\mu_2) = \phi/(q,\mu_2)$.

    \rtl Now assume that there is $(\mu_1,\mu_2) \in D(\pi)$ s.t. $(p,\mu_1) \vDash^* (\dmndback{
    \alpha}\psi)/(q,\mu_2)$. This means that $\mu_1$ and $\mu_2$ are non-empty. By comparing
    $\alpha$ with the last transition  of $\mu_2$ we can infer the form of $(\dmndback{\alpha}
    \psi) / (q,\mu_2)$.
    \begin{itemize}
        \item If the last transition of $\mu_2$ is a $\ptrans{}$ transition, i.e. $\mu_2
            = \mu_2' (q' \ptrans{} q')$ for some $\mu_2$ and $q' = \last(q,\mu_2)$, then
            we know that $(\dmndback\alpha \psi) / (q,\mu_2) = \dmndback\alpha \left(
            \psi / (q,\mu_2') \right)$. By our assumption this is satisfied by $(p,\mu_1)$
            so there exists a $\mu_1'$ s.t. $(p,\mu_1') \trans\alpha (p,\mu_1)$ and
            $(p,\mu_1') \vDash^* \psi / (q,\mu_2')$. Let $\pi'$ be $\pi$ without the
            last transition (note that $\pi$ is non-empty since $\mu_1$ and $\mu_2$ are).
            By Lemma~\ref{lemma:removing_last_trans} $(\mu_1',\mu_2')\in D(\pi')$ and by
            induction we have that $(p\parallel q,\pi') \vDash^* \psi$. From the definition
            of $D$ we can also see that the last transition of $\pi$ can only be $\trans\alpha$.
            Thus $(p\parallel q,\pi') \trans\alpha (p\parallel q, \pi)$ so $(p\parallel q,\pi)
            \vDash^* \dmndback\alpha\psi$.
        \item If the last transition of $\mu_2$ is an $\trans\alpha$ transition, i.e. one
            having the same label as the formula is testing for, then $(\dmndback\alpha\psi)
            /(q,\mu_2) = \pdmndback{}\left( \psi/(q,\mu_2') \right)$ where $\mu_2'$ is $\mu_2$
            without the last transition. Note that $(q,\mu_2')\trans\alpha(q,\mu_2)$. Since
            $(p,\mu_1)$ satisfies this formula there is a $\mu_1'$ s.t. $(p,\mu_1')\ptrans{}
            (p,\mu_1)$ and $(p,\mu_1) \vDash^* \psi/(q,\mu_2')$. By
            Lemma~\ref{lemma:removing_last_trans} $(\mu_1',\mu_2')\in D(\pi')$ where $\pi'$
            is again $\pi$ without the last transition. Also again, we can see from the
            definition of $D$ that $(p\parallel q,\pi')\trans\alpha(p\parallel q,\pi)$.
            By induction it thus holds that \mbox{$(p\parallel q, \pi')\vDash^* \psi$} and so
            $(p\parallel q,\pi) \vDash^* \dmndback{\alpha}\psi$.
        \item The only remaining case to consider is when $\mu_2$ ends with a transition
            $\trans{\beta}$ where $\beta \ne \alpha$. Then $\alpha$ can only be $\tau$,
            since otherwise the formula $(\dmndback{\alpha}\psi)/(q,\mu_2)$
            equals $\false$, which contradicts our assumption
            that $(p,\mu_1)$ satisfies it. Since $\beta\ne\alpha = \tau$ we also know $\beta$
            must be some label $a\in A$. This means that $(\dmndback\alpha\psi)/(q,\mu_2)
            = \dmndback{\bar{a}} \left( \psi / (q, \mu_2') \right)$ where $\mu_2'$ is yet
            again $\mu_2$ without the last transition. Since $(p,\mu_1)$ satisfies this formula,
            there is a $\mu_1'$ s.t. $(p,\mu_1')\trans{\bar{a}}(p,\mu_1)$ and $(p,\mu_1')
            \vDash^* \psi/(q,\mu_2')$. By
            Lemma~\ref{lemma:removing_last_trans}, $(\mu_1',\mu_2')\in D(\pi')$ where
            $\pi'$ is $\pi$ without the last transition. By induction, $(p\parallel q, \pi')
            \vDash^* \psi$ and from the definition of $D$ we can see that $(p\parallel q,\pi')
            \trans\tau (p\parallel q,\pi)$ is the only possible transition between the
            two computations. Therefore,
            $(p\parallel q,\pi) \vDash^* \dmndback\tau \psi
            = \dmndback{\alpha}\psi$.
    \end{itemize}
    In all cases $(p\parallel q,\pi) \vDash^* \dmndback\alpha\psi = \phi$.

    %\todo{I think the last case analysis can be simplifed a lot by defining $\pi',\mu_1',\mu_2'$
    %in advance and invoking Lemma~\ref{lemma:removing_last_trans} before analyzing the
    %cases.}


    \proofcase{Case $\phi = \pdmnd{}\psi$}\par\nobreak
    %
    \ltr First assume $(p\parallel q, \pi) \vDash^* \pdmnd{}\psi$ and take $(\mu_1,\mu_2)
    \in D(\pi)$. This means there exists
    a $\pi'$ s.t. $(p\parallel q, \pi) \ptrans{} (p\parallel q, \pi')$ and $(p\parallel q,
    \pi')\vDash^* \psi$. By definition
    $(\pdmnd{}\psi)/(q,\mu_2) = \pdmnd{} \left( \psi / (q, \mu_2') \right)$,
    where we let $(\mu_1',
    \mu_2') = (\mu_1(p'\ptrans{}p'),$ $\mu_2(q'\ptrans{}q'))$ with $(p' \parallel q') =
    \last{(p\parallel q, \pi)}$. This is according to the definition of $D$ so $(\mu_1',\mu_2')
    \in D(\pi')$. Thus, by induction $(p,\mu_1') \vDash^* \psi / (q,\mu_2')$. Since $(p,\mu_1)
    \ptrans{} (p,\mu_1')$ we obtain that $(p,\mu_1) \vDash^* \pdmnd{} \left(
    \psi / (q,\mu_2') \right) = (\pdmnd{}\psi)/(q,\mu_2)$.

    \rtl Assume that $\exists (\mu_1,\mu_2) \in D(\pi) : (p,\mu_1) \vDash^* (\pdmnd{}
    \psi)/(q,\mu_2)$. 
    We want to show that $(p\parallel q,\pi) \vDash^* \pdmnd{}\psi$.
    Let $p' = \last(\mu_1)$ and $q' = \last(\mu_2)$.
    The formula $\pdmnd{} \psi)/(q,\mu_2)$ is equal to 
    $\pdmnd{} \left( \psi / (q,\mu_2') \right)$
    where $\mu_2' = \mu_2(q' \ptrans{} q')$. If we let $\pi' = \pi
    (p'\parallel q' \ptrans{} p'\parallel q')$ and $\mu_1' = \mu_1(p'\ptrans{} p')$,
    then, by definition of $D$, $(\mu_1',\mu_2') \in D(\pi')$.
    Observe that $(p,\mu_1)\ptrans{}(p,\mu_1')$ and that the $\ptrans{}$ relation
    is deterministic. Therefore
    $(p,\mu_1')\vDash^* \psi/(q,\mu_2')$ for each $(\mu_1,\mu_2)$. 
    Induction gives that $(p\parallel q, \pi')
    \vDash^* \psi$. Now it follows trivially that $(p\parallel q,\pi) \vDash^* \pdmnd{}
    \psi$ because $(p\parallel q, \pi)\ptrans{}(p\parallel q, \pi')$.


    \proofcase{Case $\phi = \pdmndback{}\phi$}\par\nobreak
    %
    \ltr Assume $(p\parallel q, \pi) \vDash^* \pdmndback{}\psi$ and take $(\mu_1,\mu_2)
    \in D(\pi)$. 
    This means that $\pi = \pi'(p'\parallel q' \ptrans{} p'\parallel q')$ and
    $(p\parallel q,\pi')\vDash^*\psi$, where $p'\parallel q' = \last(\pi)$.
    It is obvious, from the definition of $D$ that $\mu_1$ and $\mu_2$ both
    end with $\ptrans{}$ since $\pi$ ends with $\ptrans{}$. Let $\pi', \mu_1', \mu_2'$
    be $\pi,\mu_1,\mu_2$ without their last transition respectively (note that our assumption
    guarantees that they are non-empty). By Lemma~\ref{lemma:removing_last_trans} we know
    that $(\mu_1',\mu_2') \in D(\pi')$. Since $(p\parallel q,\pi') \vDash^* \psi$, we have
    by induction that $(p,\mu_1')\vDash^* \psi/(q,\mu_2')$. Then $(p,\mu_1)\vDash^*
    \pdmndback{} \left( \psi/(q,\mu_2') \right) = (\pdmndback{}\psi)/(q,\mu_2)$.

    \rtl Now assume $\exists (\mu_1,\mu_2) \in D(\pi) : (p,\mu_1) \vDash^* (\pdmndback{}
    \psi)/(q,\mu_2)$. Then the last step of $\mu_2$ is $\ptrans{}$ since otherwise
    the formula would be equal to $\false$, which could not be satisfied by $(p,\mu_1)$.
    We see furthermore that the quotiented formula is $\pdmndback{}\left(\psi/(q,\mu_2')\right)$ where
    $\mu_2'$ is again $\mu_2$ without its last step. This means the last step of $\mu_1$
    is also $\pdmnd{}$ (a fact we could also have deduced from the definition of $D$).
    Let $\mu_1'$ be $\mu_1$ without this step. If we also let $\pi'$ be $\pi$ without the
    last step, then by Lemma~\ref{lemma:removing_last_trans} we have
    $(\mu_1',\mu_2')\in D(\pi')$. Since
    $(p,\mu_1')\vDash^* \psi/(q,\mu_2')$ induction gives that $(p\parallel q,\pi')\vDash^*
    \psi$. By the definition of $D$ we see that the last step of $\pi$ can only be $\ptrans{}$
    so $(p\parallel q, \pi)\vDash^* \pdmndback{} \psi$.

    \vspace{1em}

    This concludes the analysis of all structural forms for $\phi$. In each case we have
    shown by structural induction that each direction of the theorem holds.
\end{proof}

Theorem~\ref{thm:decomposition} uses the existential quantifier in the right-to-left
direction. This makes it easy to show that a composed computation satisfies a
formula, given only one witness of a decomposition with one component satisfying
a rewritten formula. Note however that the set of decompositions of any given
process is never empty, i.e. every parallel computation has a decomposition.
This allows us to write the above theorem in a more symmetrical form.
\begin{corollary}
    For CCS processes $p,q$, a parallel computation $(p \parallel q, \pi)$
    and a formula $\phi \in \HMLpp$ we have
    %\sublabon{equation}
    \begin{equation}\label{eq:decomp_bi}
        (p\parallel q, \pi) \vDash^* \phi  \quad\Leftrightarrow\quad
        \forall (\mu_1,\mu_2) \in D(\pi) : (p, \mu_1) \vDash^* \phi/(q, \mu_2)
    \end{equation}
\end{corollary}
\begin{proof}
    \ltr This case follows directly from the theorem.
    
    \rtl Assume that $\forall (\mu_1,\mu_2) \in D(\pi) : (p, \mu_1) \vDash^* \phi/(q, \mu_2)$.
    Specifically, since there exists at least one decomposition $(\mu_1',\mu_2')\in D(\pi)$,
    the above holds for that particular decomposition. By the $\Leftarrow$ part of
    Theorem~\ref{thm:decomposition}, we thus have that \mbox{$(p\parallel q,\pi)\vDash^*\phi$}.
\end{proof}
