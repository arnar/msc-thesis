\section{Introduction}
State-space explosion is a major obstacle in model-checking logical properties.
One approach to combat this problem is compositional reasoning,
where properties of a system as a whole are deduced from the properties of its components.
Decompositional reasoning \cite{Giannakopoulou05,Xie05,Andersen95,LaroussinieL95} 
often improves upon compositional reasoning by
automatically decomposing the global property to be model checked
into local properties of (possibly unknown) components.
For example, Andersen's paper shows the effectiveness of the method in the analysis of Milner's scheduler~\cite{Milner89a}.
In the context of process algebras, as the specification language, and Hennessy-Milner logic, as the logical formalism for properties,
decompositional reasoning techniques date back to the seminal work of Larsen and Xinxin in early 90's \cite{Larsen91},
which is further developed in \cite{Simpson04,Fokkink06}. % add anna87 ?
However, we are not aware of any such decomposition technique which applies to reasoning about the ``past''.
This is particularly interesting in the light of recent developments concerning reversible processes~\cite{Phillips06} and
knowledge representation (epistemic aspects) inside process algebra~\cite{Mousavi07-LPAR},
all of which involve some notion of specification and reasoning about the past.


In this chapter, we tackle this problem and present a decomposition technique for Hennesy--Milner logic with past.
As the specification language, we use a subset CCS with parallel composition, non-deterministic choice, action prefixing
and the inaction constant.
%
The rest of the chapter is structured as follows. Section~\ref{sec:preliminaries}
introduces preliminary definitions and the extension of Hennessy--Milner logic
with past. Section~\ref{sec:decomp_comp} discusses how parallel computations that maintain
their history are decomposed into their parallel components. Section~\ref{sec:decomp_hml}
presents the decompositional reasoning and the main theorem of the paper, and
Section~\ref{sec:decomp_future} discusses related work and possible extensions to our results.

