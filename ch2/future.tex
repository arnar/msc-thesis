\section{Extensions and related work} % (fold)
\label{sec:decomp_future}

So far we have stated and proven a decompositional theorem that allows us to
apply decompositional reasoning for history-based computations over CCS and
the logic $\HMLpast$. However, more work remains to be done in order to apply
this theory to meaningful examples. Part of this work is well underway already
although the details have not been worked out fully for this particular thesis
submission. In particular, we have extended the decompositional theorem to a
recursive logic $\HMLppf{\X}$ (equivalent to extending the modal $\mu$-calculus
with past), which allows a much wider class of interesting 
properties to be specified as fixed-points of systems of recursive logic equations.

In the decomposition of computations, we rely on some specific properties of 
CCS at the syntactic level, namely to detect which rule of the parallel operator
was applied. By tagging transitions with their proofs~\cite{BC88,DeganoP92}, 
or even just the last
proof used, we could eliminate this restriction and extend our approach to more
general languages involving parallel composition. Another possibility is to construct
a rule format that guarantees the properties we use at a more general level, inspired
by the work of~\cite{Fokkink06}.

In this work we have only considered parallel composition. However, decompositional
results have been shown for the more general setting of \emph{process contexts}~\cite{Larsen91}.
The two parts considered as components are then not confined to components
of the parallel construction, but a context $C[\cdot]$ (a process term with a \emph{hole})
and general process $p$ to instantiate the context with. A general property of the
instantiated context $C[p]$ can the be transformed into an equivalent property
of $p$, where the transformation depends on $C$. As the state space explosion
of model-checking problems is often due to use of the parallel construct, we
deemed our approach as a useful step towards a full decompositional result, since
the decomposition of computations will be more complex for general contexts.

The initial motivation for this work was the application of epistemic logic
to behavioural models~\cite{Mousavi07-LPAR}. We would therefore like to extend
our results to logics that include epistemic operators, reasoning about the knowledge
of agents observing a running system. This work depends somewhat on our extensions
to recursive formulae.

Our work is based on many previous results, both in decompositional reasoning
as well as history-based process systems. Yet, to the best of our knowledge, we
are the first to combine the two.

Our notion of computations comes from~\cite{DeNicola:1990}, where they are used
to compare bisimulations, that consider both forwards and backward actions, to
more conventional forward-only bisimulations. The backwards modality of \HMLpast
comes from~\cite{HennessyS85}, which shares the same notion of computations
as well.
Our methods for decompositional reasoning largely build on and adapt the methods
from~\cite{anna87,Larsen91,Fokkink03b}.
However, as the work presented in this chapter shows, the development of a theory 
of decompositional reasoning in a setting with past modalities involves subtleties 
and design decisions that do not arise in previous work on HML and Kozen's 
$\mu$-calculus~\cite{Koz83}.