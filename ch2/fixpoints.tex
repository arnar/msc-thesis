\section{Extension to $\HMLpastf{\X}$ and $\HMLppf{\X}$}

\cite{}

In this section extend our logics to include variables. Satisfiability of 
formulas with variables is determined by a declaration of equations,
mapping variables to formulas, and depends on a model for the declaration,
i.e. a mapping of variables to sets of processes (or computations). We
are usually only interested in the minimal or maximal model for a given
declaration. By doing this we can reason about properties of processes
and computations that go beyond one step of lookahead or lookback. For
example we can phrase the question ``Has the action $\alpha$ ever happened
in the past?'' as a recursive logical property under a minimal model.

\begin{definition}
    Let $\T=\langle P,A,\rightarrow \rangle$ be an LTS and $\X$ a
    set of identifiers. The set $\HMLpastf{\X}(A)$,
    or simply $\HMLpastf{\X}$, is defined by the grammar
    \[
        \phi,\psi \,::=\, \true \alt \phi\land\psi
                                \alt \neg\phi
                                \alt \dmnd{\alpha} \phi
                                \alt \dmndback{\alpha} \phi
                                %\alt \pdmnd{i} \phi
                                %\alt \pdmndback{i} \phi
                                \alt X
    \]
    where $X\in \X$. The set $\HMLppf{\X}(A)$ is defined in the same manner, 
    extending $\HMLpp(A)$ with identifiers.
\end{definition}

When reasoning about recursive formulae, it is useful to have their denotational
semantics. For convenience we rephrase part of definition \ref{dfn:hmlpast} in a 
denotational setting.

\begin{definition}[Denotational semantics of $\HMLpp$]
    \label{dfn:hmlppsemantics}

    Let $\phi$ be a $\HMLpp$ formula. Its {\em denotation} $\den{\phi}\subseteq \C_\T$ 
    is defined structurally as follows.
    \[
    \begin{array}{r@{\quad}c@{\quad}l}
        \den{\true} &=&  \C_\T \\
        \den{\phi\land\psi} &=& \den{\phi} \cap \den{\psi} \\
        \den{\neg\phi} &=& \C_\T \setminus \den{\phi} \\
        \den{\dmnd{\alpha}\phi} &=& \dendmnd{\alpha}\den{\phi} \\
        \den{\dmndback{\alpha}\phi} &=& \dendmndback{\alpha}\den{\phi} \\
        \den{\pdmnd{}\phi} &=& \denpdmnd{}\den{\phi} \\
        \den{\pdmndback{}\phi} &=& \denpdmndback{}\den{\phi} \\
    \end{array}
    \]
    where the operators $\dendmnd{\alpha}, \dendmndback{\alpha},
    \denpdmnd{}, \denpdmndback{} : \powerset{\C_\T} \rightarrow \powerset{\C_\T}$
    are defined with
    \[
    \begin{array}{r@{\quad}c@{\quad}l}
        \dendmnd{\alpha}S &=& \{ \rho \in \C_\T \,|\, \exists \rho'\in S : \rho \trans{\alpha} \rho' \} \\
        \dendmndback{\alpha}S &=& \{ \rho \in \C_\T \,|\, \exists \rho'\in S : \rho' \trans{\alpha} \rho \} \\
        \denpdmnd{i}S &=& \{ \rho \in \C_\T \,|\, \exists \rho'\in S : \rho \ptrans{i} \rho' \} \\
        \denpdmndback{i}S &=& \{ \rho \in \C_\T \,|\, \exists \rho'\in S : \rho' \ptrans{i} \rho \} \\
    \end{array}
    \]
\end{definition}

We will use the same notation for denotations of formulas in $\HMLpast$, where
the stuttering operators are simply removed from the above definition.
The the denotational semantics coincide with definition \ref{dfn:hmlpast},
although we leave the proof as an exercise for the reader.

\begin{lemma}
    Let $\phi \in \HMLpp$ and $\rho \in \C_\T$. Then $\rho\in\den{\phi}$ if
    and only if $\rho \vDash^* \phi$.
\end{lemma}

To define the satisfiability relation between computations and formulas
of this logic, we need to define the meaning of each identifier appearing
in a formula. Thus, the satisfiability relation depends on a {\em declaration}
of variables.
\begin{definition}
    Let $\X$ be a set of variables. Then 
    a {\em declaration} over $\X$ is a partial function 
    $\D : \X \hookrightarrow \HMLppf\X$,
    assigning a formula to each variable of some subset of $\X$, with the restriction
    that the formulas only contain variables in the domain of $\D$ and each occurence
    of a variable in a formula is positive, i.e. any variable is within the scope of
    an even number of negations.
\end{definition}
%
We generally write declarations as sets of defining equations, for example
\begin{align*}
    X &= \dmnd{a}\true \lor Y \\
    Y &= \dmnd{A}\true \land \bx{A}X
\end{align*}
Note that our shorthands for $\lor$ and the box operators nest their components within
an even number of negations.

To define the precise meaning of a formula involving variables, we need to
fix a declaration of the variables in question. In addition we also need
to define the denotation of each variable, which is done by a {\em model}.
\begin{definition}
    \label{dfn:model}
    Let $\X$ be a set of variables and $\D$ a declaration over $\X$. 
    A {\em model} over $\X$ for the declaration $\D$
    is a partial function $\sigma : \X \hookrightarrow \powerset{\C}$,
    assigning each variable of a set of variables a subset of computations,
    such that the assignments of variables is consistent with $\D$, i.e.
    \[
    \sigma(X) = \den{\D(X)}
    \]
    where $\sigma$ is used in the right hand side to specify the denotation
    of any nested variables in the formula $\D(X)$.
    By $\M_\X$ we denote the set of all models over $\X$.
\end{definition}
%
The idea is that a computation $\rho$ satisfies the formula $X$ if and only
if $\rho \in \sigma(X)$ for a fixed model $\sigma$. In other words, $\sigma(X)$
is the denotation of the formula $X$. The above definition is a bit vague
in the term ``consistent.'' We clarify this by 
extending the denotational semantics for $\HMLpp$
to one for $\HMLppf{\X}$.
\begin{definition}[Denotational semantics of $\HMLppf{\X}$]
    \label{dfn:hmlppfsemantics}

    Let $\phi$ be a $\HMLppf{\X}$ formula and $\D$ be a declaration assigning
    formulas to every variable appearing in $\phi$ and $\D$ itself and
    $\sigma$ a model for $\D$. 
    The denotation of $\phi$ with respect to $\D$ and $\sigma$,
    written $\den{\phi}^\D_\sigma$
    is defined structurally as follows.
    \[
    \begin{array}{r@{\quad}c@{\quad}l}
        \den{\true}^\D_\sigma &=&  \C_\T \\[4pt]
        \den{\phi\land\psi}^\D_\sigma &=& \den{\phi}^\D_\sigma \cap \den{\psi}^\D_\sigma \\[4pt]
        \den{\neg\phi}^\D_\sigma &=& \C_\T \setminus \den{\phi}^\D_\sigma \\[4pt]
        \den{\dmnd{\alpha}\phi}^\D_\sigma &=& \dendmnd{\alpha}\den{\phi}^\D_\sigma \\[4pt]
        \den{\dmndback{\alpha}\phi}^\D_\sigma &=& \dendmndback{\alpha}\den{\phi}^\D_\sigma \\[4pt]
        \den{\pdmnd{}\phi}^\D_\sigma &=& \denpdmnd{}\den{\phi}^\D_\sigma \\[4pt]
        \den{\pdmndback{}\phi}^\D_\sigma &=& \denpdmndback{}\den{\phi}^\D_\sigma \\[4pt]
        \den{X}^\D_\sigma &=& \sigma(X) \\
    \end{array}
    \]
    where the operators $\dendmnd{\alpha}, \dendmndback{\alpha},
    \denpdmnd{i}, \denpdmndback{i} : \powerset{\C_\T} \rightarrow \powerset{\C_\T}$
    are the same as in definition~\ref{dfn:hmlppsemantics}.

    The satisfiability relation $\vDash^\D_\sigma \subseteq \C_\T \times \HMLppf{\X}$
    is defined by
    \[
        \rho \vDash^\D_\sigma \phi  \quad\Leftrightarrow\quad \rho \in \den{\phi}^\D_\sigma
    \]
\end{definition}

The {\em consistency} of models can now be expressed as follows. A function 
$\sigma : \X \hookrightarrow \powerset\C$ is a model for a declaration $\D$
if and only if for all $X$ in the domain of $\sigma$
\[
\sigma(X) = \den{\D(X)}_\sigma^\D.
\]

Obviously, for every declaration there may be a choice of multiple models. 
We are usually interested in either the greatest or least models, meaning the
sets assigned to variables are as large or as small as possible. Finding such
models is done by calculating fixpoints, whose existence is given by Tarski's
Fixpoint Theorem. Tarski's theorem requires that declarations define
a monotonic function on sets of computations. Since we have negation in our logic,
monotonicity is not guaranteed in the general case, but our restriction on
declarations that its formulas may only contain positive occurrences of variables
suffices to ensure monotonicity.
Since negation is the only operator within the
denotational semantics that is non-monotonic, this suffices.

For a declaration $\D$ we look at its {\em semantic function}. 
Let $\mathcal{O}_\D$ be a function that maps models to models,
such that $\mathcal{O}_\D(\sigma)$ maps to a model defined by $X \mapsto 
\den{\D(X)}^\D_\sigma$. Since models are functions on sets of computations, they can
be pointwise partially ordered using the normal $\subseteq$ order on sets. With respect
to this order, the function $\mathcal{O}_\D$ is monotonic under our conditions on $\D$.

The monotonicity property allows us to calculate minimal and maximal models for
monotonic declarations, using Tarski's Fixpoint Theorem. To find a maximal model,
start with the model that maps every variable to $\emptyset$ and iterate the
function $\mathcal{O}_\D$ until a fixpoint is reached. A minimal model can be
found similarly by starting with a model that maps all variables to $\C$.


\subsection{Transformation of formulas in $\HMLppf{\X}$}

We now turn back to the transformation of formulas, so that we can extend 
theorem~\ref{thm:decomposition} to include formulas from $\HMLppf{\X}$. This
section largely follows the precedence of~\cite{anna}.

In Section~\ref{sec:decomp_hml} we defined how a formula $\phi$ is quotiented
with respect to a computation $\rho$. In particular the quotiented formula
$\true/\rho$ is rewritten to $\true$ for any computation $\rho$. This works
reasonably well in the non-recursive setting, but there is a hidden assumption
that we must expose before tackling recursive formulas based on a declaration
of variables. In Theorem~\ref{thm:decomposition}, the two satisfiability
relations are actually based on two different transition systems. Satisfiability
on the right hand side of the theorem, namely
\[
(p,\mu_1) \vDash \phi/(q,\mu_2)  \forall (\mu_1,\mu_2) \in D(\pi).
\]
When working with this satisfiability relation, we have implicitly assumed
it is based on the transition system of computations from $p$, which are
{\em compatible} with the computations from $q$, i.e. in the above, $\mu_1$
really is a path that is the counterpart of $\mu_2$ in a decomposition of a
path $\pi$.

Intuitively, the set of computations that satisfy rewritten formula $\phi/\rho$, is the
set of computations which {\em are compatible with $\rho$} and their {\em
composition} satisfies the formula $\phi$. However, the rewriting
\[  \true/\rho = \true \]
does not match this intuition if we take the denotational viewpoint of the formula
$\true$ on the right hand side representing {\em all} possible computations.
In fact, we expect it to represent only those computations which are compatible with $\rho$.
We will now formalize the notion of compatible computations and refine our definition
of $\true/\rho$.

\begin{definition}\label{dfn:compatible}
    Paths $\mu_1$ and $\mu_2$ are {\em compatible} with each other if and only
    if they have the same length and one of the following holds if they are non-empty.
    \begin{itemize}
        \item If $\mu_1 = \mu_1'(p'' \trans \tau p')$ then 
            $\mu_2 = \mu_2'(q' \ptrans i q')$ and 
            $\mu_1$ and $\mu_2$ are compatible.
        \item If $\mu_1 = \mu_1'(p'' \trans a p')$ then either 
            $\mu_2 = \mu_2'(q'' \trans{\bar a} q')$ or 
            $\mu_2 = \mu_2'(q' \ptrans i q')$; and in both cases
            $\mu_1$ and $\mu_2$ are compatible.
        \item If $\mu_1 = \mu_1'(p'' \ptrans i p')$ then either 
            $\mu_2 = \mu_2'(q'' \trans{\alpha} q')$ or 
            $\mu_2 = \mu_2'(q' \ptrans i q')$; and in both cases
            $\mu_1$ and $\mu_2$ are compatible.
    \end{itemize}
    We say two computations are compatible with each other if their paths
    are compatible.
\end{definition}
In a sense, the compatibility of paths is the inverse of decomposition, as stated
by the following lemma.
\begin{lemma}
    If paths $\mu_1$ and $\mu_2$ are compatible, then there exists a path $\pi$
    such that $(\mu_1,\mu_2)\in D(\pi)$.
\end{lemma}
\begin{proof}
    The path $\pi$ is constructed in the obvious way, each transition of it is
    obtained by composing the processes from the matching transitions of
    $\mu_1$ and $\mu_2$ with the parallel operator and determining
    the action according to the rules for that operator. The conditions of
    Definition~\ref{dfn:compatible} ensure that the choice of actions is
    unambiguous in every case. The rest is easy to check with the definition of $D$.
\end{proof}

We now revise our transformation of the formula $\true$. We want $T/\rho$ to be
the set of all computations that are compatible with $\rho$. It turns out this
can be expressed easily in $\HMLpp$, for which we define the following short-hand.
\begin{definition}
    Let $\pi$ be a path of transitions in the LTS
    $\T = \langle P,A,\rightarrow \rangle$.
    Then the \HMLpp{} formula $\true_\pi$ is defined as follows.
    \[ \begin{array}{rcl}
        \true_\lambda &=& \bxback A \false \land \pbxback i \false \\
        \true_{\pi'(p\trans \tau p')} &=& \pdmndback{} \true_{\pi'} \\
        \true_{\pi'(p\trans a p')} &=& \dmndback{\bar a} \true_{\pi'}  \lor \pdmndback{} \true_{\pi'} \\
        \true_{\pi'(p\ptrans{} p')} &=& \dmndback A \true_{\pi'} \lor \pdmndback{} \true_{\pi'} \\
    \end{array} \]
\end{definition}
An astute reader will notice that this is a rewording of Definition~\ref{dfn:compatible},
and it is easy to see that the computations satisfying $T_\pi$ are exactly the computations
that are compatible with the computations which have $\pi$ as their path. 
Now the revised transformation of $\true$ is
\begin{equation}
\true/(p,\pi) = \true_\pi
\end{equation}
which matches our intuition.

\hspace{1cm}

In the recursive setting, satisfiability depends not only on the formula, 
but also on a declaration
and its model, it is reasonable that we need to transform the declaration and
the model as well as formulas.
For the logic constructs of $\HMLpp$, we can reuse the transformation defined
in section~\ref{sec:decomp_hml} and define how to rewrite formulas of the form $X$. 
However, instead
of decomposing formulas of this form, we will treat the quotient $X/\rho$
as a variable, i.e. we use the set $\X\times\C$ as our set of variables. 
The intuitive idea of such variables is the following,
\[
(p,\mu_1) \vDash^{\D'}_{\sigma'} X/(q,\mu_2)
\quad \Leftrightarrow \quad
(p\parallel q, \pi) \vDash^\D_\sigma X
\quad \Leftrightarrow \quad
(p\parallel q, \pi) \in \sigma(X)
\]
where $\sigma$ is a model for the declaration $\D$ over the variables $\X$, 
$\sigma'$ and $\D'$ similarly over the variables $\X\times\C$,
and $(\mu_1,\mu_2)\in D(\pi)$. 
We will see later the relation between $\D$ and $\D'$
as well as $\sigma$ ad $\sigma'$.

Formally, the variables in our rewritten logic will be pairs $(X,\rho)\in \X\times\C$.
Formulas of the form $X$ are simply rewritten as follows.
\[
X/\rho = (X,\rho)
\]
where the $X/\rho$ on the left hand side denotes the transformation (as in
Section~\ref{sec:decomp_hml}) and the pair on the right hand side is the
variable in our adapted logic. When there is no risk of ambiguity, we will
simply use the notation $X/\rho$ to represent {\em the variable} $(X,\rho)$.

\subsubsection{Transformation of declarations}

Generating the transformed declaration $\D'$
from a declaration $\D$ is straightforward as well. For any variable $X/\rho$
encountered in any rewritten formula
\[
\D'(X/\rho) = \D(X)/\rho
\]
Note that the rewritten formula on the right hand side may introduce more variables
which obtain their values in $\D'$ in the same manner.

\subsubsection{Transformation of models}

To each model $\sigma$ for a declaration $\D$ over the set of variables $\X$,
there corresponds a model $\sigma'$ for the declaration $\D'$ over the set of
variables $\X\times\C$, where $\D'$ is defined as above, such that the following
holds.
\[
(p\parallel q, \pi) \vDash^\D_\sigma \phi
\quad \Leftrightarrow \quad
(p,\mu_1) \vDash^{\D'}_{\sigma'} \phi/(q,\mu_2)
\]
for all $(\mu_1,\mu_2)\in D(\pi)$.

We represent the definition of $\sigma'$ with the function $\Phi$, which
maps models over $\X$ to models over $\X\times\C$.
\begin{align*}
\sigma'(X/(q, \mu_2)) &= \Phi(\sigma)(X/(q, \mu_2))  \\
& = \{ (p, \mu_1) \,|\, (p\parallel q, \pi) \in \sigma(X) \textrm{ for some $\pi$ with }
(\mu_1,\mu_2) \in D(\pi) \}
\end{align*}

We intend to show the correspondence of minimal and maximal models, i.e. if $\sigma$
is a minimal model for $\D$, then $\sigma'$ is a minimal model for $\D'$ and vice versa,
and the same holds for maximal models. 
In particular, we will show that there is a bijection on the two sets of models, based
on the mapping $\Phi$. First we will attempt to define its inverse.
Consider the function $\Psi$,
which maps a model over $\X\times\C$ to a model over $\X$.
\[
\Psi(\sigma')(X) = \{ (p\parallel q, \pi) \,|\, 
\forall(\mu_1,\mu_2) \in D(\pi) :
(p,\mu_1) \in \sigma'(X/(q,\mu_2)) \}
\]

We can show that $\Psi\circ\Phi$ is the identity mapping on assignments over
$\X\times\C$. However, for now we can only prove a weaker condition about
$\Phi\circ\Psi$.

%\TODO{how do we make it clear that "`normal"' models only have processes of the
%form $p\parallel q$? do we need to}
\begin{lemma}
    \[ \Psi \circ \Phi = \id_{\M_\X}  \]
and
    \[ \forall \sigma' : (\Phi \circ \Psi)(\sigma') \sqsubseteq \sigma' \]
where $\sqsubseteq$ is the point-wise partial ordering on functions of set
generated by $\subseteq$.
\end{lemma}
\begin{proof}
    Let $\sigma\in\M_\X$. Then for any $X\in\X$
    \begin{align*}
        \phantom{\Leftrightarrow}\quad  &
        (p\parallel q, \pi) \in (\Psi(\Phi(\sigma))(X) \\
        %        
        \Leftrightarrow\quad &
        \forall (\mu_1,\mu_2) \in D(\pi) : (p,\mu_1)\in \Phi(\sigma)(X/(q,\mu_2)) \\
        %
        \Leftrightarrow\quad &
        \forall (\mu_1,\mu_2) \in D(\pi) : \exists \pi' :
             (p\parallel q, \pi') \in \sigma(X) \land (\mu_1,\mu_2)\in D(\pi')
        %
        \intertext{By lemma~\ref{thm:paths_compose_uniquely} we have that in the last 
        line, $\pi=\pi'$,
        which allows us to continue,}
        %
        \Leftrightarrow\quad &
        \forall (\mu_1,\mu_2) \in D(\pi) : (p\parallel q, \pi) \in \sigma(X) \\
        %
        \Leftrightarrow\quad &
        (p\parallel q, \pi) \in \sigma(X)
    \end{align*}
    This shows that the sets $\sigma(X)$ and $(\Psi(\Phi(\sigma)))(X)$ are equal.

    Now let $\sigma'\in \M_{\X\times\C}$. For any $X/(q,\mu_2) \in \X\times\C$
    \begin{align*}
        \phantom{\Leftrightarrow}\quad  &
        (p,\mu_1) \in (\Phi(\Psi(\sigma'))(X/(q,\mu_2)) \\
        %
        \Leftrightarrow\quad &
        \exists \pi : (\mu_1,\mu_2) \in D(\pi) \land
                (p\parallel q, \pi) \in \Psi(\sigma')(X) \\
        %
        \Leftrightarrow\quad &
        \exists \pi : (\mu_1,\mu_2) \in D(\pi) \land
                \left( \forall (\mu_1', \mu_2')\in D(\pi) : 
                    (p,\mu_1') \in \sigma'(X/(q,\mu_2')) \right) \\
        %
        \Rightarrow\quad &
        \exists \pi : (p,\mu_1) \in \sigma'(X/(q,\mu_2)) \\
        %
        \Leftrightarrow\quad &
        (p,\mu_1) \in \sigma'(X/(q,\mu_2))
    \end{align*}
    This shows that for any variable $X/(q,\mu_2)$, $(\Phi(\Psi(\sigma')))(X/(q,\mu_2))
    \subseteq \sigma'(X/(q,\mu_2))$.
\end{proof}
