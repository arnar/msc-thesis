\section{Preliminaries}\label{sec:preliminaries}
\subsection{Computations and CCS} % (fold)
The following definitions come mostly from~\cite{DeNicola:1990}.

\begin{definition}[Labelled transition system]
    A \emph{labelled transition system} (LTS)
    is a triple $\langle P,A,\trans{}\rangle$ where
    \begin{itemize}
        \item $P$ is a set of process names.
        \item $A$ is a finite set of action names, not including a \emph{silent action}
              $\tau$. We write $A_\tau$ for $A\cup \{\tau\}$.
        \item $\trans{}\subseteq P\times A_\tau \times P$ is the \emph{transition
              relation}, we call its elements \emph{transitions} and
              usually write $p\trans{\alpha}p'$ to mean that $(p,\alpha,p')\in\trans{}$.
    \end{itemize}
    We let $p,q,\dots$ range over $P$, $a,b,\dots$ over $A$ and $\alpha,\beta,\dots$
    over $A_\tau$.
\end{definition}

\begin{definition}[Sequences and computations]
    For any set $S$ we let $\seq S$ be the set of finite sequences of elements
    from $S$. Concatenation of sequences is represented by juxtaposition.
    $\lambda$ denotes the empty sequence and $\abs\sigma$ stands for the length of a
    sequence $\sigma$.

    Given an LTS $\T = \langle P,A,\trans{} \rangle$, we define a
    \emph{path from $p_0$} to be a sequence of transitions
    \[
        p_0\trans{\alpha_0}p_1,
        p_1\trans{\alpha_1}p_2, \dots ,
        p_n\trans{\alpha_{n-1}}p_n
    \]
    and usually write this as
    \[
        p_0 \trans{\alpha_0} p_1
            \trans{\alpha_1} p_2
            \trans{\alpha_2} \cdots
            \trans{\alpha_{n-1}}p_n.
    \]
    We use $\pi,\mu,...$ to range over paths.
    A \emph{computation from $p$} is a pair $(p,\pi)$ where $\pi$ is a path
    from $p$ and
    we use $\rho,\sigma,\dots$ to
    range over computations.
    $\C_\T(p)$,
    or simply $\C(p)$ when the LTS $\T$ is clear from the context,
    is the set of computations from $p$ and
    $\C_\T$ is the set of all computations in $\T$.

    For a computation $\rho = (p_0,\pi)$
    where $\pi = p_0 \trans{\alpha_0} p_1 \trans{\alpha_1} p_2 \trans{\alpha_2}
    \cdots \trans{\alpha_{n-1}}p_n$ we define
    \begin{align*}
        \first(\rho) &= p_0 \\
        \last(\rho) &= p_n \\
        \trace(\rho) &= (\alpha_0\dots\alpha_{n-1}) \in \seq{A_\tau} \\
        \abs{\rho} &= \abs{\pi} = n
    \end{align*}
    We will refer to  elements of $\seq A$
    as \emph{traces}. 
    %Note that the length
    %of a computation is the length of its trace, not the number of processes.

    Concatenation of computations $\rho$ and $\rho'$
    is denoted by their juxtaposition
    $\rho\rho'$ and is defined iff $\last(\rho)=\first(\rho')$.
    When $\last(\rho) = p$ we will
    write $\rho(p\trans{\alpha} q)$ as a shorthand for the slightly longer
    $\rho(p, p\trans{\alpha} q)$.
    We also use $\rho \trans{\alpha} \rho'$ to denote that
    there exists a computation
    $\sigma = (p, p \trans\alpha p')$, for some processes $p$ and $p'$,
    such that $\rho' = \rho\sigma$.
\end{definition}

\begin{remark}
    Representing computations with a pair $(p,\pi)$ might seem redundant at
    first, since $\pi$ must start with $p$. However, an empty computation
    $(p,\lambda)$ is also valid and must be distinguished from $(q,\lambda)$
    if $p\ne q$.
\end{remark}

\subsection{Hennessy-Milner Logic with Past}


\begin{definition}[Hennessy Milner logic with past]
    Let $\T=\langle P,A,\rightarrow \rangle$ be an LTS. The set $\HMLpast(A)$,
    or simply $\HMLpast$, of
    \emph{Hennessy-Milner logic formulas with past} is defined by the following grammar,
    where $\alpha\in A_\tau$.
    \[
        \phi,\psi \,::=\, \true \alt \phi\land\psi
                                \alt \neg\phi
                                \alt \dmnd{\alpha} \phi
                                \alt \dmndback{\alpha} \phi.
    \]
    We define the \emph{satisfaction relation} $\vDash\,\subseteq \C_\T \times \HMLpast$
    as the least relation that satisfies the following clauses:
    \begin{itemize}
        \item $\rho\vDash\true$ for all $\rho\in\C_\T$,
        \item $\rho\vDash\phi\land\psi$ iff $\rho\vDash\phi$ and $\rho\vDash\psi$,
        \item $\rho\vDash\neg\phi$ iff not $\rho\vDash\phi$,
        \item $\rho\vDash\dmnd{\alpha}\phi$ iff
              $\rho\trans{\alpha}\rho'$ and $\rho'\vDash\phi$ for some $\rho'\in\C_\T$,
        \item $\rho\vDash\dmndback{\alpha}\phi$ iff
              $\rho'\trans{\alpha}\rho$ and $\rho'\vDash\phi$ for some $\rho'\in\C_\T$.
    \end{itemize}
    The satisfaction relation $\vDash\,\subseteq P\times\HMLpast$ is defined by
    $p\vDash\phi$ if and only if $(p,\lambda)\vDash \phi$.
\end{definition}

We will make use of some standard shorthands for Hennessy-Milner type logics.
\begin{align*}
    \false &= \neg\true \\
    \phi \lor \psi &= \neg(\neg\phi \land \neg\psi) \\
    \bx{\alpha}\phi &= \neg\dmnd{\alpha}(\neg\phi) \\
    \bxback{\alpha}\phi &= \neg\dmndback{\alpha}(\neg\phi) \\
\end{align*}
For a finite set of actions $A$, we will also use the following notation.
\begin{align*}
    \dmnd{A}\phi &= \bigvee_{\alpha\in A} \dmnd{\alpha} \phi  &
    \dmndback{A}\phi &= \bigvee_{\alpha\in A} \dmndback{\alpha} \phi \\
    \bx{A}\phi &= \bigwedge_{\alpha\in A} \bx{\alpha} \phi  &
    \bxback{A}\phi &= \bigwedge_{\alpha\in A} \bxback{\alpha} \phi \\
\end{align*}
Intuitively, $\rho\vDash\dmndback{\alpha}\phi$ means the last action of $\rho$
{\em is} labelled with $\alpha$ and the preceding computation satisifies $\phi$, while
$\rho\vDash\bxback{\alpha}\phi$ means that {\em if} the last computation is labelled
with $\alpha$, then the preceding computation satisfies $\phi$. Specifically, a
computation with an empty path, i.e. the initial state, can never satisfy the former
while it will always vacuously satisfy the latter.

\subsubsection{Determinism of the past}
\label{sec:determinism_of_the_past}

It is worth mentioning that the operators $\dmnd\cdot$ and $\dmndback\cdot$ are
not entirely symmetric. The future is non-deterministic, i.e. at any point we may
have a choice of multiple ways for a computation to proceed. 
In our semantics for \HMLpast{}, the past however is
always deterministic; there is always at most one transition that was the last
transition to occur. This is by design, and we could have chosen to model the past
as nondeterministic as well, i.e. to take a possibilistic view where we would consider
all possible histories. 

This would make the forward and backwards diamond
operators symmetric in their view of the process graph. However, we are more
interested in properties about the actual past of a computation, especially
w.r.t. modelling epistemic properties, which rely on observations of some aspects
of the computation so far. (We do not discuss such properties in the
current paper.) We have thus reached the same conclusion as \cite{Laroussinie00}
that the deterministic view is more appropriate for our purposes.
Laroussinie and Schnoebelen list two other
properties of their model of the past in addition to being deterministic, namely that
the past is finite (there is a fixed initial state) and that past is cumulative (at each
transition the history gains information). We make implicit use of the latter property 
in our proofs and find that the former is a natural property of the processes we
want to model.

Treating the past as deterministic might suggest that it is unnecessary to index
the operator $\dmndback{\alpha}$ with the action and instead provide an operator
$\dmndback{\,}$ which matches any action (since there is at most one). However this
turns out to be insufficient, as such a logic could not distinguish between the
two computations
\[
    (r, r\trans\alpha t) \quad \textrm{and} \quad (r, r\trans\beta s)
\]
when $\alpha\ne\beta$. For our intended purposes, it is important that we are
able to tell these two apart.
