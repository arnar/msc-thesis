\section{Computations and CCS} % (fold)
\label{sec:preliminaries}

The following definitions come mostly from~\cite{DeNicola:1990}.

\begin{definition}[Labelled transition system]
    A \emph{labelled transition system} (LTS)
    is a triple $\langle P,A,\trans{}\rangle$ where
    \begin{itemize}
        \item $P$ is a set of process names
        \item $A$ is a set of action names, not including a \emph{silent action}
              $\tau$. We write $A_\tau$ for $A\cup \{\tau\}$.
        \item $\trans{}\subseteq P\times A\times P$ is the \emph{transition
              relation}, we call its elements \emph{transitions} and
              usually write $p\trans{a}p'$ to mean that $(p,a,p')\in\trans{}$.
    \end{itemize}
    We let $p,q,\dots$ range over $P$, $a,b,\dots$ over $A$ and $\alpha,\beta,\dots$
    over $A_\tau$.
\end{definition}

\begin{definition}[Sequences and computations]
    For any set $S$ we let $\seq S$ be the set of finite sequences of elements
    from $S$. Concatenation of sequences is represented by juxtaposition.
    $\lambda$ denotes the empty sequence and $\abs\sigma$ the length of a
    sequence $\sigma$.

    Given an LTS $\T = \langle P,A,\trans{} \rangle$, we define a
    \emph{path from $p_0$} to be a sequence of transitions
    \[
        p_0\trans{\alpha_0}p_1,
        p_1\trans{\alpha_1}p_2, \dots ,
        p_n\trans{\alpha_{n-1}}p_n
    \]
    and usually write this as
    \[
        p_0 \trans{\alpha_0} p_1
            \trans{\alpha_1} p_2
            \trans{\alpha_2} \cdots
            \trans{\alpha_{n-1}}p_n.
    \]
    We use $\pi,\mu,...$ to range over paths.
    A \emph{computation from $p$} is a pair $(p,\pi)$ where $\pi$ is a path
    from $p$ and
    we use $\rho,\sigma,\dots$ to
    range over computations.
    $\C_\T(p)$ or simply $\C(p)$ is the set of computations from $p$ and
    $\C_\T$ is the set of all computations in $\T$.

    For a computation $\rho = (p_0,\pi)$
    where $\pi = p_0 \trans{\alpha_0} p_1 \trans{\alpha_1} p_2 \trans{\alpha_2}
    \cdots \trans{\alpha_{n-1}}p_n$ we define
    \begin{align*}
        \first(\rho) &= p_0 \\
        \last(\rho) &= p_n \\
        \trace(\rho) &= (\alpha_0,\dots,\alpha_{n-1}) \in \seq{A_\tau} \\
        \abs{\rho} &= \abs{\pi} = n
    \end{align*}
    We will refer to  elements of $\seq A$
    as \emph{traces}. Note that the length
    of a computation is the length of its trace, not the number of processes.

    Concatenation of computations $\rho$ and $\rho'$
    is denoted by their juxtaposition
    $\rho\rho'$ and is defined iff $\last(\rho)=\first(\rho')$.
    When $\last(\rho) = p$ we will
    write $\rho(p\trans{\alpha} q)$ as a shorthand for the slightly longer
    $\rho(p, p\trans{\alpha} q)$.
    We also use $\rho \trans{\alpha} \rho'$ to denote that
    there exists a computation
    $\sigma = (p, p \trans\alpha p')$, for some processes $p$ and $p'$,
    such that $\rho' = \rho\sigma$.
\end{definition}

\begin{note}
    Representing computations with a pair $(p,\pi)$ might seem redundant at
    first, since $\pi$ must start with $p$. However, an empty computation
    $(p,\lambda)$ is also valid and must be distinguished from $(q,\lambda)$
    if $p\ne q$.
\end{note} 