\documentclass[10pt]{book}

% Use utf-8 encoding for foreign characters
%\usepackage[utf8]{inputenc}

% Font setup
\usepackage[T1]{fontenc}
\usepackage[medfamily,opticals,mathlf,minionint,loosequotes,fourierbb]{MinionPro}
\usepackage{MnSymbol}
\usepackage[sf,lining,scaled=0.92]{myriad}
\usepackage[T1]{fontenc}
\usepackage[scaled=.9]{blgothic}
\usepackage{microtype}

% Setup for fullpage use
%\usepackage{fullpage}
% For customization:
%\setlength{\topmargin}{0pt}
%\setlength{\textheight}{27cm}


\usepackage[paperwidth=172mm,
            paperheight=246mm,
            left=34mm,
            right=25mm,
            top=25mm,
            bottom=25mm,
            head=10mm,
            foot=10mm,
            %left=1.5cm,
            %top=2cm,
            %right=1.5cm
        ]{geometry}

% More symbols
%\usepackage{amsmath}
%\usepackage{amssymb}
%\usepackage{latexsym}
\usepackage{stmaryrd}

% For multiple columns
%\usepackage{multicol}
%\setlength{\columnsep}{1cm}
%\parindent=1em

% This is now the recommended way for checking for PDFLaTeX:
\usepackage{ifpdf}

%\newif\ifpdf
%\ifx\pdfoutput\undefined
%\pdffalse % we are not running PDFLaTeX
%\else
%\pdfoutput=1 % we are running PDFLaTeX
%\pdftrue
%\fi

\ifpdf
\usepackage[pdftex]{graphicx}
\else
\usepackage{graphicx}
\fi

\ifpdf
%\usepackage{pdfsync}

\fi

\usepackage{rotating}
\usepackage{color}

\definecolor{webblue}{rgb}{0,0,0.5}
\definecolor{darkred}{rgb}{0.63,0.16,0.16}

%\ifpdf
%	\newcommand*{\defaultcolor}{\color{black}}
%	\newcommand*{\spotcolor}{\color{darkred}}
%\else
	\let\defaultcolor\relax
	\let\spotcolor\relax
%\fi

% Other packages and settings
\usepackage{fancyvrb}
\usepackage{shortvrb}
\usepackage[plainpages=false,pdfpagelabels,pdfborder={0 0 0}]{hyperref}
\usepackage{textcomp}
\usepackage{listings}

%\usepackage{sublabel}
%\renewcommand{\substyle}[1]{--\alph{#1}}

\usepackage{fancyhdr}

\usepackage{natbib}
\setlength{\bibsep}{3pt plus .5pt minus .25pt}
\bibpunct{[}{]}{,}{A}{}{,}
\let \cite = \citep

\usepackage{amsthm}
\usepackage{ifthen}

\newtheoremstyle{definition}% name of the style to be used
  {.5em}% measure of space to leave above the theorem. E.g.: 3pt
  {.5em}% measure of space to leave below the theorem. E.g.: 3pt
  {\upshape}% name of font to use in the body of the theorem
  {}% measure of space to indent
  {}% name of head font
  {}% punctuation between head and body
  {.5em}% space after theorem head
  {{\bfseries\scshape\thmname{#1}\thmnumber{ #2}.}\thmnote{ {\itshape(#3)}}}% Manually specify head

\theoremstyle{definition}
\newtheorem{theorem}{Theorem}[chapter]
\newtheorem{lemma}[theorem]{Lemma}
\newtheorem{corollary}[theorem]{Corollary}

\theoremstyle{definition}
\newtheorem{definition}[theorem]{Definition}
\newtheorem{example}[theorem]{Example}

\newtheorem*{remark}{Remark}

% Put section numbers in the margin
\makeatletter 
\def\@seccntformat#1{\protect\makebox[0pt][r]{\csname the#1\endcsname\quad}} 
\makeatother 

% Chapter headers 
\newcommand\jointwith\relax
\newcommand{\TheAlphaChapter}{%
\ifcase\number\thechapter\or one\or two\or three\or four\or five\fi}
\makeatletter
\def\@makechapterhead#1{%
  %\vspace*{10\p@}%
  {\parindent \z@ \raggedleft \normalfont
    \ifnum \c@secnumdepth >\m@ne
      \if@mainmatter
        {\huge\sffamily\lseries\MakeLowercase{\@chapapp\space \TheAlphaChapter}}
        \par\nobreak
        \vskip 30\p@
      \fi
    \fi
    \interlinepenalty\@M
    {\Huge\sffamily\sbseries #1\par\nobreak}
    \if \jointwith \relax\relax \else
        \vskip 10\p@
        {\Large\sffamily joint work with \jointwith\par\nobreak}
    \fi
    \vskip 40\p@
  }}%
  \def\@makeschapterhead#1{%
    \vspace*{30\p@}%
    {\parindent \z@ \raggedleft
      \normalfont
      \interlinepenalty\@M
      {\Huge\sffamily\sbseries #1\par\nobreak}
      \vskip 40\p@
    }}%
\renewcommand\tableofcontents{%
    \if@twocolumn
      \@restonecoltrue\onecolumn
    \else
      \@restonecolfalse
    \fi
    \chapter*{\contentsname
        \@mkboth{%
           {\upshape\scshape\MakeLowercase\contentsname}}{{\upshape\scshape\MakeLowercase\contentsname}}}%
    \@starttoc{toc}%
    \if@restonecol\twocolumn\fi
    }
\renewenvironment{thebibliography}[1]
     {\chapter*{\bibname
        \@mkboth{\upshape\scshape\MakeLowercase\bibname}{\upshape\scshape\MakeLowercase\bibname}}%
      \list{\@biblabel{\@arabic\c@enumiv}}%
           {\settowidth\labelwidth{\@biblabel{#1}}%
            \leftmargin\labelwidth
            \advance\leftmargin\labelsep
            \@openbib@code
            \usecounter{enumiv}%
            \let\p@enumiv\@empty
            \renewcommand\theenumiv{\@arabic\c@enumiv}}%
      \sloppy\clubpenalty4000\widowpenalty4000%
      \sfcode`\.\@m}
     {\def\@noitemerr
       {\@latex@warning{Empty `thebibliography' environment}}%
      \endlist}
\renewcommand\@makefnmark{\@textsuperscript{\footnotesize\@thefnmark}}
\long\def\@makecaption#1#2{%
  \vskip\abovecaptionskip
  \hrule width \hsize height .33pt
  \vspace{4pt}
  \sbox\@tempboxa{#1: #2}%
  \ifdim \wd\@tempboxa >\hsize
    #1: #2\par
  \else
    \global \@minipagefalse
    \hb@xt@\hsize{\hfil\box\@tempboxa\hfil}%
  \fi
  \vskip\belowcaptionskip}
\makeatother




\usepackage{epigraph}
\setlength{\epigraphrule}{0pt}

\newcommand{\todo}[1]{[{\textcolor{red}{\bf TODO:}} #1]}



% Listings setup
\lstset{%
  language=Haskell,
  aboveskip=0.5\baselineskip,
  belowskip=0.5\baselineskip,
  basicstyle=\ttfamily,
  commentstyle=\itshape,
  emphstyle=\spotcolor,
  numberstyle=\scriptsize,
  breaklines=true,
  basewidth=0.5em,
  numbersep=8pt,
  fontadjust=true,
  flexiblecolumns=true,
  keepspaces=false,
  extendedchars=true,
  tabsize=4,
  upquote=true}
\lstdefinestyle{small}{}
\lstdefinestyle{numbered}{numbers=left,stepnumber=1}

\lstdefinelanguage{pseudo}{%
    morekeywords={function, if, return, try, catch, main, program, transaction, declare, sensitive, array, of},
}

% Note that listings doesn't play well with utf8. To use
% national chars with listings, the file must be saved
% in latin1 encoding and the argument to inputenc changed
% accordingly.

%------------------------------------------------------------------------------
%                                Space savers. 
%------------------------------------------------------------------------------
%
% Space saving List environment for enumerations. Does not indent any of the
% items.
\newcounter{myctrsquish}
\newenvironment{mylistsquish}{\begin{list}{(\textbf{\arabic{myctrsquish}})}
{\usecounter{myctrsquish}
\setlength{\topsep}{1mm}\setlength{\itemsep}{0.5mm}
\setlength{\parsep}{0.5mm}
\setlength{\itemindent}{0mm}\setlength{\partopsep}{0mm}
\setlength{\labelwidth}{-2mm}
\setlength{\leftmargin}{0mm}}}{\end{list}}

% This mylist environment indents items, and saves less space than the above.
\newcounter{myctr}
\newenvironment{mylist}{\begin{list}{(\textbf{\arabic{myctr}})}
{\usecounter{myctr}
\setlength{\topsep}{1mm}\setlength{\itemsep}{0.5mm}
\setlength{\parsep}{0.5mm}
\setlength{\listparindent}{\parindent}
\setlength{\itemindent}{-.5ex}\setlength{\partopsep}{0mm}
\setlength{\labelwidth}{-2mm}
\setlength{\leftmargin}{0mm}}}{\end{list}}

% Space saving List environment for itemizing.
\newenvironment{mybullet}{\begin{list}{$\bullet$}
{\setlength{\topsep}{1mm}\setlength{\itemsep}{0.5mm}
\setlength{\parsep}{0.5mm}
\setlength{\listparindent}{\parindent}
\setlength{\itemindent}{0mm}\setlength{\partopsep}{0mm}
\setlength{\labelwidth}{15mm}
\setlength{\leftmargin}{4mm}}}{\end{list}}

% Space saving List environment for subitemizing.
\newenvironment{mysubbullet}{\begin{list}{--}
{\setlength{\topsep}{1mm}\setlength{\itemsep}{0.5mm}
\setlength{\parsep}{0.5mm}
\setlength{\itemindent}{0mm}\setlength{\partopsep}{0mm}
\setlength{\labelwidth}{15mm}
\setlength{\leftmargin}{4mm}}}{\end{list}}

% Definitions for SOS rules

\newcommand{\sos}[2]{\frac{#1}{#2}}
\newcommand{\transition}[1]{\;#1\;}
\newcommand{\prg}[1]{\texttt{#1}}
\newcommand{\soslbl}[1]{\textit{(#1)}}

\newcommand{\alt}{\;\;|\;\;}

\newcommand{\tarrow}{\transition{\rightarrow}}
\newcommand{\tdarrow}{\transition{\Rightarrow}}
\newcommand{\tddarrow}{\transition{\rightharpoonup}}

\newcommand{\PP}{\mathbb{P}}
\newcommand{\EE}{\mathbb{E}}
\renewcommand{\SS}{\mathbb{S}}

\DeclareMathOperator{\img}{img}
\DeclareMathOperator{\dom}{dom}

% Comment this out if we prefer slashed o:
\renewcommand{\emptyset}{\{\}}

% for shading the background
%\definecolor{old}{gray}{0}
%\definecolor{newcolor}{gray}{0.85}
%\newcommand{\new}[1]{\colorbox{newcolor}{#1}}

% for coloring the text
%\definecolor{old}{gray}{0}
%\definecolor{newcolor}{rgb}{0.8,0,0}
%\newcommand{\new}[1]{\textcolor{newcolor}{#1}}

% new is black, old is gray
\definecolor{old}{gray}{0.5}
\newcommand{\new}[1]{\textcolor{black}{#1}}

\newcommand{\assocdesim}{\textsf{ASSOC-De Simone}}
\newcommand{\Implies}{\mathrel{\Rightarrow}}
\newcommand{\Iff}{\mathrel{\Leftrightarrow}}
\newcommand{\defs}{\mathrel{\doteq}}
\newcommand{\ident}[1]{\mathit{#1}}
\newcommand{\MAR}[1]{\textbf{\underline{Michel says}: #1}}
\newcommand{\MRM}[1]{\textbf{\underline{Mohammad says}: #1}}
\newcommand{\AB}[1]{\textbf{\underline{Arnar says}: #1}}


%\newcommand{\sosrule}[3][0]{
%\ifthenelse{\equal{#1}{0}}
%  {\begin{array}[c]{c}
%   #2\vspace{0.1em}\\\hline\rule[1em]{0pt}{0.1em}#3
%   \end{array}}
%  {\displaystyle\frac{#2}{#3}\ifthenelse{\equal{#1}{}}{}{,\quad}#1}
%}
\newcommand{\sosproof}[3][0]{
\ifthenelse{\equal{#1}{0}}
  {\begin{array}[b]{c}
   #2\vspace{0.1em}\\\hline\hline\rule[1em]{0pt}{0.1em}#3
   \end{array}}
  {\begin{array}[c]{c}
   #2\vspace{0.1em}\\\hline\hline\rule[1em]{0pt}{0.1em}#3
   \end{array}\ifthenelse{\equal{#1}{}}{}{,\quad}#1}
}

\newcommand{\eqidem}{\mathrel{\simeq_f}}
\newcommand{\isomorphic}{\mathrel{\sim_i}}

%\newcommand{\assocrules}{\mathcal{R}_\textrm{a}}
%\newcommand{\assocrulest}{\mathcal{R}_\textrm{a\term}}
%\newcommand{\tss}{\mathcal{T}}
%\newcommand{\trans}[1]{\,{\stackrel{{#1}}{\rightarrow}}\,}
\newcommand{\cando}[1]{\mathrel{\mbox{ can}_{#1}\,}}
\newcommand{\after}[1]{\mathrel{\mbox{ after}_{#1}\,}}

%\newcommand{\ntrans}[1]{\,{\stackrel{{#1}}{\nrightarrow}}\,}
\newcommand{\bisim}{\mathbin{\mbox{$\underline{\leftrightarrow}$}}}
\newcommand{\rel}{\mathcal{R}}
\newcommand{\Rel}{\,\mathcal{R}\,}
\newcommand{\cond}[1]{\textbf{Condition: }#1}
\newcommand{\term}{\checkmark}
\newcommand{\interrupt}{\vartriangleright}
\newcommand{\disrupt}{\blacktriangleright}
\newcommand{\DR}[1]{\ensuremath \mathrm{(#1)}}
\newcommand{\DRn}[1]{\ensuremath \mathrm{#1}}

\newcommand{\prem}[1]{\ensuremath \ident{prem}(#1)}
\newcommand{\conc}[1]{\ensuremath \ident{conc}(#1)}

\newcommand{\ntyft}{\textsf{Ntyft}}

\newcommand{\Terms}[1]{\mathbb{T}(#1)}
\newcommand{\CTerms}[1]{\mathbb{C}(#1)}

\newcommand{\vars}[1]{\mathit{vars}(#1)}



\newenvironment{bullets}{\begin{list}{$\bullet$}{\itemsep 0em\topsep 0em}\labelwidth 3mm}{\end{list}}


\newcommand{\assocdesim}{\textsf{ASSOC-De Simone}}
\newcommand{\Implies}{\mathrel{\Rightarrow}}
\newcommand{\Iff}{\mathrel{\Leftrightarrow}}
\newcommand{\defs}{\mathrel{\doteq}}
\newcommand{\ident}[1]{\mathit{#1}}
\newcommand{\MAR}[1]{\textbf{\underline{Michel says}: #1}}
\newcommand{\MRM}[1]{\textbf{\underline{Mohammad says}: #1}}
\newcommand{\AB}[1]{\textbf{\underline{Arnar says}: #1}}


%\newcommand{\sosrule}[3][0]{
%\ifthenelse{\equal{#1}{0}}
%  {\begin{array}[c]{c}
%   #2\vspace{0.1em}\\\hline\rule[1em]{0pt}{0.1em}#3
%   \end{array}}
%  {\displaystyle\frac{#2}{#3}\ifthenelse{\equal{#1}{}}{}{,\quad}#1}
%}
\newcommand{\sosproof}[3][0]{
\ifthenelse{\equal{#1}{0}}
  {\begin{array}[b]{c}
   #2\vspace{0.1em}\\\hline\hline\rule[1em]{0pt}{0.1em}#3
   \end{array}}
  {\begin{array}[c]{c}
   #2\vspace{0.1em}\\\hline\hline\rule[1em]{0pt}{0.1em}#3
   \end{array}\ifthenelse{\equal{#1}{}}{}{,\quad}#1}
}

\newcommand{\eqidem}{\mathrel{\simeq_f}}
\newcommand{\isomorphic}{\mathrel{\sim_i}}

%\newcommand{\assocrules}{\mathcal{R}_\textrm{a}}
%\newcommand{\assocrulest}{\mathcal{R}_\textrm{a\term}}
%\newcommand{\tss}{\mathcal{T}}
%\newcommand{\trans}[1]{\,{\stackrel{{#1}}{\rightarrow}}\,}
\newcommand{\cando}[1]{\mathrel{\mbox{ can}_{#1}\,}}
\newcommand{\after}[1]{\mathrel{\mbox{ after}_{#1}\,}}

%\newcommand{\ntrans}[1]{\,{\stackrel{{#1}}{\nrightarrow}}\,}
\newcommand{\bisim}{\mathbin{\mbox{$\underline{\leftrightarrow}$}}}
\newcommand{\rel}{\mathcal{R}}
\newcommand{\Rel}{\,\mathcal{R}\,}
\newcommand{\cond}[1]{\textbf{Condition: }#1}
\newcommand{\term}{\checkmark}
\newcommand{\interrupt}{\vartriangleright}
\newcommand{\disrupt}{\blacktriangleright}
\newcommand{\DR}[1]{\ensuremath \mathrm{(#1)}}
\newcommand{\DRn}[1]{\ensuremath \mathrm{#1}}

\newcommand{\prem}[1]{\ensuremath \ident{prem}(#1)}
\newcommand{\conc}[1]{\ensuremath \ident{conc}(#1)}

\newcommand{\ntyft}{\textsf{Ntyft}}

\newcommand{\Terms}[1]{\mathbb{T}(#1)}
\newcommand{\CTerms}[1]{\mathbb{C}(#1)}

\newcommand{\vars}[1]{\mathit{vars}(#1)}



\newenvironment{bullets}{\begin{list}{$\bullet$}{\itemsep 0em\topsep 0em}\labelwidth 3mm}{\end{list}}


\title{Topics in Structural Operational Semantics}
\author{Arnar Birgisson}

\begin{document}

\ifpdf
\DeclareGraphicsExtensions{.pdf, .jpg, .tif}
\else
\DeclareGraphicsExtensions{.eps, .jpg}
\fi

\renewcommand{\chaptermark}[1]{\markboth{\upshape\scshape chapter \thechapter. \MakeLowercase{#1}}{}}
\renewcommand{\sectionmark}[1]{\markright{\upshape\scshape\MakeLowercase{#1}}{}}

\frontmatter{}

\begin{titlepage}
\setlength{\hoffset}{0mm}
\setlength{\oddsidemargin}{5mm}
\begin{center}
\vspace*{1cm}
{\fontsize{16}{24}\selectfont\spotcolor\scshape topics in} \\
\vspace*{.5cm}
{\fontsize{24}{50}\selectfont\spotcolor\bfseries
Structural Operational Semantics}\\
\vspace*{1.5cm}
\textsc{\large thesis}\\
\vspace*{.3cm}
\textit{submitted in partial fulfillment of the \\ 
requirements for the degree of }\\
\vspace*{.5cm}
\textsc{\large master of science}\\
\vspace*{.3cm}
\textit{in}\\
\vspace*{.3cm}
\textsc{\large computer science}\\
\vspace*{2.3cm}
\textsc{\LARGE arnar birgisson}\\
\vspace*{.2cm}
\textit{Supervisor: Luca Aceto}\\
\vspace*{1.7cm}
%\textsc{\Large school of\\ computer science}\\
%\vspace*{2cm}
\begin{figure}[htbp]
    \centering
        \includegraphics[height=2cm]{RU-logo-EN.pdf}
\end{figure}
\begin{figure}[htbp]
    \centering
        \includegraphics[height=2.5cm,trim=0cm 1cm 0cm 0cm,clip]{CS-Address.pdf}
\end{figure}

\end{center}
\end{titlepage}

\tableofcontents

\mainmatter{}

\chapter{Introduction}
Languages are among the most important and ubiquitous concepts in Computer Science.
In almost every sub-field of Computer Science, one can find specific languages for
describing and reasoning about the concepts of that field. Programming languages
are the best known example of course, but file formats, network protocols,
instruction sets and even the various diagrammatic techniques used in Software
Engineering can be considered as (visual) languages. Theoretical Computer Science has
specification languages and various logics. Natural Language Processing has markup
languages for describing voicing and sentence structure. Artificial Intelligence
uses specific languages for describing behaviour, the rules of games and the
constraints of planning problems. Indeed, it seems to be common practice in 
Computer Science to invent formalisms for specific problems, and these formalisms
very often involve some kinds of languages.

Any language consists of two parts: its \emph{syntax} and its \emph{semantics}.
The syntax defines what strings of symbols are valid, i.e. part of the language,
while the semantics defines the actual \emph{meaning} of any valid string.
Formal specification of syntax is very common, even in non-academic use of 
Computer Science. However, one can argue that what really 
defines the true nature
of a a language
is its semantics. Different languages are much rather set apart by different
semantics than different syntax.
This thesis is a study, by way of example, of one specific technique of specifying
semantics formally, namely \emph{Structural Operational Semantics}. To put things
in context, we'll start by an informal overview of this field.


\section{Structural Operational Semantics}

Structural Operational Semantics~\cite{Plotkin04a, Plotkin04b},
SOS or simply Operational Semantics
\footnote{The term Operational Semantics is sometimes used as a synonym for
Structural Operational Semantics (\emph{small-step} semantics) and Natural
Semantics (\emph{large-step} semantics). In this thesis we are mostly
concerned with the former.}
for short,
is a way of defining the meaning of terms in formal languages. By
\emph{formal languages} we mean any language for specifying ideas formally in the mathematical
sense. This includes programming languages, process languages for modelling and
verification as well as many others. By \emph{terms} we usually mean programs
or specifications written in these languages.

As the name indicates, SOS describes semantics in terms of program \emph{structure}
and the \emph{operations} a program carries out as it computes. 
An SOS specification of the semantics for a certain
language is a collection of rules. These rules specify how a term with a certain
structure behaves, by describing the operation that the next step of execution
of this term performs
(often on a hypothetical machine), and what is the term that should be executed
for the next step. An example of an SOS rule is
\begin{equation}\label{eq:intro_rule1}
    \frac{t_1 \trans{a} t_1' \quad t_2 \trans{b} t_2'}{\textsf{op(}t_1,t_2\textsf{)}\trans{b} t_2'}
\end{equation}
This rule specifies how a term of the form $\textsf{op(}t_1,t_2\textsf{)}$ behaves, where $t_1$
and $t_2$ can be any sub-terms as allowed by the syntax. This is seen by looking
at the left-hand side of the \emph{conclusion}, the part appearing below the line. 
The two expressions above
the line are called \emph{premises}. These are operations of the sub-terms
that describe when the rule is applicable to derive a step of computation
of the composite program $\textsf{op(}t_1,t_2\textsf{)}$.
This particular rule only applies if the operations in the premises can be
deduced from the collection of rules. Naturally, this depends in each case on what
the terms $t_1$ and $t_2$ actually are. If the operations in all the premises
are valid for given sub-terms, the label of the arrow and the right-hand side below the line
specify what is the operation performed by executing $\textsf{op(}t_1,t_2\textsf{)}$
and what is the term that results after doing so. As one can see, both of these may
be parameterised with information from the premises.

The hypothetical execution of terms proceeds by finding a rule that matches the
term to be executed and whose premises are met. This rule then specifies an operation,
i.e. \emph{a single step} of execution, and the term to use for finding the
next step. This process is repeated to create a sequence of operations.
It is important to note that execution in this context does not necessarily mean
execution on a real machine, but rather it is a useful abstract metaphor for reasoning about
program behaviour. We say the sequences of operations are steps in the execution
of a program on a hypothetical machine.

Often it is useful to indicate when such a sequence may stop, i.e. when the
program terminates. We often do this by designating a specific set of terms as
\emph{terminal}, meaning that when a sequence of operations reaches such a term,
the application of rules stops.
Sometimes this is the empty term, e.g. a program of the form
\textsf{print "Hello"; print "World"} might perform the following sequence of
operations
\[
    \textsf{print "Hello"; print "World"}
    \;\trans{!"Hello"}\;
    \textsf{print "World"}
    \;\trans{!"World"}\;
    \epsilon
\]
where the operation $!string$ stands for writing $string$ to the screen, and
$\epsilon$ is the empty program. In other settings the terminal terms may represent
values. For example, a functional programming language might specify the meaning
of the term $50 - 4 \times 2$ with the following sequence of operations.
\[
    50 - 4 \times 2 \;\trans{}\; 50 - 8 \;\trans{}\; 42
\]
In this case, the term $42$ is terminal since it contains no operators.

There is an important difference between the two approaches; in the former case the natural meaning
of the program \textsf{print "Hello"; print "World"} is determined by the sequence
of operations that its execution goes through, while in the latter the meaning
of the program $50 - 4 \times 2$ is represented by the final value
that the sequence reaches. Which one we choose depends on the particular setting
in which we are using SOS.

For the latter interpretation, where the meaning of a term is taken to be the
final value reached by a sequence of operations, there is an important thing to
note about an SOS specification (collection of rules). In the general case, there
is nothing that prevents the specification to contain rules that allow us to
deduce \emph{multiple} sequences of operations. For example, consider a system
that contains the rule~\ref{eq:intro_rule1} above, but also contains the following
rule.
\begin{equation}\label{eq:intro_rule2}
    \frac{t_1 \trans{a} t_1' \quad t_2 \trans{b} t_2'}{\textsf{op(}t_1,t_2\textsf{)}\trans{a} t_1'}
\end{equation}
Presented with a term of the form $\textsf{op(}t_1,t_2\textsf{)}$, we can see
that both rules may apply (depending on $t_1$ and $t_2$). If they do, we have
a case of non-determinism where the term may either be executed according to
rule~\ref{eq:intro_rule1} or rule~\ref{eq:intro_rule2}. In fact, an operator
with this pair of rules is known as a \emph{choice operator}, i.e. the execution
of the term $\textsf{op(}t_1,t_2\textsf{)}$ may choose whether it behaves like
$t_1$ or like $t_2$.

It is not difficult to see that, when we take the meaning of a program to be its
final value, if such non-determinism exists in the SOS specification, this meaning
is not well defined. A term might give rise to multiple sequences of operations and
thus multiple final values. Thus, when this view of meaning is taken, which is
common when dealing with functional languages, we often make the requirement that
the language's specification given by the rules is \emph{deterministic}, 
i.e. for each term
there is at most one operation and subsequent term that can be deduced from the
collection of rules. Such collection of rules are the topic of Chapter~\ref{ch:formats}
of this thesis.

In the other setting, where meaning of a term is taken to be the sequence of
operations it gives rise to, non-determinism is generally allowed. This is for example
the case in Process Algebra, where the meaning of a term is simply taken to be
determined, in some formal sense, by
the set of all possible behaviours it may generate. Two terms might for instance be considered
equal if they generate the same set of operation sequences.

\vspace{1em}

Sometimes the terms of the language don't contain enough information themselves
to model execution. This is for example the case in programming languages that have
variables which are globally bound. To find the value of a program term $3 + x$,
one needs to know the value of $x$. In SOS specifications, this is solved by using
\emph{configurations} instead of terms in the rules. A configuration is a predefined
structure which models the state of the execution completely. In the case of languages
with variables, a common formulation is to represent the states as pairs of a
term (with the same meaning as described above) and a \emph{variable store}, written
$\langle t, \Theta \rangle$. The variable store is in turn a function from the
set of variables to actual values.\footnote{Or terms in the case of lazy languages.}
A typical rule in such a language might look like this.
\begin{equation}\label{eq:intro_rule3}
    \frac{}{\langle \textsf{x := }n, \Theta\rangle \trans{}
            \langle \epsilon, \Theta[x\mapsto n]\rangle} \qquad n\in \mathbb{N}
\end{equation}
Note that this rule has no premises, which means that it applies whenever the term
to be executed matches the left-hand side of the conclusion.
The rule specifies that the operation of executing an assignment term, e.g.
\textsf{x := 28}, under a store $\Theta$, results in a configuration with an
empty term and a store that is identical to $\Theta$ except for its value in $x$,
which is mapped to $28$ (this is the conventional meaning of the $[\cdot\mapsto\cdot]$ syntax).
Formally there is nothing special about using configurations instead of terms;
configurations can themselves be considered ``terms'' of an extended language.

Another interesting thing to note about rule~\ref{eq:intro_rule3} is that it is
in fact a \emph{rule schema}. In other words, it represents a countably infinite
number of rules, indexed by the natural number $n$. This is common when a part
of the syntax of the language comes from a large domain such as $\mathbb{N}$.

\vspace{1em}

This thesis consists of three main chapters, each of which is an independent paper.
While their topics are in essence not related to each other, they all make use of
operational semantics in a central manner. Although familiarity with SOS helps, the
informal introduction above should provide the reader with enough background to read
Chapter~\ref{ch:tmi}, which exemplifies a fairly complex use case of SOS. 
Chapters~\ref{ch:decomp} and especially~\ref{ch:formats} use semantics in a more
formal way; these chapters will each introduce the necessary preliminaries needed
for their discussion. The following section introduces each chapter and highlights
their ties to operational semantics.


\section{Thesis contributions}

Over the course of 12 months, rather than working solely on one single MSc study project,
I have participated in several research projects at
Reykjavik University and at the Technical University of Eindhoven. The result of this work
are research contributions made by me and my co-authors to a few different fields
of Computer Science. Each of these projects have built on the theory of SOS; in fact
one of the projects (Chapter~\ref{ch:formats}) is only about the theory of SOS rule
systems, independent of their use.

The papers are arranged in order of increasing abstraction. Chapter~\ref{ch:tmi} uses SOS
to specify the semantics of an authorisation framework in a functional programming language.
The SOS specification presented is non-trivial, but is intended solely for clarifying
the semantics of this particular framework. The specification is accompanied by a
detailed discussion of the semantics as well as an implementation of the framework
in question.

Chapter~\ref{ch:decomp} is in the field of Process Algebra. It 
uses SOS to provide quotienting techniques a la~\cite{Larsen91} for extensions
to the process specification language CCS and Hennessy-Milner logic. CCS has a simple
operational semantics and the paper proves, using the semantics, a powerful theorem
for studying properties that include past modalities in a decompositional manner.

Finally, Chapter~\ref{ch:formats} goes one abstraction level above SOS specifications
and provides so-called \emph{meta-theorems} about rule systems that guarantee their
determinism and the idempotence of certain operators. The meta-theorems consist of
syntactic conditions on the rules themselves, such conditions are generally referred
to as \emph{SOS rule formats}.

In order to give the reader enough background for each chapter, we will now describe
the general field of each paper, its contributions as well as highlight the specific
contributions I made to each.


\subsection{Semantics of Transactional Memory Introspection}

Chapter~\ref{ch:tmi} builds on previous work of \cite{tmi}. In that paper we present
an authorisation architecture called Transactional Memory Introspection, or TMI.
The motivation for this architecture comes from the fact that Software Transactional
Memory has recently become a popular way of avoiding race conditions in concurrent
programs. Software Transactional Memory, or STM for short, tackles the issue of
shared memory by replacing programmer managed locks with transactions. Where programmers
would conventionally manage access to shared resources by careful lock placement,
they may use STM instead to offload this responsibility to a machine controlled
framework.

When using STM, programmers do away with lock management and instead mark sections
of code as \emph{atomic}. At run-time, an STM system will, as part of the program
in question, ensure that the accesses a single thread makes to shared resources
inside such atomic sections, appear atomic to other threads and moreover, are
completely isolated from the effects concurrent threads may have on this shared
state. Semantically this is equivalent to enforcing a rule which says that only
one thread may be running in an atomic section at each time.

The beauty of STM comes from the fact that the actual implementation 
does not enforce such strict policies, 
as that would hurt performance. Instead, multiple threads are allowed to
execute simultaneously inside atomic sections. Meanwhile, the STM system will carefully
monitor the actions of each one of the threads. Generally, the threads will be
accessing disjoint sets of resources, so most of the time this simultaneous execution
poses no problems. However, in the cases where threads in atomic sections do conflict
in their accesses, the STM system will notice and simply roll back some or all
of the threads involved, and restart their execution at the start of their atomic
sections. A rollback consists of undoing all work done by the threads, and will
be triggered in cases when the execution of a thread has violated the isolation
guarantee of the STM system. In practice, such violations happen in the minority
of cases, so often the overhead of this approach will be paid for by the overhead
saved in not using fine grained locking.

To implement the above, an STM system generally must provide
\begin{itemize}
    \item isolation of concurrent threads in atomic sections,
    \item monitoring of resource access to detect conflicts,
    \item the ability to abort and rollback execution of an atomic section.
\end{itemize}
In~\cite{tmi} we argue that these mechanisms can be very beneficial to the problem
of \emph{policy enforcement}. Policy enforcement (or authorisation) is required
in programs that handle sensitive data, to ensure that no illegal operations are
performed, such as releasing confidential data or otherwise violate the applications
policy. Traditionally this is done by careful code scrutiny and great effort on
the programming side. Just as with locking, this practice is prone to errors.

Since programs that use STM systems for synchronisation purposes are already paying
the price of monitoring and maintaining the ability to abort code, we conjectured
that these mechanisms could be used to simplify policy enforcement at a relatively
little extra cost. We identified three common problems (or errors) in modern policy
enforcement code.
\begin{itemize}
    \item \emph{Time of check to time of use} (TOCTTOU) bugs. These happen when
    a policy decision is made prior to access, but the state used for the decisions
    is mutated in between by a concurrent thread.
    \item Difficulty in guaranteeing \emph{complete mediation}, i.e. ensuring that
    any access, explicit or implicit, is accompanied by the relevant policy check.
    This is non-trivial in complex systems and empirical studies show that this
    is a source of several security holes in critical software.
    \item Difficulty in dealing with authorisation errors, when a policy violation
    has been detected, the system state must be carefully reset in order not to leak
    sensitive information or implicitly cause other policy violations.
\end{itemize}
The first of these is simply a synchronisation issue, and could be solved with locking.
However, STM systems provide a synchronisation mechanism with added benefits; we
can make use of its careful monitoring and abort capabilities to severely reduce
the second and third difficulties.

When an application that makes use of TMI (which implies the use of STM) runs,
any accesses made inside atomic sections are inspected by the STM system. TMI
hooks into this inspection and also notifies an application specific security
manager, which checks if the access is allowed by the application policy. At any
time, the security manager has the capability to veto an access due to policy
violation, in which case the abort mechanism of the STM is invoked. In one fell
swoop this solves the issue of complete mediation, since the STM diligently
inspects every access, as well as the issue of error handling since the rollback
puts the system back into a consistent state and the isolation guarantees of the
STM make sure that no concurrent thread gained knowledge of the actions leading
up to the policy violation.

Our previous work of~\cite{tmi} consists of an extended discussion of the above,
accompanied by a proof-of-concept implementation based on a prototype STM framework
for Java~\cite{hlm06}. However, while working with TMI and STM systems in general,
we discovered that there are a great number of subtleties in the behaviour of
unusual edge cases. An informal discussion, and even an implementation, did not
provide a thorough understanding of the semantics of TMI. Thus the contribution
presented in Chapter~\ref{ch:tmi} consists of the formal specification of the semantics
of our architecture, in the form of an extension to the semantics of the Haskell
STM system~\cite{haskellstm}. The semantics is accompanied by a matching implementation.

My specific contributions to Chapter~\ref{ch:tmi} consist of most of the technical
work involved. I built the extension of the Haskell STM semantics, which went
through several iterations of discussions with my co-author and revisions. In
parallel I wrote the implementation in Haskell, which provided a lot of insight
into the design decisions behind the semantic specification. I wrote the initial
versions of most of the text, except for the introduction and the background on
STM and TMI. All sections underwent a rewriting phase carried out jointly by
me and my co-author.

This work has been submitted to the ACM SIGPLAN Fourth Workshop on Programming
Languages and Analysis for Security (PLAS 2009), scheduled for June 15th 2009
in Dublin, Ireland. The author notification will arrive before the final version
of this thesis is prepared. The submission is identical to the version presented
here.

\renewcommand{\jointwith}{\'Ulfar Erlingsson}
\chapter{Semantics of Transactional Memory Introspection}
%\markboth{\scshape\upshape chapter \thechapter. semantics of tmi}{}
\section{Introduction} % (fold)
\label{sec:introduction}

The implementation of security enforcement mechanisms requires special care,
as any flaw may open the door to malicious attacks. This is especially true
in the case of multithreaded software, as the designer must consider all possible
interleavings of code paths. In~\cite{tmi} we presented Transactional Memory Introspection
(TMI), an architecture that greatly simplifies the implementation of correct
reference monitors on 
mechanisms that
implement Software Transactional Memory (STM)~\cite{harrisFraserSTM,hm93} support.
%
In this paper we present a formal semantics for TMI,
    as well as a reference TMI implementation over the Haskell STM.
These specifications clarify the TMI architecture
    and helps identify and resolve ambiguities 
    in its implementation.


STM systems provide many useful guarantees that make the implementation of multithreaded
software easier and less error-prone. In particular, STM offers 
atomicity and isolation
through optimistically executing concurrent code and monitoring
for conflicting accesses to resources.
By providing
rollback mechanisms, STM systems can resolve conflicting accesses by undoing the work of a transaction
and retrying that transaction again, from the beginning.

All STM implementations must perform bookkeeping of accesses (such as reads and writes)
to shared resources. By interposing on 
this bookkeeping, and the necessary monitoring and validation steps,
TMI provides facilities to support the creation of robust and correct
enforcement mechanisms. 
TMI provides \emph{complete mediation} by enhancing the STM
runtime checks against conflicting, concurrent accesses, and TMI
adds the requirement that all accesses must have been successfully
authorized before a transaction is committed.
TMI also \emph{simplifies error handling}. When unauthorized accesses are detected in a transaction,
the transaction is rolled back and not retried. 
This saves the programmer from the onerous and error-prone task of performing clean-up after
a failed authorization.

Another common problem in traditional authorization is {\em time of check to time of use} bugs.
Such bugs arise when an authorization check is used to decide if a
dangerous operation should be performed, and when the interleaving
of code execution may cause state changes that invalidate that
decision, before the dangerous operation is actually performed.
TMI resolves this problem, by making use of STM mechanisms to
execute both the authorization check and the dangerous operation
within a single transaction.

In~\cite{tmi} we give a comprehensive, informal overview of the TMI architecture and also evaluate 
a Java TMI implementation built on the DSTM2 library~\cite{hlm06}.
In order to provide clear, well-defined semantics for TMI, this paper provides a more
formal treatment and defines a structural operational semantics~\cite{plotkin_04_structural} 
for the TMI architecture.
For this we need an STM system with equally well founded semantics. This is provided by the Haskell
STM, whose semantics is given in~\cite{haskellstm}. Our semantics of TMI builds on this foundation
and is also accompanied by an implementation over the Haskell STM system. We believe having formal
semantics for TMI is important to guide possible implementations and disambiguate design choices.

The structure of the paper is as follows. In section 2 we give the necessary background, including
STM systems and how TMI makes use of their mechanisms, as well as an overview of the Haskell STM
implementation. Section 3 covers TMI in greater detail and describes its Haskell implementation
from the user standpoint. Section 4 defines the formal semantics of TMI, building on existing
semantics for STM Haskell. Section 5 describes the key elements of our Haskell implementation
and section 6 discusses future work.

% section introduction (end)

\section{Background} % (fold)
\label{sec:background}

\subsection{STM and TMI} % (fold)
\label{sub:stm_and_tmi}

STM provides attractive guarantees for multithreaded software; namely atomicity,
consistency and isolation of specifically marked blocks of code in {\em transactions}.
In general, STM implementations must do so by performing

\begin{mybullet}
    \item careful monitoring of the resources that are accessed within a transaction,
    \item validation of the accesses of concurrent transactions, and
    \item complete rollback of the effects of aborted transactions.
\end{mybullet}

TMI builds on this machinery and allows security enforcement to benefit from
the STM guarantees. TMI helps the programmer to write correct enforcement mechanisms
and simplifies error-handling. In~\cite{tmi} we outline three main benefits of TMI:

\paragraph{Complete mediation.} TMI provides complete mediation by implicitly invoking
the reference monitor before any effects of a transaction are permanently committed. The
reference monitor validation checks
are able to inspect the resource access logs of the STM and may veto the
commit if an application specific policy is violated. 
In general, this requires that STM mechanisms provide
strong atomicity, i.e. resources marked for transactional scrutiny may not be accessed outside
the scope of a transaction.

\paragraph{Freedom from TOCTTOU bugs.} {\em Time of check to time of use} (TOCTTOU) bugs arise
in conventional enforcement mechanisms when interleaved threads may affect the policy decisions
of each other. For example, a thread may make a policy-based decision to allow access to a
certain resource, e.g. reading a memory location. Before that operation is actually performed, 
execution may be preempted by another thread. That thread can change the global state so that the 
policy decision becomes invalid,
e.g. by writing privileged information into the memory location.

This problem is implicitly solved by using STM, which guarantees that transactions are 
isolated and cannot affect the policy decisions of each other.

\paragraph{Simplified error handling.} In the event of an authorization failure, TMI uses the STM
facilities to completely roll back the effects of the transaction in question and raise an appropriate
exception to the code that initiated the transaction. This frees the programmer from having to
undo state changes leading up to the unauthorized operation, a common source of 
errors~\cite{errorHandlingMistakes}.

% subsection stm_and_tmi (end)

\subsection{Haskell STM} % (fold)
\label{sub:haskell_stm}

For a formal treatment, we build our semantics and implementation on those of the Haskell STM~\cite{haskellstm},
which in turn is built on Concurrent Haskell~\cite{concurrenthaskell}. Concurrent Haskell is an extension to Haskell 98,
a lazy (i.e. call-by-name), pure, functional language. It supports concurrent threads and communications
between them. Non-pure computations are modelled with {\em monads}~\cite{monads}; this includes computations
with side-effects such as input/output and mutable state.

The main entry to a Haskell program is an instantiation of the I/O monad, i.e. a value that represents an
{\em action} of the type \lstinline+IO ()+. An action of this type can, in addition to performing pure
computation, perform other I/O actions by way of composing smaller actions into larger ones. For an example,
Haskell standard libraries define the basic I/O actions \lstinline+getChar+ and \lstinline+putChar+, which
read from standard input and write to standard output, respectively. The most common composition is simple
sequencing. For example the composed I/O action
\begin{lstlisting}[style=small]
main = do { c <- getChar; putChar c; putChar c }
\end{lstlisting}
defines an action that, when executed, will perform the three actions listed in sequence.

In general a value of type \lstinline+IO a+ represents an action that when executed, may perform some
I/O operations as defined by the Haskell libraries and then result in a value of type \lstinline+a+.
Pure functions cannot execute such actions without jeopardizing their purity and this is neatly enforced
by the Haskell type system. Naturally, I/O actions are however free to run pure computations. Thus
the only way to get at the value of an action is if it is a part of a bigger I/O action. The Haskell
runtime bootstraps the whole process by executing the special I/O action called \lstinline+main+.

In addition to conventional input and output, I/O actions can perform reads and updates of mutable memory
cells. The type \lstinline+IORef a+ represents a mutable cell that contains a value of type \lstinline+a+. 
Haskell provides the basic I/O actions \lstinline+newIORef+, \lstinline+readIORef+ and \lstinline+writeIORef+
for manipulation of such cells. As with other I/O actions, these operations can only be used when composing
larger I/O actions.

Concurrent Haskell supports explicit forking of threads through the I/O action \lstinline+forkIO+.
\begin{lstlisting}[style=small]
forkIO :: IO a -> IO ThreadID
\end{lstlisting}
\lstinline+forkIO+ takes another I/O action as a parameter and spawns a new thread to execute the action, 
immediately
returning a newly allocated thread identifier. For further discussion of concurrency we refer to~\cite{awkward}
or tutorials such as~\cite{partutorial}.

The Haskell STM is based on a monadic type similar to the one for I/O actions, namely \lstinline+STM a+.
A value of this type represents an {\em STM action}, which when executed may perform smaller STM actions and
result in a value of type \mbox{\lstinline+a+.} %  dot falls on a second line. I tried mbox (dunno if it works)
STM actions may contain pure computations as well, but note that
they {\em cannot} contain e.g. I/O actions. The main STM actions provided are actions that allow manipulations
of another kind of memory cells, which have the type \mbox{\lstinline+TVar a+}, % This breaks accross lines, tried mbox
where \lstinline+a+ is the type of
value that the cell holds. The actions are \lstinline+newTVar+, \lstinline+readTVar+ and \lstinline+writeTVar+,
so they have the same power as their I/O counterparts. The important thing to note is that sets of \lstinline+IORef+s
and \lstinline+TVar+s are kept separate; one can only be used in I/O actions and the other in STM actions.

STM actions can be composed. Similar to I/O actions, the most common composition is sequencing,
but in addition Haskell STM provides the basic STM action \lstinline+retry+ and a combinator
\lstinline+orElse+. The action \lstinline+retry+ is a blocking operation for 
STM actions, which restarts the current transaction with potentially updated \lstinline+TVar+ contents. 
By issuing \lstinline+retry+, the programmer is stating
that the current transaction cannot finish for the state of \lstinline+TVar+s it started in. 
The Haskell STM provides an optimized implementation of \lstinline+retry+. This implementation captures
the set of \lstinline+TVar+s that a transaction has read before the retry, and suspends the transaction
until at least one of those \lstinline+TVar+s has been updated.
This makes sense because the {\em only} outside factors that can
affect the execution of an STM action are the values of the \lstinline+TVar+s it reads.

If \lstinline+t1+ and \lstinline+t2+ are STM actions, then \lstinline+t1 `orElse` t2+
is an STM action that first tries performing \lstinline+t1+ on its own. 
If \lstinline+t1+ invokes the \lstinline+retry+ action,
then the combined action rolls back the effects of \lstinline+t1+ 
and tries \lstinline+t2+ instead. If that one retries also, the whole
action retries, but waits for updates on the variables read by {\em both} \lstinline+t1+ and \lstinline+t2+.

For an example how the above can be used for synchronization primitives such as communication channels, see Section
4 of~\cite{haskellstm}.

While the basic STM actions and their compositions give us a way to build larger STM actions, we have not
discussed how those actions can be run or how they relate to transactions. For this, STM Haskell provides us
with the \lstinline+atomically+ function,\footnote{While~\cite{haskellstm} uses the name \lstinline+atomic+, the actual implementation
of the Glasgow Haskell Compiler uses \lstinline+atomically+.} whose type is
\begin{lstlisting}[style=small]
atomically :: STM a -> IO a
\end{lstlisting}
This function gives an I/O action that, when performed, will execute the input STM action.
The atomicity comes from the fact that STM Haskell will guarantee that the effects that the STM action has on
\lstinline+TVar+s are atomic, i.e. they all become visible at once and that what happens
inside the STM action is not affected by concurrent threads.  

The Haskell STM system does this by monitoring concurrent invocations of STM actions, taking care of rolling
them back if they conflict with each other and retrying them.
As an example, the following program creates
a transactional variable holding a counter and spawns three threads that each increments the counter atomically.
\begin{lstlisting}[style=small]
increment :: TVar Int -> STM ()
increment counter = do x <- readTVar counter
                       writeTVar counter (x + 1)

main = do c <- atomically (newTVar 0)
          forkIO (atomically (increment c))
          forkIO (atomically (increment c))
          forkIO (atomically (increment c))
\end{lstlisting}
The \lstinline+increment+ action is a classical example of where a race condition might occur 
in a traditional setting,
but in our situation the STM system will guarantee the atomicity of each invocation.

% subsubsection haskell_stm (end)

% section background (end)

\section{Transactional Memory Introspection} % (fold)
\label{sec:transactional_memory_introspection}

In this section we give an overview of the TMI architecture and how it is implemented. We then describe
our Haskell implementation from a user standpoint.

\subsection{Overview of TMI} % (fold)
\label{ssub:overview_of_tmi}

As described in our previous work~\cite{tmi}, the TMI architecture aims to  raise the level
of abstraction in the implementation of security enforcement mechanisms. It allows the programmer
to decouple application logic from security enforcement. Just as STM frees the programmer from worrying
about lock acquisition order and other synchronization efforts, TMI can be used to eliminate concerns about check
placement, race conditions and exceptional execution paths.

TMI provides these guarantees by imposing on the STM system. 
The programmer marks certain variables as security
sensitive. This implicitly indicates to the STM system that these variables are shared,
and ensures that the STM system will protect against race conditions in accesses to the variables.
TMI enhances the monitoring of these security sensitive variables
by ensuring that an access-control reference monitor is invoked every time
that the variables are accessed.

\paragraph{\emph{Time of policy evaluation with TMI:}} % Changed proposed header
% Didn't apply changes in below para
The TMI architecture only loosely constrains when policy must be evaluated,
and in~\cite{tmi} we consider a number of alternatives. In particular, TMI enforcement
can be \emph{eager} or \emph{lazy}.
With {\em eager} enforcement,
every access to a variable triggers the reference monitor, which immediately checks it against the
relevant policy. If authorization is denied, the transaction is immediately aborted. With lazy enforcement,
accesses to variables are simply logged (often they are already logged by the STM) and 
the logs are inspected by the reference monitor only at the end
of the transaction. If any of the logged accesses
are invalid, the whole transaction is aborted. 

% Applied some, not all, changes in below para
A key property of TMI enforcement is that policy decisions
can be evaluated at any time, as long as they are evaluated in a serialized fashion,
and evaluation is fully complete before the transaction commits.
A good STM system will ensure that each transaction is executed in isolation,
such that aborting one will have the same semantics as not having started it. 
This said, for our formal treatment and Haskell implementation, we focus 
on lazy enforcement only. Thus, the following
discussion only deals with the lazy variant unless otherwise noted.

\paragraph{\emph{Utilizing TMI enforcement:}}  % Changed proposed header
% Didn't apply changes in below para
To use TMI, the programmer declares a set of variables as security relevant.
This implicitly indicates to the underlying STM system
that those variables should be protected against race conditions.
This means the values held by these variables can only
be read or modified within a transaction, and that the STM system takes care of resolving conflicting accesses
by concurrent transactions. 
This also means that, upon every variable access,
TMI appends information identifying the variable in 
question to a transaction-specific {\em introspection log}. 
In particular, the introspection log will contain information about the creation, reading, and writing
of the security sensitive variables.

% Changed below para
In addition, TMI requires that all sections
of code that access security-sensitive variables
must be explicitly marked as \emph{atomic}.
To execute such atomic code sections,
programmers initiate a TMI transaction 
and provide a reference to the atomic block and a
{\em security manager}. The security manager is a block of code (or closure) that 
encodes the intended, application-specific security policy, 
and is able to determine
whether a transaction introspection log
satisfies the security policy.
The security manager closure includes the active principal, and other  auxiliary information that is 
needed to check policy compliance.

% Changed below para
% NOTE: I don't have a good spell checker, so please spell check.
Finally, transaction commit plays a special role in TMI enforcement.
TMI runs atomic blocks as transactions in the underlying STM system,
but changes the semantics of transaction commit.
After a TMI atomic block has finished execution, but before it is committed,
TMI ensures that the security manager has fully evaluated whether the transaction introspection log 
complies with the intended security policy.
Importantly, this evaluation occurs within the same STM transaction
as the execution of the atomic block,
and a commit of the transaction is attempted only if 
the security manager returns success.

Even when the transaction has complied with the security policies,
the attempted commit may still fail,
and the transaction is retried,
in the case when the STM system 
finds conflicting concurrent accesses.
Also,
if the security manager finds the transaction in violation of policy,
all state changes are rolled back---including changes to the security manager state,
in the case of history-based policies---and,
instead of retrying the transaction,
an exception is raised to the invoker of the atomic block.

\paragraph{\emph{A simple example:}} 
The following pseudo code shows what software
that makes use of TMI-based security enforcement might look like.
(Note that the code makes use of function-argument currying.)
\begin{lstlisting}[style=small,language=pseudo]
declare sensitive accounts = array of Account

function withdraw(account, amount):
    account.balance = account.balance - amount

function security_manager(user, log):
    if log contains <withdrawal from account>:
        if account.owner == user:
            return Allowed
    return Denied

main program:
    user = aquire_login_credentials()
    try:
        transaction with security_manager(user):
            withdraw(get_account(123456), 42)
    catch AuthorizationFailed:
        tell user about error
\end{lstlisting}
In this code, two aspects are especially noteworthy. First, 
security enforcement code is completely decoupled from the application logic
and the function \lstinline+withdraw+ performs no authorization. 
Even so, complete mediation is ensured,
since the introspection log is implicitly updated by the TMI reference monitor
upon each access to account variables.

Second, in the case of 
authorization failure (e.g. a withdrawal from a different user's account), the error handler 
need only consider how to indicate the error to the user.
The error handler need not clean up any mess:
the state changes that happened during the
transaction (if any)
have already been rolled back when the error handler starts execution. 
Although perhaps not apparent in this simplified example, 
there is ample evidence that writing correct cleanup code is difficult,
especially when multiple security-relevant operations are involved~\cite{errorHandlingMistakes}.

While the above observations form the two main benefits of the TMI architecture,
the third is freedom from TOCTTOU bugs.
Without TMI,
this example code 
might suffer from 
TOCTTOU race conditions,
e.g., if accounts could change owners. 
However, with TMI,
such account-ownership changes would be isolated,
and a transaction 
would be guaranteed to see the same owner throughout its execution.

% subsubsection overview_of_tmi (end)

\subsection{TMI in Haskell} % (fold)
\label{ssub:tmi_in_haskell}

We saw earlier how Concurrent Haskell uses the type system to confine operations on shared variables to
STM actions, and provides a single function to wrap STM actions into an atomic I/O action.
For TMI, we do something very similar. We confine operations on security sensitive variables to {\em TMI
actions}, and provide a single function to turn a TMI action into an STM action and associating it with a
security manager at the same time. 

Figure~\ref{fig:syntax2} shows the extensions of STM Haskell with the TMI extensions (highlighted).
We define a new monad that represents TMI actions and operations on sensitive variables. In addition,
we lift all standard STM functions to their TMI counterparts. This is done so that an existing Haskell STM program can
be easily adapted to TMI with minimal changes to their code. The TMI monad also encapsulates state, namely the
introspection log of a transaction. The introspection log contains entries which specify the access type (create,
read or write) of a variable and the {\em security descriptor} of a variable. Security descriptors are provided by the
programmer when she creates sensitive variables and contain the metadata about the variable 
that is necessary for authorization,
such as the owner of an account, permissions of a file, etc.

\begin{figure}
\flushleft{\begin{minipage}{\linewidth}
\small
\color{old}
\begin{gather*}
\begin{array}{rcl}
    x,y & \in & Variable \\
    r,t & \in & Name \\
    c & \in & Char \\
\\
V & ::= & r \alt c \alt \backslash x\prg{->}M \\
& | & \prg{return }M \alt M \,\prg{>{}>=}, N \\
& | & \prg{putChar }c \alt \prg{getChar} \\
& | & \prg{throw }M \alt \prg{catch }M\;N \\
& | & \prg{retry} \alt M \;\textrm{\lstinline{`orElse`}}\; N \\
& | & \prg{forkIO }M \alt \prg{atomically }M \\
& | & \prg{newTVar }M \\
& | & \prg{readTVar } r \alt \prg{writeTVar }r\;M \\
& | & \new{ \prg{newTMIVar }N\;M } \\
& | & \new{ \prg{readTMIVar } r } \alt \new{ \prg{writeTMIVar }r\;M } \\
& | & \new{ \prg{authorized }N\;M } \alt \new{ \prg{liftSTM } M} \\
& | & \new{ \prg{getlog} } \alt \new{ \prg{UnauthorizedError} } \\
% \alt \new{ \prg{abort} }\\
\\
M,N & ::= & x \alt V \alt M\;N \alt \dots \\
\end{array}
\end{gather*}
\end{minipage}}\caption{Syntax of values ($V$) and terms ($N,M$)}
\label{fig:syntax2}
\end{figure}



Since the type of security descriptors is application specific, our new monad type
is polymorphic,
\begin{lstlisting}[style=small]
TMI d a
\end{lstlisting}
where \lstinline+d+ is the type of descriptors and \lstinline+a+ is the type returned by the action. For
security sensitive variables, we have a type similar to \lstinline+IORef a+ and \lstinline+TVar a+,
\begin{lstlisting}[style=small]
TMIVar d a
\end{lstlisting}
An instance of this type is a cell with a security descriptor of type \lstinline+d+ and a value of type \lstinline+a+.
While the value can change over time, the security descriptor is specified when the cell is created and cannot
change after that. Creation, reading and writing of cells is performed with the following set of functions.
\begin{lstlisting}[style=small]
  newTMIVar :: d -> a -> TMI d (TMIVar d a)
 readTMIVar :: TMIVar d a -> TMI d a
writeTMIVar :: TMIVar d a -> a -> TMI d ()
\end{lstlisting}

For an example, the following code defines a descriptor type for a bank account. The account itself is represented
by a simple integer.
\begin{lstlisting}[style=small]
-- Security descriptor for accounts
data AccountDescr = AccountDescr {
    acctOwner  :: String,
    acctNumber :: Int
}
type Account = TMIVar AccountDescr Int

createAccount :: String -> Int -> Int 
                               -> TMI Account
createAccount owner number balance = 
    newTMIVar (AccountDescr owner number) balance
\end{lstlisting}
The next function demonstrates reading and writing of the security-relevant account variables.
\begin{lstlisting}[style=small]
deposit :: Account -> Int -> TMI AccountDescr ()
deposit acct amount = 
    do balance <- readTMIVar acct
       writeTMIVar acct (balance + amount)
\end{lstlisting}

To turn a TMI action into an STM action, we need to associate it with a security manager, i.e. a boolean function
that evaluates the transaction introspection log of security-relevant accesses and determines if the transaction should be aborted. As an input
to this function, TMI defines the type of an introspection log.
\begin{lstlisting}[style=small]
data AccessType = CreateVar | ReadVar | WriteVar
type TMILog d = [(AccessType, d)]
\end{lstlisting}
To specify the application specific policy, the programmer must supply the security manager, a function
of the type \lstinline+TMILog d -> Bool+. This function, along with a TMI action is passed to the
\lstinline+authorized+ function. The simplest security manager is one that performs no authorization and
simply allows all operations.
\begin{lstlisting}[style=small]
allowAll :: forall d. TMI d a -> STM a
allowAll tx = authorized (const True) tx
\end{lstlisting}
A slightly more complex example is a security manager that looks at all \lstinline+Account+s touched
by a transaction and verifies that they belong to the current user. The current user is passed to the
security manager as the first argument, and this currying ensures 
that we satisfy the type required by \lstinline+authorized+.
% NOTE: the comment in the code said ``Defined my the TMI module'' and I changed to ``by''
\begin{lstlisting}[style=small]
auth :: String -> TMILog AccountDescr -> Bool
auth user thelog = all checkowner thelog
    where
      checkowner :: (AccessType, AccountDescr) 
                 -> Bool
      checkowner (_,descr) = 
            user == (acctOwner descr)

-- Defined by the TMI module:
-- authorized :: (TMILog d -> Bool) 
--            -> TMI d a 
--            -> STM a

main = 
  do acct <- atomically (allowAll mkAccount)
     atomically (doDeposit acct "alice")  -- OK
     atomically (doDeposit acct "bob")    -- FAILS
  where
     mkAccount = createAccount "alice" 123456 0
     doDeposit acct user =
         authorized (auth user) (deposit acct 42)
\end{lstlisting}

Since TMI actions are ultimately executed as STM actions, we also provide a lifting operation to lift
STM operations into TMI operations, \lstinline+liftSTM+. 
This allows for the embedding of an STM action inside a TMI action.
Once the TMI action is turned into an STM action via \lstinline+authorized+, the embedded action is
just composed with it in the normal way. An interesting effect of this is that it allows for nested
calls to \lstinline+authorized+. While this might cause ambiguity for other implementations,
in this Haskell-based implementation such nesting has 
clear and well-defined semantics, and can therefore be permitted.
In fact, we will make explicit use of such nesting in the following sections to
implement {\em privilege amplification}.
% �etta h�r a� ofan er eitt sem v�ri f�nt a� s�na � merkingafr��inni.
% Ef �etta er svona s�per clear, �� g�tum vi� s�nt �a� og gert a� s�luv�ru.
% Eins og �a� er n�na, �� er �etta ekkert s�per clear fyrir mig, btw.

% Hvar eru orElseTMI og vinir � k��anum?  Eiga �eir ekki a� vera neinsta�ar?
TMI actions are also composable in the same way STM actions are. This means the monadic bind acts as
sequential composition and we provide \lstinline+orElseTMI+ and \lstinline+retryTMI+ that behave as
their STM counterparts. When TMI actions composed with \lstinline+orElseTMI+ are turned into STM actions,
via \lstinline+atomically+, the security manager only sees the log entries for \lstinline+TMIVar+-actions
that are actually committed or could have affected the committed actions.
% subsubsection tmi_in_haskell (end)

% section transactional_memory_introspection (end)

\section{Formal semantics of TMI} % (fold)
\label{sec:formal_semantics_of_tim}

\begin{figure}
\flushleft{\begin{minipage}{\linewidth}
    \small
\color{old}
\vspace{-1.5ex}
\begin{gather*}
\begin{array}{rrcl}
\textrm{Thread soup} & P,Q & ::= & M_t \alt (P\,|\,Q) \\
\new{\textrm{Descriptors}} & \new{D_\bot} & \new{::=} & \new{M \cup \{\bot\}} \\
\textrm{Heap} & \Theta & ::= & r \hookrightarrow M \new{\,\times\, D_\bot} \\
\textrm{Allocations} & \Delta & ::= & r \hookrightarrow M \new{\,\times\, D_\bot} \\
\new{\textrm{Access types}} & \new{T} & \new{::=} & \new{\{ \textsc{create}, \textsc{read}, \textsc{write} \}} \\
\textrm{\new{Log}} & \new{\Sigma} & \new{::=} & \new{\textrm{list monoid } ([], \oplus) \textrm{ over } T\times D} \\
\\
\textrm{Evaluation} & \EE & ::= & [\cdot] \alt \EE \,\prg{>{}>=}\,M \alt \prg{catch }\EE\;M \\
\textrm{contexts} & \SS & ::= & [\cdot] \alt \SS \,\prg{>{}>=}\,M \\
                  & \PP & ::= & \SS_t \alt (\PP|P) \alt (P|\PP) \\
\textrm{Action} & a & ::= & \prg{!}c \alt \prg{?}c \alt \epsilon
\end{array}
\end{gather*}
\end{minipage}}\caption{Program state and evaluation contexts}
\vspace{2ex}
\label{fig:contexts}
\end{figure}




To formalize the semantics of TMI, we build on the semantics for the Haskell STM
presented in~\cite{haskellstm}. The semantics is a structural operational semantics
in the style of Plotkin~\cite{Plotkin04a}. 
For the sake of completeness and to help the reader understand our extensions,
we give a cursory
explanation of the concepts of the semantics from~\cite{haskellstm} 
so that a reader not familiar with it may understand our extensions.

Figures~\ref{fig:sosadmin} through \ref{fig:sos2} give the operational rules that describe the steps a program may
take. At the top level, a program transforms a state of the form $P;\Theta$ via labelled transition.
\[
    P;\,\Theta \stackrel{a}{\tarrow} Q;\,\Theta'
\]
$P$ represents a program term in the syntax of Figure~\ref{fig:syntax2} while $\Theta$ stands for a
memory store, a partial function from variable names to annotated terms. An annotated value is a tuple
$(t,d)$ where $t$ is a program term and $d$ is a value that holds the security relevant description of
the relevant variable. The labels on transitions represent the programs input and output actions. $Q$
and $\Theta'$ represent the term that is left unevaluated and the updated store after a transition, 
respectively.

To model atomicity of transactions, separate relations represent the top level I/O transitions
and the STM actions. We extend this by adding a third relation representing the security relevant TMI
actions. Furthermore we add a simple relation for evaluation of security managers under the context of
an immutable transaction log.

Execution of a program proceeds by
non-deterministically
picking a program term from a collection of terms, each representing
a separate thread of execution. One I/O transition of this term combined with the current store is performed
and then the process is repeated. This models interleaved concurrency at the level of I/O transitions.
STM transitions however can only be performed as a required premise of the \lstinline+atomically+ operator
at the I/O level, and thus appear in this model as a single atomic step.

As mentioned in~\cite{haskellstm} there is no need to represent rollback, but contrary to the semantics
in that paper,
our extensions do need to formalize the notion of the transaction log as it is no longer purely an 
implementation detail. For simplicity though, we only model the log for security sensitive operations as
they are the only ones relevant to the semantics of TMI.

\begin{figure}[t]
\flushleft{\begin{minipage}{\linewidth}
    \small
\color{old}

\center{\framebox{Administrative transitions $\qquad M \tarrow N $}}
\begin{gather*}
    M \quad\tarrow\quad V  \quad \textrm{if $\,\mathcal{V}\lsem M \rsem = V$ and $M \nequiv V$}
    \qquad\soslbl{EVAL}
\\
\begin{array}{rcll}
\prg{return }N \,\prg{>{}>=}\, M
    & \tarrow
    & M\;N
    & \soslbl{BIND}
    \\
\prg{throw }N \,\prg{>{}>=}\, M
    & \tarrow
    & \prg{throw }N
    & \soslbl{THROW}
    \\
\prg{retry >{}>=}\, M
    & \tarrow
    & \prg{retry}
    & \soslbl{RETRY}
%\prg{catch retry }N
%    & \tarrow
%    & \prg{retry}
%    & \soslbl{CATCH3}
%    \\
%\prg{abort }N\,\prg{>{}>=}\, M
%    & \tarrow
%    & \prg{abort }N
%    & \soslbl{ABORT}
%    & &
%\prg{catch }(\prg{abort }M)\,N
%    & \tarrow
%    & \prg{abort }M
%    & \soslbl{CATCH4}
\end{array}
%
\end{gather*}

\rule{0.95\linewidth}{0.33pt}

\center{\framebox{I/O transitions $\qquad P;\,\Theta \stackrel{a}{\tarrow} Q;\,\Theta'$}}
\begin{gather*}
\begin{array}{rcll}
    \PP[\prg{putChar } c];\, \Theta & \stackrel{!c}{\tarrow}
                                       & \PP[\prg{return ()}];\, \Theta
                                       & \soslbl{PUTC} \\
    \PP[\prg{getChar }];\, \Theta& \stackrel{?c}{\tarrow}
                                       & \PP[\prg{return } c];\, \Theta
                                       & \soslbl{GETC} \\
    \PP[\prg{forkIO } M];\, \Theta& \tarrow
                                       & (\PP[\prg{return } t] \,|\, M_t);\, \Theta
                                            \quad t\notin \PP, \Theta, M
                                       & \soslbl{FORK} \\
    \PP[\prg{catch }(\prg{return }M)\, N];\,\Theta& \tarrow 
                                       & \PP[\prg{return } M];\,\Theta
                                       & \soslbl{CATCH1} \\
    \PP[\prg{catch }(\prg{throw }M)\, N];\,\Theta& \tarrow 
                                      & \PP[N\,M];\,\Theta
                                      & \soslbl{CATCH2}
\end{array}
\\
\sos{M \tarrow N}
    {\PP[M];\, \Theta \tarrow \PP[N];\, \Theta}
\quad\soslbl{ADMIN}
\\
%
% <<< Atomic
%
\sos{M;\,\Theta,\emptyset
     \stackrel{*}{\tdarrow}
     \prg{return }N;\, \Theta',\Delta'}
   {\PP[\prg{atomically }M];\,\Theta \tarrow
    \PP[\prg{return }N];\,\Theta'}
  \quad\soslbl{ARET}
\\
\sos{M;\,\Theta,\emptyset
     \stackrel{*}{\tdarrow}
     \prg{throw }N;\, \Theta',\Delta'}
   {\PP[\prg{atomically }M];\,\Theta \tarrow
    \PP[\prg{throw }N];\,\Theta\cup\Delta'}
  \quad\soslbl{ATHROW}
%\\
%\sos{M;\,\Theta,\emptyset
%     \stackrel{*}{\tdarrow}
%     \prg{abort }N;\, \Theta',\Delta',\Sigma'}
%   {\PP[\prg{atomically }M];\,\Theta \tarrow
%    \PP[\prg{throw }N];\,\Theta\cup\Delta'}
%  \quad\soslbl{ATHROW1}
% >>>
\end{gather*}

\end{minipage}}\caption{Evaluation of terms and monad operations and IO actions}
\label{fig:sosadmin}
\end{figure}



\subsection{Syntax, states and evaluation contexts} % (fold)
\label{sub:syntax}

The syntax of terms for a subset of STM Haskell is given in Figure~\ref{fig:syntax2} with our TMI-related
extensions (highlighted). Terms and values are standard except that the application of some monadic
operators are considered values, a technique again lifted from~\cite{haskellstm}. The \lstinline+do+-notation
used up until now is standard syntactic sugar for the monad bind and return operations.

\vspace{0.5em}
\begin{tabular}{rcl}
    \lstinline[mathescape=true]+do {$x$<-$e$; $Q$}+ & $\equiv$ &
    \lstinline[mathescape=true]+$e$ >>= ($\backslash x$ -> do {$Q$})+ \\

    \lstinline[mathescape=true]+do {$e$; $Q$}+ & $\equiv$ &
    \lstinline[mathescape=true]+$e$ >>= ($\backslash$_ -> do {$Q$})+ \\

    \lstinline[mathescape=true]+do {$e$}+ & $\equiv$ &
    \lstinline[mathescape=true]+$e$+
\end{tabular}
\vspace{0.5em}

Figure~\ref{fig:contexts} defines some symbols used in the semantics. The metavariable $D$ represents a set
of terms used to describe the security properties of variables. We extend this set with an invalid value $\bot$ and
write $D_\bot$ for the extended set. A state of a computation is a pair $(M,\Theta)$ of a term that remains
to be evaluated and a store $\Theta$. The store maps variable names to terms and their variable descriptor. If
a variable does not have a suitable descriptor, we use $\bot$ as a fill-in.

The set of access types, $T$, consists of three constants, each
representing an operation performed on variables. An introspection log $\Sigma$
is a list monoid of pairs $(t,d)$ where $t$ is an access type and $d$ is a descriptor term; we use $[]$ for the empty
list and $\oplus$ for concatenation, and in the semantics we use $[\cdot]$ as a constructor.
Other symbols are conventional and taken from~\cite{haskellstm}.

For a (partial) function $f$ whose co-domain is a cross-product of two or more sets, and an integer $i$,
we write $f_i$ instead of $\pi_i \circ f$ where $\pi_i$ is the standard $i$-th projection function.
For convenience, we introduce the following notation for filtering logs. If $\Delta$ is a store and $\Sigma$ is
an introspection log, we define the \emph{$\Delta$-restriction of $\Sigma$}, indicated by $\Sigma |_\Delta$, thus
\[
    \begin{array}{rl}
        [] |_\Delta &= [] \\
        ([(t,d)] \oplus \Sigma')|_\Delta &=
        \begin{cases}
            [(t,d)] \oplus \Sigma'|_\Delta & \textrm{if $d \in \img(\Delta_2)$} \\
            \Sigma'|_\Delta & \textrm{otherwise}
        \end{cases}
    \end{array}
\]
Intuitively, $\Sigma|_\Delta$ is the list of entries from $\Sigma$ which apply to variables defined by $\Delta$,
where variables are identified by their security descriptors.

Interleaving of operations is modelled with the evaluation context $\mathbb{P}$, often referred to as
a {\em thread soup}. Through this evaluation context the semantics can non-deterministically choose a
term for reduction from the parallel construct, each term representing a thread. Haskell terms are usually
reduced according to the evaluation context $\mathbb{E}$, which allows for reductions of the right
hand side of the \lstinline+>>=+ operator as well as within the body of a \lstinline+catch+ term. However,
since we want to handle exceptions in a specific manner for STM and TMI actions, we will use the
simpler context $\mathbb{S}$ which requires the operational semantics rules to specify explicitly
how \lstinline+catch+ terms are handled.

% subsection syntax (end)

\begin{figure*}
\flushleft{\begin{minipage}{\textwidth}
    \small

\color{old}

\center{\framebox{STM transitions $\qquad M;\,\Theta, \Delta\new{,\Sigma}
                                 \tdarrow N;\,\Theta',\Delta'\new{,\Sigma'}$}}
\begin{gather*}
\begin{array}{rclll}
%
   \SS[\prg{readTVar } r];\,\Theta,\Delta\new{,\Sigma}
 & \tdarrow
 & \SS[\prg{return }\Theta_1(r)];\,\Theta,\Delta\new{,\Sigma}
 & \\ & & \hspace{1cm}
   \textrm{if $r\in \dom(\Theta)$ \new{and $\Theta_2(s) = \bot$}}
   \hspace{.7cm} \soslbl{READ} \\[.7em]
%
   \SS[\prg{writeTVar } r\;M];\,\Theta,\Delta\new{,\Sigma}
 & \tdarrow
 & \SS[\prg{return ()}];\,\Theta[r\mapsto \new{(}M\new{,\bot)}],\Delta\new{,\Sigma}
 & \\ & & \hspace{1cm}
   \textrm{if $r\in \dom(\Theta)$ \new{and $\Theta_2(s) = \bot$}}
   \hspace{.7cm} \soslbl{WRITE} \\[.7em]
%
   \SS[\prg{newTVar } M];\,\Theta,\Delta\new{,\Sigma}
 & \tdarrow
 & \SS[\prg{return }r];\,\Theta[r\mapsto \new{(}M\new{,\bot)}],\Delta[r\mapsto \new{(}M\new{,\bot)}]\new{,\Sigma}
 & \\ & & \hspace{1cm}
   \textrm{$r\notin \dom(\Theta)$}
   \hspace{2.9cm} \soslbl{NEW} \\[.7em]
%
\end{array}
%
\\
\sos{M \tarrow N}
    {\SS[M];\,\Theta,\Delta\new{,\Sigma} \tdarrow \SS[N];\,\Theta,\Delta\new{,\Sigma}}
    \quad\soslbl{AADMIN}
\\[5pt]
\sos{M_1;\,\Theta,\Delta\new{,\Sigma} \stackrel{*}{\tdarrow} 
      \prg{return }N;\,\Theta',\Delta'\new{,\Sigma'}}
      {\SS[M_1 \prg{ `orElse` } M_2];\,\Theta,\Delta\new{,\Sigma} \tdarrow 
      \SS[\prg{return }N];\,\Theta',\Delta'\new{,\Sigma'}} 
    \quad\soslbl{OR1}
\\[5pt]
\sos{M_1;\,\Theta,\Delta\new{,\Sigma} \stackrel{*}{\tdarrow} 
      \prg{throw }N;\,\Theta',\Delta'\new{,\Sigma'}}
      {\SS[M_1 \prg{ `orElse` } M_2];\,\Theta,\Delta\new{,\Sigma} \tdarrow 
      \SS[\prg{throw }N];\,\Theta',\Delta'\new{,\Sigma'}} 
    \quad\soslbl{OR2}
\\[5pt]
\sos{M_1;\,\Theta,\Delta\new{,\Sigma} \stackrel{*}{\tdarrow} 
      \prg{retry};\,\Theta',\Delta'\new{,\Sigma'}}
      {\SS[M_1 \prg{ `orElse` } M_2];\,\Theta,\Delta\new{,\Sigma} \tdarrow 
      \SS[M_2];\,\Theta,\Delta\new{,\Sigma}} 
    \quad\soslbl{OR3}
\\[5pt]
%
% <<< XSTM catch
%
\sos{M;\,\Theta,\emptyset\new{,[]}
     \stackrel{*}{\tdarrow}
     \prg{return }M';\, \Theta', \Delta'\new{,\Sigma'}}
    {\SS[\prg{catch }M\;N];\,\Theta,\Delta\new{,\Sigma} \tdarrow
     \SS[\prg{return }M'];\, \Theta',\Delta\cup\Delta'\new{,\Sigma\oplus\Sigma'}}
    \quad\soslbl{XSTM1}
\\[5pt]
\sos{M;\,\Theta,\emptyset\new{,[]}
     \stackrel{*}{\tdarrow}
     \prg{throw }M';\, \Theta', \Delta'\new{, \Sigma'}}
    {\SS[\prg{catch }M\;N];\,\Theta,\Delta\new{,\Sigma} \tdarrow
     \SS[N \,M'];\, \Theta\cup\Delta',\Delta\cup\Delta'\new{,\Sigma\oplus(\Sigma'|_{\Delta'})}}
    \quad\soslbl{XSTM2}
\\[5pt]
\sos{M;\,\Theta,\emptyset\new{,[]}
     \stackrel{*}{\tdarrow}
     \prg{retry};\, \Theta', \Delta'\new{, \Sigma'}}
    {\SS[\prg{catch }M\;N];\,\Theta,\Delta\new{,\Sigma} \tdarrow
     \SS[\prg{retry}];\, \Theta,\Delta\new{,\Sigma}}
    \quad\soslbl{XSTM3}
\\[5pt]
% >>>
% <<< Authorized
%
\new{
\sos{M;\,\Theta,\emptyset,[]
     \stackrel{*}{\tddarrow}
     \prg{return }M';\, \Theta',\Delta',\Sigma'
   \qquad
   \Sigma' \vdash N \stackrel{*}{\leadsto} \prg{return }N' }
   {\SS[\prg{authorized }N\;M];\,\Theta,\Delta,\Sigma \tdarrow
    \SS[\prg{return }M'];\Theta',\Delta',\Sigma}
  \quad\soslbl{AURET1}
}
\\[5pt]
\new{
\sos{M;\Theta,\emptyset,[]
     \stackrel{*}{\tddarrow}
     \prg{throw }M';\, \Theta',\Delta',\Sigma'
   \qquad
   \Sigma' \vdash N \stackrel{*}{\leadsto} \prg{return }N' }
   {\SS[\prg{authorized }N\;M];\,\Theta,\Delta,\Sigma \tdarrow
    \SS[\prg{throw }M'];\,\Theta',\Delta',\Sigma}
  \quad\soslbl{AUTHROW1}
}
\\[5pt]
\new{
\sos{M;\,\Theta,\emptyset,[]
     \stackrel{*}{\tddarrow}
     \prg{return }M';\, \Theta',\Delta',\Sigma'
   \qquad
   \Sigma' \vdash N \stackrel{*}{\leadsto} \prg{throw }N' }
   {\SS[\prg{authorized }N\;M];\,\Theta,\Delta,\Sigma \tdarrow
    \SS[\prg{throw UnathorizedError}];\Theta',\Delta',\Sigma}
  \quad\soslbl{AURET2}
}
\\[5pt]
\new{
\sos{M;\Theta,\emptyset,[]
     \stackrel{*}{\tddarrow}
     \prg{throw }M';\, \Theta',\Delta',\Sigma'
   \qquad
   \Sigma' \vdash N \stackrel{*}{\leadsto} \prg{throw }N' }
   {\SS[\prg{authorized }N\;M];\,\Theta,\Delta,\Sigma \tdarrow
    \SS[\prg{throw UnauthorizedError}];\,\Theta',\Delta',\Sigma}
  \quad\soslbl{AUTHROW1}
}
\\[5pt]
\new{
\sos{M;\Theta,\emptyset,[]
     \stackrel{*}{\tddarrow}
     \prg{retry};\, \Theta',\Delta',\Sigma'
   }
   {\SS[\prg{authorized }N\;M];\,\Theta,\Delta,\Sigma \tdarrow
    \SS[\prg{retry}];\,\Theta',\Delta'\Sigma}
  \quad\soslbl{AURETRY}
}
% >>>
%
\end{gather*}

\end{minipage}}
\caption{Operational semantics for STM actions}
\label{fig:sos1}
\end{figure*}


\begin{figure*}
\flushleft{\begin{minipage}{\textwidth}
    \small

\center{\framebox{TMI transitions $\qquad M;\,\Theta, \Delta, \Sigma
                                \tddarrow N;\,\Theta',\Delta',\Sigma'$}}
\begin{gather*}
\begin{array}{rcll}
%
   \SS[\prg{readTMIVar } r];\,\Theta,\Delta,\Sigma
 & \!\!\tddarrow\!\!
 & \SS[\prg{return }\Theta_1(r)];\,\Theta,\Delta,\Sigma\oplus[(\textsc{read},\Theta_2(r))]
 & \\ & & \hspace{1cm}
   \textrm{if $r\in \dom(\Theta)$ and $\Theta_2(r)) \ne \bot$}
   \hspace{.7cm} \soslbl{TMIREAD} \\[.7em]
%
   \SS[\prg{writeTMIVar } r\;M];\,\Theta,\Delta,\Sigma
 & \!\!\tddarrow\!\!
 & \SS[\prg{return ()}];\,\Theta[r\mapsto (M,\Theta_2(r))],\Delta,\Sigma\oplus[(\textsc{write},\Theta_2(r))]
 & \\ & & \hspace{1cm}
   \textrm{if $r\in \dom(\Theta)$ and $\Theta_2(r)) \ne \bot$}
   \hspace{.7cm} \soslbl{TMIWRITE} \\[.7em]
%
   \SS[\prg{newTMIVar }N\;M];\,\Theta,\Delta,\Sigma
 & \!\!\tddarrow\!\!
 & \SS[\prg{return }r];\,\Theta[r\mapsto (M,N)],\Delta[r\mapsto (M,N)],\Sigma\oplus[(\textsc{create},N)]
 & \\ & & \hspace{1cm}
   \textrm{$r\notin \dom(\Theta)$}
   \hspace{.7cm} \soslbl{TMINEW} \\[.7em]
%
\end{array}
%
\\
\sos{M \tarrow N}
    {\SS[M];\,\Theta,\Delta\new{,\Sigma} \tddarrow \SS[N];\,\Theta,\Delta\new{,\Sigma}}
    \quad\soslbl{TADMIN}
\\[5pt]
\sos{M;\,\Theta,\Delta,\Sigma \stackrel{*}{\tdarrow} N;\,\Theta',\Delta',\Sigma'}
    {\SS[\prg{liftSTM }M];\,\Theta,\Delta,\Sigma \tddarrow
     \SS[N];\,\Theta',\Delta',\Sigma'}
    \quad\soslbl{LIFTSTM}
\\[5pt]
\sos{M_1;\,\Theta,\Delta\new{,\Sigma} \stackrel{*}{\tddarrow} 
      \prg{return }N;\,\Theta',\Delta'\new{,\Sigma'}}
      {\SS[M_1 \prg{ `orElse` } M_2];\,\Theta,\Delta\new{,\Sigma} \tddarrow 
      \SS[\prg{return }N];\,\Theta',\Delta'\new{,\Sigma'}} 
    \quad\soslbl{TOR1}
\\[5pt]
\sos{M_1;\,\Theta,\Delta\new{,\Sigma} \stackrel{*}{\tddarrow} 
      \prg{throw }N;\,\Theta',\Delta'\new{,\Sigma'}}
      {\SS[M_1 \prg{ `orElse` } M_2];\,\Theta,\Delta\new{,\Sigma} \tddarrow 
      \SS[\prg{throw }N];\,\Theta',\Delta'\new{,\Sigma'}} 
    \quad\soslbl{TOR2}
\\[5pt]
\sos{M_1;\,\Theta,\Delta\new{,\Sigma} \stackrel{*}{\tddarrow} 
      \prg{retry};\,\Theta',\Delta'\new{,\Sigma'}}
      {\SS[M_1 \prg{ `orElse` } M_2];\,\Theta,\Delta\new{,\Sigma} \tddarrow 
      \SS[M_2];\,\Theta,\Delta\new{,\Sigma}} 
    \quad\soslbl{TOR3}
\\[5pt]
%
% <<< XSTM catch
%
\sos{M;\,\Theta,\emptyset\new{,[]}
     \stackrel{*}{\tddarrow}
     \prg{return }M';\, \Theta', \Delta'\new{,\Sigma'}}
    {\SS[\prg{catch }M\;N];\,\Theta,\Delta\new{,\Sigma} \tddarrow
     \SS[\prg{return }M'];\, \Theta',\Delta\cup\Delta'\new{,\Sigma\oplus\Sigma'}}
    \quad\soslbl{XTMI1}
\\[5pt]
\sos{M;\,\Theta,\emptyset\new{,[]}
     \stackrel{*}{\tddarrow}
     \prg{throw }M';\, \Theta', \Delta'\new{, \Sigma'}}
    {\SS[\prg{catch }M\;N];\,\Theta,\Delta\new{,\Sigma} \tddarrow
     \SS[N \,M'];\, \Theta\cup\Delta',\Delta\cup\Delta'\new{,\Sigma\oplus(\Sigma'|_{\Delta'})}}
    \quad\soslbl{XTMI2}
\\[5pt]
\sos{M;\,\Theta,\emptyset\new{,[]}
     \stackrel{*}{\tddarrow}
     \prg{retry};\, \Theta', \Delta'\new{, \Sigma'}}
    {\SS[\prg{catch }M\;N];\,\Theta,\Delta\new{,\Sigma} \tddarrow
     \SS[\prg{retry}];\, \Theta,\Delta\new{,\Sigma}}
    \quad\soslbl{XTMI3}
\end{gather*}

\rule{0.95\linewidth}{0.33pt}

\center{\framebox{Authorization transitions $\qquad \Sigma \vdash M \leadsto N$}}
\begin{gather*}
\sos{M\tarrow N}
    {\Sigma \vdash \EE[M] \,\leadsto\, \EE[N]}
\quad\soslbl{AUADMIN}
\\[0.5em]
\Sigma \vdash \EE[\prg{getlog}] \,\leadsto\, \EE[\prg{return } hs(\Sigma)]
\quad\soslbl{GETLOG}
\\[0.5em]
\Sigma \vdash \EE[\prg{catch }(\prg{return }M)\, N]
    \,\leadsto\,
    \EE[\prg{return } M]
    \quad\soslbl{ACATCH1}
\\[0.5em]
\Sigma \vdash \EE[\prg{catch }(\prg{throw }M)\, N]
    \,\leadsto\,
    \EE[N\,M]
    \quad\soslbl{ACATCH2}
\end{gather*}

\end{minipage}}
\caption{Operational semantics for TMI actions}
\label{fig:sos2}
\end{figure*}


\subsection{Operational semantics} % (fold)
\label{sub:operational_semantics}

Figures~\ref{fig:sosadmin} through \ref{fig:sos2} detail the transition relations of our semantics.
Figures~\ref{fig:sosadmin} and \ref{fig:sos1} are mostly the same as in the semantics
of~\cite{haskellstm}, parts added for TMI are indicated with a darker ink. 
The semantics uses several different
transition systems that are layered such that a sequence of reductions in one layer becomes one reduction
in the next layer above. This makes a sequence of transitions in a lower layer appear as 
one atomic transition at the higher level.
There are three main layers - the top level I/O context, the STM context
and the TMI context. An auxiliary transition system is used to reduce authorization functions.

\paragraph{\emph{Values and I/O transitions:}}
The {\em admin} transitions of Figure~\ref{fig:sosadmin} 
define the evaluation of terms to values via a function $\mathcal{V}$. This
function is standard and its definition omitted here. Administrative transitions also include the
behaviour of the monadic bind operator \lstinline+>>=+.

The top level I/O actions are described
by the labelled $\tarrow$ relation. They 
operate on the $\mathbb{P}$ context, which allows for picking any program
term from the thread soup for reduction. The first two rules are I/O primitives. The rule {\em FORK} is used
to create a new thread and enter it into the thread soup, choosing a fresh thread id $t$.
The rules {\em CATCH1} and {\em CATCH2} 
deal with exception
handling as described in the appendix of the post-publication, extended version of the Haskell
semantics~\cite{haskellstm}.
The {\em ADMIN} rule
allows for lifting of administrative transitions to the I/O transition relation.
This is done to reduce
repetition, as the administrative 
rules also apply to the STM and TMI transition relations, which have a similar
lifting rule.
Finally, the rules {\em ARET} and {\em ATHROW} 
enable the use of the \lstinline+atomic+ combinator to lift a sequence
of reductions in the STM transition relation to a single I/O transition.

If the series of STM reductions results in a \lstinline+return+ value, the effects on the store are retained.
If it however results in an exception (i.e. a \lstinline+throw+ value), the modifications to existing variables
are discarded but any new allocations are retained. This is necessary as the exception value may hold references
to newly allocated variables.

\paragraph{\emph{STM transitions:}}
The STM transitions define the behaviour of STM actions. The states used in these transitions are extended
from the I/O transitions by adding a separate store for new allocations $\Delta$, and an introspection log
$\Sigma$. A transition of the form
\[
M;\,\Theta, \Delta,\Sigma \tdarrow N;\,\Theta',\Delta',\Sigma'
\]
represents a reduction of the term $M$ to the term $N$. Some variables in $\Theta$ may be introduced
or altered to yield $\Theta'$. $\Delta$ is a store similar to $\Theta$, that only tracks newly allocated
variables while $\Sigma$ is a log of accesses to TMI variables. $\Delta$ and $\Sigma$ are transaction
local, i.e. they are reset at the start of each atomic sequence of reductions in the STM system, see
e.g. rule~$\soslbl{ARET}$.

The first three rules define actions on transactional variables. They operate on the store
$\Theta$ but only on those variables where the second component (i.e. the security descriptor) is $\bot$.
This is what differentiates regular transactional variables from security sensitive variables. 

Other rules in the STM transition system define the behavior of STM combinators as described in
Section~\ref{sub:haskell_stm}. We have used the revised semantics of exception handling from the
later versions of~\cite{haskellstm}, namely the rule {\em XTM2} ensure that when a term reduction
results in a caught exception, all effects of that reduction are rolled back except for new 
allocations.

% �g er ekkert rosa happy me� �essi paragraph heading, en �a� ver�ur a� brj�ta eitthva� upp �essa 
% bla�s��u einhvernvegin
\paragraph{\emph{TMI transitions:}}
A key TMI addition to the STM transitions is the handling of \lstinline+authorized+.
The rules {\em AURETn} and {\em AUTHROWn} specify that for a term of the form {\tt (authorized}
$N\;M${\tt )}
if $M$ evaluates to a \lstinline+return+ or \lstinline+throw+ value, then that value is propagated only
if $N$ evaluates to a \lstinline+return+ value under the authorization relation (see later).
If evaluation of the {\em authorization term} $N$ raises an exception, a fixed exception containing
no information about the local transaction state is thrown. This triggers a rollback of any updates
performed during that invocation of \lstinline+authorized+.

We should note that in the case of an exception, including authorization failure, new allocations
are retained for the reason described above. In the implementation, this is not done explicitly
as deallocation of references is handled by the garbage collector. Any new allocations that are
actually referenced by exception values are thus retained, but others are discarded. Thus, since
we don't allow any variable references in our special exception for authorization failures, no
new allocations will leak in practice.

Figure~\ref{fig:sos2} shows the two new TMI transition relations. The first one deals with
TMI actions and is indicated by the symbol $\tddarrow$. The configurations of this transition
system are identical to those of the STM system.
Indeed, TMI actions behave very much like STM actions, the main difference is that variable
operations in TMI actions can operate on security sensitive variables. A variable $v$ is security
sensitive if and only if $\Theta_2(v) \neq \bot$.
The first three rules of
Figure~\ref{fig:sos2} describe the variable operations. 
When these operations are performed, a log entry is added to
$\Sigma$ with the contents of the variable's security descriptor. Another addition over STM
behavior is rule {\em LIFTSTM}. This rule states that any sequence of STM reductions can be lifted
to the TMI level. 
This is necessary to allow TMI code to access regular transactional variables, and should be 
possible, since TMI actions are always performed in the context of an enclosing STM action.

Finally, we add a separate transition system to evaluate authorization functions. In this system
a transition of the form
\[
\Sigma \vdash M \leadsto N
\]
represents the reduction of term $M$ to term $N$, under the context of an introspection log $\Sigma$.
The reason for this notation is that the introspection log is fixed, i.e. read-only, for these
transitions.
This system only allows pure operations and monad binding via the administrative transitions, the
usual exception handling and one special term \lstinline+getlog+. 
The \lstinline+getlog+ term is reduced to a list representation (in the Haskell sense) of the access
log $\Sigma$. The terms reduced with this system can thus examine the log and make decisions based
on its contents.

\paragraph{\emph{An example:}}
As an example of reading and applying the rules, consider the program
\begin{lstlisting}[style=small]
atomically 
  (
     authorized (assert (isEmpty getlog)) 
                (writeTMIVar x 10)
  )
\end{lstlisting}
Working from the inside out, we can see that the innermost expression of \lstinline+writeTMIVar x 10+
will update the value of $x$ in $\Theta$ as well as enter an entry to the introspection log $\Sigma$, by applying
rule $\soslbl{TMIWRITE}$. The resulting term is \lstinline+return ()+. % breaks accross lines
As the resulting log is non-empty, the authorization term 
\lstinline+assert (isEmpty getlog)+ will throw an exception. Thus, for the \lstinline+authorized+
term, rule $\soslbl{AURET2}$ is the only applicable one, so that term evaluates to
\lstinline+throw UnauthorizedError+. The \lstinline+atomically+ term is therefore evaluated to
the same result via rule $\soslbl{ATHROW}$, but this rule does not preserve updates to the store $\Theta$,
meaning that the transaction has been aborted.

\paragraph{\emph{Nested TMI actions:}} As we mentioned in the previous section, the capability
of lifting STM actions up to the TMI levels allows us to nest TMI actions. An inner TMI action
can be authorized with a separate authorization term. Consider the following example of an
action that provides a student with information about her grade for a course, as well as the
average of all grades of other students. 
Naturally, the student doesn't have access to other students' grades
but for the purpose of calculating the average we may allow such access in a nested action.

Assume that we have defined the following terms.
\begin{mybullet}
\item \lstinline+ownGradesRead s+ is an authorization term that succeeds only if
    the input log only contains reading of grades that belong to student \lstinline+s+
\item \lstinline+allGradesRead+ is an authorization term that succeeds only if the log only contains
    reading of grades, but regardless of the owner of the grades. This may be considered
    a kind of a {\em system} read access to grades.
\item \lstinline+readGrade s+ is a TMI action that reads a grade of a student from the
    relevant \lstinline+TMIVar+ and returns it. The introspection log will contain an 
    appropriate entry afterwards.
\item \lstinline+averageGrades+ is a TMI action that reads grades of all students from the
    appropriate \lstinline+TMIVars+ and returns their average. The introspection log will 
    contain an entry for every read grade.
\end{mybullet}
Now it is possible to define the following TMI action that provides a student with her
own grade as well as the average grades of all students.
\begin{lstlisting}[style=small]
gradeInformation :: Student -> TMI (Grade, Grade)
gradeInformation s =
  do own <- readGrade s
     avg <- liftSTM (authorized allGradesRead averageGrades)
     return (own, avg)
\end{lstlisting}
This function may be called with the appropriate authorization function,
namely one that only allows a student access to her own grades.
\begin{lstlisting}[style=small]
atomically (authorized (ownGradesRead s) (gradeInformation s))
\end{lstlisting}

By applying the operational semantics rules to this term, one can find that the innermost action
\lstinline+averageGrades+ will by authorized be \lstinline+allGradesRead+ {\em before}
turning it into an STM transition. This may, for example, happen through rule $\soslbl{AURET1}$.
Note that such a rule does not keep the log entries of already authorized actions, i.e.
the $\Sigma$ is not affected in the $\Rightarrow$ transition below the line. 
Thus the log of the nested
action is not contained in the log authorized by the outer authorization function
\lstinline+ownGradesRead+.

This use of nesting constitutes a {\em privilege amplification} 
in a manner similar to stack inspection~\cite{GordonFournetPOPL}.

% subsection operational_semantics (end)


% section formal_semantics_of_tim (end)

\section{Implementation}
\label{sec:implementation}

Our Haskell implementation is comprised of one module, \verb+TMI+. 
The most important components are the
monad \lstinline+TMI d a+ and the type for TMI variables, \lstinline+TMIVar d a+. 
Both are parameterized on the descriptor type \lstinline+d+, which is chosen by the user
of this module. A TMI variable is represented by a descriptor value and an STM \lstinline+TVar+,
\begin{lstlisting}[style=small]
data TMIVar d a = TMIVar {
      getTVar       :: TVar a,
      getDescriptor :: d
}
\end{lstlisting}
The field accessors \lstinline+getTVar+ and \lstinline+getDescriptor+ are not exported and
only available inside the \lstinline+TMI+ module.

The TMI monad is a stack of the standard {\em writer monad} on top of the regular STM
monad. The writer monad has the ability of collecting accumulating information in a
sequence, which is exactly what we need to maintain the introspection log.
In the TMI module, the regular STM module is imported under the name \lstinline+T+.
\begin{lstlisting}[style=small]
newtype TMI d a = TM {
   unwrapTM :: WriterT (TMILog d) T.STM a
} deriving (Monad)
\end{lstlisting}
The type \lstinline+TMILog d+ represents a log of all accesses to TMI variables.
It is defined by the following declarations.
\begin{lstlisting}[style=small]
data TMIAccess = CreateVar | ReadVar | WriteVar
type TMILog d = [(TMIAccess, d)]
\end{lstlisting}
For inserting entries in the log, we define the following shortcut, where \lstinline+tell+
is the standard function that the writer monad uses to collect information.
\begin{lstlisting}[style=small]
log :: (TMIAccess, d) -> STM d ()
log entry = (TM . tell) [entry]
\end{lstlisting}
\lstinline+log+ returns an STM action which has the
only effect of appending its argument to the introspection log. Note that this
helper is not exported, so users of the TMI module cannot append to the log
directly.
As expected, a log entry of the form \lstinline+(ReadVar, x)+ just means that a variable with descriptor
value \lstinline+x+ was read. 

The \lstinline+liftSTM+ function lifts an STM operation to a TMI operation. The log
is not affected by the work performed in the STM action.
\begin{lstlisting}[style=small]
liftSTM :: STM a -> TMI d a
liftSTM = TM . lift
\end{lstlisting}

The functions to create, read and write TMI variables are now simple to define. They
all enter the relevant entries to the log and then call the underlying functions
from the STM module.
\begin{lstlisting}[style=small]
newTMIVar :: d -> a -> TMI d (TMIVar d a)
newTMIVar description val =
    do log (CreateVar, description)
       var <- liftSTM (T.newTVar val)
       return (TMIVar var description)

readTMIVar :: TMIVar d a -> TMI d a
readTMIVar tv = 
    do log (ReadVar, getDescriptor tv)
       liftSTM (T.readTVar (getTVar tv))

writeTMIVar :: TMIVar d a -> a -> TMI d ()
writeTMIVar tv val = 
    do log (WriteVar, getDescriptor tv)
       liftSTM (T.writeTVar (getTVar tv) val)
\end{lstlisting}

The combinators from the STM world are defined thus.
\begin{lstlisting}[style=small]
retryTMI = liftSTM T.retry

runTMI = runWriterT . unwrapTM   -- helper

orElseTMI t1 t2 = TM . WriterT (runTMI t1 `T.orElse` runTMI t2)
\end{lstlisting}

What is left is to define the crucial \lstinline+authorized+ function. This function
accepts an authorization function with the type \lstinline+TMILog d -> Bool+
and a TMI action; it should return an STM action that performs the operation
of the TMI action, validates the resulting log with the authorization function and
either returns the result or throws an exception. With this description in mind,
the implementation is pretty straight-forward.
\begin{lstlisting}[style=small]
authorized :: (TMILog d -> Bool) -> TMI d a -> T.STM a
authorized auth act =  do
   (result, log) <- runTMI act
   if not (auth log)
      then throw (AssertionFailed "Access denied")
      else return result
\end{lstlisting}
Note that a custom exception may be more appropriate but for sake of clarity
we simply use the standard assertion failure to trigger a transaction abort.

Our Haskell TMI implementation can support variants of
history-based policy enforcement~\cite{HBAC},
in particular
allowing \emph{privilege amplification} with nested TMI actions as described in
the previous section.
%
In particular a call to \lstinline+authorized+ will return
an STM action which {\em contains} a TMI operation and an associated
authorization closure. When
code inside a TMI action needs increased privileges it can nest a call to \lstinline+authorized+
with a different authorization manager and use \lstinline+liftSTM+ to lift the resulting
STM operation back to the TMI level.

As in stack inspection~\cite{GordonFournetPOPL},
privilege amplification provides 
TMI security managers with
a useful 
escape hatch to perform operations as a more powerful ``application principal''.
%
% For instance, we have used it 
% in cases where applications need to compute aggregates, 
% such as in an grade-management application
% that shows a student their grade and a class grade average,
% but prevents access to other student's grades.


\section{Discussion and future work}
\label{sec:conclusion}

We have presented both a formal semantics and an implementation of TMI over the
Haskell STM system. During this work, we discovered that there are many design
decisions to be made and the design we have presented here is only one of many
possibilities. The variants we experimented with in the design process did not
always exhibit the behaviour that we expected or wanted. For this task, defining
the formal semantics proved to be an essential tool to understand and evaluate
different design decisions as well as spotting special cases that were not so
obvious in the actual implementation. 
Indeed, constructing the semantics helped us discover bugs
and unexpected behaviour in the code, even after weeks of careful consideration.
In addition the formal semantics gives a clear and unambiguous
description of the TMI architecture.

%
% What do you mean by 'the design' ?  Do you mean the TMI architecture (which
% we describe and clearly isn't limited in this way), or the formal semantics,
% or the Haskell implementation?  If it is the last (which I think it is), then
% it isn't quite true, since we have the privilege stuff.
%
% Also, this is kind of a downer to end on.  Why not just skip, or say that we would
% like to explore better implementations for stateful policies (although our semantics supports it).
%

The TMI architecture, as described in Section~\ref{ssub:overview_of_tmi}, 
can support the enforcement of stateful security policies that depend on the 
execution history over multiple transactions.  In our paper \cite{tmi}, we have experimented 
with such policies in another implementation of the TMI architecture.  
However, we have not explored the addition of such facilities here, 
in order to simplify the exposition of our semantics and Haskell implementation.

The privilege amplification by nested TMI actions naturally relies on the programmer
to ensure that the nested authorization manager does not violate the enclosing
policy. The objective of the TMI architecture is to provide facilities for writing
policy enforcement code, not to prevent injection of malicious code. In the case of library
development where one deals with untrusted code, such nesting may not be desirable and can
be disabled. We have experimented with other ways of implementing privilege amplification
without relying on this nesting, with good results.
% The policy state must be protected by the STM mechanisms to avoid
% race conditions or interference between threads, but we must take care that the accesses
% to this state are only performed by trusted code, as they cannot be subject to authorization,
% lest we introduce a circular dependency.

In our implementation we are maintaining the introspection log by hand. This works well for
prototypical purposes, but we would like to investigate the possibility of making use
of the real underlying transaction log. This requires modifications to the STM framework
provided in the GHC runtime library. Having the formal semantics as the definition of the desired
behaviour should make such an implementation easier to construct and check.

Future work in this context also involves writing or porting complex software to the
architecture to obtain realistic performance measurements.
Also, our semantics may still be simplified, while allowing the same behavior; 
for example, the transaction log seems redundant in STM transitions, 
and may possibly be eliminated.


Most importantly, we think that having clear semantics for TMI and an implementation
over a production-ready STM system, further validates our claim that TMI architecture
is very relevant to practical software development.



\renewcommand{\jointwith}{Luca Aceto, Anna Ing\'olfsdottir,\\ and MohammadReza Mousavi}
\chapter[Decompositional Reasoning about the History of Parallel Processes]{Decompositional Reasoning \\ about the History of \\ Parallel Processes}
\markboth{\upshape\scshape chapter \thechapter. decompositional reasoning about history}{}
\section{Introduction}
State-space explosion is a major obstacle in model-checking logical properties.
One approach to combat this problem is compositional reasoning,
where properties of a system as a whole are deduced from the properties of its components.
Decompositional reasoning \cite{Giannakopoulou05,Xie05} improves upon compositional reasoning by
automatically decomposing the global property into local properties of (possibly unknown) components.
In the context of process algebras, as the specification language, and Hennessy-Milner logic, as the logical formalism for properties,
decompositional reasoning techniques date back to the seminal work of Larsen and Xinxin in early 90's \cite{Larsen91},
which is further developed in \cite{Simpson04,Fokkink06}.
However, we are not aware of any such decomposition technique which applies to reasoning about ``past''.
This is particularly interesting in the light of recent developments concerning reversible processes and
knowledge representation (epistemic aspects) inside process algebra,
all of which involve some notion of specification and reasoning about past.


In this paper, we tackle this problem and present a decomposition technique for Hennesy Milner logic with past.
As the specification language, we use a subset CCS with parallel composition, non-deterministic choice, action prefixing
and the inaction constant.

The rest of this paper is structured as follows.


\section{Preliminaries}\label{sec:preliminaries}
\subsection{Computations and CCS} % (fold)
The following definitions come mostly from~\cite{DeNicola:1990}.

\begin{definition}[Labelled transition system]
    A \emph{labelled transition system} (LTS)
    is a triple $\langle P,A,\trans{}\rangle$ where
    \begin{itemize}
        \item $P$ is a set of process names.
        \item $A$ is a finite set of action names, not including a \emph{silent action}
              $\tau$. We write $A_\tau$ for $A\cup \{\tau\}$.
        \item $\trans{}\subseteq P\times A_\tau \times P$ is the \emph{transition
              relation}, we call its elements \emph{transitions} and
              usually write $p\trans{\alpha}p'$ to mean that $(p,\alpha,p')\in\trans{}$.
    \end{itemize}
    We let $p,q,\dots$ range over $P$, $a,b,\dots$ over $A$ and $\alpha,\beta,\dots$
    over $A_\tau$.
\end{definition}

\begin{definition}[Sequences and computations]
    For any set $S$ we let $\seq S$ be the set of finite sequences of elements
    from $S$. Concatenation of sequences is represented by juxtaposition.
    $\lambda$ denotes the empty sequence and $\abs\sigma$ stands for the length of a
    sequence $\sigma$.

    Given an LTS $\T = \langle P,A,\trans{} \rangle$, we define a
    \emph{path from $p_0$} to be a sequence of transitions
    \[
        p_0\trans{\alpha_0}p_1,
        p_1\trans{\alpha_1}p_2, \dots ,
        p_n\trans{\alpha_{n-1}}p_n
    \]
    and usually write this as
    \[
        p_0 \trans{\alpha_0} p_1
            \trans{\alpha_1} p_2
            \trans{\alpha_2} \cdots
            \trans{\alpha_{n-1}}p_n.
    \]
    We use $\pi,\mu,...$ to range over paths.
    A \emph{computation from $p$} is a pair $(p,\pi)$ where $\pi$ is a path
    from $p$ and
    we use $\rho,\sigma,\dots$ to
    range over computations.
    $\C_\T(p)$,
    or simply $\C(p)$ when the LTS $\T$ is clear from the context,
    is the set of computations from $p$ and
    $\C_\T$ is the set of all computations in $\T$.

    For a computation $\rho = (p_0,\pi)$
    where $\pi = p_0 \trans{\alpha_0} p_1 \trans{\alpha_1} p_2 \trans{\alpha_2}
    \cdots \trans{\alpha_{n-1}}p_n$ we define
    \begin{align*}
        \first(\rho) &= p_0 \\
        \last(\rho) &= p_n \\
        \trace(\rho) &= (\alpha_0\dots\alpha_{n-1}) \in \seq{A_\tau} \\
        \abs{\rho} &= \abs{\pi} = n
    \end{align*}
    We will refer to  elements of $\seq A$
    as \emph{traces}. 
    %Note that the length
    %of a computation is the length of its trace, not the number of processes.

    Concatenation of computations $\rho$ and $\rho'$
    is denoted by their juxtaposition
    $\rho\rho'$ and is defined iff $\last(\rho)=\first(\rho')$.
    When $\last(\rho) = p$ we will
    write $\rho(p\trans{\alpha} q)$ as a shorthand for the slightly longer
    $\rho(p, p\trans{\alpha} q)$.
    We also use $\rho \trans{\alpha} \rho'$ to denote that
    there exists a computation
    $\sigma = (p, p \trans\alpha p')$, for some processes $p$ and $p'$,
    such that $\rho' = \rho\sigma$.
\end{definition}

\begin{remark}
    Representing computations with a pair $(p,\pi)$ might seem redundant at
    first, since $\pi$ must start with $p$. However, an empty computation
    $(p,\lambda)$ is also valid and must be distinguished from $(q,\lambda)$
    if $p\ne q$.
\end{remark}

\subsection{Hennessy-Milner Logic with Past}


\begin{definition}[Hennessy Milner logic with past]
    Let $\T=\langle P,A,\rightarrow \rangle$ be an LTS. The set $\HMLpast(A)$,
    or simply $\HMLpast$, of
    \emph{Hennessy-Milner logic formulas with past} is defined by the following grammar,
    where $\alpha\in A_\tau$.
    \[
        \phi,\psi \,::=\, \true \alt \phi\land\psi
                                \alt \neg\phi
                                \alt \dmnd{\alpha} \phi
                                \alt \dmndback{\alpha} \phi.
    \]
    We define the \emph{satisfaction relation} $\vDash\,\subseteq \C_\T \times \HMLpast$
    as the least relation that satisfies the following clauses:
    \begin{itemize}
        \item $\rho\vDash\true$ for all $\rho\in\C_\T$,
        \item $\rho\vDash\phi\land\psi$ iff $\rho\vDash\phi$ and $\rho\vDash\psi$,
        \item $\rho\vDash\neg\phi$ iff not $\rho\vDash\phi$,
        \item $\rho\vDash\dmnd{\alpha}\phi$ iff
              $\rho\trans{\alpha}\rho'$ and $\rho'\vDash\phi$ for some $\rho'\in\C_\T$,
        \item $\rho\vDash\dmndback{\alpha}\phi$ iff
              $\rho'\trans{\alpha}\rho$ and $\rho'\vDash\phi$ for some $\rho'\in\C_\T$.
    \end{itemize}
    The satisfaction relation $\vDash\,\subseteq P\times\HMLpast$ is defined by
    $p\vDash\phi$ if and only if $(p,\lambda)\vDash \phi$.
\end{definition}

We will make use of some standard shorthands for Hennessy-Milner type logics.
\begin{align*}
    \false &= \neg\true \\
    \phi \lor \psi &= \neg(\neg\phi \land \neg\psi) \\
    \bx{\alpha}\phi &= \neg\dmnd{\alpha}(\neg\phi) \\
    \bxback{\alpha}\phi &= \neg\dmndback{\alpha}(\neg\phi) \\
\end{align*}
For a finite set of actions $A$, we will also use the following notation.
\begin{align*}
    \dmnd{A}\phi &= \bigvee_{\alpha\in A} \dmnd{\alpha} \phi  &
    \dmndback{A}\phi &= \bigvee_{\alpha\in A} \dmndback{\alpha} \phi \\
    \bx{A}\phi &= \bigwedge_{\alpha\in A} \bx{\alpha} \phi  &
    \bxback{A}\phi &= \bigwedge_{\alpha\in A} \bxback{\alpha} \phi \\
\end{align*}
Intuitively, $\rho\vDash\dmndback{\alpha}\phi$ means the last action of $\rho$
{\em is} labelled with $\alpha$ and the preceding computation satisifies $\phi$, while
$\rho\vDash\bxback{\alpha}\phi$ means that {\em if} the last computation is labelled
with $\alpha$, then the preceding computation satisfies $\phi$. Specifically, a
computation with an empty path, i.e. the initial state, can never satisfy the former
while it will always vacuously satisfy the latter.

\subsubsection{Determinism of the past}
\label{sec:determinism_of_the_past}

It is worth mentioning that the operators $\dmnd\cdot$ and $\dmndback\cdot$ are
not entirely symmetric. The future is non-deterministic, i.e. at any point we may
have a choice of multiple ways for a computation to proceed. 
In our semantics for \HMLpast{}, the past however is
always deterministic; there is always at most one transition that was the last
transition to occur. This is by design, and we could have chosen to model the past
as nondeterministic as well, i.e. to take a possibilistic view where we would consider
all possible histories. 

This would make the forward and backwards diamond
operators symmetric in their view of the process graph. However, we are more
interested in properties about the actual past of a computation, especially
w.r.t. modelling epistemic properties, which rely on observations of some aspects
of the computation so far. (We do not discuss such properties in the
current paper.) We have thus reached the same conclusion as \cite{Laroussinie00}
that the deterministic view is more appropriate for our purposes.
Laroussinie and Schnoebelen list two other
properties of their model of the past in addition to being deterministic, namely that
the past is finite (there is a fixed initial state) and that past is cumulative (at each
transition the history gains information). We make implicit use of the latter property 
in our proofs and find that the former is a natural property of the processes we
want to model.

Treating the past as deterministic might suggest that it is unnecessary to index
the operator $\dmndback{\alpha}$ with the action and instead provide an operator
$\dmndback{\,}$ which matches any action (since there is at most one). However this
turns out to be insufficient, as such a logic could not distinguish between the
two computations
\[
    (r, r\trans\alpha t) \quad \textrm{and} \quad (r, r\trans\beta s)
\]
when $\alpha\ne\beta$. For our intended purposes, it is important that we are
able to tell these two apart.

\section{Decomposing Computations} % (fold)
\label{sec:decomp_comp}

We seek a definition of a \emph{formula quotient w.r.t. a process/state}
following the work of~\cite{Larsen91}.
This might be written as $\phi / p$ where $\phi$ is an \HMLpast{} formula
and $p$ is a process. The theorem we then seek to prove is this:
\[
    p \parallel q \vDash \phi  \quad\Leftrightarrow\quad p \vDash \phi / q
\]
where the \emph{parallel composition operator $\parallel$} is suitably defined
over LTSs.

Given our definition of $\vDash$ for \HMLpast, this requires us to prove
a theorem of the form
\[
    \rho \vDash \phi  \quad\Leftrightarrow\quad \rho_1 \vDash \phi / \rho_2
\]
where $\rho,\rho_1,\rho_2$ are computations such that $\rho$ is 
a computation of a process of the form $p \parallel q$ and that is,
in some sense,
the ``parallel composition'' of $\rho_1$ and $\rho_2$. In the 
standard setting it is straightforward to identify the components of a
parallel composition. In the case of computations, however, this is not so
obvious. A computation composed of two processes run in parallel has the form
\[
    (p\parallel q, \pi)
\]
where $p\parallel q$ is a syntactic representation of the initial state and
$\pi$ is the path leading up to the current state. The path $\pi$ however
may involve contributions from both of the parallel components.
Separating the contributions of the components
for the purposes of decompositional model checking requires us to \emph{unzip}
these paths into separate paths that might have been observed by considering
only one argument of the composition. This means that we have to find two paths $\pi_p$
and $\pi_q$ such that the computations
\[
    (p, \pi_p) \quad\textrm{and}\quad (q, \pi_q)
\]
are in some sense independent computations that run in parallel will
yield \mbox{$(p\parallel q, \pi)$}.

% subsection the_theorem_we_seek (end)

\subsection{Decomposition of computations} % (fold)
\label{sub:decomposition_of_computations}

In the setting of HML without past, parallel composition may be defined 
directly on LTSs independent
of the syntax or semantics of the underlying process algebra. When dealing
with computations, this does not provide enough information to find the two
computations that make up the parallel composition. For this information, one
needs to look into the syntax and semantics of the processes themselves and
moreover their semantics have to follow some restrictions.

For this study, 
in order to highlight the main ideas and technical tools in our approach,
we will restrict ourselves to a subset of CCS, namely CCS without
renaming, restriction or recursion. 
(We will discuss possible extensions of our results in 
Section~\ref{sec:decomp_future}.)
Processes are thus defined by the following
grammar.
\[
    p,q \quad::=\quad 0 \alt \alpha.p \alt p + q \alt p \parallel q
\]
with the following operational semantics
\begin{gather*}
    \sosrule{}{\alpha.p \trans{\alpha} p} \qquad
    \sosrule{p\trans{\alpha}p'}{p + q \trans{a} p'} \qquad
    \sosrule{q\trans{\alpha}q'}{p + q \trans{a} q'} \\
    \sosrule{p\trans{\alpha}p'}{p \parallel q \trans{\alpha} p' \parallel q} \qquad
    \sosrule{q\trans{\alpha}q'}{p \parallel q \trans{\alpha} p \parallel q'} \qquad
    \sosrule{p\trans{a}p'\quad q\trans{\bar{a}}q'}{p \parallel q \trans{\tau} p' \parallel q'}
\end{gather*}
We write $p\trans{\alpha}q$ to denote that this transition is provable by
these rules. We assume also that $\bar{\cdot} : A\rightarrow A$ is a function on action names
such that $\bar{\bar{a}} = a$.
The LTS associated with a CCS process $p$ is the largest transition system 
generated by these
rules starting from the process $p$. 
%We will use the notation $\C(p)$ to represent the set of computations in the
%LTS associated with $p$.

The decomposition of a computation running two parallel components must retain the
information about the order of steps in the interleaved computation. We do this
by modelling the decomposition using {\em stuttering computations}. These are
computations that are not only sequences of transition triplets, but may also involve
pseudo steps labelled with $\ptrans{}$. Intuitively, $p\ptrans{} p$ means that
process $p$ has remained idle in the last transition performed by a parallel process
having $p$ as one of its parallel components.
We denote
the set of stuttering computations with $\C_\T^*$ or simply $\C^*$.
For example, the computation
\[
(a.0\parallel b.0,
a.0\parallel b.0 \trans{a}
0\parallel b.0 \trans{b}
0\parallel 0)
\]
is decomposed into the stuttering computations
\begin{gather*}
    (a.0,a.0 \trans{a}0 \ptrans{}0) \quad\textrm{and}\\
    (b.0,b.0 \ptrans{} b.0 \trans{b} 0).\quad\phantom{\textrm{and}}
\end{gather*}
%
% However, we must be careful when decomposing computations that
% already involve stuttering steps.
% In this case we must tag the stuttering steps appropriately in order to prevent ambiguities.
% For example, consider the following computation.
% \[
% ((a.0\parallel b.0) \parallel c.0,\
%  (a.0\parallel b.0) \parallel c.0 \trans{a}
%  (0 \parallel b.0) \parallel c.0 \trans{b}
%  (0 \parallel 0) \parallel c.0 \trans{c}
%  (0 \parallel 0) \parallel 0)
% \]
% We first decompose this into
% \begin{gather*}
% (a.0\parallel b.0,\
% a.0\parallel b.0 \trans{a}
% 0\parallel b.0 \trans{b}
% 0\parallel 0 \ptrans{}
% 0\parallel 0)
% \\
% (c.0,\ c.0 \ptrans{} c.0 \ptrans{} c.0 \trans{c} 0)
% \end{gather*}
% Now we must decompose the first computation further, but it is already stuttering.
% To indicate at which decomposition step the stuttering steps are introduced,
% we tag them with a natural number. When decomposing a computation that is not
% stuttering, all stuttering steps get tagged with $0$, denoted by $\ptrans 0$. This is what
% we have simply called $\ptrans{}$ until now. When decomposing a computation that
% is already stuttering, new stuttering steps are also labelled with 
% $\ptrans 0$ but any existing
% stuttering steps coming from the composed computation
% have their tag increased by one.
% Returning to our example, which with tagged stuttering steps is
% \begin{gather*}
% (a.0\parallel b.0,\
% a.0\parallel b.0 \trans{a}
% 0\parallel b.0 \trans{b}
% 0\parallel 0 \ptrans{0}
% 0\parallel 0),
% \\
% (c.0,\ c.0 \ptrans{0} c.0 \ptrans{0} c.0 \trans{c} 0).
% \end{gather*}
% Its decomposition is
% \begin{gather*}
%     (a.0,\ a.0 \trans{a} 0 \ptrans{0} 0 \ptrans{1} 0) \\
%     (b.0,\ b.0 \ptrans{0} b.0 \trans{b} 0 \ptrans{1} 0) \\
%     (c.0,\ c.0 \ptrans{1} c.0 \ptrans{1} c.0 \trans{c} 0)
% \end{gather*}
%
However, the decomposition of a parallel computation 
is not in general unique, as there may be several possibilities
stemming from different synchronisation patterns.
For example consider a computation with the following trace.
\[
     (a.0+b.0)\parallel(\bar{a}.0+\bar{b}.0) \trans{\tau} 0\parallel 0
\]
From this computation it is not possible to distinguish if the transition
labelled with $\tau$ was the result of communication of the $a$ and $\bar a$ actions,
or of the $b$ and $\bar b$ actions.
For our purposes, this is not necessarily a problem because no expression
of our logic can differentiate between the two synchronisations, given only the composed
computation. We thus consider all possibilities simultaneously,
i.e. a decomposition of a computation will actually
be \emph{a set} of pairs of components.

The following function over paths defines the decomposition of a computation.
%To define the possible decompositions of a computation of the form $(p\parallel q, \pi)$
%we define the following function on paths.
\begin{align*}
    D(\lambda) &= \{(\lambda,\lambda)\} \\
    D(\pi'(p'\parallel q' \ptrans{} p'\parallel q')) &=
        \{(\mu_1(p'\ptrans{}p'),\mu_2(q'\ptrans{}q')) \alt (\mu_1,\mu_2)\in D(\pi')\} \\
    D(\pi'(p'\parallel q' \trans{\alpha} p''\parallel q'')) &=
        \begin{cases}
            \{(\mu_1(p' \trans{\alpha} p''), \mu_2(q' \ptrans{} q')) \\ \quad\quad
                \alt (\mu_1,\mu_2) \in D(\pi')\} & \textrm{if $q'\equiv q''$} \\[0.7em]
            \{(\mu_1(p' \ptrans{} p'), \mu_2(q'\trans{\alpha} q'')) \\ \quad\quad
                \alt (\mu_1,\mu_2) \in D(\pi')\} & \textrm{if $p'\equiv p''$} \\[0.7em]
            \{(\mu_1(p'\trans{a}p''), \mu_2(q'\trans{\bar{a}}q'')) \\ \quad\quad
                \alt (\mu_1,\mu_2) \in D(\pi'), a\in A, \\
                \qquad\qquad p'\trans{a}p'', q'\trans{\bar{a}}q''\}
                & \textrm{otherwise and $\alpha=\tau$}
        \end{cases}
\end{align*}
We should make a note of the fact that if $(\mu_1,\mu_2)$ is a decomposition
of a computation $\pi$, then the three computations have the same length.
Furthermore
\begin{equation}\label{eq:last_equals_last}
    \last(\pi) = \last(\mu_1) \parallel \last(\mu_2).
\end{equation}
%
Also of interest
is that, even though the above definition yields a set of decompositions of $\pi$, the only
case where multiple possibilities are generated is the last case where both components
evolve, and where there is ambiguity in the processes as to which actions actually
contributed to the communication. As we mentioned above, there is no expression
of the \HMLpast{} logic that can resolve such ambiguity only by looking at a composed
computation.
Since our goal is to model check of such expressions, the
existence of multiple decompositions of one computation will not pose any problem.
%Looking at a $\tau$ step in a composed computation, the computation does
%may not contain enough information to see precisely which actions were performed
%by each component. 
%Consider again the trace
%\[
%     (a.0+b.0)\parallel(\bar{a}.0+\bar{b}.0) \trans{\tau} 0\parallel 0
%\]
%In this case it is impossible to tell if the $\tau$ step is a result of matching
%$a$ with $\bar{a}$ or of matching $b$ with $\bar{b}$. 
%In reality, one or the other
%must have happened, but since the result is identical we cannot tell which.
%While this ambiguity may seem problematic at first, we note that no expression
%of \HMLpast{} logic can differentiate between the two ``realities'' only by looking
%at a composed computation.

Another notable property of path decomposition, is that its inverse is unique, i.e.
a pair $(\mu_1,\mu_2)$ can only be the decomposition of a single path. We formalise
this as a lemma which will come in handy later.
\begin{lemma}\label{thm:paths_compose_uniquely}
    Let $\pi_1$ be a path of a parallel computation and $(\mu_1,\mu_2)\in D(\pi_1)$. 
    If $\pi_2$ is a path such
    that $(\mu_1,\mu_2)\in D(\pi_2)$ also, then $\pi_1=\pi_2$.
\end{lemma}
\begin{proof}
    We start by noting that $\pi_1$ and $\pi_2$ cannot differ in length, as they
    are both equal in length to $\mu_1$ (and $\mu_2$). We apply induction on
    their common length.

    If both are empty, $\pi_1=\pi_2=\lambda$, then there is nothing to prove.
    Now assume they are non-empty and that
    \begin{align*}
        \pi_1 &= \pi_1' (p_1' \parallel q_1' \ R_1\  p_1 \parallel q_1) \\
        \pi_2 &= \pi_2' (p_2' \parallel q_2' \ R_2\  p_2 \parallel q_2)
    \end{align*}
    where $R_1,R_2$ are relations of the form $\trans\alpha$ or $\ptrans{}$.
    The induction hypothesis states that $\pi_1' = \pi_2'$, which also means
    that $p_1'=p_2'$ and $q_1'=q_2'$. Property~\eqref{eq:last_equals_last} above
    furthermore gives that $p_1=p_2$ and $q_1=q_2$.
    Thus we only need to show
    that the final steps coincide also, i.e. that $R_1=R_2$.
    % and the processes
    % $p_1, q_1$ are equal to $p_2,q_2$, respectively.
    The proof proceeds by case analysis on the last steps of $\mu_1$ and $\mu_2$.    
    \begin{itemize}
        \item If both $\mu_1$ and $\mu_2$ end with a pseudo-step, then we see from the definition
        of $D$ that both $R_1$ and $R_2$ must be pseudo-transitions. 
        % Since processes do
        %     not evolve during such transitions, it is obvious that $p_1,q_1$ are respectively
        %     equal to $p_2,q_2$.
        
        \item If only one of $\mu_1$ and $\mu_2$ ends with a pseudo-step, then the action
        of the other one must be the same as the last action of both $\pi$ and $\pi'$.
        % Since the transition system for CCS is deterministic for a fixed label, it
        % follows that the right hand sides of the last transitions are also equal.
        
        \item If both $\mu_1$ and $\mu_2$ end with a proper transition, we note that by
        the definition of $D$ the actions must complement each other. Then
        the last step of both $\pi$ and $\pi'$ must thus be labelled with $\tau$.
        % and by the same argument as before the right hand sides must match.
        
        \item If both $\mu_1$ and $\mu_2$ end with a proper transition, we note that by
        the definition of $D$ the actions must complement each other. Then
        the last step of both $\pi$ and $\pi'$ must thus be labelled with $\tau$.
        % and by the same argument as before the right hand sides must match.
    \end{itemize}
    This covers all the cases and thus we have shown that $R_1=R_2$, $p_1=p_2$
    and $q_1=q_2$. Coupled with the induction hypothesis, this means that $\pi=\pi'$.
\end{proof}

Note that the definition of $D$ relies on some properties of CCS specifically.
\begin{enumerate}
    \item We must have that $p \trans{a} p'$ leads to $p\not\equiv p'$. This is
          necessary so that the case-definitions are well defined, i.e. that
          they are mutually exclusive. This means that we can rely on
          $\equiv$-testing to determine if one side of the composition took
          a step or not.

          We should also note that this requirement means that we can actually
          remove the text ``and $\alpha=\tau$'' from the last case condition. To
          see why, note that the condition $q'\not\equiv q'' \land p'\not\equiv p''$
          (as implied by the word ``otherwise'') means that both must have taken
          a step simultaneously and communicated,  and therefore the only possible
          result action is indeed $\tau$. This means that the definition properly
          covers all cases.
% TODO: probably not needed...
%     \item The parallel operator needs to be commutative. Consider the process
%           $p = (a.0 + \bar{a}.0)$ and then consider $p \parallel p$. A
%           computation after one step of execution will be
% \[
%     (p\parallel p, p\parallel p \trans{\tau} 0 \parallel 0)
% \]
%           From this computation it is not possible to determine which side did
%           $a$ and which did $\bar{a}$. The commutativity of $\parallel$
%           means this is not really an issue (I think) as for any $p,q$ we can
%           consider $p\parallel q$ equal to $q\parallel p$, and we just pick
%           one side for the $a$ and the other for $\bar{a}$. However, when the
%           communication results in a composition of two distinct components,
%           then we can use the syntax of these components to deduce what were
%           the actions of each side.
    \item The only possible result of a communication is $\tau$, and $\tau$
          can never act as one partner of the communication.
\end{enumerate}

We now want to define quotient on \HMLpast-formulae such that a property of the
form
\[
    (p\parallel q, \pi) \vDash \phi  \quad\Leftrightarrow\quad
    (p, \mu_1) \vDash \phi/(q, \mu_2)
\]
where $(\mu_1,\mu_2) \in D(\pi)$. However, since we are dealing with sets of
decompositions, we need to quantify over these sets. It turns out that a
natural way that also gives a strong result is the following. Given that a composed
computation satisfies a formula, we can prove that one component of {\em every}
decomposition satisfies a formula quotiented with the other component,
\[
    (p\parallel q, \pi) \vDash \phi  \quad\Rightarrow\quad
    \forall (\mu_1,\mu_2) \in D(\pi) : (p, \mu_1) \vDash \phi/(q, \mu_2).
\]
On the other hand, to show the other direction, we need only one witness of a
decomposition that satisfies a quotiented formula to deduce that the composed
computation satisfies the original one,
\[
    \exists (\mu_1,\mu_2) \in D(\pi) : (p, \mu_1) \vDash \phi/(q, \mu_2)
    \quad\Rightarrow\quad
    (p\parallel q, \pi) \vDash \phi.
\]

Before defining the quotienting transformation we need define what $\vDash$ 
means with respect to stuttering computations.
We do this by extending \HMLpast{} to \HMLpp{} by adding
two operators.

\begin{definition}[Stuttering Hennessy Milner logic with past]
    \label{dfn:hmlpast}
    Let $\T=\langle P,A,\rightarrow \rangle$ be an LTS. The set $\HMLpp(A)$,
    or simply $\HMLpp$, of
    \emph{stuttering Hennessy-Milner logic formulae with past}
    is defined by the grammar
    \[
        \phi,\psi \,::=\, \true \alt \phi\land\psi
                                \alt \neg\phi
                                \alt \dmnd{\alpha} \phi
                                \alt \dmndback{\alpha} \phi
                                \alt \pdmnd{} \phi
                                \alt \pdmndback{} \phi
    \]
    where $i\in\mathbb{N}$ and $\alpha\in A_\tau$.
    We define the \emph{satisfaction relation} $\vDash^*\,\subseteq \C_\T^* \times \HMLpp$
    as the least relation that satisfies the following,
    \begin{itemize}
        \item $\rho\vDash^*\true$ for all $\rho\in\C_\T^*$,
        \item $\rho\vDash^*\phi\land\psi$ iff $\rho\vDash^*\phi$ and $\rho\vDash^*\psi$,
        \item $\rho\vDash^*\neg\phi$ iff not $\rho\vDash^*\phi$,
        \item $\rho\vDash^*\dmnd{\alpha}\phi$ iff for some
              $\rho'\in\C_\T^* : \rho\trans{\alpha}\rho'$ and $\rho'\vDash^*\phi$
        \item $\rho\vDash^*\dmndback{\alpha}\phi$ iff for some
              $\rho'\in\C_\T^* : \rho'\trans{\alpha}\rho$ and $\rho'\vDash^*\phi$
        \item $\rho\vDash^*\pdmnd{}\phi$ iff
              $\rho(p\ptrans{} p) \vDash^* \phi$ where $p=\last(\rho)$.
        \item $\rho\vDash^*\pdmndback{}\phi$ iff
              $\rho' \vDash^* \phi$ where $\rho = \rho'(p \ptrans{} p)$ for some $p$.
    \end{itemize}
    The satisfaction relation $\vDash^*\,\in P\times\HMLpp$ is defined by
    $p\vDash^*\phi$ if and only if $(p,\lambda)\vDash^* \phi$.
\end{definition}

\begin{remark}
    The satisfaction relations $\vDash^*$ and $\vDash$ coincide over $\C_\T \times \HMLpast$.
\end{remark}
% \paragraph{\bf Problem:} There is a problem here which makes this not well defined.
% Consider the processes $p = (a.c.0 + b.c.0)$ and $q = (\bar{a}.d.0 + \bar{b}.d.0)$.
% A computation after one step may look like this:
% \[
%     (p \parallel q, p\parallel q \trans{\tau} c.0 \parallel d.0)
% \]
% The problem here is that the above definition fails both of the computations
% \[
%     (p, p\trans{a}c.0) \quad\textrm{and}\quad (p, p\trans{b}c.0)
% \]
% are possible candidates for the left component. Possible solution: we extend
% the semantics of CCS to annotate $\tau$ actions resulting from communication
% with the actions that were performed by each side. (Would this solution perhaps
% also address 2 above, i.e. $\parallel$ would not need to be commutative?)
% Now the property we want is actually
% \[
%     (p\parallel q, \pi) \vDash^* \phi  \quad\Leftrightarrow\quad
%     \forall (\mu_1,\mu_2)\in D(\pi) : (p, \mu_1) \vDash^* \phi/(q, \mu_2).
% \]

% subsection decomposition_of_computations (end)

\subsection{Why are the stutters necessary?}

One may ask why we need to extend both computations and the logic to include the
notion of stuttering steps. The reason for doing so is to capture the information
about the interleaving order in component computations. This in turn is necessary because
the original logic can differentiate between different interleavings of parallel
processes.

For an example, let $p$ be a process that cannot perform an $a$ action, but
$p\trans{b} p'$ for some $p'$. Consider the computation $(a.0 \parallel p, \pi)$
where
\begin{equation}\label{eq:interleave1}
    \pi = a.0 \parallel p \trans{a} 0 \parallel p \trans{b} 0 \parallel p'
\end{equation}
Clearly this computation does {\bf not} satisfy the formula $\dmndback{a}\true$.

Another interleaving of the same parallel composition is the computation
$(a.0\parallel p, \pi')$ where
\begin{equation}\label{eq:interleave2}
    \pi' = a.0 \parallel p \trans{b} a.0 \parallel p' \trans{a} 0\parallel p'.
\end{equation}
This computation however does satisfy $\dmndback a \true$. Since the logic can
distinguish between different interleaving orders of a parallel computation, it
is vital to maintain information about interleaving order in our decomposition.
If the decomposition of the above computations only consider the actions contributed
by each component, this information is lost and both decompose to the same pair
of computations and we cannot reasonably expect to test if they satisfy the
formula $\dmndback a \true$ in a decompositional manner.

% section notes_to_be_structured_ (end)

\section{Decompositional Reasoning} % (fold)
\label{sec:decomp_hml}

We now define the quotienting construction over formulae structurally. 
Quotienting distributes over the boolean operators.
\begin{align*}
\true / \rho &= \true \\
(\phi_1 \land \phi_2)/\rho &= \phi_1/\rho  \land \phi_2 / \rho \\
(\neg\phi)/\rho &= \neg(\phi/\rho) \\
\end{align*}
The modal operators however need more attention. We start with $\dmnd{\alpha}\phi$
and consider separately the cases where $\alpha\in A$ and $\alpha = \tau$. In
the following we assume $p' = \last(\rho)$.
\begin{align*}
    (\dmnd{a}\phi) / \rho &= \dmnd{a} \left(\phi / \rho(p'\ptrans{0} p')\right)
        \lor \left(\bigvee_{\rho' : \rho\transshort{a}\rho'} \pdmnd{0} (\phi/\rho') \right)
    \\[1em]
    (\dmnd{\tau}\phi) / \rho &= \dmnd{\tau} \left(\phi / \rho(p'\ptrans{0} p')\right)
        \lor \left(\bigvee_{\rho' : \rho\transshort{\tau}\rho'} \pdmnd{0} (\phi/\rho') \right)
        \lor \left( \bigvee_{\rho',a : \rho \transshort{a} \rho'} \dmnd{\bar{a}} (\phi/\rho')  \right)
\end{align*}
Intuitively, the first case states that when we expect the composed computation to be
able to perform an $a$-transition, there are two possibilities. The first possibility
is that the component
we intend to test with the quotient formula can perform an $a$-transition. The rest of the
formula must then be quotiented with $\rho$ plus a pseudo-step representing that this
component remained idle.
The second possibility is that there is an $a$-transition from $\rho$. In this case
the component we want to test must proceed with a pseudo-step.
The same holds when we look
for a $\tau$-transition, with one addition.  If $\rho$ can advance with a non-$\tau$
action, then we should look for a matching action in the other component that may
have caused the two components to communicate.

To define the transformation for formulas of the form $\dmndback{\alpha}\phi$, we
again look at several cases separately. First we consider the case when $\rho$ has
the empty path. In this case it is obvious that no backward step is possible.
\[
    (\dmndback{\alpha}\phi) / (p,\lambda) = \false
\]
The second case to consider is when $\rho$ ends with a pseudo-transition, or a ``gap''.
In this case the only possibility is that the other component (the one we are testing)
is able to perform the backward transition, but only if the gap is tagged with $0$.
Otherwise both components end with a gap with a matching tag, as can be seen from the
definition of the decomposition function $D$. In this case the computation we are testing
obviously cannot end with an $\alpha$ transition.
\[
    (\dmndback{\alpha}\phi) / \rho'(p'\ptrans{i} p') =
    \begin{cases}
        \dmndback{\alpha} (\phi / \rho') & \textrm{if $i=0$} \\
        \false & \textrm{if $i>0$} \\
    \end{cases}
\]
The third case applies when $\rho$ does indeed end with the transition we look for.
In this case the other component must end with a matching gap.
\begin{equation}\label{eq:rewrite_back_dmnd_same}
    (\dmndback{\alpha}\phi) / \rho'(p''\trans{\alpha} p') = \pdmndback{0} (\phi / \rho')
\end{equation}
The only remaining case to consider is when $\rho$ ends with a transition different from
the one we look for. We split this case further and consider
again separately the cases when $\alpha\in A$ and when $\alpha = \tau$. The former case
is simple: if $\rho$ indicates that the last transition has a label other than the one
specified
in the diamond operator, there is no way that the composed computation satisfies it.
\[
    (\dmndback{a}\phi) / \rho'(p'' \trans{\beta} p') = \false \quad\textrm{where $a\ne \beta$}
\]
If however the diamond operator mentions a $\tau$ transition, then we must look for a
transition in the other component that can synchronize the last one of $\rho$. Note that this
case does not include computations ending with a $\tau$ transition, as that case is
covered by equation~\ref{eq:rewrite_back_dmnd_same}.
\[
    (\dmndback{\tau}\phi) / \rho'(p'' \trans{b} p') = \dmndback{\bar{b}} (\phi/\rho')
\]
This covers all possible cases for $\dmndback{\alpha}\phi / \rho$.

For the new operators, $\pdmnd{i}$ and $\pdmndback{i}$, the transformation is simple.
First, if a composed computation should satisfy $\pdmnd{i}\phi$, then it must be
because the components can add a pseudo-step tagged with $i+1$. I.e.
\[
    (\pdmnd{i}\phi) / \rho = \pdmnd{i+1} \left( \phi / \rho(p'\ptrans{i+1} p') \right)
\]
where again $p' = \last(\rho)$. If the composition should end with a $\ptrans{i}$
pseudo-transition, then both components must end with $\ptrans{i+1}$. Thus, we
have to consider two cases, where $\rho$ does end with such a pseudo-transition
and when it doesn't.
\[
    (\pdmndback{i}\phi) / \rho = \begin{cases}
        \pdmndback{i+1} (\phi / \rho') & \textrm{if $\rho = \rho'(p' \ptrans{i+1} p')$} \\
        \false & \textrm{otherwise}
    \end{cases}
\]

\todo{Do we need this?}
The complete quotienting transformation is summarised in the following table.
    \begin{align*}
    \true / \rho &= \true \\
    (\phi_1 \land \phi_2)/\rho &= \phi_1/\rho  \land \phi_2 / \rho \\
    (\neg\phi)/\rho &= \neg(\phi/\rho) \\
    %
    (\dmnd{a}\phi) / \rho &= \dmnd{a} \left(\phi / \rho(p'\ptrans{0} p')\right)
        \lor \left(\bigvee_{\rho' : \rho\transshort{a}\rho'} \pdmnd{0} (\phi/\rho') \right)
    \\[1em]
    (\dmnd{\tau}\phi) / \rho &= \dmnd{\tau} \left(\phi / \rho(p'\ptrans{0} p')\right)
        \lor \left(\bigvee_{\rho' : \rho\transshort{\tau}\rho'} \pdmnd{0} (\phi/\rho') \right) \\
        &\phantom{= \dmnd{\tau} \left(\phi / \rho(p'\ptrans{0} p')\right)\;}
        \lor \left( \bigvee_{\rho',a : \rho \transshort{a} \rho'} \dmnd{\bar{a}} (\phi/\rho')  \right) \\
    %
    (\dmndback{\alpha}\phi) / (p,\lambda) &= \false \\
    (\dmndback{\alpha}\phi) / \rho'(p'\ptrans{i} p') &=
    \begin{cases}
        \dmndback{\alpha} (\phi / \rho') & \textrm{if $i=0$} \\
        \false & \textrm{if $i>0$} \\
    \end{cases} \\
    (\dmndback{\alpha}\phi) / \rho'(p''\trans{\alpha} p') &= \pdmndback{0} (\phi / \rho') \\
    (\dmndback{a}\phi) / \rho'(p'' \trans{\beta} p') &= \false \quad\textrm{where $a\ne \beta$} \\
    (\dmndback{\tau}\phi) / \rho'(p'' \trans{b} p') &= \dmndback{\bar{b}} (\phi/\rho') \\
    %
    (\pdmnd{i}\phi) / \rho &= \pdmnd{i+1} \left( \phi / \rho(p'\ptrans{i+1} p') \right) \\
    %
    (\pdmndback{i}\phi) / \rho &= \begin{cases}
        \pdmndback{i+1} (\phi / \rho') & \textrm{if $\rho = \rho'(p' \ptrans{i+1} p')$} \\
        \false & \textrm{otherwise}
    \end{cases} \\
\end{align*}

% subsection transformation_of_formulas (end)

\subsection{Decomposition theorem}\label{sub:theorem}

Before we state and prove our main theorem, we establish a few useful lemmas.

\begin{lemma}\label{lemma:both_step_means_tau}
    If $p\parallel q \trans{\alpha} p'\parallel q'$ where $p\not\equiv p'$ and
    $q\not\equiv q'$ then $\alpha = \tau$.
\end{lemma}
\begin{proof}
    Consider the proof tree for the transition $p\parallel q \trans{\alpha}
    p'\parallel q'$ and, in particular, the last rule used in the proof. This
    rule can be one of the three rules for the parallel operator. The first two,
    where only one component advances, are ruled out since then either $p\equiv p'$
    or $q\equiv q'$ must hold. Therefore the last rule used in the proof
    must be the communication rule, in which case the
    label of the proved transition can only be $\tau$.
\end{proof}

\begin{lemma}\label{lemma:last_step_decomp}
    Let $p,q$ be processes, $(p\parallel q, \pi)\in\C(p\parallel q)$ and
    $(\mu_1,\mu_2)\in D(\pi)$.

    \noindent (i) If $(p\parallel q, \pi) \trans{\alpha} (p\parallel q, \pi')$
    then there exists a pair $(\mu_1', \mu_2')\in D(\pi')$ such that one of the following
    holds.
    \begin{enumerate}
        \item $(p,\mu_1) \trans\alpha (p,\mu_1')$ and $(q,\mu_2) \ptrans0 (q,\mu_2')$,
        \item $(p,\mu_1) \ptrans0 (p,\mu_1')$ and $(q,\mu_2) \trans\alpha (q,\mu_2')$ or
        \item $\alpha = \tau$, $(p,\mu_1) \trans{a} (p,\mu_1')$ and $(q,\mu_2) \trans{\bar{a}} (q,\mu_2')$ for some $a\in A$.
    \end{enumerate}

    \noindent (ii) Symmetrically,
    \begin{enumerate}
        \item If there exists a $\mu_1'$ s.t. $(p, \mu_1) \trans\alpha (p,\mu_1')$
            then there exists a $\pi'$ s.t. $(p\parallel q, \pi) \trans\alpha
            (p\parallel q, \pi')$ and $(\mu_1', \mu_2(q'\ptrans{0}q')) \in D(\pi')$
            where $q' = \last(\mu_2)$.
        \item If there exists a $\mu_2'$ s.t. $(q, \mu_2) \trans\alpha (q,\mu_2')$
            then there exists a $\pi'$ s.t. $(p\parallel q, \pi) \trans\alpha
            (p\parallel q, \pi')$ and $(\mu_1(p' \ptrans{0} p'), \mu_2') \in D(\pi')$
            where $p' = \last(\mu_1)$.
        \item If there exist $\mu_1'$ and $\mu_2'$ s.t. $(p,\mu_1) \trans{a} (p,\mu_1')$
            and $(q,\mu_2) \trans{\bar{a}} (q,\mu_2')$ for some $a\in A$, then there
            exists $\pi'$ s.t. $(p\parallel q, \pi) \trans{\tau} (p\parallel q, \pi')$
            and $(\mu_1', \mu_2') \in D(\pi')$.
    \end{enumerate}
\end{lemma}
\begin{proof}
    (i) Assume that $(p\parallel q, \pi) \trans{\alpha} (p\parallel q, \pi')$ and let
    $(\mu_1,\mu_2)\in D(\pi)$. This means there exist processes $p',q',p'',q''$
    with $\pi' = \pi(p''\parallel q'' \trans{\alpha} p'\parallel q')$,
    $p''=\last(\mu_1), q''=\last(\mu_2)$.
    Since $p''\parallel q'' \not\equiv
    p'\parallel q'$ we observe that $p''\equiv p'$ and $q''\equiv q'$ cannot hold
    simultaneously, so we consider the remaining cases.
    \begin{enumerate}
        \item $p''\not\equiv p'$ and $q''\equiv q'$. In this case the transition
            $p''\parallel q' \trans{\alpha} p'\parallel q'$ was proven using the first
            rule for $\parallel$. Its only premise must hold, namely $p''\trans{\alpha} p'$.
            We therefore let $\mu_1' = \mu_1(p''\trans{\alpha}p')$ and $\mu_2' = \mu_2
            (q' \ptrans{0} q')$. From the inductive definition of $D$ it is easy to see
            that $(\mu_1',\mu_2') \in D(\pi')$.
        \item $p''\equiv p'$ and $q''\not\equiv q'$. This case is entirely symmetric
            to the previous one where the proof is based on the second rule for $\parallel$.
        \item $p''\not\equiv p'$ and $q''\not\equiv q'$. Here the proof
            of the transition $p''\parallel q'' \trans{\alpha} p'\parallel q'$ must be based
            on the third rule for $\parallel$, 
            namely the communication rule and $\alpha=\tau$, as seen by
            Lemma~\ref{lemma:both_step_means_tau}. By the premises of this rule there
            exists an $a\in A$ such that $p''\trans{a}p'$ and $q''\trans{\bar{a}}q'$. We simply
            let $\mu_1' = \mu_1(p''\trans{a}p')$ and $\mu_2' = \mu_2(q''\trans{\bar{a}}q')$.
            Again it is clear from the definition of $D$ that $(\mu_1',\mu_2')\in D(\pi')$.
    \end{enumerate}

    (ii) \todo{TBD.} Follows directly from definition of $D$ -- but perhaps we could introduce
    some notation and tweak definitions to make this obvious. This is very similar to arguments
    made in the main proof.. maybe just a matter of rewording/extending the lemma or using
    a simpler notation.

\end{proof}

\begin{lemma}\label{lemma:removing_last_trans}
    Let $(p\parallel q,\pi) \in \C(p\parallel q)$ with $\pi$ non-empty and $(\mu_1,\mu_2)
    \in D(\pi)$. Let $\pi',\mu_1'$ and $\mu_2'$ be the prefixes of length $\abs{\pi}-1$ of $\pi,
    \mu_1$ and $\mu_2$ respectively. Then $(\mu_1',\mu_2') \in D(\pi')$.
\end{lemma}
\begin{proof}
    Follows directly from the definition of $D$.
\end{proof}

\begin{theorem}\label{thm:decomposition}
    For CCS processes $p,q$ and a computation $(p \parallel q, \pi)\in \C(p\parallel q)$
    and a formula $\phi \in \HMLpp$ we have
    %\sublabon{equation}
    \begin{equation}\label{eq:decomp_ltr}
        (p\parallel q, \pi) \vDash^* \phi  \quad\Rightarrow\quad
        \forall (\mu_1,\mu_2) \in D(\pi) : (p, \mu_1) \vDash^* \phi/(q, \mu_2)
    \end{equation}
    and conversely,
    \begin{equation}\label{eq:decomp_rtl}
        (p\parallel q, \pi) \vDash^* \phi
        \quad\Leftarrow\quad
        \exists (\mu_1,\mu_2) \in D(\pi) : (p, \mu_1) \vDash^* \phi/(q, \mu_2).
    \end{equation}
    %\sublaboff{equation}
\end{theorem}
\begin{proof}
    We prove both implications simultaneously by induction on the structure of $\phi$. 
    In the following text, the terms
    {\em ``the left-hand side''} and {\em ``the right-hand side''} refer respectively
    to the left- and right-hand sides of the above implications where the quantifier
    used in the right-hand side will be made clear by the context.

    \proofcase{Case $\phi = \true$} Then $\phi/(q, \mu_2) = \true$ and both sides
    of both~\eqref{eq:decomp_ltr} and~\eqref{eq:decomp_rtl} are trivially satisfied.


    \proofcase{Case $\phi = \psi_1 \land \psi_2$}\par\nobreak
    %
    \ltr First assume $(p\parallel q, \pi)
    \vDash^* \psi_1 \land \psi_2$
    and let $(\mu_1,\mu_2)\in D(\pi)$.
    Since both $\psi_1$ and $\psi_2$ are smaller than $\phi$ and both are satisfied by
    $(p\parallel q, \pi)$ we have by induction that $(p,\mu_1) \vDash^* \psi_i/(q, \mu_2)$
    for $i\in\{1,2\}$. Since $\phi/(q,\mu_2) = (\psi_1 \land \psi_2)/(q,\mu_2) =
    \psi_1 / (q, \mu_2) \land \psi_2 / (q,\mu_2)$ we obtain $(p,\mu_1) \vDash^*
    \phi/(q,\mu_2)$.

    \rtl Now assume the right side of~\eqref{eq:decomp_rtl}, 
    \[
        \exists (\mu_1,\mu_2)\in D(\pi) : (p,\mu_1)
        \vDash^* (\psi_1 \land \psi_2)/(q, \mu_2).
    \] 
    By definition the formula is equal
    to $\psi_1/(q,\mu_1) \land \psi_2/(q,\mu_2)$. By induction \mbox{$(p\parallel q, \pi)$}
    satisfies both $\psi_1$ and $\psi_2$ and thus also $\psi_1 \land \psi_2 = \phi$.


    \proofcase{Case $\phi = \neg \psi$}\par\nobreak
    %
    \ltr First assume the left side $(p\parallel q, \pi) \vDash^* \neg\psi$.
    %Since $\psi$ is smaller than $\phi$ the
    %implication~\eqref{eq:decomp_rtl} holds for it by induction. 
    Assume towards
    contradiction that there does exist a decomposition $(\mu_1',\mu_2')\in D(\pi)$
    such that $(p, \mu_1') \vDash^* \phi/(q, \mu_2')$. 
    Then by induction~\eqref{eq:decomp_rtl} gives $(p\parallel q, \pi) \vDash^* \psi$,
    which is in direct contradiction with our assumption. Since no such decomposition
    can exist, it holds for all $(\mu_1,\mu_2)\in D(\pi)$ that
    $(p,\mu_1) \vDash^* \neg\psi/(q,\mu_2) = \phi/(q,\mu_2)$.

    \rtl Assume the right side of~\eqref{eq:decomp_rtl}, namely there exists a
    decomposition $(\mu_1,\mu_2)\in D(\pi)$ such that $(p,\mu_1)\vDash^* \neg\psi/(q,\mu_2)$.
    Assume, again towards a contradiction, that $(p\parallel q,\pi)\vDash^*\psi$.
    By induction, \eqref{eq:decomp_ltr} now gives that for all $(\mu_1',\mu_2')\in D(\pi)$,
    $(p,\mu_1')\vDash^* \neg\psi/(q,\mu_2')$. In particular, this holds for the
    decomposition $(\mu_1,\mu_2)$, which contradicts our assumption. Therefore
    we must have that $(p\parallel q,\pi) \vDash^* \neg\psi = \phi$.


    \proofcase{Case $\phi = \dmnd\alpha\psi$}\par\nobreak
    %
    \ltr Again, first assume the left side and take $(\mu_1,\mu_2)\in D(\pi)$. Then
    there exists a computation $(p\parallel q, \pi')$ s.t. $(p\parallel q, \pi) \trans{\alpha}
    (p\parallel q, \pi')$ and $(p\parallel q, \pi') \vDash^* \psi$.
    % This means that $\pi' = \pi (p''\parallel q'' \trans{\alpha}
    % p' \parallel q')$ for some processes $p'', q'', p'$ and $q'$.
    By part (i) of
    Lemma~\ref{lemma:last_step_decomp} there exists a pair $(\mu_1',\mu_2')\in D(\pi')$.
    Since $\psi$ is smaller than $\phi$ we have by induction that
    \begin{equation}\label{eq:dmnd_psi}
        (p,\mu_1') \vDash^* \psi / (q,\mu_2')
    \end{equation}
    Lemma~\ref{lemma:last_step_decomp} also states that one of the following three cases holds.
    \begin{enumerate}
        \item $(p,\mu_1)\trans{\alpha}(p,\mu_1')$ and $(q,\mu_2)\ptrans{0}(q,\mu_2')$.
            %The first clause of $\phi/(q,\mu_2)$ is $\dmnd{\alpha}(\psi/(q,\mu_2')$.
            From~\eqref{eq:dmnd_psi} we have that $(p,\mu_1) \vDash^*
            \dmnd{\alpha}(\psi/(q,\mu_2'))$ and since that is the first clause of the
            disjunction $\phi/(q,\mu_2)$ then also $(p,\mu_1) \vDash^* \phi/(q,\mu_2)$.
        \item $(p,\mu_1)\ptrans{0}(p,\mu_1')$ and $(q,\mu_2)\trans{\alpha}(q,\mu_2')$.
            Again from~\eqref{eq:dmnd_psi} we have that $(p,\mu_1) \vDash^*
            \pdmnd{0}(\psi/(q,\mu_2'))$, and again this is a clause of the disjunction
            $\phi/(q,\mu_2)$ so $(p,\mu_1) \vDash^* \phi/(q,\mu_2)$.
        \item $\alpha = \tau$, $(p,\mu_1) \trans{a} (p,\mu_1')$ and $(q,\mu_2)
            \trans{\bar{a}} (q,\mu_2')$ for some $a\in A$. Then the disjunction
            $\phi/(q,\mu_2)$ has a clause $\dmnd{a}\left( \psi / (q,\mu_2') \right)$
            (note that $\bar{\bar{a}} = a$). By~\eqref{eq:dmnd_psi} we get that
            $(p,\mu_1) \vDash^* \phi / (q,\mu_2)$.
    \end{enumerate}
    In all cases the result is the same, namely $(p,\mu_1) \vDash^* \phi / (q,\mu_2)$
    which is what we wanted to prove.

    \rtl Now assume the right side of~\eqref{eq:decomp_rtl}, i.e. there exists a $(\mu_1,\mu_2)
    \in D(\pi)$ such that that $(p,\mu_1) \vDash^* \dmnd{\alpha}\psi / (q,\mu_2)$.
    We know $\dmnd{\alpha}\psi / (q,\mu_2)$ is a disjunction of one or more clauses
    so $(p,\mu_1)$ must satisfy at least one of them. Each clause has one of three
    forms, and we analyze the possible cases.
    Let $\phi'$ be the clause that $(p,\mu_1)$ satisfies.
    \begin{enumerate}
        \item $\phi' = \dmnd{\alpha} \left(\psi / (q,\mu_2)(q' \ptrans{0} q') \right)$
            where $q' = \last(q,\mu_2)$.
            Then there is a $\mu_1'$ such that $(p,\mu_1) \trans{\alpha} (p,\mu_1')$
            and $(p,\mu_1') \vDash^* \psi/(q,\mu_2(q'\ptrans{0}q'))$. If
            we let $\mu_2' = \mu_2(q' \ptrans{0} q')$ then 
            part (ii) of
            Lemma~\ref{lemma:last_step_decomp}
            gives that there exists a $\pi'$ with $(\mu_1',\mu_2')\in D(\pi')$ and
            $(p\parallel q, \pi) \trans{\alpha} (p\parallel q, \pi')$.
            Since $(p,\mu_1')
            \vDash^* \psi / (q,\mu_2')$ then by induction, since $\psi$ is smaller than
            $\phi$, $(p\parallel q, \pi') \vDash^* \psi$. This in turn means that
            $(p\parallel q, \pi) \vDash^* \dmnd{\alpha}\psi = \phi$.
        \item $\phi' = \pdmnd{0} \left(\psi / (q,\mu_2')\right)$ for some $\mu_2'$
            such that $(q,\mu_2) \trans{\alpha} (q,\mu_2')$. Let $\mu_1' = \mu_1(p'
            \ptrans{0} p')$ where $p' = \last(p,\mu_1)$. Lemma~\ref{lemma:last_step_decomp}
            gives the existence of $\pi'$ with $(\mu_1',\mu_2')\in D(\pi')$ and $(p\parallel
            q,\pi) \trans{\alpha} (p\parallel q, \pi')$. Since $(p,\mu_1')\vDash^* \psi
            / (q,\mu_2')$ then by induction $(p\parallel q,\pi') \vDash^* \psi$ and thus
            $(p\parallel q, \pi) \vDash^* \dmnd{\alpha}\psi = \phi$.
        \item $\alpha = \tau$ and $\phi' = \dmnd{\bar{a}} \left( \psi / (q,\mu_2') \right)$
            for some $\mu_2'$ s.t. $(q,\mu_2)\trans{a}(q,\mu_2')$ and $a\in A$.
            This means there is a
            $\mu_1'$ with $(p,\mu_1) \trans{\bar{a}} (p,\mu_1')$.
            Lemma~\ref{lemma:last_step_decomp} then says that there exists $\pi'$ with $(p
            \parallel q, \pi) \trans{\tau} (p\parallel q, \pi')$ and $(\mu_1',\mu_2')
            \in D(\pi')$. Since $(p,\mu_1')\vDash^* \psi / (q,\mu_2')$ then by induction
            $(p\parallel q,\pi')\vDash^* \psi$ and thus $(p\parallel q, \pi) \vDash^*
            \dmnd{\tau}\psi = \phi$.
    \end{enumerate}
    In all cases we obtain what we wanted to prove, namely $(p\parallel q, \pi) \vDash^* \phi$.


    \proofcase{Case $\phi = \dmndback\alpha\psi$}\par\nobreak
    %
    \ltr Assume that $(p\parallel q, \pi) \vDash^* \dmndback{\alpha}\psi$ and take
    $(\mu_1,\mu_2)\in D(\pi)$. Since \mbox{$(p\parallel q, \pi') \trans{\alpha} (p\parallel
    q, \pi)$} for some $\pi'$ such that $(p\parallel q,\pi') \vDash^* \psi$,
    there exist processes $p',q',p'',q''$ such that $\pi
    = \pi'(p''\parallel q'' \trans{\alpha} p'\parallel q')$. By analyzing the definition
    of $D$, we can gain some information about $\mu_1$ and $\mu_2$, in particular by
    comparing $p''$ to $p'$ and $q''$ to $q'$. Since $p''\parallel q'' \not\equiv
    p'\parallel q'$ we must consider three cases.
    \begin{enumerate}
        \item $p''\not\equiv p'$ and $q''\equiv q'$. Then $(\mu_1,\mu_2) = (\mu_1'
            (p'' \trans{\alpha} p'), \mu_2'(q' \ptrans{0} q'))$ for some $(\mu_1',\mu_2')
            \in D(\pi')$. Given this form of $\mu_2$ we also know that $(\dmndback{\alpha}
            \psi) / (q,\mu_2) = \dmndback{\alpha} \left( \psi / (q,\mu_2') \right)$. Since
            $(p\parallel q, \pi') \vDash^* \psi$, we get by induction that $(p,\mu_1')
            \vDash^* \psi / (q,\mu_2')$, which in turn means that $(p,\mu_1)\vDash^*
            \dmndback{\alpha} \left(\psi / (q,\mu_2)\right)$ and since the last step of
            $\mu_2$ is a pseudo-step, $\dmndback{\alpha} \left(\psi / (q,\mu_2)\right)
            = (\dmndback{\alpha}\psi)/(q,\mu_2)$.
        \item $p''\equiv p'$ and $q''\not\equiv q'$. In this case $(\mu_1,\mu_2)
            = (\mu_1'(p'\ptrans{0}p'), \mu_2'(q''\trans{\alpha}q'))$ where $(\mu_1',
            \mu_2') \in D(\pi')$. This form of $\mu_2$ means that $\dmndback{\alpha}
            \psi / (q,\mu_2) = \pdmndback{0} \left( \psi / (q,\mu_2') \right)$.
            By induction, the fact that $(p\parallel q, \pi') \vDash^* \psi$ gives
            that $(p,\mu_1') \vDash^* \psi / (q,\mu_2')$, so $(p,\mu_1) \vDash^*
            \pdmndback{0} \left( \psi / (q,\mu_2') \right) = (\dmndback{\alpha}\psi)
            / (q,\mu_2)$.
        \item $p''\not\equiv p'$ and $q''\not\equiv q'$. By Lemma~\ref{lemma:both_step_means_tau}
            $\alpha$ must be equal to $\tau$. Thus we have that $(\mu_1,\mu_2) =
            (\mu_1'(p''\trans{a}p'), \mu_2'(q''\trans{a}q'))$ for some $a\in A$ and
            $(\mu_1',\mu_2') \in D(\pi')$. This also means that $(\dmndback{\alpha}\psi)
            / (q,\mu_2) = \dmndback{a}\left( \psi / (q,\mu_2') \right)$. Since $(p\parallel q,
            \pi') \vDash^* \psi$ we again have by induction that $(p,\mu_1')\vDash^*\psi/(q,
            \mu_2')$. We therefore obtain that $(p,\mu_1) \vDash^* \dmndback{a}\left(
            \psi / (q,\mu_2') \right) = (\dmndback{\alpha}\psi) / (q,\mu_2)$.
    \end{enumerate}
    In all cases we obtain the same result, namely $(p,\mu_1) \vDash^* (\dmndback{\alpha}\psi)
    / (q,\mu_2) = \phi/(q,\mu_2)$.

    \rtl Now assume that there is $(\mu_1,\mu_2) \in D(\pi)$ s.t. $(p,\mu_1) \vDash^* (\dmndback{
    \alpha}\psi)/(q,\mu_2)$. This means that $\mu_1$ and $\mu_2$ are non-empty. By comparing
    $\alpha$ with the last transition  of $\mu_2$ we can infer the form of $(\dmndback{\alpha}
    \psi) / (q,\mu_2)$.
    \begin{itemize}
        \item If the last transition of $\mu_2$ is a $\ptrans{i}$ transition, we first
            observe that $i=0$. If $i>0$ then the last transition of $\mu_1$ must match
            it which contradicts that it is $\trans{\alpha}$. This means that $\mu_2
            = \mu_2' (q' \ptrans{0} q')$ for some $\mu_2$ and $q' = \last(q,\mu_2)$. Then
            we know that $(\dmndback\alpha \psi) / (q,\mu_2) = \dmndback\alpha \left(
            \psi / (q,\mu_2') \right)$. By our assumption this is satisfied by $(p,\mu_1)$
            so there exists a $\mu_1'$ s.t. $(p,\mu_1') \trans\alpha (p,\mu_1)$ and
            $(p,\mu_1') \vDash^* \psi / (q,\mu_2')$. Let $\pi'$ be $\pi$ without the
            last transition (note that $\pi$ is non-empty since $\mu_1$ and $\mu_2$ are).
            By Lemma~\ref{lemma:removing_last_trans} $(\mu_1',\mu_2')\in D(\pi')$ and by
            induction we have that $(p\parallel q,\pi') \vDash^* \psi$. From the definition
            of $D$ we can also see that the last transition of $\pi$ can only be $\trans\alpha$.
            Thus $(p\parallel q,\pi') \trans\alpha (p\parallel q, \pi)$ so $(p\parallel q,\pi)
            \vDash^* \dmndback\alpha\psi$.
        \item If the last transition of $\mu_2$ is an $\trans\alpha$ transition, i.e. one
            having the same label as the formula is testing for, then $(\dmndback\alpha\psi)
            /(q,\mu_2) = \pdmndback{0}\left( \psi/(q,\mu_2') \right)$ where $\mu_2'$ is $\mu_2$
            without the last transition. Note that $(q,\mu_2')\trans\alpha(q,\mu_2)$. Since
            $(p,\mu_1)$ satisfies this formula there is a $\mu_1'$ s.t. $(p,\mu_1')\ptrans{0}
            (p,\mu_1)$ and $(p,\mu_1) \vDash^* \psi/(q,\mu_2')$. By
            Lemma~\ref{lemma:removing_last_trans} $(\mu_1',\mu_2')\in D(\pi')$ where $\pi'$
            is again $\pi$ without the last transition. Also again, we can see from the
            definition of $D$ that $(p\parallel q,\pi')\trans\alpha(p\parallel q,\pi)$.
            By induction it thus holds that $(p\parallel q, \pi')\vDash^* \psi$ and so
            $(p\parallel q,\pi) \vDash^* \dmndback{\alpha}\psi$.
        \item The only remaining case to consider is when $\mu_2$ ends with a transition
            $\trans{\beta}$ where $\beta \ne \alpha$. Then $\alpha$ can only be $\tau$,
            since otherwise the formula $(\dmndback{\alpha}\psi)/(q,\mu_2)$
            equals $\false$, which contradicts our assumption
            that $(p,\mu_1)$ satisfies it. Since $\beta\ne\alpha = \tau$ we also know $\beta$
            must be some label $a\in A$. This means that $(\dmndback\alpha\psi)/(q,\mu_2)
            = \dmndback{\bar{a}} \left( \psi / (q, \mu_2') \right)$ where $\mu_2'$ is yet
            again $\mu_2$ without the last transition. Since $(p,\mu_1)$ satisfies this,
            there is a $\mu_1'$ s.t. $(p,\mu_1')\trans{\bar{a}}(p,\mu_1)$ and $(p,\mu_1')
            \vDash^* \psi/(q,\mu_2')$. By
            Lemma~\ref{lemma:removing_last_trans}, $(\mu_1',\mu_2')\in D(\pi')$ where
            $\pi'$ is $\pi$ without the last transition. By induction, $(p\parallel q, \pi')
            \vDash^* \psi$ and from the definition of $D$ we can see that $(p\parallel q,\pi')
            \trans\tau (p\parallel q,\pi)$ is the only possible transition between the
            two. In particular,
            $(p\parallel q,\pi) \vDash^* \dmndback\tau \psi
            = \dmndback{\alpha}\psi$.
    \end{itemize}
    In all cases $(p\parallel q,\pi) \vDash^* \dmndback\alpha\psi = \phi$.

    %\todo{I think the last case analysis can be simplifed a lot by defining $\pi',\mu_1',\mu_2'$
    %in advance and invoking Lemma~\ref{lemma:removing_last_trans} before analyzing the
    %cases.}


    \proofcase{Case $\phi = \pdmnd{i}, i\in \mathbb{N}$}\par\nobreak
    %
    \ltr First assume $(p\parallel q, \pi) \vDash^* \pdmnd{i}\psi$ and take $(\mu_1,\mu_2)
    \in D(\pi)$. This means there exists
    a $\pi'$ s.t. $(p\parallel q, \pi) \ptrans{i} (p\parallel q, \pi')$ and $(p\parallel q,
    \pi')\vDash^* \psi$. By definition
    $(\pdmnd{i}\psi)/(q,\mu_2) = \pdmnd{i+1} \left( \psi / (q, \mu_2') \right)$,
    where we let $(\mu_1',
    \mu_2') = (\mu_1(p'\ptrans{i+1}p'),$ $\mu_2(q'\ptrans{i+1}q'))$ with $(p' \parallel q') =
    \last{(p\parallel q, \pi)}$. This is according to the definition of $D$ so $(\mu_1',\mu_2')
    \in D(\pi')$. Thus, by induction $(p,\mu_1') \vDash^* \psi / (q,\mu_2')$. Since $(p,\mu_1)
    \ptrans{i+1} (p,\mu_1')$ we obtain that $(p,\mu_1) \vDash^* \pdmnd{i+1} \left(
    \psi / (q,\mu_2') \right) = (\pdmnd{i}\psi)/(q,\mu_2)$.

    \rtl Assume that $\exists (\mu_1,\mu_2) \in D(\pi) : (p,\mu_1) \vDash^* (\pdmnd{i}
    \psi)/(q,\mu_2)$. 
    We want to show that $(p\parallel q,\pi) \vDash^* \pdmnd{i}\psi$.
    The formula is equal to $\pdmnd{i+1} \left( \psi / (q,\mu_2') \right)$
    where $\mu_2' = \mu_2(q' \ptrans{i+1} q')$, $q' = \last(q,\mu_2)$. Let $\pi' = \pi
    (p'\parallel q' \ptrans{i} p'\parallel q')$ and $\mu_1' = \mu_1(p'\ptrans{i+1} p')$
    where $p' = \last(p,\mu_1)$. Then, by definition of $D$, $(\mu_1',\mu_2') \in D(\pi')$.
    Observe that $(p,\mu_1)\ptrans{i+1}(p,\mu_1')$ and that the $\ptrans{i+1}$ relation
    is deterministic. Therefore
    $(p,\mu_1')\vDash^* \psi/(q,\mu_2')$ for each $(\mu_1,\mu_2)$. 
    Induction gives that $(p\parallel q, \pi')
    \vDash^* \psi$. Now it follows trivially that $(p\parallel q,\pi) \vDash^* \pdmnd{i}
    \psi$ because $(p\parallel q, \pi)\ptrans{i}(p\parallel q, \pi')$.


    \proofcase{Case $\phi = \pdmndback{i}, i\in \mathbb{N}$}\par\nobreak
    %
    \ltr Assume $(p\parallel q, \pi) \vDash^* \pdmndback{i}\psi$ and take $(\mu_1,\mu_2)
    \in D(\pi)$. It is obvious, from the definition of $D$ that $\mu_1$ and $\mu_2$ both
    end with $\ptrans{i+1}$ since $\pi$ ends with $\ptrans{i+1}$. Let $\pi', \mu_1', \mu_2'$
    be $\pi,\mu_1,\mu_2$ without their last transition respectively (note that our assumption
    guarantees that they are non-empty). By Lemma~\ref{lemma:removing_last_trans} we know
    that $(\mu_1',\mu_2') \in D(\pi')$. Since $(p\parallel q,\pi') \vDash^* \psi$, we have
    by induction that $(p,\mu_1')\vDash^* \psi/(q,\mu_2')$. Then $(p,\mu_1)\vDash^*
    \pdmndback{i+1} \left( \psi/(q,\mu_2') \right) = (\pdmndback{i}\psi)/(q,\mu_2)$.

    \rtl Now assume $\exists (\mu_1,\mu_2) \in D(\pi) : (p,\mu_1) \vDash^* (\pdmndback{i}
    \psi)/(q,\mu_2)$. Then the last step of $\mu_2$ is $\ptrans{i+1}$ since otherwise
    the formula would be equal to $\false$, which could not be satisfied by $(p,\mu_1)$.
    We see furthermore that the formula is $\pdmndback{i+1}\left(\psi/(q,\mu_2')\right)$ where
    $\mu_2'$ is again $\mu_2$ without its last step. This means the last step of $\mu_1$
    is also $\pdmnd{i+1}$ (a fact we could also have deduced from the definition of $D$),
    let $\mu_1'$ be $\mu_1$ with out this step. If we also let $\pi'$ be $\pi$ without the
    last step, and by Lemma~\ref{lemma:removing_last_trans} we have
    $(\mu_1',\mu_2')\in D(\pi')$. Since
    $(p,\mu_1')\vDash^* \psi/(q,\mu_2')$ induction gives that $(p\parallel q,\pi')\vDash^*
    \psi$. By the definition of $D$ we see that the last step of $\pi$ can only be $\ptrans{i}$
    so $(p\parallel q, \pi)\vDash^* \pdmndback{i} \psi$.

    \vspace{1em}

    This concludes the analysis of all structural forms for $\phi$. In each case we have
    shown by structural induction that each direction of the theorem holds.
\end{proof}

Theorem~\ref{thm:decomposition} uses the existential quantifier in the right-to-left
direction. This makes it easy to show that a composed computation satisfies a
formula, given only one witness of a decomposition with one component satisfying
a rewritten formula. Note however that the set of decompositions of any given
process is never empty, i.e. every parallel computation has a decomposition.
This allows us to write the above theorem in a more symmetrical form.
\begin{corollary}
    For CCS processes $p,q$, a parallel computation $(p \parallel q, \pi)$
    and a formula $\phi \in \HMLpp$ we have
    %\sublabon{equation}
    \begin{equation}\label{eq:decomp_bi}
        (p\parallel q, \pi) \vDash^* \phi  \quad\Leftrightarrow\quad
        \forall (\mu_1,\mu_2) \in D(\pi) : (p, \mu_1) \vDash^* \phi/(q, \mu_2)
    \end{equation}
\end{corollary}
\begin{proof}
    \ltr This case follows directly from the theorem.
    
    \rtl Assume that $\forall (\mu_1,\mu_2) \in D(\pi) : (p, \mu_1) \vDash^* \phi/(q, \mu_2)$.
    Specifically, since there exists at least one decomposition $(\mu_1',\mu_2')\in D(\pi)$,
    the above holds for that particular decomposition. By $\Leftarrow$ part of the
    theorem, we thus have that \mbox{$(p\parallel q,\pi)\vDash^*\phi$}.
\end{proof}

\section{Extensions and related work} % (fold)
\label{sec:decomp_future}

So far we have stated and proven a decompositional theorem that allows us to
apply decompositional reasoning for history-based computations over CCS and
the logic $\HMLpast$. However, more work remains to be done in order to apply
this theory to meaningful examples. Part of this work is well underway already
although the details have not been worked out fully for this particular thesis
submission. In particular, we have extended the decompositional theorem to a
recursive logic $\HMLppf{\X}$, which allows a much wider class of interesting 
properties to be specified as fix-points of systems of recursive logic equations.


\renewcommand{\trans}[1]{\,{\stackrel{{#1}}{\rightarrow}}\,}

\renewcommand{\jointwith}{Luca Aceto, Anna Ing\'olfsdottir,\\ MohammadReza Mousavi and Michel Reniers}
\chapter[Rule Formats for Determinism and Idempotency]{Rule Formats for \\ Determinism and Idempotency}
%\epigraph{I like deadlines. I like the whooshing sound they make as they pass by.}%
%         {\textit{Douglas Adams}}
\section{Introduction}
Structural Operational Semantics (SOS) \cite{Plotkin04a} is a popular method for assigning a rigorous meaning to
specification and programming languages.
The meta-theory of SOS provides powerful tools for proving semantic properties for such languages
without investing too much time on the actual proofs; it offers syntactic templates for SOS rules, called
\emph{rule formats},
which guarantee semantic properties once the SOS rules conform to the templates
(see, e.g., the references~\cite{Aceto01,Mousavi07-TCS} for surveys on the meta-theory of SOS).
There are various rule formats  in the literature for many different semantic properties, 
ranging from basic properties such as commutativity \cite{Mousavi05-IPL} and 
associativity \cite{Mousavi08-CONCUR} of operators, and congruence of behavioral 
equivalences (see, e.g., \cite{Verhoef95}) to more technical and involved ones such as 
non-interference \cite{Tini04} and (semi-)stochasticity \cite{Lanotte05}.
In this paper, we propose rule formats for two (related) properties, namely, determinism and idempotency.

Determinism is a semantic property of (a  fragment of) a language
that specifies that a program cannot evolve operationally in several different ways.
It holds for sub-languages of many process calculi and programming languages,
and it is also a crucial property for many formalisms for the description of timed systems, where time transitions are required to be deterministic, because the passage of time should not resolve any choice.

Idempotency is a property of binary composition operators requiring
that the composition of two identical specifications or programs
will result in a piece of specification or program that is equivalent to the original components.
Idempotency of a binary operator $f$ is concisely expressed by the following algebraic equation.
\[
f(x, x) = x
\]
Determinism and idempotency may seem unrelated at first sight.
However, it turns out that in order to obtain a powerful rule format
for idempotency, we need to have the determinism of certain transition relations in place. Therefore,
having a syntactic condition for determinism, apart from its intrinsic value,
results in a powerful, yet syntactic framework for idempotency.


To our knowledge, our rule format for idempotency has no precursor in the literature.
As for determinism, in \cite{Fokkink03a}, a rule format for bounded nondeterminism is presented but the case for determinism is not studied.
Also, in \cite{Ulidowski97b} a rule format is proposed to guarantee several time-related properties, including time determinism, in the settings of
Ordered SOS. In case of time determinism, their format corresponds to a subset of our rule format when translated to the setting of ordinary SOS, by means of the recipe given in \cite{Mousavi06-FSTTCS}.

We made a survey of existing deterministic process calculi and
of idempotent binary operators in the literature and we have applied our formats to them.
Our formats could cover all practical cases that we have discovered so far,
which is an indication of its expressiveness and relevance.

The rest of this paper is organized as follows. In Section \ref{sec::pre} we recall some basic definitions from the meta-theory of SOS.
In Section \ref{sec::det}, we present our rule format for determinism and prove that it does guarantee determinism for certain transition relations.
Section \ref{sec:idempotency} introduces a rule format for idempotency and proves it correct.
In Sections \ref{sec::det} and \ref{sec:idempotency},  we also provide several examples to motivate
the constraints of our rule formats and to demonstrate their practical applications.
Finally, Section \ref{sec::conc} concludes the paper and presents some directions for future research. 
\section{\label{sec::pre}Preliminaries}

In this section we present, for sake of completeness, some standard definitions from
the meta-theory of SOS that will be used in the remainder of the paper.

\begin{definition}[Signature and terms]
    We let $V$ represent an infinite set of variables and use $x,x',x_i,y,y',y_i,\dots$ to range over elements of $V$.
    A \emph{signature} $\Sigma$ is a set of function symbols, each with a fixed arity. We call these symbols
    \emph{operators} and usually represent them by $f,g,\dots$. An operator with arity zero is called a
    \emph{constant}. We define the set $\Terms\Sigma$ of \emph{terms} over $\Sigma$ as the smallest set satisfying the following constraints.
    \begin{bullets}
        \item A variable $x\in V$ is a term.
        \item If $f\in \Sigma$ has arity $n$ and $t_1,\dots,t_n$ are terms, then $f(t_1,\dots,t_n)$ is a term.
    \end{bullets}
    We use $t,t',t_i,\dots$ to range over terms. We write $t_1 \equiv t_2$ if $t_1$ and $t_2$ are syntactically equal.
    The function $vars : \Terms\Sigma \rightarrow 2^V$ gives the set of variables appearing in a term.
    The set $\CTerms\Sigma \subseteq \Terms\Sigma$ is the set of \emph{closed terms}, i.e., terms that contain no variables.
    We use $p,p',p_i,\dots$ to range over closed terms.
    A \emph{substitution} $\sigma$ is a function of type $V \rightarrow \Terms\Sigma$. We extend the domain of substitutions to terms
    homomorphically. If the range of a substitution lies in $\CTerms\Sigma$, we say that it is a \emph{closing substitution}.
\end{definition}

\begin{definition}[Transition System Specifications (TSS), formulae and transition relations]
    A \emph{transition system specification} is a triplet $(\Sigma, L, D)$ where
    \begin{bullets}
        \item $\Sigma$ is a signature.
        \item $L$ is a set of labels. If $l \in L$, and $t,t'\in \Terms\Sigma$
              we say that $t \trans{l} t'$ is a \emph{positive formula} and
              $t \ntrans{l}$ is a \emph{negative formula}. A formula, typically denoted by $\phi$, $\psi$, $\phi'$, $\phi_i$, $\ldots$
              is either a negative formula or a positive one.
        \item $D$ is a set of \emph{deduction rules}, i.e., tuples of the form $(\Phi,\phi)$ where $\Phi$ is a set of
              formulae and $\phi$ is a positive formula. We call the formulae contained in $\Phi$ the \emph{premises} of the rule and $\phi$ the
              \emph{conclusion}.
    \end{bullets}
    We write $\vars{r}$ to denote the set of variables appearing in a deduction rule $\DR{r}$.
    We say a formula is \emph{closed} if all of its terms are closed. Substitutions are also extended to formulae
    and sets of formulae in the natural way. A set of positive closed formulae is called a \emph{transition relation}.
\end{definition}

We often refer to a formula $t \trans{l} t'$ as a \emph{transition} with $t$ being its \emph{source}, $l$ its label, and $t'$
its \emph{target}.
A deduction rule $(\Phi,\phi)$ is typically written as $\frac{\Phi}{\phi}$. For a deduction rule $r$, we write $\conc{r}$ to denote its conclusion and $\prem{r}$ to denote its premises.
We call a deduction rule $f$-\emph{defining} when the outermost function symbol appearing in its source of the conclusion is $f$.


The meaning of a TSS is defined by the following notion of least three-valued stable model.
To define this notion, we need two auxiliary definitions, namely provable transition rules and contradiction, which are given below.

\begin{definition}[Provable Transition Rules]
A deduction rule is called a \emph{transition rule} when it is of the form $\frac{N}{\phi}$ with $N$ a set of {\em negative formulae}. A TSS $\tss$ \emph{proves} $\frac{N}{\phi}$, denoted by $\tss \vdash \frac{N}{\phi}$, when there is a well-founded upwardly branching tree with formulae as nodes and of which
\begin{itemize}
\item the root is labelled by $\phi$;
\item if a node is labelled by $\psi$ and the nodes above it form the set $K$ then:
\begin{itemize}
\item $\psi$ is a negative formula and $\psi \in N$, or
\item $\psi$ is a positive formula and $\frac{K}{\psi}$ is an instance of a deduction rule in $\tss$.
\end{itemize}
\end{itemize}
\end{definition}

\begin{definition}[Contradiction and Contingency]
Formula $t \trans{l} t'$ is said to \emph{contradict} $t \ntrans{l}$, and vice versa.
For two sets $\Phi$ and $\Psi$ of formulae,
$\Phi$ \emph{contradicts} $\Psi$, denoted by $\Phi \nvDash \Psi$, when there is a $\phi \in \Phi$ that contradicts a $\psi \in \Psi$.
$\Phi$ is \emph{contingent} w.r.t.\ $\Psi$, denoted by $\Phi \vDash \Psi$, when $\Phi$ does not contradict $\Psi$.
\end{definition}

It immediately follows from the above definition that contradiction and contingency are symmetric relations on (sets of) formulae.
We now have all the necessary ingredients to define the semantics of TSSs in terms of three-valued stable models.

\begin{definition}[The Least Three-Valued Stable Model]
A pair $(C, U)$ of sets of positive closed transition formulae is called a \emph{three-valued stable model} for a TSS $\tss$ when
\begin{itemize}
\item  for all $\phi \in C$, $\tss \vdash \frac{N}{\phi}$ for a set $N$ such that $C \cup U \vDash N$, and
\item  for all $\phi \in U$, $\tss \vdash \frac{N}{\phi}$ for a set $N$ such that $C \vDash N$.
\end{itemize}
$C$ stands for {\em Certainly} and $U$ for {\em Unknown}; the third value is determined by the formulae not in $C \cup U$.
The \emph{least} three-valued stable model is a three valued stable model which is the least with respect to the ordering on
pairs of sets of formulae defined as $(C,U) \leq (C', U')$ iff $C \subseteq C'$ and $U' \subseteq U$.
When for the least three-valued stable model it holds that $U=\emptyset$, we say that $\tss$ is \emph{complete}.
\end{definition}


Complete TSSs univocally define a transition relation, i.e., the $C$ component of their least three-valued stable model.
Completeness is central to almost all meta-results in the SOS meta-theory and,
as it turns out, it also plays an essential role in our meta-results concerning determinism and idempotency.
All practical instances of TSSs are complete and
there are syntactic sufficient conditions guaranteeing completeness, see for example \cite{Groote93}.

\section{\label{sec::det}Determinism}
\begin{definition}[Determinism]
A transition relation $T$ is called deterministic for label $l$, when if
$p \trans{l} p' \in T$ and $p \trans{l} p'' \in T$, then $p' \equiv p''$.
\end{definition}

Before we define a format for determinism, we need two auxiliary definitions.
The first one is
the definition of source dependent variables,
which we borrow from~\cite{Mousavi05-ICALP} with minor additions.
%\MAR{It is not clear what are those additions and why we have those.}

\begin{definition}[\label{def::varDepend}Source dependency]
For a deduction rule, we define the set of \emph{source dependent} variables as the smallest set that contains
\begin{enumerate}
    \item all variables appearing in the source of the conclusion, and
    \item all variables that appear in the target of a premise where all variables in the source of that premise are source dependent.
\end{enumerate}
For a source dependent variable $v$, let $\mathcal{R}$ be the collection of transition relations appearing in a set of premises needed to show source dependency through condition 2. We say that $v$ is source dependent \emph{via} the relations in $\mathcal{R}$.
\end{definition}

Note that for a source dependent variable, the set $\mathcal{R}$ is not necessarily unique. For example, in the rule
\[
    \sosrule{y\trans{l_1}y' \quad x\trans{l_2}z \quad z\trans{l_3}y'}{f(x,y) \trans{l} y'}
\]
the variable $y'$ is source dependent both via the set $\{\trans{l_1}\}$ as well as $\{\trans{l_2},\trans{l_3}\}$.

%We also note that for any source dependent variable, it is possible to construct a chain of transition relations and
%other source dependent variables, such that by following these relations backwards through the premises

The second auxiliary definition needed for our determinism format is the definition of determinism-respecting substitutions.

\begin{definition}[\label{def::determinismRespecting}Determinism-Respecting Pairs of Substitutions]
Given a set $L$ of labels, a pair of substitutions $(\sigma,\sigma')$ is determinism-respecting w.r.t.\
a pair of sets of formulae $(\Phi, \Phi')$ and $L$ when for all two positive formulae $s \trans{l} s' \in \Phi$ and $t \trans{l} t' \in \Phi'$ such that $l \in L$, $\sigma(s) \equiv \sigma'(t)$ only if  $\sigma(s') \equiv \sigma'(t')$.
%\MAR{What is $\equiv$?}
\end{definition}

%\MAR{Would it be possible to avoid the term ``format'' for the following constraint. It is not syntactic and thus should not be called a format.}

We now have all the necessary ingredients to define a rule format that guarantees
determinism.

\begin{definition}[\label{def::detHard}Determinism Format]
A TSS $\tss$ is in the determinism format w.r.t.\ a set of labels $L$,
when for each $l \in L$ the following conditions hold.
\begin{enumerate}
    \item In each deduction rule $\frac{\Phi}{t\trans{l}t'}$, each variable $v\in \vars{t'}$ is source dependent
          via a subset of $\{ \trans{l} \mid l \in L\}$, and
    \item for each pair of distinct deduction rules $\frac{\Phi_0}{t_0\trans{l}t_0'}$ and $\frac{\Phi_1}{t_1\trans{l}t_1'}$ and
          for each determinism-respecting pair of substitutions $(\sigma, \sigma')$ w.r.t. $(\Phi_0, \Phi_1)$ and $L$
          such that $\sigma(t_0)\equiv\sigma'(t_1)$, it holds that
          either $\sigma(t_0')\equiv\sigma'(t_1')$ or $\sigma(\Phi_0)$ contradicts $\sigma'(\Phi_1)$.
\end{enumerate}
\end{definition}

The first constraint in the definition above ensures that each rule in a TSS
in the determinism format, with some $l \in L$ as its label of conclusion,
can be used to prove at most one outgoing transition for each closed term.
The second requirement guarantees that no two different rules can be used to prove two distinct
$l$-labelled transitions for any closed term.

\begin{theorem}\label{th::det}
Consider a TSS with $(C, U)$ as its least three-valued stable model and a subset $L$ of its labels. If the TSS is in the determinism format w.r.t.\ $L$, then $C$ is deterministic for each $l\in L$.
\end{theorem}
\begin{proof}
Instead of proving that $C$ is deterministic for each $l \in L$, we establish the following more general result.
We prove that, for each $l \in L$, 
\begin{equation}\label{eq:det_claim1}
    \textrm{if} \;  p \trans{l} p' \in C \cup U   \;
    \textrm{and} \; p \trans{l} p'' \in C      \quad
    \textrm{then} \quad p' \equiv p''.
\end{equation}

Assume the first two statements.
Since $p \trans{l} p' \in C \cup U$, then there exists a provable transition rule, such that
$\tss \vdash \frac{N}{p \trans{l} p'},$
for some set $N$ of negative formulae such that $C \vDash N$.
We show the claim~\eqref{eq:det_claim1} by an induction on the proof structure
for the transition rule $\frac{N}{p \trans{l} p'}$.
Let $\DR{r}$ be the last deduction rule, and $\sigma$ the substitution,
used in the proof structure for $\frac{N}{p \trans{l} p'}$.

%\AB{I need a little help in stating the induction hypothesis explicitly here, if needed.}

Similarly, since $p \trans{l} p'' \in C$,
there also exists a proof structure such that 
$\tss \vdash \frac{N'}{p \trans{l} p''}$
for some set $N'$ of negative formulae such that $C \cup U \vDash N'$.
Let $\DR{r'}$ be the last deduction rule, and $\sigma'$ the substitution,
used in the proof structure for $\tss \vdash \frac{N'}{p \trans{l} p''}$.

The proof is split in two main cases, the case when both proofs are based on the
same rule ($\DR{r} = \DR{r'}$) and the case when they are based on two distinct
rules ($\DR{r} \ne \DR{r'}$).

\paragraph{Case $\DR{r} = \DR{r'}$.}
%We first consider the case when $\DR{r}$ and $\DR{r'}$ are the same rule, 
In this case, say the rules $\DR{r}$ and $\DR{r'}$ are both the rule $\frac{\Phi}{t\trans{l}t'}$.
Obviously $\sigma(t)\equiv\sigma'(t)$ since both must be equal to $p$.
Since $\sigma(t')$ and $\sigma'(t')$ are equal to $p'$ and $p''$ respectively, 
to show our claim~\eqref{eq:det_claim1} we thus need to show that $\sigma(t') \equiv \sigma'(t')$.

% If the proof structure for $\frac{N}{p\trans{l}p'}$ consists of only one rule,
% $\DR{r}$ is an axiom and thus contains no premises. This means, since all variables
% in $t'$ are source dependent, that they must appear in the source $t$.
% Since $\sigma(t)\equiv p \equiv\sigma'(t)$ we obtain
% that $\sigma(t')\equiv\sigma'(t')$ also holds.

%If the proof structure is more complex, i.e. the set of premises $\Phi$ is non-empty.

We define the \emph{distance} of a source-dependent variable as the length of the shortest 
backward path from the variable, via premises with a label in $L$, to the variables 
in the source of conclusion. A variable in the source of the conclusion is thus 
of distance $0$.

By induction on its distance, we now show that any variable $v$, which is
source-dependent via a subset of $\{ \trans{l} \mid l \in  L\}$,
is assigned the same value by $\sigma$ and $\sigma'$, i.e. $\sigma(v)\equiv\sigma'(v)$.
The first constraint of our rule format dictates that any variable appearing in
the target $t'$ of the rule, is source-dependent via this set. Therefore if
$\sigma$ and $\sigma'$ agree on all such variables, they must also agree on $t'$.

Let $v$ be a variable appearing in the rule, which is source-dependent via some
subset of $\{ \trans{l} \mid l \in  L\}$. The base case of the induction is simple:
If the distance of $v$ is zero, this means $v$ appears in the source $t$. Since
$\sigma(t)\equiv\sigma'(t)$ it must be the case that $\sigma(v)\equiv\sigma'(v)$.

For the inductive step, assume $v$ has a non-zero distance. According
to Definition~\ref{def::varDepend} this means that $v$ appears in the target of some premise
$t_i \trans{l_i} t_i' \in \Phi$ where $l_i\in L$, and all variables appearing
in $t_i$ are also source dependent via the set $\{ \trans{l} \mid l \in L \}$.
However, each variable $w\in vars(t_i)$ has a smaller distance than $v$. By the
induction hypothesis (on variable distances), we thus have that $\sigma(w)\equiv\sigma'(w)$.

At this point, we can invoke the outer induction hypothesis, namely that of our
induction on the proof structure of $p \trans{l} p'$. Since $t_i\trans{l_i}t_i'$
is a premise of the first rule used, it must be provable with a smaller proof structure,
using either the substitution $\sigma$ or $\sigma'$.
By the induction hypothesis, the claim~\eqref{eq:det_claim1} holds for it. In other words
the target of the premise is the same whether we use $\sigma$ or $\sigma'$, i.e.
$\sigma(t_i')\equiv\sigma'(t_i')$. Since the variable $v$ appears in $t_i'$, it
must thus hold that $\sigma(v)\equiv\sigma'(v)$.

% \begin{enumerate}
%     \item Assume that $v$ appears in $t$. In this case, $\sigma(v)\equiv\sigma'(v)$ since $\sigma(t)\equiv\sigma'(t)$.
%     \item Assume that $v$ appears in the target of some premise $t_i \trans{l_i} t_i' \in \Phi$ where $l_i \in L$
%           and all variables in $t_i$ are source dependent via  a subset of $\{ \trans{l} \mid l \in  L\}$.
%           Each variable $w \in vars(t_i)$ has a distance smaller than that of $v$. Therefore, the induction
%           hypothesis (on the distance of variables) applies and we have that $\sigma(w) \equiv\sigma'(w)$.
%           This means that $\sigma(t_i)\equiv\sigma'(t_i)$. This allows
%           us to apply the induction hypothesis on the proof structure, since $\sigma(t_i \trans{l_i} t_i')$ has a proof structure
%           that is smaller than the one for $p \trans{l} p'$, to conclude that $\sigma(t'_i) \equiv \sigma'(t'_i)$.
%           Since $v$ appears in $t_i'$ it must hold that $\sigma(v)\equiv\sigma'(v)$.
% \end{enumerate}
We have thus showed that $\sigma$ and $\sigma'$ agree on the value of $v$ in all cases. 
As noted above, this holds specifically for all variables of $t'$
and we can conclude that $\sigma(t')\equiv\sigma'(t')$, or $p'\equiv p''$, 
which proves the claim~\eqref{eq:det_claim1} in the case of $\DR{r}=\DR{r'}$.

\paragraph{Case $\DR{r}\ne\DR{r'}$.}
We now consider the case where the rules $\DR{r}$ and $\DR{r'}$ are distinct.
Let $\DR{r} = \frac{\Phi}{s\trans{l}s'}$ and $\DR{r'}=\frac{\Phi'}{t\trans{l}t'}$.
We first show that the pair $(\sigma, \sigma')$ is 
determinism-respecting w.r.t.\ $(\Phi, \Phi')$ and $L$.

Assume, towards a contradiction, that the pair is not determinism-respecting.
Then there exist two positive formulae $s_i \trans{l'} s'_i$ and $t_i \trans{l'} t'_i$ 
for some $l' \in L$ among the premises of $\DR{r}$ and$\DR{r'}$, respectively, such that
$\sigma(s_i) \equiv \sigma'(t_i)$ but $\sigma(s'_i) \nequiv \sigma'(t'_i)$.
Since $s_i \trans{l} s'_i$  is a premise of $\DR{r}$, we know that $\sigma(s_i \trans{l} s'_i) \in C \cup U$ 
and it has a smaller proof structure than $p \trans{l} p' \in C \cup U$. 
Following a similar reasoning, $\sigma'(t_i \trans{l} t'_i) \in C$.
But the induction hypothesis (on the proof structure) applies and hence, we have $\sigma(s'_i) \equiv \sigma'(t'_i)$, 
which contradicts our earlier conclusion that $\sigma(s'_i) \nequiv \sigma'(t'_i)$ does not hold. 
Hence, we conclude that our assumption is false and that $(\sigma, \sigma')$ is 
determinism-respecting w.r.t.\ $(\Phi, \Phi')$ and $L$.

Since we have shown that $(\sigma, \sigma')$ is determinism respecting,
it then follows from the second condition of the determinism format that either $\sigma(s') \equiv \sigma'(t')$,
which was to be shown, or
there exist premises $\phi_i \equiv u_i \trans{l_i} u'_i$  in one deduction rule and
$\phi'_i \equiv w_i \ntrans{l_i}$ in the other deduction rule such that
$\sigma(\phi_i)$ contradicts $\sigma'(\phi'_i)$.
We show that the latter possibility leads to a contradiction, thus completing the proof.
Assume $\sigma(\phi_i)$ contradicts $\sigma'(\phi'_i)$, then we have that $\sigma(u_i) \equiv \sigma'(w_i)$.
We distinguish the following two cases based on the status of the positive and negative contradicting 
premises with respect to $\DR{r}$ and $\DR{r'}$.

\begin{enumerate}
\item
Assume that the positive formula is a premise of $\DR{r}$.
Then, $\sigma(u_i \trans{l_i} u'_i) \in C \cup U$ but  from $C \cup U \vDash N'$ and $\sigma'(\phi'_i) \in N'$, it follows that
for no $p''$, we have that $\sigma(u_i) \equiv \sigma'(w_i) \trans{l_i} p'' \in C \cup U$, thus reaching a contradiction.

\item
Assume that the positive formula is a premise of $\DR{r'}$.
Then, $\sigma'(u_i) \trans{l_i} \sigma(u'_i) \in C$ but from $C \vDash N$ and $\sigma(\phi'_i) \in N$, it follows that
for no $p_1$, we have that $\sigma(w_i) \equiv \sigma'(u_i) \trans{l_i} p_1 \in C$, hence reaching a contradiction. \qedhere
\end{enumerate}
\end{proof}

For a TSS in the determinism format with $(C, U)$ as its least three-valued stable model, $U$ and thus $C \cup U$ need not be deterministic.
The following counter-example illustrates this phenomenon.

\begin{example}
Consider the TSS given by the following deduction rules.
\[
\sosrule[]{a \trans{l} a}{a \trans{l} b} \quad \sosrule[]{a \ntrans{l}}{a \trans{l} a}
\]
The above-given TSS is in the determinism format since $a \trans{l} a$ and $a \ntrans{l}$ contradict each other (under any substitution).
Its least three-valued stable model is, however, $(\emptyset, \{a \trans{l} a, a \trans{l} b\})$ and $\{a \trans{l} a, a \trans{l} b\}$ is not deterministic.
\end{example}

\begin{corollary}
Consider a complete TSS with $L$ as a subset of its labels. If the TSS is in the determinism format w.r.t. $L$, then its defined transition relation is deterministic for each $l \in L$.
\end{corollary}

Constraint 2 in Definition \ref{def::detHard} may seem difficult to verify,
since it requires checks for all possible (determinism-respecting) substitutions.
However, in practical cases, to be quoted in the remainder of this chapter,
variable names are chosen in such a way that constraint 2
can be checked syntactically.
For example, consider the following two deduction rules.
\[
\sosrule{x \trans{a} x'}{f(x,y) \trans{a} x'} \qquad
\sosrule{y \ntrans{a} \quad x \trans{b} x'}{f(y,x) \trans{a} x'}
\]
If in both deduction rules $f(x,y)$ (or symmetrically $f(y,x)$) was used,
it could have been easily seen from the syntax of the rules that the premises of one deduction rule
always (under all pairs of substitutions agreeing on the value of $x$) contradict the premises of the other deduction rule and, hence, constraint 2 is trivially satisfied.
Based on this observation, we next present a rule format, whose constraints
have a purely syntactic form,
and that is sufficiently powerful to handle all the examples we discuss
in Section \ref{sec::detExamples}.
(Note that, for the examples in Section \ref{sec::detExamples}, checking the constraints of Definition \ref{def::detHard} is
not too hard either.)



\begin{definition}[Normalized TSSs]\label{def::normalized}
A TSS is normalized w.r.t. $L$ if each deduction rule is $f$-defining for some function symbol $f$, and for each label $l \in L$, each function symbol $f$ and each pair of deduction rules of the form
\[
\raisebox{1em }{\DR{r}}\;  \sosrule{\Phi_r}{f(\overrightarrow{s}) \trans{l} s'} \qquad
\raisebox{1em }{\DR{r'}}\; \sosrule{\Phi_{r'}}{f(\overrightarrow{t}) \trans{l} t'}
\]
the following constraints are satisfied:
\begin{enumerate}
\item the sources of the conclusions coincide, i.e., $f(\overrightarrow{s}) \equiv f(\overrightarrow{t})$,
\item each variable $v \in \vars{s'}$ (symmetrically $v \in \vars{t'}$) is source dependent in $\DR{r}$ (respectively in $\DR{r'}$) via some subset of $\{ \trans{l} \mid l \in L\}$,

\item for each variable $v \in \vars{r} \cap \vars{r'}$ there is a set of formulae in $\Phi_r \cap \Phi_{r'}$ proving its source dependency (both in $\DR{r}$ and $\DR{r'})$) via some subset of $\{ \trans{l} \mid l \in L\}$.
\end{enumerate}
\end{definition}

The second and third constraint in Definition \ref{def::simplifiedDet} guarantee that
the syntactic equivalence of relevant terms (the target of the conclusion or the premises negating each other)
will lead to syntactically equivalent closed terms under all determinism-respecting pairs of substitutions.



%\begin{definition}[Source-normalized rules]
%Given a countably infinite list $\widetilde{x} \equiv x_0, x_1, \ldots$ of variables.
%A $\tss$ contains only {\em source-normalized rules}, when all its deduction rules
%are of the form:
%\[
%\sosrule[]{\Phi}{f(\overrightarrow{x}) \trans{l} t'}
%\]
%where $\overrightarrow{x}$ is a prefix of $\widetilde{x}$.
%Henceforth, we assume a fixed yet arbitrary list $\widetilde{x} \equiv x_0, x_1, \ldots$ and only speak
%of source-normalized rules without mentioning the list.
%
%We refer to each rule with $f$ as the main operator of the source and $l$ as the label of the conclusion as an $(f,l)$\emph{-defining rule}.
%\end{definition}
%
%\MAR{We should indicate that for any TSS there is an equivalent (definition?) TSS with only source-normalized rules.}
The reader can check that all the examples quoted from the literature in Section \ref{sec::detExamples} are indeed normalized TSSs.


\begin{definition}[\label{def::simplifiedDet}Syntactic Determinism Format]
A normalized TSS is in the (syntactic) determinism format w.r.t.\ $L$, when
%\begin{enumerate}
%    \item in each deduction rule $\frac{\Phi}{t\trans{l}t'}$, each variable $v\in vars(t')$ is source dependent
%          via some subset of $\{ \trans{l} \mid l \in L\}$, and
%    \item
    for each two deduction rules $\frac{\Phi_0}{f(\overrightarrow{s}) \trans{l} s'}$ and $\frac{\Phi_1}{f(\overrightarrow{s})  \trans{l} s''}$, it holds that $s' \equiv s''$ or $\Phi_0$ contradicts $\Phi_1$.
%\end{enumerate}
\end{definition}

The following theorem states that for normalized TSSs, Definition \ref{def::simplifiedDet} implies Definition \ref{def::detHard}.


\begin{theorem}\label{th::simplifiedImpliesHard}
Each normalized TSS in the syntactic determinism format w.r.t.\ $L$ is also in the determinism format w.r.t.\ $L$.
\end{theorem}
\begin{proof}
Let $\tss$ be a normalized TSS in the syntactic determinism format w.r.t. $L$.
Condition 1
of Definition~\ref{def::detHard} is satisfied since $\tss$ is normalized. To
see this, consider item 2 of Definition~\ref{def::normalized}, by taking $\DR{r}$ and $\DR{r'}$ to be the same rule.

To prove condition 2 of Definition~\ref{def::detHard}
let $(r) = \frac{\Phi_0}{t_0\trans{l}t_0'}$ and
$(r') = \frac{\Phi_1}{t_1\trans{l}t_1'}$ be distinct rules of $\tss$ and $(\sigma,\sigma')$
be a determinism-respecting pair of substitutions w.r.t. $(\Phi_0, \Phi_1)$ and $L$
such that $\sigma(t_0) \equiv \sigma'(t_1)$. Since $\tss$ is normalized,
both $\DR{r}$ and $\DR{r'}$ are $f$-defining for some function symbol $f$, i.e., $t_0 = f(\overrightarrow{s})$
and $t_1 = f(\overrightarrow{t})$.
Furthermore, since $f(\overrightarrow{s}) \equiv f(\overrightarrow{t})$ we have that
$\sigma$ and $\sigma'$
agree on all variables appearing
in $f(\overrightarrow{s}) = f(\overrightarrow{t})$.

For each variable $v \in \vars{r} \cap \vars{r'}$, we define
its {\em common source distance} to be the source distance of $v$
when only taking the formulae in $\Phi_0 \cap \Phi_1$ into account.
Note that such a source distance exists since by constraint 3 of Definition \ref{def::normalized}
all $v \in \vars{r} \cap \vars{r'}$ are source dependent via a subset of $\{ \trans{l} \mid l \in  L\}$ included in $\Phi_0 \cap \Phi_1$.

We prove for each $v \in \vars{r} \cap \vars{r'}$ that $\sigma(v) \equiv \sigma'(v)$ by
an induction on the common source distance of variables $v$.
Suppose that we show the above claim, then we can prove the theorem as follows.
It follows from Definition \ref{def::simplifiedDet} that either $t_0' \equiv t'_1$ or $\Phi_0$ contradicts $\Phi_1$.
If $t'_0 \equiv t'_1$, then variables in $\vars{t'_0} = \vars{t'_1}$ are all source dependent via transitions in $L$
that are common to both $\Phi_0$ and $\Phi_1$
(by constraint 3 of Definition \ref{def::normalized}).
By the above-mentioned claim, $\sigma(t'_0) \equiv \sigma'(t'_1)$, thus, constraint 2 of Definition \ref{def::detHard} follows, which was to be shown.
If  $\Phi_0$ contradicts $\Phi_1$, then assume that the premises negating each other are $\phi_j \equiv s_j \trans{l_j} s'_j$ and $\phi_{j'} \equiv t_{j'} \ntrans{l_j}$ and it holds that $s_j \equiv t_{j'}$.
All variables in $t_j \equiv s_j$ are source dependent via transitions in $L$ (by constraint 3   of Definition \ref{def::normalized}).
It follows from the claim that $\sigma(s_j) \equiv \sigma'(t_{j'})$
and thus, $\sigma(\phi_j)$ contradicts $\sigma'(\phi_{j'})$, which implies constraint 2 of Definition \ref{def::detHard}.


Hence, it only remains to prove, by an induction on the common source distance of $v$, that $\sigma(v) \equiv \sigma'(v)$.
If $v\in vars(f(\overrightarrow s))$ then we know that $\sigma(v) \equiv \sigma'(v)$ (since $t_0 \equiv t_1$ and $\sigma(t_0) \equiv \sigma'(t_1)$.
Otherwise, since $v$ is source dependent in $(r)$
via transitions with labels in $L$,
there is a positive premise
$u \trans{l} u'$ in $\Phi_0$ with $l\in L$
such that $v\in vars(u')$ and all variables in $u$ are source
dependent with a shorter common source distance.
Furthermore,
since $v$ appears in both rules, i.e., $v \in vars(r) \cap vars(r')$,
this premise also appears in $\Phi_1$ according
to item 3 of Definition~\ref{def::normalized} and thus $\vars{u} \subseteq vars(r) \cap vars(r')$.
By the induction hypothesis we have that $\sigma(u)\equiv \sigma'(u)$ and since $(\sigma,\sigma')$
is determinism-respecting w.r.t. $(\Phi_0, \Phi_1)$ and $L$, we know that
$\sigma(u') \equiv \sigma'(u')$. Specifically, the substitutions
must agree on the value of $v$, i.e.
$\sigma(v) \equiv \sigma'(v)$.
\end{proof}

The following statement is a corollary to Theorems \ref{th::simplifiedImpliesHard} and \ref{th::det}.

\begin{corollary}
Consider a normalized TSS with $(C, U)$ as its least three-valued stable model and a subset $L$ of its labels. If the TSS is in the (syntactic) determinism format w.r.t.\ $L$ (according to Definition \ref{def::simplifiedDet}), then $C$ is deterministic w.r.t.\ any $l\in L$.
\end{corollary}


\subsection{Examples\label{sec::detExamples}}
In this section, we present some examples of various TSSs from the literature and apply our (syntactic) determinism format to them.
Some of the examples we discuss below are based on TSSs with predicates. The extension of our formats with predicates is
straightforward and we discuss it in Section \ref{sec::pred} to follow.


\begin{example}[Conjunctive Nondeterministic Processes]\label{cnp}
Hennessy and Plotkin, in \cite{Hennessy87},  define a language, called conjunctive nondeterministic processes,
for studying logical characterizations of processes.
The signature of the language consists of a constant $0$, a unary action prefixing operator $a.\_$ for each $a \in A$, and
a binary conjunctive nondeterminism operator $\lor$.
The operational semantics of this language is defined by the following deduction rules.
\[
\sosrule{}{0 \cando{a}}
\qquad
\sosrule{}{a.x \cando{a}}
\qquad
\sosrule{x \cando{a}}{x \lor y \cando{a}}
\qquad
\sosrule{y \cando{a}}{x \lor y \cando{a}}
\]
\[
\sosrule{}{0 \after{a} 0}
\quad
\sosrule{}{a.x \after{a} x}
\quad
\sosrule{}{a.x \after{b} 0} ~a \neq b
\quad
\sosrule{x \after{a} x' \quad y \after{a} y'}{x \lor y  \after{a} x' \lor y'}
\]
The above TSS is in the (syntactic) determinism format with respect to the transition relation $\after{a}$ (for each $a \in A$).
Hence, we can conclude that the transition relations $\after{a}$ are deterministic.
\end{example}

\begin{example}[Delayed choice]\label{dc}
The second example we discuss is a subset of the process algebra $\mathrm{BPA}_{\delta\epsilon}+\mbox{DC}$ \cite{BaetenMauw94}, i.e., Basic Process Algebra with deadlock and empty process extended with delayed choice. First we restrict attention to the fragment of this process algebra without non-deterministic choice $+$ and with action prefix $a.\_$ instead of general sequential composition $\cdot$. This altered process algebra has the following deduction rules, where $a$ ranges over the set of actions $A$:
\[
\sosrule{}{\epsilon\downarrow}
\qquad
\sosrule{}{a.x \trans{a} x}
\qquad
\sosrule{x \downarrow}{x \mp y \downarrow}
\qquad
\sosrule{y \downarrow}{x \mp y \downarrow}
\]
\[
\sosrule{x \trans{a} x' \qquad y \trans{a} y'}{x \mp y \trans{a} x' \mp y'}
\qquad
\sosrule{x \trans{a} x' \qquad y \ntrans{a}}{x \mp y \trans{a} x'}
\qquad
\sosrule{x \ntrans{a} \qquad y \trans{a} y'}{x \mp y \trans{a} y'}
\]
In the above specification, predicate $p \downarrow$ denotes the possibility of termination for process $p$.
The intuition behind the delayed choice operator, denoted by $\_\ \mp\ \_$, is that the choice between
two components is only resolved when one performs an action that the other cannot perform.
When both components can perform an action, the delayed choice between them remains unresolved and
the two components synchronize on the common action.
This transition system specification is in the (syntactic) determinism format w.r.t.\ $\{ a \mid a \in A\}$.

Addition of non-deterministic choice $+$ or sequential composition $\cdot$ results in deduction rules that do not satisfy the determinism format. For example, addition of sequential composition comes with the following deduction rules:
\[ \sosrule{x \trans{a} x'}{x \cdot y \trans{a} x' \cdot y}
\qquad
\sosrule{x \downarrow \quad y \trans{a} y'}{x \cdot y \trans{a} y'}
\]
The sets of premises of these rules do not contradict each other.
The extended TSS is indeed non-deterministic since, for example, $(\epsilon \mp (a.\epsilon))\cdot (a.\epsilon) \trans{a} \epsilon$ and
$(\epsilon \mp (a.\epsilon))\cdot (a.\epsilon) \trans{a} \epsilon \cdot (a.\epsilon)$.
\end{example}

\begin{example}[Time transitions I]\label{ex:timetrans1}
This example deals with the Algebra of Timed Processes, ATP, of Nicollin and Sifakis \cite{Nicollin94}.
In the TSS given below, we specify the time transitions (denoted by label $\chi$) of delayable deadlock $\delta$,
non-deterministic choice $\_~\oplus~\_$, unit-delay operator $\lfloor\_\rfloor\_$ and parallel composition $\_~\parallel~\_$.
\[
\sosrule{}{\delta \trans{\chi} \delta}
\qquad
\sosrule{x \trans{\chi} x' \quad y \trans{\chi} y'}{x \oplus y \trans{\chi} x' \oplus y'}
\qquad
\sosrule{}{\lfloor x \rfloor (y) \trans{\chi} y}
\qquad
\sosrule{x \trans{\chi} x' \quad y \trans{\chi} y'}{x \parallel y \trans{\chi} x' \parallel y'}
\]
These deduction rules all trivially satisfy the determinism format for time transitions since the sources of conclusions of different deduction rules cannot be unified. Also the additional operators involving time, namely, the delay operator $\lfloor \_ \rfloor^{d}\_$, execution delay operator
$\lceil \_ \rceil^{d}\_$ and unbounded start delay operator $\lfloor \_ \rfloor^{\omega}$, satisfy the determinism format for time transitions. The deduction rules are given below, for $d \geq 1$:
%start delay within d
\[
\sosrule{}{\lfloor x \rfloor^1(y) \trans{\chi} y}
\qquad
\sosrule{x \trans{\chi} x'}{\lfloor x \rfloor^{d+1} (y) \trans{\chi} \lfloor x' \rfloor^{d}(y)}
\qquad
\sosrule{x \ntrans{\chi}}{\lfloor x \rfloor^{d+1} (y) \trans{\chi} \lfloor x \rfloor^{d}(y)}
\]
%unbounded start delay
\[
\sosrule{x \trans{\chi} x'}{\lfloor x \rfloor^{\omega} \trans{\chi} \lfloor x' \rfloor^{\omega}}
\qquad
\sosrule{x \ntrans{\chi}}{\lfloor x \rfloor^{\omega} \trans{\chi} \lfloor x \rfloor^{\omega}}
\]
%execution delay within d
\[
\sosrule{x \trans{\chi} x'}{\lceil x \rceil^1(y) \trans{\chi} y}
\qquad
\sosrule{x \trans{\chi} x'}{\lceil x \rceil^{d+1} (y) \trans{\chi} \lceil x' \rceil^{d}(y)}
\]
\end{example}

\begin{example}[Time transitions II]\label{ex:timetrans2}
Most of the timed process algebras that originate from the Algebra of Communicating Processes (ACP) from \cite{Bergstra84,Baeten90} such as those reported in \cite{Baeten02} have a deterministic time transition relation as well.

In the TSS given below, the time unit delay operator is denoted by $\sigma_{\mathrm{rel}} \_$, nondeterministic choice is denoted by $\_~+~\_$,
and sequential composition is denoted by $\_ \cdot \_$.
The deduction rules for the time transition relation for this process algebra are the following:
\[ \sosrule{}{\sigma_{\mathrm{rel}}(x) \trans{1} x}
\qquad
\sosrule{x \trans{1} x' \quad y \trans{1} y'}{x + y \trans{1} x'+y'}
\qquad
\sosrule{x \trans{1} x' \quad y \ntrans{1} }{x + y \trans{1} x'}
\qquad
\sosrule{x \ntrans{1} \quad y \trans{1} y'}{x + y \trans{1} y'}
\]
\[
\sosrule{x \trans{1} x' \quad x \not\downarrow}{x \cdot y \trans{1} x' \cdot y}
\qquad
\sosrule{x \trans{1} x' \quad y \ntrans{1}}{x \cdot y \trans{1} x' \cdot y}
\qquad
\sosrule{x \trans{1} x' \quad x \downarrow \quad y \trans{1} y'}{x \cdot y \trans{1} x' \cdot y + y'}
\]
\[
\sosrule{x \ntrans{1} \quad x \downarrow \quad y \trans{1} y'}{x \cdot y \trans{1} y'}
\]
Note that here we have an example of deduction rules, the first two deduction rules for time transitions of a sequential composition, for which the premises do not contradict each other. Still these deduction rules satisfy the determinism format since the targets of the conclusions are identical. In the syntactically richer framework of \cite{Reniers08}, where arbitrary first-order logic formulae over transitions are allowed, those two deduction rules are presented by a single rule with premise $x \trans{1} x' \land (x \not\downarrow \lor y \ntrans{1})$.

Sometimes such timed process algebras have an operator for specifying an arbitrary delay, denoted by $\sigma^*_{\mathrm{rel}} \_$, with the following deduction rules.
\[ \sosrule{x \ntrans{1}}{\sigma^*_{\mathrm{rel}}(x) \trans{1} \sigma^*_{\mathrm{rel}}(x)}
\qquad
\sosrule{x \trans{1} x'}{\sigma^*_{\mathrm{rel}}(x) \trans{1} x' + \sigma^*_{\mathrm{rel}}(x)}
\]
The premises of these rules contradict each other and so, the semantics of this operator also satisfies the constraints of our (syntactic)
determinism format.
\end{example}

\section{\label{sec:idempotency}Idempotency}
Our order of business in this section is to present a rule format
that guarantees the idempotency of certain binary operators.
In the definition of our rule format, we rely implicitly on the work presented in the previous section.


\subsection{Format}

\begin{definition}[Idempotency]
\label{def:idem}
A binary operator $f \in \Sigma$ is \emph{idempotent w.r.t.\ an equivalence} $\sim$ on closed terms if and only if  for each $p \in \CTerms{\Sigma}$, it holds that $f(p,p) \sim p$.
\end{definition}

Idempotency is  defined with respect to a notion of behavioral equivalence.
There are various notions of behavioral equivalence defined in the literature,
which are by and large, weaker than bisimilarity defined below.
Thus, to be as general as possible, we prove our idempotency result
for all notions that contain, i.e., are weaker than, bisimilarity.


\begin{definition}[Bisimulation]
\label{def:bisim}
Let $\tss$ be a TSS with signature $\Sigma$.
A relation $\rel \subseteq \CTerms{\Sigma} \times \CTerms{\Sigma}$ is a \emph{bisimulation relation}
if and only if $\rel$ is symmetric
%\leaveout{(i.e.\ $\forall_{p_0,p_1 \in \CTerms{\Sigma}}~p_0 \rel  p_1 \Leftrightarrow p_1\rel p_0$)}
and for all $p_0,p_1,p'_0 \in \CTerms{\Sigma}$ and $l \in L$
$$(p_0 \Rel p_1 \wedge \tss \vdash p_0 \trans{l} p_0') \Rightarrow \exists_{p_1' \in \CTerms{\Sigma}} (\tss \vdash p_1 \trans{l} p_1'\wedge p_0' \Rel p_1' ).$$
Two terms $p_0, p_1 \in \CTerms{\Sigma}$ are called \emph{bisimilar}, denoted by $p_0 \bisim p_1$, when there exists a bisimulation relation $\Rel$ such that
$p_0 \rel p_1$.
\end{definition}


\begin{definition}[The Idempotency Rule Format]
\label{def::format}
Let $\gamma: L \times L \rightarrow L$ be a partial function such that $\gamma(l_0,l_1) \in \{l_0,l_1\}$ if it is defined.
We define the following two rule forms.

\paragraph{$1_l$. Choice rules}
\[
    \sosrule[i\in\{0,1\}]{\{x_i\trans{l}t\}\cup\Phi}{f(x_0,x_1)\trans{l}t} \\
\]
\paragraph{$2_{l_0,l_1}$. Communication rules}
\[
    \sosrule[t_0\equiv t_1 \mbox{ or } (l_0=l_1 \mbox{ and }  \trans{l_0}\textrm{ is deterministic } )]{\{x_0\trans{l_0}t_0,\, x_1\trans{l_1}t_1\}\cup\Phi}{f(x_0,x_1)\trans{\gamma(l_0,l_1)}f(t_0,t_1)}
\]
In each case, $\Phi$ can be an arbitrary, possibly empty set of (positive or negative) formulae.

In addition, we define the starred version of each form, $1_l^*$ and $2_{l_0,l_1}^*$. The starred version of each rule is the
same as the unstarred one except that $t, t_0$ and $t_1$ are restricted to be concrete variables and the set $\Phi$ must be
empty in each case.

A TSS is in \emph{idempotency format w.r.t. a binary operator $f$} if each \mbox{$f$-defining} rule, i.e., a deduction rule with $f$ appearing in the source of the conclusion, is of the forms $1_l$
or $2_{l_0,l_1}$, for some $l, l_0, l_1 \in L$, and for each label $l \in L$ there exists at least one rule of the forms $1_l^*$ or $2_{l,l}^*$.
\end{definition}

We should note that the starred versions of the forms are special cases of their unstarred counterparts; for example a rule
which has form $1_l^*$ also has form $1_l$.


\begin{theorem}\label{thm:idempotent}
Assume that  a TSS is complete and is in the idempotency format with respect to a binary operator $f$. Then,  $f$ is idempotent w.r.t. to
any equivalence $\sim$ such that  $\bisim \subseteq{\sim}$.
% I.e. for all $p\in\CTerms\Sigma$ it holds that $f(p,p) \sim p$.
\end{theorem}
\begin{proof}
First define the relation $\eqidem \subseteq \CTerms\Sigma \times \CTerms\Sigma$ as follows.
\[
    \eqidem = \{ (p,p), (p,f(p,p)), (f(p,p),p) \,|\, p \in \CTerms\Sigma \}
\]
To prove the theorem it suffices to show that $\eqidem$ is a bisimulation relation. If it is, then $f(p,p)\bisim p$ for any closed term $p$ and since $\bisim \subseteq \sim$ we obtain the theorem.

Let $(C,U)$ be the least three-valued stable model for the TSS under consideration.
First consider a closed term $p$ s.t. $p \trans{l} p' \in C$ for some $l$ and $p'$ (note that $U=\emptyset$ since the TSS is complete).
Next, we argue that $f(p,p) \trans{l} p''$ for some $p''$ such that  $p' \eqidem p''$.
Since $p \trans{l} p' \in C$, there exists a provable transition rule of the form $\frac{N}{p \trans{l} p'}$ for some set of negative formulae $N$ such that $C \vDash N$.
In particular, that means that $p \ntrans{l} \notin N$.
In this case we make use of the requirement that there exists at least one rule of a starred form for label $l$. If there exists a
rule of the form $1_l^*$, i.e.
\[
    \sosrule[i\in\{0,1\}]{x_i\trans{l}x'}{f(x_0,x_1)\trans{l}x'}
\]
then we can instantiate it to prove that $f(p,p)\trans{l}p'\in C$.
In particular, it does not matter if $i=0$ or $i=1$.
Since $\eqidem$ is reflexive, $p'\eqidem p'$ holds.
If there exists a rule of the form $2_{l,l}^*$, we observe that $\gamma(l,l) = l$ so the transition rule becomes
\[
    \sosrule{x_0\trans{l}x_0' \quad x_1\trans{l}x_1'}{f(x_0,x_1) \trans{l} f(x_0',x_1')},
\]
where $x'_0 \equiv x'_1$ or $\trans{l}$ is deterministic.
Now we can use the existence of $p\trans{l}p'$ to satisfy both premises and obtain that $f(p,p)\trans{l}f(p',p')$.
By the definition of $\eqidem$ we also have that $p' \eqidem f(p',p')$.
In either case, if $p \trans{l} p'\in C$, then there exists a $p''$ s.t. $f(p,p) \trans{l} p'' \in C$ and $p' \eqidem p''$.

Now assume that $f(p,p) \trans{k} p' \in C$. Then there exists a provable transition rule $\frac{N}{f(p,p) \trans{k} p'}$
for some set of negative formulae $N$ such that $C\vDash N$. Since all rules for $f$ are either of the form $1_l$ or $2_{l_0,l_1}$,
this provable transition rule must be based on a rule of those forms. We analyze each possibility separately, showing that
in each case $p \trans{k} p''$ for some $p''$ such that $p' \eqidem p''$.

If the rule is based on a rule of form $1_l$, its positive premises must also be provable. In particular it must hold that
$p\trans{k} p' \in C$ since both $x_0$ and $x_1$ in the rule are instantiated to $p$. The other premises are of no
consequence to this conclusion and, again, we observe that $p'\eqidem p'$.

Now consider the case where the transition is a consequence of a rule of the form $2_{l_0,l_1}$.
If $t_0\equiv t_1$, say both are equal to $p''$, we must consider two cases, namely $k=l_0$ and $k=l_1$.
If $k=l_0$ then the first premise of the rule actually states that $p\trans{k}p''$.
If $k=l_1$ then the second premise similarly states that $p\trans{k}p''$.
In either case, we note that $p'\equiv f(p'',p'')$ must hold and again by the definition of $\eqidem$ we have
that $f(p'',p'')\eqidem p''$.

If however $t_0\not\equiv t_1$ the side condition requires that $l_0=l_1 = k$, which also implies $\gamma(l_0,l_1)=l_0=k$,
and that the transition relation $\trans{l_0}$ is deterministic.
In this case it is easy to see that the right-hand sides of the first two premises, namely $t_0$ and $t_1$, evaluate to
the same closed term in the proof structure, say $p''$.
The conclusion then states that $k=l_0$ and $p'\equiv f(p'',p'')$.
It must thus hold that $p\trans{k}p''\in C$ and $f(p'',p'')\eqidem p''$ as before.

From this we obtain that if $f(p,p) \trans{k} p' \in C$ then there exists
a $p''$ such that $p \trans{k} p'' \in C$ and $p' \eqidem p''$.
Thus, $\eqidem$ is a bisimulation.
\end{proof}


\subsection{Relaxing the restrictions} % (fold)

In this section we consider the constraints of the idempotency rule format (Definition~\ref{def::format}) and 
show that they cannot be dropped without jeopardizing the meta-theorem.

% Rule forms $1_l$ and $2_l$ only allow for variables in the targets. To see why arbitrary terms are not allowed, consider the
% following rule where $a$ is a process term.
% \[
%     \sosrule{x \trans{l} a}{f(x,y) \trans{l} a}
% \]
% If this is the only \mbox{$f$-defining} rule and the only outgoing transition for $a$ is $a \trans{l} b$, then the processes
% $a$ and $f(a,a)$ are not bisimilar as the former can do an \mbox{$l$-transition} but the latter is stuck.

First of all we note that, in rule form $1_l$, it is necessary that the label of the premise matches
the label of the conclusion.
If it does not, in general, we cannot prove that $f(p,p)$ simulates $p$ or vice versa.
This requirement can be stated more generally for both rule forms in Definition \ref{def::format};
the label of the conclusion must be among the labels of the premises.
The requirement that $\gamma(l,l') \in \{l,l'\}$ exists to ensure this constraint for form $2_{l,l'}$.
A simple synchronization rule provides a counter-example that shows why this restriction is needed.
Consider the following TSS with constants $0$, $\tau$, $a$ and $\bar{a}$ and two binary operators $+$ and $\parallel$.
\[
\begin{array}{c}
    \sosrule{}{\alpha \trans{\alpha} 0} \quad
    \sosrule{x \trans{\alpha} x'}{x + y \trans{\alpha} x' } \quad \sosrule{y \trans{\alpha} y'}{x + y \trans{\alpha} y' }
    \quad \sosrule{x \trans{a} x' \quad  y \trans{\bar{a}} y'}{x \parallel y \trans{\tau} x' \parallel y'}
\end{array}
\]
where $\alpha$ is $\tau$, $a$ or $\bar{a}$.
Here it is easy to see that although
$(a + \bar{a}) \parallel (a + \bar{a})$ has an outgoing \mbox{$\tau$-transition}, $a + \bar{a}$ does not afford such a transition.

The condition that for each $l$ at least one rule of the forms $1_l^*$ or $2_{l,l}^*$ must exist comprises a few constraints on the rule format.
First of all, it says there must be at least one \mbox{$f$-defining} rule.
If not, it is easy to see that there could exist a process $p$ where $f(p,p)$
deadlocks (since there are no \mbox{$f$-defining} rules) but $p$ does not.
It also states that there must be at
least one rule in the starred form,
where the targets are restricted to variables.
To motivate these constraints, consider the following TSS.
\[
    \sosrule{}{a \trans{a} 0} \quad
    \sosrule{x \trans{a} a}{f(x,y) \trans{a} a}
\]
The processes
$a$ and $f(a,a)$ are not bisimilar as the former can do an \mbox{$a$-transition} but the latter is stuck.
The starred forms also require that $\Phi$ is empty, i.e. there is no testing. This is necessary in the proof because in the presence
of extra premises, we cannot in general instantiate such a rule to show that $f(p,p)$ simulates $p$.
Finally, the condition requires that if we rely on a rule of the form $2_{l,l'}^*$
and $t_0 \equiv\!\!\!\!\!/~ t_1$, then the labels $l$ and $l'$ in the premises of the rule must coincide. To
see why, consider a TSS containing a {\em left synchronize} operator $\rrfloor$, one that synchronizes a step from each operand
but uses the label of the left one. Here we let $\alpha\in\{a,\bar{a}\}$.
\begin{equation*}
    \sosrule{}{\alpha \trans{\alpha} 0} \quad
    \sosrule{x \trans{\alpha} x'}{x + y \trans{\alpha} x' } \quad \sosrule{y \trans{\alpha} y'}{x + y \trans{\alpha} y' } \quad
    \sosrule{x \trans{a} x' \quad y \trans{\bar{a}} y'}{x\rrfloor\, y \trans{a} x'\rrfloor\, y'}
\end{equation*}
In this TSS the processes $(a + \bar{a})$ and $(a + \bar{a})\rrfloor\, (a + \bar{a})$ are not bisimilar
since the latter does not
afford an $\bar{a}$-transition whereas the former does.

% \MAR{Can the above problem be solved by requiring presence of a rule $5_{l',l}$? In the example, presence of a deduction rule
% \begin{equation}
%     \sosrule[]{x \trans{l'} x' \quad y \trans{l} y'}{f(x,y) \trans{l'} f(x',y')}
% \end{equation}
% solves the problem (given the requirement that $x' \equiv y'$!}
%
% \AB{Yes, I believe it can. The condition would then be: $1_l^* \lor 2_{l,l}^* \lor (2_{l,l'}^* \land 2_{l',l}^*)$. Should we go
% with that instead, or present it as an extension later?}

% AB: I liked this example of the trace operator, but it doesn't really apply to the new format
%
% For rules of type $3_{l,l'}$ we require that $\gamma(l,l')=l$ or that $t\equiv t'$. If neither of these conditions hold the format supports the following rule, defining the {\sf trace} operator found in some functional programming languages.
% \[
% \sosrule{x \trans{l} x' \quad y \trans{l'} 0}{\textsf{trace}(x,y) \trans{l'} x'}
% \]
% The idea here is that $\textsf{trace}(p,q)$ performs the input and output of $q$ while evaluating to the value of $p$.
% This operator  is not idempotent as witnessed by the (CCS style) process $p = a.0 + b.c.0$, because while
% $\textsf{trace}(p,p) \trans{a} c.0$ the only possible $a$-successor of $p$ is the nil process. The similar constraint
% on form $4_{l,l'}$ is necessary for the same reasons.

For rules of form $2_{l,l'}$ we require that either $t_0\equiv t_1$, or that the mentioned labels are the same and the
associated transition
relation is deterministic. This requirement is necessary in the proof to ensure that the
target of the conclusion fits our definition of $\eqidem$, i.e. the operator is applied to two identical terms.
Consider the following TSS where $\alpha\in\{a,b\}$.
\[
    \sosrule{}{a\trans{a} a} \quad
    \sosrule{}{a\trans{a} b} \quad
    \sosrule{}{b\trans{b} b} \quad
    \sosrule{x\trans\alpha x' \quad y\trans\alpha y'}{x | y \trans\alpha x' | y'}
\]
For the operator $|$, this violates the condition $t_0\equiv t_1$ (note that $\trans{a}$ is not deterministic).
We observe that $a|a \trans{a} a|b$. The only possibilities for $a$ to simulate this $a$-transition
is either with $a\trans{a}a$
or with $a\trans{a}b$. However, neither $a$ nor $b$ can be bisimilar to $a|b$ because both $a$ and $b$ have outgoing
transitions while $a|b$ is stuck. Therefore $a$ and $a|a$ cannot be bisimilar.
%
If $t_0 \not\equiv t_1$ we must require that the labels match, $l_0 = l_1$, and that $\trans{l_0}$ is deterministic.
We require the labels to match because if they do not, then given only $p\trans{l}p'$ it is impossible to prove that $f(p,p)$
can simulate it using only a $2_{l,l'}^*$ rule. The determinacy of the transition with that label
is necessary when proving that transitions from $f(p,p)$ can, in general, be simulated by $p$;
if we assume that $f(p,p)\trans{l}p'$ then we must be able to conclude that $p'$ has the shape $f(p'',p'')$ for some $p''$,
in order to meet the bisimulation condition for $\eqidem$. Consider the standard choice operator $+$
and prefixing operator $.$ of CCS
with the $|$ operator from the last example, with $\alpha\in\{a,b,c\}$.
\[
    \sosrule{}{\alpha \trans{\alpha} 0} \quad
    \sosrule{}{\alpha.x \trans{\alpha} x} \quad
    \sosrule{x \trans{\alpha} x'}{x + y \trans{\alpha} x' } \quad \sosrule{y \trans{\alpha} y'}{x + y \trans{\alpha} y' } \quad
    \sosrule{x\trans\alpha x' \quad y\trans\alpha y'}{x | y \trans\alpha x' | y'}
\]
If we let $p = a.b + a.c$, then $p|p \trans{a} b|c$ and $b|c$ is stuck.
However, $p$ cannot simulate this transition w.r.t. $\eqidem$.
Indeed, $p$ and $p|p$ are not bisimilar.


% subsection relaxing_the_restrictions (end)

\subsection{Predicates\label{sec::pred}}

There are many examples of TSSs where predicates are used.
The definitions presented in Section \ref{sec::pre} and  \ref{sec:idempotency} can be easily adapted to deal with predicates as well.
In particular, two closed terms are called bisimilar in this setting when,
in addition to the transfer conditions of bisimilarity, they satisfy the same predicates.
%Note that the notion of bisimulation (Definition \ref{def:bisim}), and thus the notion of idempotency (Definition \ref{def:idem}), are
%extended naturally to the setting with predicates,
%by requiring that for each two closed terms and each predicate, one term satisfies the predicate if and only if the other one satisfies the predicate.
To extend the idempotency rule format to a setting with predicates, the following types of rules for predicates are introduced:

\paragraph{$3_P$. Choice rules for predicates}
\[
    \sosrule[i\in\{0,1\}]{\{P x_i\}\cup\Phi}{P f(x_0,x_1)} \\
\]
\paragraph{$4_{P}$. Synchronization rules for predicates}
\[
    \sosrule{\{P x_0,\, P x_1\}\cup\Phi}{P f(x_0,x_1)}
\]

As before, we define the starred version of these forms, $3_P^*$ and $4_{P}^*$. The starred version of each rule is the same as the unstarred one except that the set $\Phi$ must be
empty in each case. With these additional definition the idempotency format is defined as follows.

A TSS is in \emph{idempotency format w.r.t. a binary operator $f$} if each \mbox{$f$-defining} rule, i.e., a deduction rule with $f$ appearing in the source of the conclusion, is of one the forms $1_l$, $2_{l_0,l_1}$, $3_P$ or $4_P$ for some $l, l_0, l_1 \in L$, for each label $l \in L$  and predicate symbol $P$. Moreover, for each $l \in L$, there exists at least one rule of the forms $1_l^*$ or $2_{l,l}^*$, and for each predicate symbol $P$ there is a rule of the form $1^*_P$ or $2_P^*$.

\subsection{Examples} % (fold)

\begin{example}
The most prominent example of an idempotent operator is non-deterministic choice, denoted $+$. It typically has the following deduction rules:
\[
\sosrule{x_0 \trans{a} x_0'}{x_0 + x_1 \trans{a} x_0'} \qquad
\sosrule{x_1 \trans{a} x_1'}{x_0 + x_1 \trans{a} x_1'}
\]
Clearly, these are in the idempotency format w.r.t.\ $+$.
\end{example}

\begin{example}[External choice]
The well-known external choice operator $\Box$ from CSP \cite{Hoare85} has the following deduction rules
\[
\sosrule{x_0 \trans{a} x'_0}{x_0 \Box x_1 \trans{a} x'_0}
\quad
\sosrule{x_1 \trans{a} x'_1}{x_0 \Box x_1 \trans{a} x'_1}
\quad
\sosrule{x_0 \trans{\tau} x'_0}{x_0 \Box x_1 \trans{\tau} x'_0 \Box x_1}
\quad
\sosrule{x_1 \trans{\tau} x'_1}{x_0 \Box x_1 \trans{\tau} x_0 \Box x'_1}
\]
Note that the third and fourth deduction rule are not instances of any of the allowed types of deduction rules.
Therefore, no conclusion about the validity of idempotency can be drawn from our format.
In this case this does not point to a limitation of our format,
because this operator is not idempotent in strong bisimulation semantics \cite{dArgenio95}.
\end{example}

\begin{example}[Strong time-deterministic choice]
The choice operator that is used in the timed process algebra ATP \cite{Nicollin94} has the following deduction rules.
\[
\sosrule{x_0 \trans{a} x'_0}{x_0 \oplus x_1 \trans{a} x'_0}
\qquad
\sosrule{x_1 \trans{a} x'_1}{x_0 \oplus x_1 \trans{a} x'_1}
\qquad
\sosrule{x_0 \trans{\chi} x'_0 \quad x_1 \trans{\chi} x'_1}{x_0 \oplus x_1 \trans{\chi} x'_0 \oplus x'_1}
\]
The idempotency of this operator follows from our format since the last rule for $\oplus$ fits
the form $2_{\chi,\chi}^*$ because, as we remarked in Example~\ref{ex:timetrans1}, the transition relation
$\trans{\chi}$ is deterministic.
\end{example}

\begin{example}[Weak time-deterministic choice]
The choice operator $+$ that is used in most ACP-style timed process algebras~\cite{Baeten02} has the following deduction rules:
\[
\sosrule{x_0 \trans{a} x_0'}{x_0 + x_1 \trans{a} x_0'} \qquad
\sosrule{x_1 \trans{a} x_1'}{x_0 + x_1 \trans{a} x_1'}\]
\[\sosrule{x_0 \trans{1} x'_0 \quad x_1 \trans{1} x'_1}{x_0 + x_1 \trans{1} x'_0 + x'_1} \qquad
\sosrule{x_0 \trans{1} x'_0 \quad x_1 \ntrans{1}}{x_0 + x_1 \trans{1} x'_0} \qquad
\sosrule{x_0 \ntrans{1}  \quad x_1 \trans{1} x'_1}{x_0 + x_1 \trans{1} x'_1}
\]
The third deduction rule is of the form $2^*_{1,1}$, the others are of forms $1^*_a$ and $1^*_1$.
This operator is idempotent (since the transition relation $\trans{1}$ is deterministic, as remarked
in Example~\ref{ex:timetrans2}).
\end{example}

\begin{example}[Conjunctive nondeterminism]
The operator $\lor$ as defined in Example \ref{cnp} by means of the deduction rules
\[
%\sosrule{}{0 \cando{a}}
%\qquad
%\sosrule{}{a.x \cando{a}}
%\qquad
\sosrule{x \cando{a}}{x \lor y \cando{a}}
\qquad
\sosrule{y \cando{a}}{x \lor y \cando{a}}
%\]
%\[
%\sosrule{}{0 \after{a} 0}
%\quad
%\sosrule{}{a.x \after{a} x}
%\quad
%\sosrule{}{a.x \after{b} 0} ~a \neq b
\quad
\sosrule{x \after{a} x' \quad y \after{a} y'}{x \lor y  \after{a} x' \lor y'}
\]
satisfies the idempotency format (extended to a setting with predicates). The first two deduction rules are of the form $3^*_{\cando{a}}$ and the last one is of the form $2^*_{a,a}$. Here we have used the fact that the transition relations $\after{a}$ are deterministic as concluded in Example \ref{cnp}.
\end{example}

\begin{example}[Delayed choice]
Delayed choice can be concluded to be idempotent in the restricted setting without $+$ and $\cdot$ by using the idempotency format and the fact that in this restricted setting the transition relations $\trans{a}$ are deterministic. (See Example \ref{dc}.)
\[
\sosrule{x \downarrow}{x \mp y \downarrow}
\qquad
\sosrule{y \downarrow}{x \mp y \downarrow}
\qquad
\sosrule{x \trans{a} x' \quad y \trans{a} y'}{x \mp y \trans{a} x' \mp y'}
\]
\[
\sosrule{x \trans{a} x' \quad y \ntrans{a}}{x \mp y \trans{a} x'}
\qquad
\sosrule{x \ntrans{a} \quad y \trans{a} y'}{x \mp y \trans{a} y'}
\]
The first two deduction rules are of form $3^*_{\downarrow}$, the third one is a $2^*_{a,a}$ rule, and the others are $1_{a}$ rules. Note that for any label a starred rule is present.

For the extensions discussed in Example \ref{dc} idempotency cannot be established using our rule format
since the transition relations are no longer deterministic. In fact, delayed choice is not idempotent in these cases.
\end{example}
% subsection examples (end)

\section{\label{sec::conc}Conclusions}
In this paper, we presented two rule formats  guaranteeing determinism of certain transitions and idempotency of binary operators.
Our rule formats cover all practical cases of determinism and idempotency that we have thus far encountered in the literature.

We plan to extend our rule formats with the addition of data/store.
Also, it is interesting to study the addition of structural congruences pertaining to idempotency
to the TSSs in our idempotency format.  

\bibliographystyle{plainnat}
\bibliography{references}

\end{document}
